% =============================================================================

% PACKAGES ---------------------------------------------------------------- {{{

% DOCUMENT CLASS , ENCODING , TITLE -------------------------------------- {{{

\documentclass[a4paper,twocolumn,10pt]{article}
\setlength{\columnsep}{20pt}
%\setlength{\columnseprule}{1pt}
\usepackage[utf8]{inputenc}
\usepackage[english]{babel}

\title{This will be some super impressive title}
\author{Ishan Tiwathia}
\date{\today}

% }}}
% PAGE DIMENSIONS -------------------------------------------------------- {{{


\usepackage{courier}

\usepackage[document]{ragged2e} % Text alignment package
\usepackage{enumitem}           %
\usepackage{setspace}           %
\singlespacing                  %

% \onehalfspacing
% \doublespacing
% \setstretch{1.1}

\usepackage{geometry}           %
\geometry{
	a4paper,
	total = {170mm,257mm},
	top=20mm,
	left=20mm,
	right=20mm,
	bottom=20mm
}

\usepackage{verbatim}
\makeatletter
\newcommand{\verbatimfont}[1]{\def\verbatim@font{#1}}%
\makeatother%\verbatimfont{courier}

% }}}
% HYPERLINKS ------------------------------------------------------------- {{{

\usepackage{hyperref} % auto hyperlinks toc , refrences
                      % others can be manually specified

\hypersetup{
colorlinks = true,
linktoc    = all,
citecolor  = purple,
filecolor  = black,
linkcolor  = black,
urlcolor   = black
}


% }}}
% HEADER , FOOTER -------------------------------------------------------- {{{

\usepackage{fancyhdr} % allows for header and footer customizations
\pagestyle{fancy}     %
\fancyhf{}            %

\renewcommand{\headrulewidth}{0.2pt} % draw line at header
\lhead{\textit{\leftmark}}           % LEFT  : show section name at header
\rhead{\textit{\thepage}}            % RIGHT : Show page number
% \chead{ }

\renewcommand{\footrulewidth}{0pt} % draw line at footer
%\lfoot{\textit{Last Edited : \today}}
%\cfoot{center foot}
%\rfoot{\textit{tiwathia \thepage}}

\pagenumbering{arabic} % Specify type of number characters to use

% }}}
% COLORS ----------------------------------------------------------------- {{{

\usepackage  {xcolor, colortbl}

% Section Colors {{{

\definecolor {section-bg}         {HTML} { 767676}
\definecolor {subsection-bg}      {HTML} { aaaaaa}
\definecolor {subsubsection-bg}   {HTML} { dddddd}
\definecolor {section-font}       {RGB}  { 0,0,0}
\definecolor {subsection-font}    {RGB}  { 0,0,0}
\definecolor {subsubsection-font} {RGB}  { 0,0,0}

% }}}

% Table Colors {{{

\definecolor {cell-lightblue}     {HTML} { b2cce5}
\definecolor {cell-lightgray}     {HTML} { d8deda}
\definecolor {cell-lightorange}   {HTML} { F6C396}
\definecolor {cell-lightred}      {HTML} { F1A099}
\definecolor {cell-lightgreen}    {HTML} { C6DA7F}
\definecolor {cell-lightpurple}   {HTML} { CCB2E5}
\definecolor {cell-lightyellow}   {HTML} { FFF09A}

% }}}

% Tcolorbox Tables {{{

\definecolor {defn-bg}       {HTML} {F2F3F4}
\definecolor {defn-title}    {HTML} {A9A9A9}
\definecolor {defn-theword}  {HTML} {EDA72D}

\definecolor {note-bg}       {HTML} {F2F3F4}
\definecolor {note-theword}  {HTML} {E25A22}

\definecolor {table-bg}      {HTML} {F2F3F4}
\definecolor {table-title}   {HTML} {A9A9A9}
\definecolor {table-theword} {HTML} {9ACD32}

\definecolor {image-bg}      {HTML} {F2F3F4}
\definecolor {image-title}   {HTML} {A9A9A9}
\definecolor {image-theword} {HTML} {0067A5}

% }}}

\definecolor {gray-dark}          {RGB}  { 71,77,80}
\definecolor {gray-medium}        {RGB}  { 95,103,107}
\definecolor {gray-light}         {RGB}  { 228,230,221}

\definecolor {green-goethe}       {RGB}  { 160,200,20}
\definecolor {green-goethe-light} {RGB}  { 219,243,134}

% }}}
% TABLES & IMAGES -------------------------------------------------------- {{{

\usepackage{booktabs}   %
\usepackage{multirow}   %
\usepackage{tabularx}   % allow tables to stretch to page length
%\usepackage{longtable}  % allows tables to span pages
%\usepackage{ltablex}    % combination of longtable and tabularx
\usepackage{xtab} % allows page breaking tables inline

\usepackage{graphicx}   %
\usepackage{subcaption} %
\usepackage{wrapfig}    % wrap images around text
\usepackage{capt-of}    % define captions independent of figures
\usepackage{float}
\usepackage{varwidth}
\graphicspath{{images/}} % define folder for images

% }}}
% CODE / CODE DISPLAY ---------------------------------------------------- {{{

\usepackage{listings}


\definecolor{codebackground}{RGB}{239,239,239}
\definecolor{codecomments}{RGB}{169,169,169}
\definecolor{codekeyword}{RGB}{249,38,114}
\definecolor{codestrings}{HTML}{ECE47E}
\definecolor{coderegular}{RGB}{39,40,34}

\lstset{
basicstyle       = \footnotesize\ttfamily,
backgroundcolor  = \color{codebackground},
commentstyle     = \color{codecomments}, % comment style
keywordstyle     = \color{codekeyword},  % keyword style
stringstyle      = \color{codestrings},  % string literal style
rulecolor        = \color{black},        % if not set, the frame-color may be changed on line-breaks
frame            = single,               % adds a frame around the code
basicstyle       = \footnotesize,        % the size of the fonts that are used for the code
keepspaces       = true,                 % keeps spaces in text, useful for keeping indentation of code
tabsize          = 2,                    % sets default tabsize
breaklines       = true,                 % sets automatic line breaking
captionpos       = b,                    % sets the caption-position to bottom
numbers          = left,
numberstyle      = \tiny,
numbersep        = 10pt,
frame            = tb,
columns          = fixed,
showstringspaces = false,
showtabs         = false,
keepspaces,
% escapeinside={\%*}{*)},  % if you want to add LaTeX within your code
framextopmargin=10pt,    % margin for the top background border
framexbottommargin=10pt, % margin for the bot background border
framexleftmargin=0pt,    % margin for the left background border
framexrightmargin=0pt    % margin for the right background border
}


% }}}
% MATH / GRAPHING  ------------------------------------------------------- {{{

\usepackage{amsmath} % basic math package
\usepackage{amssymb} % allows more math symbols
\usepackage{amsthm}  % allows custom therorem,defn,corll etc... definitions

\newtheorem{mydef}{DEFINITION}[section]
\newtheorem{myimage}{IMAGE}[section]
\newtheorem{mytable}{TABLE}[section]

% TIKZ -------------------------------------------------------------------- {{{
\usepackage{tikz}
% }}}

% }}}
% SECTION TITLES --------------------------------------------------------- {{{

\usepackage[export]{adjustbox}
\usepackage{changepage}
\usepackage[compact,explicit]{titlesec} % Allows customization of section head

% compact : reduces spaces before and after sections
% explicit : allows for expicit positioning of title statement with #1


% title_format : command,shape,format,label,sep,before,after
% title-brackets : {}[]{}{}{}{}[] , basically only shape and aftercode have []

% Section Title Settings {{{
\titleformat {\section}
	[hang]
	{\color{black}\Large\bfseries}
	{}
	{0em}
	{
	\nolinenumbers
	\begin{section-box}
		\thesection. #1
	\end{section-box}
	}
[
\linenumbers
]

% left before-sep after-sep right-sep
\titlespacing{\section}{0cm}{0cm}{0cm}[0cm]

% }}}

% Sub-Section Title Settings {{{
\titleformat {\subsection}
	[hang]
	{\color{black}\small\bfseries}
	{}
	{0em}
	{
	\nolinenumbers
	\begin{subsection-box}
		\thesubsection. #1
	\end{subsection-box}
	}
[
\linenumbers
]

% left before-sep after-sep right-sep
\titlespacing*{\subsection}{0cm}{0cm}{0cm}[0em]

% }}}

% Sub-Sub-Section Title Settings {{{
\titleformat {\subsubsection}
	[hang]
	{\color{black}\small\bfseries}
	{}
	{0em}
	{
	\nolinenumbers
	\begin{subsubsection-box}
		\thesubsubsection. #1
	\end{subsubsection-box}
	}
[
\linenumbers
]

% left before-sep after-sep right-sep
\titlespacing{\subsubsection}{0cm}{0cm}{0cm}[0em]

% }}}

% }}}
% BIBLIOGRAPHY / FOOTNOTES ----------------------------------------------- {{{

\usepackage[nottoc,numbib]{tocbibind} 
\renewcommand{\thefootnote}{\roman{footnote}} % footnote style
% }}}
% TCOLORBOX -------------------------------------------------------------- {{{

\usepackage[skins,breakable]{tcolorbox}
% skins allows use of enhanced options
% breakable allows breaking boxes between pages

% 	IMAGE TABLES (TCOLORBOX) {{{

\newtcolorbox{image-bg}[2][]{
	enhanced,
	colback           = image-bg,
	colframe          = image-bg,
	fonttitle         = \bfseries,
%	width             = 0.98\linewidth,
	beforeafter skip  = 0.5cm,
	drop fuzzy shadow = gray,
%	boxrule         = 0mm,
%	top             = 0mm,
%	bottom          = 0mm,
%	left            = 0mm,
%	right           = 0mm,
	title = #2,#1
}

\newtcolorbox{image-title}[2][]{
	enhanced,
	colback           = image-title,
	colframe          = image-title,
	fonttitle         = \bfseries,
	height            = 0.6cm,
	drop fuzzy shadow = gray,
	beforeafter skip  = 0pt,
	grow to left by   = 0.7cm,
	boxrule           = 0mm,
	top               = 0.5mm,
	bottom            = 0mm,
	left              = 1mm,
	right             = 0mm,
	sharp corners,
	title = #2,#1
}

\newtcolorbox{image-theword}{
	enhanced,
	colback           = image-theword,
	colframe          = image-theword,
	fonttitle         = \bfseries,
	drop fuzzy shadow = gray,
	width             = 1.3cm,
	height            = 0.5cm,
	beforeafter skip  = 0pt,
	grow to left by   = 0.7cm,
	boxrule           = 0mm,
	top               = 0.5mm,
	bottom            = 0mm,
	left              = 1mm,
	right             = 0mm,
	sharp corners,
}

\newtcolorbox{image-content}{
	enhanced,
	colback         = image-bg,
	colframe        = image-bg,
	fonttitle       = \bfseries,
%	enlarge top by  = -0.5cm,
	enlarge right by = 5cm,
	width           = \linewidth,
	boxrule         = 0mm,
	top             = 2mm,
	bottom          = 0mm,
	left            = 0mm,
	right           = 0mm,
%	show bounding box
}


\newtcolorbox{image-caption}[2][]{
	enhanced,
	colback           = defn-title,
	colframe          = defn-title,
	fonttitle         = \bfseries,
	halign            = center,
	height            = 0.6cm,
	drop fuzzy shadow = gray,
	before skip       = 5pt,
%	grow to right by  = 1.055\linewidth,
	enlarge bottom by = -1cm,
	boxrule           = 0mm,
	top               = 0.5mm,
	bottom            = 0mm,
	left              = 2mm,
	right             = 0mm,
	sharp corners,
	title = #2,#1
}


% }}}

% 	REGULAR TABLES (TCOLORBOX) {{{

\newtcolorbox{table-bg}[2][]{
	enhanced,
%	float,
%	breakable,
	colback           = table-bg,
	colframe          = table-bg,
	fonttitle         = \bfseries,
%	width             = 0.98\linewidth,
	beforeafter skip  = 0.5cm,
	drop fuzzy shadow = gray,
%	boxrule         = 0mm,
%	top             = 0mm,
%	bottom          = 0mm,
%	left            = 0mm,
%	right           = 0mm,
	title = #2,#1
}

\newtcolorbox{table-theword}{
	enhanced,
	colback           = table-theword,
	colframe          = table-theword,
	fonttitle         = \bfseries,
	drop fuzzy shadow = gray,
	width             = 1.5cm,
	height            = 0.5cm,
	beforeafter skip  = 0pt,
	grow to left by   = 0.7cm,
	boxrule           = 0mm,
	top               = 0.5mm,
	bottom            = 0mm,
	left              = 1mm,
	right             = 0mm,
	sharp corners,
}

\newtcolorbox{table-title}[2][]{
	enhanced,
	colback           = table-title,
	colframe          = table-title,
	fonttitle         = \bfseries,
	height            = 0.6cm,
	drop fuzzy shadow = gray,
	beforeafter skip  = 0pt,
	grow to left by   = 0.7cm,
	boxrule           = 0mm,
	top               = 0.5mm,
	bottom            = 0mm,
	left              = 1mm,
	right             = 0mm,
	sharp corners,
	title = #2,#1
}


\newtcolorbox{table-content}{
	enhanced,
	colback         = table-bg,
	colframe        = table-bg,
	fonttitle       = \bfseries,
	before skip = 0.5cm,
%	enlarge top by  = -0.5cm,
%	enlarge right by = 5cm,
	width           = \linewidth,
	boxrule         = 0mm,
	top             = 2mm,
	bottom          = 0mm,
	left            = 0mm,
	right           = 0mm
	%show bounding box
}

% }}}

% 	DEFINITION TABLE (TCOLORBOX) {{{

% \newtcbox[init options]{name}[number][default]{options}

\newtcolorbox{defn-bg}{
	enhanced,
	colback           = defn-bg,
	colframe          = defn-bg,
	fonttitle         = \bfseries,
	drop fuzzy shadow = gray,
	width             = 0.95\linewidth,
	beforeafter skip  = 0.5cm,
	arc is angular,
}

\newtcolorbox{defn-theword}{
	enhanced,
	colback           = defn-theword,
	colframe          = defn-theword,
	fonttitle         = \bfseries,
	drop fuzzy shadow = gray,
	width             = 2.5cm,
	height            = 0.5cm,
	beforeafter skip  = 0pt,
	grow to left by   = 0.7cm,
	boxrule           = 0mm,
	top               = 0.5mm,
	bottom            = 0mm,
	left              = 1mm,
	right             = 0mm,
	sharp corners,
}

\newtcolorbox{defn-title}[2][]{
	enhanced,
	colback           = defn-title,
	colframe          = defn-title,
	fonttitle         = \bfseries,
	height            = 0.6cm,
	drop fuzzy shadow = gray,
	beforeafter skip  = 0pt,
	grow to left by   = 0.7cm,
	boxrule           = 0mm,
	top               = 0.5mm,
	bottom            = 0mm,
	left              = 1mm,
	right             = 0mm,
	sharp corners,
	title = #2,#1
}

\newtcolorbox{defn-content}{
	enhanced,
	colback         = defn-bg,
	colframe        = defn-bg,
	fonttitle       = \bfseries,
%	enlarge top by  = -0.5cm,
%	enlarge right by = -5cm,
%	width           = 13cm,
%	boxrule         = 0mm,
%	top             = 1mm,
%	bottom          = 0mm,
%	left            = 4mm,
%	right           = 10mm,
%	show bounding box
}

% }}}

% 	NOTE TABLE (TCOLORBOX) {{{

\newtcolorbox{note-bg}{
	enhanced,
	colback           = note-bg,
	colframe          = note-bg,
	fonttitle         = \bfseries,
	drop fuzzy shadow = gray,
	width             = 0.95\linewidth,
	before skip  = 0.5cm,
	after skip  = 0.5cm,
	arc is angular,
}

\newtcolorbox{note-theword}{
	enhanced,
	colback           = note-theword,
	colframe          = note-theword,
	fonttitle         = \bfseries,
	drop fuzzy shadow = gray,
	width             = 0.7cm,
	height            = 0.5cm,
	beforeafter skip  = 0pt,
	grow to left by   = 0.7cm,
	boxrule           = 0mm,
	top               = 0.5mm,
	bottom            = 0mm,
	left              = 1mm,
	right             = 0mm,
	sharp corners,
}

\newtcolorbox{note-content}{
	enhanced,
	colback         = note-bg,
	colframe        = note-bg,
	fonttitle       = \bfseries,
%	enlarge top by  = -0.6cm,
%	enlarge left by = 1.5cm,
%	width           = 13cm,
%	boxrule         = 0mm,
%	top             = 0mm,
%	bottom          = 0mm,
%	left            = 0mm,
%	right           = 0mm
}


% }}}

% 	SECTION TITLES (TCOLORBOX) {{{

\newtcolorbox{section-box}{
	enhanced,
	colback           = section-bg,
	colframe          = section-bg,
	fonttitle         = \bfseries,
	width = \linewidth,
	height = 1cm,
	beforeafter skip  = 0pt,
	sharp corners
}

\newtcolorbox{subsection-box}{
	enhanced,
	colback           = subsection-bg,
	colframe          = subsection-bg,
	fonttitle         = \bfseries,
	width             = \linewidth,
	height = 0.6cm,
	sharp corners,
	beforeafter skip  = 0pt,
	boxrule           = 0mm,
	top               = 0.5mm,
	bottom            = 0mm,
	left              = 1mm,
	right             = 0mm
}

\newtcolorbox{subsubsection-box}{
	enhanced,
	colback   = subsubsection-bg,
	colframe  = subsubsection-bg,
	fonttitle = \bfseries,
	width     = \linewidth,
	height = 0.6cm,
	sharp corners,
	beforeafter skip  = 0pt,
	boxrule           = 0mm,
	top               = 0.5mm,
	bottom            = 0mm,
	left              = 1mm,
	right             = 0mm
}

% }}}

% }}}
% TABLE OF CONTENTS ------------------------------------------------------ {{{

\usepackage{titletoc}
% margin from RHS
%\contentsmargin{1cm}

% \dottedcontents {section}[left]{above}{label-width}{leader-width}
%\dottedcontents{section}[1.8cm]{\bfseries}{3.2em}{1pc}
%\dottedcontents{subsection}[1.8cm]{}{3.2em}{1pc}
%\dottedcontents{subsubsection}[1.8cm]{}{2.8em}{1pc}


% }}}
% MARGIN LINE NUMBERS ---------------------------------------------------- {{{

\usepackage[switch,displaymath,mathlines]{lineno}
%\modulolinenumbers[5]
\linenumberfont{\normalfont\large\sffamily}
\renewcommand\thelinenumber{\color{gray}\arabic{linenumber}}
\setlength\linenumbersep{0.5cm}


% }}}
% LETTRINE --------------------------------------------------------------- {{{

\usepackage{type1cm}  % Allows font resizing
\usepackage{lettrine} % Allows font calligraphy (enlarge first char)
\renewcommand{\LettrineTextFont}{\rmfamily}

% }}}
% TESTING ---------------------------------------------------------------- {{{

\usepackage{lipsum}  % generates filler text
\usepackage{blindtext} % generates non-latin filler text

% }}}

% }}}

\usepackage[activate={true,nocompatibility},
final,
tracking=true,
kerning=true,
spacing=true,
factor=1100,
stretch=10,
shrink=10]{microtype}
% prevents a certiain amount of overfull hbox badness
% helps with other margin stuff

\newcommand{\bulletpoint}
{ $\bullet$  }

\newcommand{\newpar}
{\par \vspace{0.3cm}}

\newcommand{\subsubsectionend}
{\nolinenumbers \begin{center} $\blacksquare$ \end{center} \linenumbers}

\newcommand{\subsectionend}
{\nolinenumbers \begin{center} $\blacksquare$ \hspace{0.2cm} $\blacksquare$ \end{center} \linenumbers}

\newcommand{\sectionend}
{\nolinenumbers \begin{center} $\blacksquare$ \hspace{0.2cm} $\blacksquare$ \hspace{0.2cm} $\blacksquare$ \end{center} \linenumbers \clearpage}

\newcommand{\theworddefinition}
{ \begin{defn-theword} { \footnotesize \begin{mydef} \end{mydef} } \end{defn-theword}}

\newcommand{\tcolorboxstart}
{\vspace{0.2cm} \centering \nolinenumbers }

\newcommand{\tcolorboxend}
{\justifying \linenumbers \vspace{0.2cm}}





% ============================================================================ 
\begin{document}
% ----------------------------------------------------------------------------- 
% TOC & SETUP {{{
\raggedbottom
\onecolumn

\tableofcontents
\pagebreak

\listoftables
\clearpage
\twocolumn
\justifying

% SECTION : preamble {{{
\section{{Preamble}}
\label{sec:preamble}


The following things should be noted before reading this document :\newpar

\textbf{Numbering Schemes \& Sectioning}

\begin{itemize}[noitemsep]
	\item Definition / Table / Image numbering scheme looks like :\\

		Section.Definition Number\\

		The definition / table / image numbers are reset every section.
		Essentially the exact same scheme as a subsection, but independent of
		the sub-section or the sub-sub-section numbers.\\
	\item Every section ending is indicated with a :  $\blacksquare$ \\
	\item Line numbers are only provided for main body text, and will not be
		listed for lists / tables / images etc \ldots

\end{itemize}

\textbf{Bibliography / References / Footnotes}


\begin{itemize}[noitemsep]
	\item Footnotes always use either roman or alphabet based caharacters.\\
	\item Citations will always use arabic numerals.\\
		
		The two choices above were made in order to easily distinguish between the two.\\

	\item Footnotes are provided as references for things that are used in only
		once in the current specific instance, e.g.\ some definition that I
		quote from wikipedia.\\
	
	\item A bibliography entry/ Citation is provided
		when books / articles / websites are used extensively throughout the
		document.\\

	\item Footnotes are also sometimes used as an aside, to talk about the
		current sentence outside of the current context , e.g.\ as a  fourth
		wall breaking personal opinion / note on the given topic.\\
		
	\item The distinction between when I use a footnote vs.\ when I use the NOTE
		box is extremely blurry.\\

\end{itemize}




\nolinenumbers \vspace{0.2cm}
\begin{center} $\blacksquare$ \end{center}
\linenumbers

\clearpage
% }}} END SECTION : preamble

\linenumbers

% }}}
% ----------------------------------------------------------------------------- 

% SECTION : nouns {{{
\section{{Nouns}}
\label{sec:nouns}

% DEFINITION : substantives {{{
\tcolorboxstart
\begin{defn-bg}

	\begin{defn-title}[width=7cm]{}
		{\normalsize \textbf{\textit{Substantives / Nouns}} }
	\end{defn-title}

	\theworddefinition

	\begin{defn-content}
		\justify
		Substantives/Nouns are people, animals, things, concepts and ideas.
	\end{defn-content}

\end{defn-bg}

\tcolorboxend

% }}} END DEFINITION : substantives

\lettrine[lines=3, findent=3pt, nindent=0pt]{S}{ubstantives}\footnote{In german
	a substantive is spelled as substantiv so over the course of this document I
	will probably end up using both, so yeah, if someone besides me (or more
	probably my future self) is reading this, dont give me shit for having
	spelling mistakes everywhere. This statement applies for a whole bunch of
	words.} are more commonly known as a nouns. Calling something a substantive
is just a more grammatical jargony way of referring to a noun. The reason that
there are two names for the same thing goes back to latin , where the phrase
\textit{Nomen Substantivus} or \textit{the name of substance} was used. I assume
lazy humans just split this into two words meaning the same thing over the
course of history.\newpar

German Nouns / Substantives have two defining characteristics that will help you
identify them in a german sentence. They are :

\begin{enumerate}[noitemsep]
	\item The first character of a noun is always uppercase.
	\item Every noun is preceded by a grammatically gendered article.
	\item German Nouns are declined.
\end{enumerate}

% 	SUB-SECTION : gendered_nouns {{{
\subsection{Gendered Nouns}
\label{ssec:gendered_nouns}

German is a gendered language therefore every substantive comes with one of
three genders. In german the gender is known as \textit{Genus}. German
dileniates between three grammatical genders and they are :

\nolinenumbers
\begin{enumerate}[noitemsep]
	\item \textbf{Maskulin }: der
	\item \textbf{Feminin} : die
	\item \textbf{Neuter}  : das
\end{enumerate}
\linenumbers

The words attached to each gender der , die , and das are what we use to
indicate that a particular noun belongs to a certain gender class. These three
words are called \textit{articles}. These are talked about in more detail in the
next sub section.\newpar

It is very important to note that the gender of a noun is NOT related to its
physical or biological gender, so please keep this in mind. As an exaple a young
girl is : \textit{das Madchen} , which is the article for a neuter noun, even
though we would assume that a young girl would be assigned the feminine article.
It is important to keep the differnce between grammatical gender and physical
gender distict in your mind to avoid making mistakes.


% }}} END SUB-SECTION : gendered_nouns

% 	SUB-SECTION : articles {{{
\subsection{Articles}
\label{ssec:articles}

As mentioned in Section~\ref{ssec:gendered_nouns}, every German noun has a corresponding
grammatical article. There are two types of article a noun can have  , and they
are :

\nolinenumbers
\begin{itemize}[noitemsep]
	\item \textbf{Definite Articles} : The english equivalent is the word ``{\textit{the}}''
	\item \textbf{In-definite Articles} : The english equivalent is the word ``{\textit{a}}''
\end{itemize}
\linenumbers

There are vairous things that affect what the exact article is for the word that
we are using. The main things to keep into consideration for each German noun
are :

\nolinenumbers
\begin{itemize}[noitemsep]
	\item Grammatical Gender
	\item Count (Singular / Plural)
	\item Case
\end{itemize}
\linenumbers

Based on these three things the article we are using for each noun becomes very
specific and gives a detailed description of the function this noun is serving
in the sentence.\newpar

Since there is no noun without an article in german, the basis for discussing
articles only arises when we understand the german cases. So cases and articles
changes are discussed in Section~\ref{sssec:article_tips_neuter_das_} which is
exclusively about articles and Cases.\newpar

So we need to learn every noun in german along with its corresponding article. I
really dont expect most sane humans will bother sitting around memorizing the
article for each word, so the next couple of subsections have some tricks to
help in guessing them.


% NOTE : article_importance {{{
\nolinenumbers
\vspace{0.2cm}
\centering
\begin{note-bg}

	\begin{note-theword}
		{\footnotesize \textbf{NOTE} }
	\end{note-theword}

	\begin{note-content}
		\justifying
		The importance of articles in German cannot be emphasized enough. In no
		correctly formed german sentence will there exist a noun without its
		definite or indefinite article.\newpar

		Every single noun, in every single sentence MUST be written with a
		corresponding grammatical article (in the correct case of course).
	\end{note-content}

\end{note-bg}
\linenumbers
\justifying

% }}} END NOTE : article_importance

% 		SUB-SUB-SECTION : article_tips_masculine_der_ {{{
\subsubsection{Article Tips : Masculine (Der)}
\label{sssec:article_tips_masculine_der_}

The following list provides some common roots / endings of words that will
recieve the \textbf{\textit{der (Maskulin)}} article.

% LIST : der (Maskulin) {{{

\nolinenumbers

\vspace{0.2cm}

\begin{tabular}{l r l}

\rowcolor{white} $\bullet$ -ant   & e.g. & der Konsonant\\
\rowcolor{white} $\bullet$ -ast   & e.g. & der Gast     \\
\rowcolor{white} $\bullet$ -ich   & e.g. & der Teppich  \\
\rowcolor{white} $\bullet$ -ismus & e.g. & der Marxismus\\
\rowcolor{white} $\bullet$ -ling  & e.g. & der Häftling \\
\rowcolor{white} $\bullet$ -us    & e.g. & der Rythmus  \\
\rowcolor{white} $\bullet$ -er    & e.g. & der Sommer   \\

\end{tabular}

\vspace{0.2cm}

\linenumbers

% }}} End LIST : der (Maskulin)


A note about the last one with the -er ending. This one not only means that the
grammatical gender of the noun is masculine, but most of the time often is also
reffering to the phyisical gender. E.g.\ der Lehrer (the male teacher), der
Amerikaner (the male american), der Fahrer (the male driver).\newpar


% }}} END SUB-SUB-SECTION : article_tips_masculine_der_

% 		SUB-SUB-SECTION : article_tips_feminine_die_ {{{
\subsubsection{Article Tips : Feminine (Die)}
\label{sssec:article_tips_feminine_die_}

The following list provides some common roots / endings of words that will
recieve the \textbf{\textit{die (feminin)}} article.

% LIST : die_article_tricks {{{

\nolinenumbers

\vspace{0.2cm}

\begin{xtabular}{l r l}

\rowcolor{white}  $\bullet$ -ung    & e.g. & die Entscheidung \\
\rowcolor{white}  $\bullet$ -tät    & e.g. & die Universität  \\
\rowcolor{white}  $\bullet$ -tion   & e.g. & die Explosion    \\
\rowcolor{white}  $\bullet$ -sion   & e.g. & die              \\
\rowcolor{white}  $\bullet$ -schaft & e.g. & die Gesellschaft \\
\rowcolor{white}  $\bullet$ -heit   & e.g. & die Schönheit    \\
\rowcolor{white}  $\bullet$ -keit   & e.g. & die              \\
\rowcolor{white}  $\bullet$ -ie     & e.g. & die Geographie   \\
\rowcolor{white}  $\bullet$ -enz    & e.g. & die              \\
\rowcolor{white}  $\bullet$ -anz    & e.g. & die Toleranz     \\
\rowcolor{white}  $\bullet$ -ei     & e.g. & die Schlägerei   \\
\rowcolor{white}  $\bullet$ -ur     & e.g. & die Natur        \\
\rowcolor{white}  $\bullet$ -in     & e.g. & die Boxerin      \\
\rowcolor{white}  $\bullet$ -itis   & e.g. & \\
\rowcolor{white}  $\bullet$ -sis    & e.g. & \\
\rowcolor{white}  $\bullet$ -ik     & e.g. & \\
\rowcolor{white}  $\bullet$ -ade    & e.g. & \\
\rowcolor{white}  $\bullet$ -age    & e.g. & \\
\rowcolor{white}  $\bullet$ -ine    & e.g. & \\
\rowcolor{white}  $\bullet$ -ere    & e.g. & \\
\rowcolor{white}  $\bullet$ -isse   & e.g. & \\
\rowcolor{white}  $\bullet$ -ive    & e.g. & \\
\rowcolor{white}  $\bullet$ -se     & e.g. & \\

\end{xtabular}

\vspace{0.2cm}

\linenumbers

% }}} End TABLE : die_article_tricks

Just like the note about the -er in the der section, the last point with the -in
ending not only means that the grammatical gender of the noun is
feminine, but most of the time often is also reffering to the phyisical gender.
E.g.\ die Lehrerin (the female teacher) , die Fahrerin (the female driver), die
Amerikanerin (the female american)

% NOTE : die article note {{{
\nolinenumbers
\vspace{0.2cm}
\centering
\begin{note-bg}

	\begin{note-theword}
		{\footnotesize \textbf{NOTE} }
	\end{note-theword}

	\begin{note-content} \justifying The large majority of nouns which end in -e
		are feminine, e.g. : die Lampe (the lamp), die Rede (the speech), and
		die Bühne (the stage).\newpar

		This is true roughly 80\% of the time opposed to being a grammatical
		rule, which is why this information is in a note as opposed to in the
		list above.  \end{note-content}

\end{note-bg}
\linenumbers
\justifying

% }}} END NOTE : die_article_note

% }}} END SUB-SUB-SECTION : article_tips_feminine_die_

% 		SUB-SUB-SECTION : article_tips_neuter_das_ {{{
\subsubsection{Article Tips : Neuter (Das)}
\label{sssec:article_tips_neuter_das_}

The following list provides some common roots / endings of words that will
recieve the \textbf{\textit{das (neuter)}} article.

% LIST : das (neutrum)  {{{

\nolinenumbers

\vspace{0.2cm}

\begin{tabularx}{\linewidth}{l r l}

\rowcolor{white} $\bullet$ -chen & e.g. & Das Häuschen \\
\rowcolor{white} $\bullet$ -lein & e.g. & Das Büchlein \\
\rowcolor{white} $\bullet$ -um & e.g. & Das Wachstum \\

\end{tabularx}

\vspace{0.2cm}

\linenumbers


% }}} End LIST : das (neutrum)

% NOTE : das article note {{{
\nolinenumbers
\vspace{0.2cm}
\centering
\begin{note-bg}

	\begin{note-theword}
		{\footnotesize \textbf{NOTE} }
	\end{note-theword}

	\begin{note-content} \justifying Similar to the note in the femnine article
		tips section, a lot of the german nouns that begin with Ge- are neuter
		but not all, which is why you are reading this in a note right now and
		not the main list above.

	\end{note-content}

\end{note-bg}
\linenumbers
\justifying

% }}} END NOTE : das_article_note

% }}} END SUB-SUB-SECTION : article_tips_neuter_das_


% }}} END SUB-SECTION : articles

% 	SUB-SECTION : plurals {{{
\subsection{Plurals}
\label{ssec:plurals}

German just like English build plurals out of nouns by appending certain endings
to the noun. Its a bit more involved than english however, since english has
only the -s ending, german has a few more. The ending appended is realtively
arbitrary so there are few choices but to learn the plural formation along with
the original noun, although after a while you should get a feel for what kind of
word will get what kind of plural formation.\newpar

Below are some general guidelines to building the plural formation for a noun
:\cite{em},\cite{germanveryeasy}

% 		SUB-SUB-SECTION : plurals_masculine_nouns {{{
\textbf{\subsubsection{Plurals : Masculine Nouns}}
\label{sec:plurals_masculine_nouns}

The following list shows some common transformations to make masculine nouns
plural :

% TABLE : plural_masculine_nouns {{{


\nolinenumbers

\vspace{0.2cm}

\begin{xtabular}{lll}


		\rowcolor{white} $\bullet$ -ich  & -e     & \textcolor{gray}{\textit{der Teppich , die Teppiche} }\\
		\rowcolor{white} $\bullet$ -ig   & -e     & \textcolor{gray}{\textit{der König, die Könige} }\\
		\rowcolor{white} $\bullet$ -ling & -e     & \textcolor{gray}{\textit{der Schmetterling, die Schmetterlinge} }\\
		\rowcolor{white} $\bullet$ -är   & -e     & \textcolor{gray}{\textit{der Veterinär,die Veterinäre} }\\
		\rowcolor{white} $\bullet$ -eur   & -e    & \textcolor{gray}{\textit{der Friseur,die Friseure} }\\
		\rowcolor{white} $\bullet$       & \"{-}e & \textcolor{gray}{ \textit{der Platz , die Plätze} }\\
		\rowcolor{white}                 &        & \textcolor{gray}{ \textit{der Kuss , die Küsse} }\\
		\rowcolor{white}                 &        & \textcolor{gray}{ \textit{der Arzt , die Ärzte} }\\
		\rowcolor{white} $\bullet$       & -      & \textcolor{gray}{ \textit{der Schüler , die Schüler} }\\
		\rowcolor{white} $\bullet$ -er   & \"{-}  & \textcolor{gray}{ \textit{der Vater , die Väter} }\\
		\rowcolor{white} $\bullet$ -el   & \"{-}  & \textcolor{gray}{ \textit{der Mantel , die Mäntel} }\\
		\rowcolor{white} $\bullet$ -us   & -usse  & \textcolor{gray}{ \textit{der Bus , die Busse} }\\



\end{xtabular}

\vspace{0.2cm}

\linenumbers

% }}} End TABLE : plural masculine nouns

Please keep in mind that the list above shows trasformations in general and is
not meant to serve as a list of rules for converting nouns with the given
endings into thier respective plural forms. They are only meant to serve as
educated guesses.\newpar

A lot of masculine nouns ending in -e also follow the rules for n-Declination.
So check out Section~\ref{ssec:n_declension} for more details.

% }}} END SUB-SUB-SECTION : plurals_masculine_nouns

% 		SUB-SUB-SECTION : plurals_feminine_nouns {{{
\textbf{\subsubsection{Plurals : Feminine Nouns}}
\label{sec:plurals_feminine_nouns}

The following tables illustrates the plural formations for feminine nouns :

% TABLE : plurals_feminine_nouns {{{

\nolinenumbers

\vspace{0.2cm}

\begin{xtabular}{lll}

		\rowcolor{white} $\bullet$  -ei    & -en & \textcolor{gray}{ \textit{die Datei , die Dateien} }\\
		\rowcolor{white} $\bullet$ -ung    & -en & \textcolor{gray}{ \textit{} }\\
		\rowcolor{white} $\bullet$ -heit   & -en & \textcolor{gray}{ \textit{} }\\
		\rowcolor{white} $\bullet$ -keit   & -en & \textcolor{gray}{ \textit{} }\\
		\rowcolor{white} $\bullet$ -ion    & -en & \textcolor{gray}{ \textit{} }\\
		\rowcolor{white} $\bullet$ -schaft & -en & \textcolor{gray}{ \textit{} }\\
		\rowcolor{white} $\bullet$ -ik     & -en & \textcolor{gray}{ \textit{} }\\
		\rowcolor{white} $\bullet$ -eur    & -en & \textcolor{gray}{ \textit{} }\\
		\rowcolor{white} $\bullet$ -enz    & -en & \textcolor{gray}{ \textit{} }\\
		\rowcolor{white} $\bullet$ -tät    & -en & \textcolor{gray}{ \textit{} }\\
		\rowcolor{white} $\bullet$ -itis   & -en & \textcolor{gray}{ \textit{} }\\
		\rowcolor{white} $\bullet$ -sis    & -en & \textcolor{gray}{ \textit{} }\\
		\rowcolor{white} $\bullet$ -ung    & -en & \textcolor{gray}{ \textit{} }\\
		\rowcolor{white} $\bullet$ -ung    & -en & \textcolor{gray}{ \textit{} }\\

		\midrule
		\rowcolor{white} $\bullet$ -ie  & -n & \textcolor{gray}{ \textit{die Fantasie, die Fantasien} }\\
		\rowcolor{white} $\bullet$ -ade  & -n & \textcolor{gray}{ \textit{..} }\\
		\rowcolor{white} $\bullet$ -age  & -n & \textcolor{gray}{ \textit{..} }\\
		\rowcolor{white} $\bullet$ -ere  & -n & \textcolor{gray}{ \textit{..} }\\
		\rowcolor{white} $\bullet$ -ine  & -n & \textcolor{gray}{ \textit{..} }\\
		\rowcolor{white} $\bullet$ -isse & -n & \textcolor{gray}{ \textit{..} }\\
		\rowcolor{white} $\bullet$ -ive  & -n & \textcolor{gray}{ \textit{..} }\\
		\rowcolor{white} $\bullet$ -se   & -n & \textcolor{gray}{ \textit{..} }\\

		\midrule
		\rowcolor{white} $\bullet$ -in   & -nen    & \textcolor{gray}{ \textit{..} }\\
		\rowcolor{white} $\bullet$ -      & \"{-}e & \textcolor{gray}{ \textit{..} }\\
		\rowcolor{white} $\bullet$ -nis  & -nisse  & \textcolor{gray}{ \textit{..} }\\
		\rowcolor{white} $\bullet$ -xis  & -xien   & \textcolor{gray}{ \textit{..} }\\
		\rowcolor{white} $\bullet$ -itis & -iden   & \textcolor{gray}{ \textit{..} }\\
		\rowcolor{white} $\bullet$ -aus  & -äuse   & \textcolor{gray}{ \textit{..} }\\
		\rowcolor{white} $\bullet$ -      & \"{-}   & \textcolor{gray}{ \textit{die Mutter,die Mütter} }\\
		\rowcolor{white} $\bullet$ -      & \"{-}en & \textcolor{gray}{ \textit{die Werkstatt , die Werkstätten } }\\

	\end{xtabular}

\vspace{0.2cm}

\linenumbers

% }}} End TABLE : plurals_feminine_nouns


% }}} END SUB-SUB-SECTION : plurals_feminine_nouns

% 		SUB-SUB-SECTION : plurals_neuter_nouns {{{
\textbf{\subsubsection{Plurals Neuter Nouns}}
\label{sec:plurals_neuter_nouns}



% }}} END SUB-SUB-SECTION : plurals_neuter_nouns

% }}} END SUB-SECTION : plurals

% 	SUB-SECTION : cases {{{
\subsection{Cases}
\label{ssec:cases}

There is no such thing as a case in English, therefore it is difficult to form
an equivalence relation with something that you might already know. The easiest
way to go about understanding cases is to consider the following questions when
constructing sentences in german :

\nolinenumbers
\begin{enumerate}[noitemsep]
	\item Who is doing the action ?
	\item Who or what is being directly affected by the action ?
	\item Who or what is being indirectly affected by the action ?
	\item Who is indicating ownership of what ?
\end{enumerate}
\linenumbers

Based on the answers to the questions above, every german noun falls into a
category called a case \textit{(fall , die falle)} . We already know that german
is a gendered language, and based on the grammatical gender of the noun, the
article of the changes between masculine, feminine and neuter.\newpar

Now, an added degree of complication is that based on the case, the article of
the noun will further change. Essentially cases just serve as extremely specific
articles (instead of just the simple 3 masculine , feminine and neuter) when
talking about German nouns.Basically, in every sentence  when we have a person
or a thing (noun) performing some actions (verbs). Depending on how the noun
(person/thing) is interacting with the verb(action) the article (how we refer to
the person / thing) will slightly change. This slight change is called the
application of a case to that noun.\newpar

A case is called a Falle in German.  In German we have four cases :

% LIST : german_cases_basic_list {{{
\nolinenumbers

\vspace{0.2cm}

\begin{tabularx}{0.95\linewidth}{lllX}

\rowcolor{white} \bulletpoint & \cellcolor{cell-lightred} Nominative    & : & Der Nominativ\\
\rowcolor{white}              &                                         &   & \textcolor{gray}{ \textit{Der Werfall} }\\
\rowcolor{white} \bulletpoint & \cellcolor{cell-lightyellow} Accusative & : & Der Akkusativ\\
\rowcolor{white}              &                                         &   & \textcolor{gray}{ \textit{Der Wenfall} }\\
\rowcolor{white} \bulletpoint & \cellcolor{cell-lightgreen} Dative      & : & Der Dativ\\
\rowcolor{white}              &                                         &   & \textcolor{gray}{ \textit{Der Wemfall} }\\
\rowcolor{white} \bulletpoint & \cellcolor{cell-lightblue} Genetive     & : & Der Genetiv\\
\rowcolor{white}              &                                         &   & \textcolor{gray}{ \textit{Der Wesfall} }\\

\end{tabularx}

\linenumbers

\vspace{0.2cm}
% }}} End LIST : german cases basic list

The cases are color coded to make it easier to identify which object, is in
which case in the examples given in the following sections. The tables in the
pronouns section will also use the same colors in order to maintain consistency
throughout the document.\newpar

A short summary of when to use the cases is in the bullet points below, and a
thorough explanation is further below in the sepreate sub-sections.

% LIST : cases_basic_explanation {{{

\nolinenumbers

\vspace{0.2cm}

\begin{tabularx}{0.95\linewidth}{lllX}


\rowcolor{white} \bulletpoint Nominativ & : &  Subject \\
\rowcolor{white} & &\textcolor{gray}{ \textit{Who is performing the action} }\\

\rowcolor{white} \bulletpoint Akkusativ & : &  Direct object\\
\rowcolor{white} & &\textcolor{gray}{ \textit{Who / what is the action being} }\\
\rowcolor{white} & &\textcolor{gray}{ \textit{being performed on ?} }\\

\rowcolor{white}  \bulletpoint Dativ & : &  Indirect object \\

\rowcolor{white} & &\textcolor{gray}{ \textit{Who / what is the action } }\\
\rowcolor{white} & &\textcolor{gray}{ \textit{ affecting aside from the direct } }\\
\rowcolor{white} & &\textcolor{gray}{ \textit{ object ?} }\\

\rowcolor{white}  \bulletpoint Genetiv & : &  Possession\\

\end{tabularx}

\vspace{0.2cm}

\linenumbers

% }}} End LIST : cases basic explanation

A full table of article changes according to the specific case is shown below.
This table can be super useful when making new sentences as a reference guide.
It will only make complete sense however after you have been through all of the
following sections explaining all four cases in detail.\newpar

A good thing to meintion here would be that cases change articles as both
definite and indefinite articles for the given noun change.


% 		SUB-SUB-SECTION : nominative {{{
\textbf{\subsubsection{Nominative}}
\label{sec:nominative}

To understand the nominative case, and any subsequent cases we have to
understand the two main parts that make up any sentence both in English and in
German. These two things are : \textbf{\textit{the grammatical subject}} , and ,
\textbf{\textit{the predicate}} .  Both of these things are defined below :

% DEFINITION : grammatical_subject {{{

\vspace{0.2cm}
\centering
\nolinenumbers
\begin{defn-bg}


	\begin{defn-title}[width=7cm]{}
		{\normalsize \textbf{\textit{Grammatical Subject}} }
	\end{defn-title}

	\begin{defn-theword}
	{
		\footnotesize \begin{mydef}\label{def:grammatical_subject} \end{mydef}
	}
	\end{defn-theword}

	\begin{defn-content}
		\justify
The grammmatical subject is the person or thing about whom the current statement
is being made.\newpar

In lingustic jargon : The subject is the word or phrase that controls the
verb \textit{(Def~\ref{def:verb})} or the clause \textit{(Def~\ref{def:clause}).}~\footnote{\url{https://en.wikipedia.org/wiki/Subject\_(grammar)}}
	\end{defn-content}

\end{defn-bg}

\justifying
\linenumbers

% }}} END DEFINITION : grammatical_subject

% DEFINITION : predicate {{{

\vspace{0.2cm}
\centering
\nolinenumbers
\begin{defn-bg}

\label{def:predicate}
	\begin{defn-title}[width=7cm]{}
		{\normalsize \textbf{\textit{Predicate}} }
	\end{defn-title}

	\begin{defn-theword}
	{
		\footnotesize \begin{mydef} \end{mydef}
	}
	\end{defn-theword}

	\begin{defn-content}
		\justify
		The predicate is the part of the sentence (or clause), that tells us
		where the subject is, or what the subject does or is.\newpar

		Basically the predicate is everything that is not the subject
		itself.~\footnote{\url{https://www.grammar-monster.com/glossary/predicate.htm}}
	\end{defn-content}

\end{defn-bg}

\justifying
\linenumbers

% }}} END DEFINITION : predicate

 The subject in the sentence will always take the nominative case and therefore
 the corresponding gendered nominative article. The main subject (as is defined
 above) is the noun that is performing some action. The action here will be
 specified by the verb. The list of both definite and indefinite nominative
 articles in german is shown below :\newpar

% TABLE : Nominative Articles {{{
\nolinenumbers

\vspace{0.5cm}

\begin{tabularx}{0.94\linewidth}{l|XXXX}

		&
		\cellcolor{lightgray} \textbf{\textit{MAS.}} &
		\cellcolor{lightgray} \textbf{\textit{NEU.}}  &
		\cellcolor{lightgray} \textbf{\textit{FEM.}}  &
		\cellcolor{lightgray} \textbf{\textit{PLU.}} \\
		\midrule

		\cellcolor{lightgray} \textbf{\textit{NOM.}} &
		\cellcolor{cell-lightpurple}  der            &
		\cellcolor{cell-lightorange}  das            &
		\cellcolor{cell-lightblue} die               &
		\cellcolor{cell-lightblue} die \\

		\midrule

		\cellcolor{lightgray} \textbf{\textit{NOM.}} &
		\cellcolor{cell-lightpurple}  ein            &
		\cellcolor{cell-lightorange}  ein            &
		\cellcolor{cell-lightblue} eine              &
		\cellcolor{table-bg} - \\

\end{tabularx}

\vspace{0.5cm}

\linenumbers
% }}} END-TABLE : Nominative Articles

An example is shown below to further clarify this concept :\newpar

\noindent
\textit{Der Hund beißt den Mann.}\\
\textcolor{gray} { \textit{( The dog bites the man. )} } \newpar

The action (verb) being performed is beißen (to bite). The thing doing the
action is the dog. Therefore the dog will be in the nominative case and will
have the ‘normal’ masculine article of \textit{der}.\newpar

Overall, the nominative case is used in the following situations :\newpar

\nolinenumbers
\begin{itemize}[noitemsep]
	\item If the word is isolated , e.g. One word answers like
		``{\textit{Name}}''.
	\item If the word makes up part of the subject.
	\item If the word forms part of the object of the predicate, and the
			sentence is formed with the copulative verb
			\textit{(Definition~\ref{def:copulative_verb})}. 
\end{itemize}
\linenumbers

% NOTE : nominative_clarification {{{
\nolinenumbers
\vspace{0.2cm}
\centering
\begin{note-bg}

	\begin{note-theword}
		{\footnotesize \textbf{NOTE} }
	\end{note-theword}

	\begin{note-content}
		\justifying
		Every single sentence in german will always have a nominative object in
		it. The nominative is the only case where this is fact is true, due to
		fact that there is no such thing as a nominative preposition.
	\end{note-content}

\end{note-bg}
\linenumbers
\justifying

% }}} END NOTE : nominative_clarification

% }}} END SUB-SUB-SECTION : nominative

% 		SUB-SUB-SECTION : accusative {{{
\textbf{\subsubsection{Accusative}}
\label{sec:accusative}

The second german case is called \textbf{\textit{the Accusative (Der Akkusativ /
		Der Wenfall)}} .  The accusative case (or more specifically the
accusative grammatical article) applies on the direct object / person (noun)
that the action (verb) is being performed on by the subject (nominative noun).
Read the last sentence again, because it is a little dense the first time
around, but nonetheless important. \newpar

The accusative gendered articles are mainly the same as the nominative case,
with the only exception being the masculine accusative. All the articles are
shown in a table below :

% TABLE : Accusative Articles {{{

\nolinenumbers
\vspace{0.5cm}
\begin{tabularx}{0.94\linewidth}{l|XXXX}

		&
		\cellcolor{lightgray} \textbf{\textit{MAS.}} &
		\cellcolor{lightgray} \textbf{\textit{NEU.}}  &
		\cellcolor{lightgray} \textbf{\textit{FEM.}}  &
		\cellcolor{lightgray} \textbf{\textit{PLU.}} \\
		\midrule

		\cellcolor{lightgray} \textbf{\textit{ACC.}} &
		\cellcolor{cell-lightgreen} den              &
		\cellcolor{cell-lightorange}  das            &
		\cellcolor{cell-lightblue}  die              &
		\cellcolor{cell-lightblue} die \\

		\midrule

		\cellcolor{lightgray} \textbf{\textit{ACC.}} &
\cellcolor{cell-lightgreen} einen            &
\cellcolor{cell-lightorange}  ein            &
\cellcolor{cell-lightblue}  eine             &
\cellcolor{table-bg} - \\


\end{tabularx}

\vspace{0.5cm}

\linenumbers

% }}} END-TABLE : Accusative Articles

The distinction between the direct object, and the subject is easily explained
through the same example that we dealt with in the nominative section, which is
:\newpar

\noindent
\textit{Der Hund beißt den Mann.}\\
\textcolor{gray} { \textit{( The dog bites the man. )} } \newpar

As mentioned earlier, the thing doing the action is the dog, so the dog gets the
nominative article. The action being done is biting (beißen), but who is the
action being done to , or, who/what is being directly affected by the action ?
In this specific example : Who is being bitten ? : The Mann (Der Mann). The
normal article for the man is the masculine \textit{der} , but in this sentence,
we will use the masculine der, but in its accusative form \textit{der} . The
reason the above is a good example is that we have two nouns, which are both
masculine , both the two take different cases. This allows us to see the
nominative and the accusative case functioning simultaneously.\newpar


Although the accusative object is defined as the direct object, there are also
other ways that an object might be assigned the accusative grammatical article,
even though they are not being directly affected by the verb. The most common
way that this occurs is through certain prepositions in German. When a word is
being affected by these prepositions in a sentence, it will always take the
accusative case, therfore now we have the following situations in which we will
give an object its acccusative grammatival article :\newpar

\nolinenumbers
\begin{itemize}[noitemsep]

	\item If the word is a direct object in the english version of the sentence
		(i.e.\ it is the noun that the verb is acting on), then 90\% of the time
		this word will take the accusative case in german.

	\item If the word is being affected by either an accusative preposition
		\textit{( Table~\ref{table:prepositions_accusative} )} or a wechsel
		preposition \textit{( Table~\ref{table:prepositions_wechsel} )} , then
		it will take the accusative case.~\footnote{More information about how
			to distingush between when to use wechsel prepositions for
			accusative and when for dative is provided in the prepositions
			section.} These prepositions are listed here and then again in the
		Prepositions Section. \textit{(Section~\ref{sec:prepositions})}
	

	\begin{itemize}[noitemsep]

		\item durch, für , entlang, gegen, ohne, um \ldots herum, hinter, in,
			neben, über, unter, vor, zwischen, wider

	\end{itemize}

\end{itemize}
\linenumbers

Rememember that just because there is not direct object in the sentence, does
not mean that there is no accusative object. Since we can have an accusative
object in a sentence as is defined by the preposition.\newpar

This also implies that we can have two ` \textit{Accusative Objects} ' in one
sentnce since, we can have a regular direct object and we also have another
object that is accusative according to one of the accusative prepositions listed
above, or a movement based wechsel preposition.\newpar

For the sake of illustration, here is another example which has two accusative
objects in it :

% }}} END SUB-SUB-SECTION : accusative

% 		SUB-SUB-SECTION : dative {{{
\textbf{\subsubsection{Dative}}
\label{sec:dative}

The third german case is called \textbf{\textit{the Dative Case (Der Dativ Fall
		/ Der Wemfall)}} If you have understood and internalized the accusative
case, then the dative case should not be too much of a stretch to master. The
dative case applies when there is an object in the sentence that is being
indirectly impacted by the action (verb).\newpar

Unlike in the accusative where only the masculine article changes, all the
articles change for an object that is  under the dative case which it is worth
mentioning includes the plural, since up until people learn about the Dative
case most people take for granted that plural article will always be
\textit{`die'}, and this is no longer the case. The dative case articles along
with the articles for all the other cases that we have learned so far are shown
in a table below :\newpar

% TABLE : Dative Articles {{{

\vspace{0.5cm}

\nolinenumbers

\begin{tabularx}{0.94\linewidth}{l|XXXX}

		&
		\cellcolor{lightgray} \textbf{\textit{MAS.}} &
		\cellcolor{lightgray} \textbf{\textit{NEU.}}  &
		\cellcolor{lightgray} \textbf{\textit{FEM.}}  &
		\cellcolor{lightgray} \textbf{\textit{PLU.}} \\

		\midrule

		\cellcolor{lightgray} \textbf{\textit{DAT.}} &
		\cellcolor{cell-lightred} dem               &
		\cellcolor{cell-lightred} dem               &
		\cellcolor{cell-lightpurple} der               &
		\cellcolor{cell-lightgreen} den \\

		\midrule

		\cellcolor{lightgray} \textbf{\textit{DAT.}} &
		\cellcolor{cell-lightred} einem              &
		\cellcolor{cell-lightred} einem              &
		\cellcolor{cell-lightpurple} einer           &
		\cellcolor{table-bg} - \\

\end{tabularx}

\vspace{0.5cm}

\linenumbers

% }}} END-TABLE : Dative Articles

Similar to the accusative case, there are also situations where an object will
take the dative grammatical article and not be the inderect object in the
sentence. This occurs when the noun in question is subject to a dative
preposition. Therefore now we have the following situations in which a noun will
have the dative grammatical article :\newpar

\nolinenumbers
\begin{itemize}[noitemsep]

	\item If the word is an \textbf{\textit{in-direct object}} in the english
		version of the sentence (i.e.\ it is the noun that is being affected by
		the verb, but not the noun that is being directly acted on), then 90\%
		of the time this word will take the Dative case in german.

   \item If the word is being mentioned along with either a Dative preposition
	   \textit{( Table~\ref{table:prepositions_dative} )} or a wechsel
	   preposition \textit{( Table~\ref{table:prepositions_wechsel} )} , then it
	   will take the Dative case.~\footnote{See previous footnote.} 

\end{itemize}
\linenumbers

Just like the previous sections let us consider an example for clarification
:\newpar

\noindent
\textit{Ich schenke dir das Heft.}\\
\textcolor{gray} { \textit{( I gift you the notebook. )} } \newpar

The sentence above can be broken down in the following way :

% TABLE : dative_example_clarification {{{

\nolinenumbers

\vspace{0.2cm}

\begin{tabularx}{0.95\linewidth}{lllX}

\rowcolor{white} \bulletpoint Subject (Nominative) & : & I (ich)   & \\
\rowcolor{white} \bulletpoint Verb                 & : & schenken (to gift) & \\
\rowcolor{white} \bulletpoint Direct Object        & : & das Heft (the notebook)  & \\
\rowcolor{white} \bulletpoint Indirect Object      & : & you (dir)\footnote{more
about pronouns (i.e.\ why dir and not du) in \textit{(Section~\ref{sec:pronouns})}  } & \\

\end{tabularx}

\vspace{0.2cm}

\linenumbers

% }}} End TABLE : dative example clarification

Another example, just in case the one above was not clear is as follows :
\newpar

\noindent
\textit{Wir machen das mit einem Computer}\\
\textcolor{gray} { \textit{( We are doing that with a computer. )} } \newpar

Before ananlysing the sentence, please note that the das used in the example is
not actually a grammatical gender for any noun in the sentence, rather it is the
german equivalent word for that, as is evident in the translation. Das will also
always take the accusative case, since it is always the direct object when the
word that appears in a sentence.\newpar

That being said, this example is a little easier because we know (if we have
already been through the prepositions section) that the preposition mit is a
dative preposition, therefore the thing that is being accompanied with the
dative preposition must take its grammatical gender in the dative case.\newpar

However if we did not know about mit being a dative preposition then the breakup
of the sentence would look something like the following :\newpar

% TABLE : dative_example_clarification_2 {{{

\nolinenumbers

\vspace{0.2cm}

\begin{tabularx}{0.95\linewidth}{llX}

\rowcolor{white} \bulletpoint Subject (Nominative) & : & Wir (we)   \\
\rowcolor{white} \bulletpoint Verb                 & : & machen (to do) \\
\rowcolor{white} \bulletpoint Direct Object        & : & das (that)  \\
\rowcolor{white} \bulletpoint Indirect Object      & : & the computer (einem
Computer) \\

\end{tabularx}

\vspace{0.2cm}

\linenumbers

% }}} End TABLE : dative example clarification_2

So the only thing I think that is worth mentioning here is the computer. The
computer takes the masculine gender in german , so der computer, and it is in
the dative case here so dem computer. Also worth noting in the above example is
that I have used the indefinite (einem) version of the grammatical gender as
opposed to the definite version(dem), just to spice things up a little and to
provide examples for as many scenarios as I can.\newpar

To finish things off, here is an example that has no accusative object, and two
dative objects : \newpar





% }}} END SUB-SUB-SECTION : dative

% 		SUB-SUB-SECTION : genetive {{{
\textbf{\subsubsection{Genetive}}
\label{sec:genetive}

The fourth and last case in German , is called \textbf{\textit{the Genetive Case
		(Der Genetiv Fall /  Der Wesfall)}} . The genetive case is mainly used
when we are trying to indicate posession of something. As is with the previously
mentioned cases (except nominative) , the Genetive has its own set of
prepositoins, the creatively name : genetive prepositions. If an object is being
affected by these prepositions, then it will take the genetive grammtical
article.\newpar

The table for the definite and indefinite genetive articles is as follows
:\newpar

The english equivalent of displaying possession is when we add the apostrophe s
to the end of a word to show belonging.\ e.g.\ John's book , Mary's car , etc
\ldots

% TABLE : Genetive_Articles {{{

\vspace{0.5cm}

\nolinenumbers

\begin{tabularx}{0.94\linewidth}{l|XXXX}

		&
		\cellcolor{lightgray} \textbf{\textit{MAS.}} &
		\cellcolor{lightgray} \textbf{\textit{NEU.}}  &
		\cellcolor{lightgray} \textbf{\textit{FEM.}}  &
		\cellcolor{lightgray} \textbf{\textit{PLU.}} \\

		\midrule

		\cellcolor{lightgray} \textbf{\textit{GEN.}} &
		\cellcolor{cell-lightyellow} des               &
		\cellcolor{cell-lightyellow} des               &
		\cellcolor{cell-lightpurple} der               &
		\cellcolor{cell-lightpurple} der \\

		\midrule

		\cellcolor{lightgray} \textbf{\textit{GEN.}} &
		\cellcolor{cell-lightyellow} eines               &
		\cellcolor{cell-lightyellow} eines               &
		\cellcolor{cell-lightpurple} einer               &
		\cellcolor{cell-lightpurple} einer \\


\end{tabularx}

\vspace{0.5cm}

\linenumbers

% }}} END-TABLE : Genetive_Articles

A noun will recieve Genetive grammatical article in the following situations
:\newpar 
\nolinenumbers
\begin{itemize}[noitemsep]

	\item     If the word is after the word ` \textit{of} ' in English

	\item If it follows a preposition that is Genitive (anstatt, aufgrund,
	außerhalb, dank, statt, während, wegen)

\end{itemize}
\linenumbers

An example to aid in the clarification of the Genetive case is :\newpar


\noindent
\textit{Die Zukunft des Buches ist schwer}\\
\textcolor{gray} { \textit{( The future of the book is difficult )} } \newpar

Just like the previous section here is a breakdown of all of the objects in the
sentence.


*(In English genitive’s expressed with of or by adding an apostrophe to show
possession. Des Buches is translated as of the book or the book’s)

The genitive is not used as often by Germans as the three
other previous cases.


Often, a noun object is made with the preposition von +
Dative and the genitive preposition are sometimes used incorrectly as if they
were dative.


You have to keep in mind that one word can fit the rules of different cases
simultaneously.


For example, it can be a subject while being a
part of a noun object and follow a preposition that is dative. Which case would
it be then?  Nominative because it’s the subject, Genitive, because it’s the
noun object or dative because it is after a preposition?  The answer is that the
priorities are in this order: Following a preposition (governing with
Accusative, Dative or Genitive) Being part of a genitive object (Genitive) The
rest of the rules

% }}} END SUB-SUB-SECTION : genetive

Following we have all the cases with thier correponding articles in two tables,
one for all the definite articles and one for the indefinite ones :\newpar


% TABLE : cases_definite_articles {{{

\nolinenumbers

\vspace{0.2cm}

\begin{table-bg}[width=\linewidth]{}

	\begin{table-title}[width=6.5cm]{}
		\captionsetup{labelformat=empty}
		\captionof{table}{Cases : Definite Articles}
	\end{table-title}

	\begin{table-theword}
		\footnotesize \begin{mytable}\label{table:cases_definite_articles} \end{mytable}
	\end{table-theword}

	\begin{table-content}
	\begin{tabularx}
		{\textwidth}{l|XXXX}

		&
		\cellcolor{lightgray} \textbf{\textit{MAS.}}  &
		\cellcolor{lightgray} \textbf{\textit{NEU.}}  &
		\cellcolor{lightgray} \textbf{\textit{FEM.}}  &
		\cellcolor{lightgray} \textbf{\textit{PLU.}} \\

		\midrule

		\cellcolor{lightgray} \textbf{\textit{NOM.}} &
		\cellcolor{cell-lightpurple}  der            &
		\cellcolor{cell-lightorange}  das            &
		\cellcolor{cell-lightblue}    die            &
		\cellcolor{cell-lightblue}    die \\

		\cellcolor{lightgray} \textbf{\textit{ACC.}} &
		\cellcolor{cell-lightgreen}   den            &
		\cellcolor{cell-lightorange}  das            &
		\cellcolor{cell-lightblue}    die            &
		\cellcolor{cell-lightblue}    die \\

		\cellcolor{lightgray} \textbf{\textit{DAT.}} &
		\cellcolor{cell-lightred}    dem             &
		\cellcolor{cell-lightred}    dem             &
		\cellcolor{cell-lightpurple} der             &
		\cellcolor{cell-lightgreen}  den \\

		\cellcolor{lightgray} \textbf{\textit{GEN.}} &
		\cellcolor{cell-lightyellow} des               &
		\cellcolor{cell-lightyellow} des               &
		\cellcolor{cell-lightpurple} der               &
		\cellcolor{cell-lightpurple} der \\



	\end{tabularx}
	\end{table-content}

\end{table-bg}

\linenumbers

% }}} End TABLE : cases_definite_articles

% TABLE : cases_indefinite_articles {{{

\nolinenumbers

\begin{table-bg}[width=\linewidth]{}

	\begin{table-title}[width=6.5cm]{}
		\captionsetup{labelformat=empty}
		\captionof{table}{Cases : Indefinite Articles}
	\end{table-title}

	\begin{table-theword}
		\footnotesize \begin{mytable}\label{table:cases_indefinite_articles} \end{mytable}
	\end{table-theword}

	\begin{table-content}
	\begin{tabularx}
		{\textwidth}{l|XXXX}

		&
		\cellcolor{lightgray} \textbf{\textit{MAS.}} &
		\cellcolor{lightgray} \textbf{\textit{NEU.}}  &
		\cellcolor{lightgray} \textbf{\textit{FEM.}}  &
		\cellcolor{lightgray} \textbf{\textit{PLU.}}~\footnote{In german
			sometimes the word \textit{einige (some) } is used to refer to an indefinite
number of objects in plural.} \\

\midrule

\cellcolor{lightgray} \textbf{\textit{NOM.}} &
\cellcolor{cell-lightpurple}  ein            &
\cellcolor{cell-lightorange}  ein            &
\cellcolor{cell-lightblue} eine              &
\cellcolor{table-bg} - \\

\cellcolor{lightgray} \textbf{\textit{ACC.}} &
\cellcolor{cell-lightgreen} einen            &
\cellcolor{cell-lightorange}  ein            &
\cellcolor{cell-lightblue}  eine             &
\cellcolor{table-bg} - \\

\cellcolor{lightgray} \textbf{\textit{DAT.}} &
\cellcolor{cell-lightred} einem              &
\cellcolor{cell-lightred} einem              &
\cellcolor{cell-lightpurple} einer           &
\cellcolor{table-bg} - \\

\cellcolor{lightgray} \textbf{\textit{GEN.}} &
\cellcolor{cell-lightyellow} eines           &
\cellcolor{cell-lightyellow} eines           &
\cellcolor{cell-lightpurple} einer           &
\cellcolor{table-bg} - \\

	\end{tabularx}
	\end{table-content}

\end{table-bg}

\linenumbers

% }}} End TABLE : cases_definite_articles


% }}} END SUB-SECTION : cases

% 	SUB-SECTION : contractions {{{
\subsection{Contractions}
\label{ssec:contractions}



The definite articles are contracted with prepositions in these cases:

an + das = ans, an + dem = am, auf + das = aufs, bei + dem = beim, durch + das =
durchs, für + das = fürs, in + das = ins, in + dem = im, um + das = ums, von +
dem = vom, zu + der = zur, zu + dem = zum

\subsectionend

% }}} END SUB-SECTION : contractions


\sectionend

% }}} END SECTION : nouns

% SECTION : verbs {{{
\section{{Verbs}}
\label{sec:verbs}


% DEFINITION : verbs {{{

\vspace{0.2cm}
\centering
\nolinenumbers
\begin{defn-bg}

\label{def:verbs_infinitives}
	\begin{defn-title}[width=7cm]{}
		{\normalsize \textbf{\textit{Verbs}} }
	\end{defn-title}

	\begin{defn-theword}
	{
		\footnotesize \begin{mydef} \end{mydef}
	}
	\end{defn-theword}

	\begin{defn-content}
		\justify
	
\textbf {Verbs} are words that refer to actions. These actions are happening to
the nouns (substantives).

	\end{defn-content}

\end{defn-bg}

\justifying
\linenumbers

% }}} END DEFINITION : verbs


\lettrine[lines=3, findent=3pt, nindent=0pt]{A} verb is basically any word that
allows us to describe an action, an ongoing process or a state of being. E.g. :
to walk (laufen), to sing (singen).  Verbs are made up of two parts : the
stem which is the main body of the verb, and the ending.\\

\noindent
E.g : Laufen , Singen , Haben , Machen\\

Every verb must be \textbf{\textit{Conjugated}}. To conjugate a verb means to
change it from it's base infinitive form to a form matching the pronoun that we
are using in the current sentence.



% DEFINITION : infinitive {{{

\vspace{0.2cm}
\centering
\nolinenumbers
\begin{defn-bg}

\label{def:infinitive}
	\begin{defn-title}[width=7cm]{}
		{\normalsize \textbf{\textit{Infinitive}} }
	\end{defn-title}

	\theworddefinition

	\begin{defn-content}
		\justify
		Infinitive is the base form of any verb. This is the unconjugated form.
	\end{defn-content}

\end{defn-bg}

\justifying
\linenumbers

% }}} END DEFINITION : infinitive


When we say we are conjugating / declining a verb, we mean that we are changing
the stems and endings of the verbs in order to suit the pronoun that we are
currently using.\\

In german verbs can be broken up into two different forms according to the way
that we are going to conjugate them. These two forms are depend on wether we
change only the ending during conjugation , or wether we change both the ending
and the stem during the conjugation process. All verbs in german fall into one
of the two following categories :


% LIST : Regular Verbs Conjugation {{{

\vspace{0.3cm}
\begin{tabular}{l l l l}

\rowcolor{white} $\bullet$ Change & e  & to & ie\\
\rowcolor{white} $\bullet$ Change & e  & to & i\\
\rowcolor{white} $\bullet$ Change & el & to & il\\
\rowcolor{white} $\bullet$ Change & eh & to & im\\
\rowcolor{white} $\bullet$ Change & a  & to & ä\\

\end{tabular}
\vspace{0.3cm}
\newline

% }}} End LIST : Regular verbs Conjugation
% LIST : Irregular Verbs Conjugation {{{

\vspace{0.3cm}
\begin{tabular}{l l l l}

\rowcolor{white} $\bullet$ Change & e  & to & ie\\
\rowcolor{white} $\bullet$ Change & e  & to & i\\
\rowcolor{white} $\bullet$ Change & el & to & il\\
\rowcolor{white} $\bullet$ Change & eh & to & im\\
\rowcolor{white} $\bullet$ Change & a  & to & ä\\

\end{tabular}
\vspace{0.3cm}
\newline

% }}} End LIST : Irregular verbs Conjugation

% 	SUB-SECTION : imperative {{{
\subsection{Imperative}
\label{ssec:imperative}

The german imperative is a way of basically ordering someone to do something
instead of asking them politely. The imperative can be formed using three
differnt ways, depending on the count and politeness that we wish to employ. The
pronoun is never used when we are trying to build an imperative sentence. The
conjugation of the verb is also a bit different. The three main forms that will
change are as follows :\newpar


% TABLE : imperative_explanation {{{

\nolinenumbers

\vspace{0.5cm}

\begin{tabularx}{0.95\linewidth}{llX}

\rowcolor{white} \bulletpoint du  & : & The -st(or just -t) ending on the conjugation will
go away.  \\

\rowcolor{white} & e.g  & \textcolor{gray}{iss deine Essen.}\\
\rowcolor{white} \bulletpoint Sie & : & The conjugation stays the same. \\
\rowcolor{white} \bulletpoint ihr & : & The conjugation stays the same. \\

\end{tabularx}

\vspace{0.5cm}

\linenumbers

% }}} End TABLE : Imperative Explanation

if the umlaut is in the infinitive, then the du form will remain the same in the
imperative , i.e.\ it will still have the umlaut.\

if the umlaut is not there in the infinitive, but it appears due to it being a
irregular verb, then it will go away in the imperative form.\

if the umlaut is not there in the infinitive, but it appears due to it being a
irregular verb, then it will go away in the imperative form.\


% }}} END SUB-SECTION : imperative

% 	SUB-SECTION : present_tense {{{
\textbf{\subsection{Present Tense}}
\label{sec:present_tense}



% }}} END SUB-SECTION : present_tense

% 	SUB-SECTION : past_tense {{{
\textbf{\subsection{Past Tense}}
\label{sec:past_tense}



% }}} END SUB-SECTION : past_tense

% 	SUB-SECTION : future_tense {{{
\textbf{\subsection{Future Tense}}
\label{sec:future_tense}



% }}} END SUB-SECTION : future_tense

% 	SUB-SECTION : special_verbs {{{
\textbf{\subsection{Special Verbs}}
\label{sec:special_verbs}

% 		SUB-SUB-SECTION : werden {{{
\textbf{\subsubsection{Werden}}
\label{sec:werden}



% }}} END SUB-SUB-SECTION : werden

% 		SUB-SUB-SECTION : lassen {{{
\textbf{\subsubsection{Lassen}}
\label{sec:lassen}



% }}} END SUB-SUB-SECTION : lassen

% }}} END SUB-SECTION : special_verbs

% 	SUB-SECTION : modal_verbs {{{
\textbf{\subsection{Modal Verbs}}
\label{sec:modal_verbs}


% DEFINITION : modal_verbs {{{

\vspace{0.2cm}
\centering
\nolinenumbers
\begin{defn-bg}

\label{def:modal_verbs}
	\begin{defn-title}[width=7cm]{}
		{\normalsize \textbf{\textit{Modal Verbs}} }
	\end{defn-title}

	\begin{defn-theword}
	{
		\footnotesize \begin{mydef} \end{mydef}
	}
	\end{defn-theword}

	\begin{defn-content} \justify Modal Verbs are verbs that allow us to change
		/ modify the original sentence to add degrees of permission and
		necessity.

	\end{defn-content}

\end{defn-bg}

\justifying
\linenumbers

% }}} END DEFINITION : modal_verbs

Modal Verbs are always used in conjunction with a another verb. The modal verbs
action is to indication to what extent the action specified by the other verb is
necessary or allowed in the current sentence.\\

Modal verbs are also often called \textbf{\textit{auxiliary / helping verbs}}
(Die Hilfsverben)\\

When we are building a sentence with a modal verb, we put the modal verb in its
conjugated form in the second place of the sentence, and send the original verb
that was supposed to be in the second place all the way to the end of the
sentence. Another thing to note is that we should not conjugate the second
verb.\\

The general sentence structure in the present tense is as follows :\\

Pronoun - Modal Verb (conj.) - frequency/time - other words - second verb
(unconj.)\\

In the case of W-questions , or Ja / Nien Questions the sentence structure
remains the same as before with the modal verb in place 1 for Ja / Nien
Questions, and in position 2 in W-Style questions. The most common modal verbs
are :

% LIST : modal_verbs_list {{{

\vspace{0.3cm}
\begin{tabular}{l l}

\rowcolor{white} $\bullet$ dürfen & (to be allowed)  \\
\rowcolor{white} $\bullet$ können & (to be able)  \\
\rowcolor{white} $\bullet$ mögen & (would like to)  \\ 
\rowcolor{white} $\bullet$ wollen & (want to)  \\
\rowcolor{white} $\bullet$ sollen & (should)\\
\rowcolor{white} $\bullet$ mussen & (must) \\

\end{tabular}
\vspace{0.3cm}
\newline

% }}} End LIST : modal_verbs_list

Be careful with saying „Ich will“ as it sounds impolite if you are asking for
something. It is more appropriate to say “Ich möchte“ or “Ich hätte
gern“.\\

A table of all modal verbs with conjugations is shown below :




% TABLE : modal_verbs {{{

\nolinenumbers

\begin{table-bg}[width=\textwidth]{}

\label{table:modal_verbs}

	\begin{table-title}[width=6.5cm]{}
		\captionsetup{labelformat=empty}
		\captionof{table}{Modal Verbs}
	\end{table-title}

	\begin{table-theword}
		\footnotesize \begin{mytable} \end{mytable}
	\end{table-theword}

	\begin{table-content}
	\begin{tabularx}{\textwidth}{X|X|X|X|XXX}
&
\cellcolor{gray-light} \textbf{ich} &
\cellcolor{gray-light} \textbf{du} &
\cellcolor{gray-light} \textbf{er/sie/es} &
\cellcolor{gray-light} \textbf{wir} &
\cellcolor{gray-light} \textbf{ihr} &
\cellcolor{gray-light} \textbf{sie/Sie} \\
\midrule

\cellcolor{gray-light} \textbf{\textit{dürfen}} &
\cellcolor{white}                  &
\cellcolor{white}                  &
\cellcolor{white}                  &
\cellcolor{white}                  &
\cellcolor{white}                  &
\cellcolor{white} \\

\cellcolor{gray-light} \textbf{\textit{können}} &
\cellcolor{white}                  &
\cellcolor{white}                  &
\cellcolor{white}                  &
\cellcolor{white}                  &
\cellcolor{white}                  &
\cellcolor{white} \\

\cellcolor{gray-light} \textbf{\textit{mögen}} &
\cellcolor{white}                  &
\cellcolor{white}                  &
\cellcolor{white}                  &
\cellcolor{white}                  &
\cellcolor{white}                  &
\cellcolor{white} \\

\cellcolor{gray-light} \textbf{\textit{wollen}} &
\cellcolor{white}                  &
\cellcolor{white}                  &
\cellcolor{white}                  &
\cellcolor{white}                  &
\cellcolor{white}                  &
\cellcolor{white} \\

\cellcolor{gray-light} \textbf{\textit{sollen}} &
\cellcolor{white}                  &
\cellcolor{white}                  &
\cellcolor{white}                  &
\cellcolor{white}                  &
\cellcolor{white}                  &
\cellcolor{white} \\

\cellcolor{gray-light} \textbf{\textit{mussen}} &
\cellcolor{white}                  &
\cellcolor{white}                  &
\cellcolor{white}                  &
\cellcolor{white}                  &
\cellcolor{white}                  &
\cellcolor{white} \\



	\end{tabularx}
	\end{table-content}

\end{table-bg}

\linenumbers

% }}} End TABLE : modal_verbs


% }}} END SUB-SECTION : modal_verbs

% 	SUB-SECTION : reflexive_verbs {{{
\textbf{\subsection{Reflexive Verbs}}
\label{sec:reflexive_verbs}


% DEFINITION : copulative_verb {{{

\vspace{0.2cm}
\centering
\nolinenumbers
\begin{defn-bg}


	\begin{defn-title}[width=7cm]{}
		{\normalsize \textbf{\textit{Copulative Verb}} }
	\end{defn-title}

	\begin{defn-theword}
	{
		\footnotesize \begin{mydef}~\label{def:copulative_verb}\end{mydef}
	}
	\end{defn-theword}

	\begin{defn-content}
		\justify
		A copulative verb is one that serves to link the subject of the sentence
		with the predicate in the
		sentence.\newpar
		The word ``{\textit{is}}'' in the sentence:\newpar
		The sky is blue.~\footnote{
			\url{https://en.wikipedia.org/wiki/Copula\_(linguistics)}}~\footnote{In the german context the main copulative verbs are :
			\textit{sein} , \textit{werden} and \textit{bleiben} }

	\end{defn-content}

\end{defn-bg}

\justifying
\linenumbers

% }}} END DEFINITION : copulative_verb

% }}} END SUB-SECTION : reflexive_verbs

% 	SUB-SECTION : conjunctive_ii {{{
\textbf{\subsection{Conjunctive II}}
\label{sec:conjunctive_ii}



% }}} END SUB-SECTION : conjunctive_ii


\sectionend


% }}} END SECTION : verbs

% SECTION : prepositions {{{
\section{{Prepositions}}
\label{sec:prepositions}


\lettrine[lines=3, findent=3pt, nindent=0pt]{P}{repositions} indicate the
relationship of a noun (or pronoun) to another element in the sentence.
Prepositions tend to be some of the most commonly used words in a language.
Following are some examples of prepositions : We are going to the
Apartment. (Wir gehen zu die Wohnung) The food is inside the refrigerator. (Das
Essen ist innen die Kuhlschrank.) It is behind the chair. (Es ist hinter den
Stühl) Prepositions work in much the same way in German, except for the added
complication that the nouns and pronouns that they are refereeing to are going
to change according to the articles of the nouns that we are using the
preposition to refer to. Some prepositions are specific to certain cases, i.e.,
certain prepositions will always invoke the dative or the accusative case. These
are explained in more detail in separate sections below.


Depending on the article of the noun that we are using the preposition with, the
form of the preposition might change. The change is basically that the
preposition gets contracted (see contractions section above) with the article of
the noun in the dative case.
Normally “bei dem,” “von dem,” “zu dem,” and “zu der” are the ones that are most
commonly contracted.  They are contracted into the following forms :
bei + dem = beim
von + dem = vom
zu + dem = zum
zu + der = zur
If we have any other article preposition combination except for the ones listed
above, just write the full preposition and article out. For example :
zu + die =/= zuie , it will remain as zu die
bei + der =/= bier , it will remain as bei der


Unlike dative or accusative prepositions that we learned earlier, which can only
be used in their respective cases, Wechsel Prepositions are prepositions that
can be used in two different cases, namely wechsel prepositions can be used with
objects that are in the dative case (indirect objects) and in the accusative
case (direct objects). Wechsel prepositions are known as dual prepositions in
English since they can be used with two cases.
The easiest way to determine if in a given sentence we are using the dative or
the accusative version of the wechsel preposition is by looking the question
that the sentence is answering. Extremely simply If the sentence is answering
a wo (where) question about the object, then we are using the wechsel
preposition in the accusative case. If the sentence is answering a wohin (where
to) question about the object then we are using the wechsel preposition in the
dative case.
One thing to clarify when talking about wechsel prepositions is – the fact that
when we say we are using the preposition in the accusative or dative case, the
preposition itself is not changing. What we actually mean is that the article of
the noun that the preposition is talking about will get changed into either it’s
accusative case form or the dative case form.
To further understand the point made earlier about wo and wohin questions, it
helps to think about the movement of the object in the sentence. A way to think
about it in English is using the two phrases “he jumps into the water”
versus “he is swimming in the water.” The first answers a “where to” question:
Where is he jumping? Into the water. Or in German, in das Wasser or ins Wasser.
He is changing location by moving from the land into the water. The second
phrase represents a “where” situation. Where is he swimming? In the water.
In German, in dem Wasser or im Wasser. He is swimming inside the body of water
and not moving in and out of that one location.


So basically :
Use the accusative if there is a significant change of location / position
happening to the object in the sentence, i.e., if the action (verb) is resulting
in the object being moved from one place to a different place then we will use
the accusative case with the wechsel preposition.
If there is no significant change in movement then, the action is occurring in a
confined space and little or no movement is taking place. Then we will use the
dative article for the object with the wechsel preposition.




% 	SUB-SECTION : accusative_prepositions {{{
\textbf{\subsection{Accusative prepositions}}
\label{ssec:accusative_prepositions}

% TABLE : prepositions_accusative {{{

\nolinenumbers

\begin{table-bg}[width=\linewidth]{}



	\begin{table-title}[width=6.5cm]{}
		\captionsetup{labelformat=empty}
		\captionof{table}{Prepositions : Accusative}
	\end{table-title}

	\begin{table-theword}
		\footnotesize \begin{mytable}\label{table:prepositions_accusative} \end{mytable}
	\end{table-theword}

	\begin{table-content}
	\begin{tabularx}{\textwidth}{l|XX}

		& PREPOSITION & TRANSLATION \\

		\midrule

		\cellcolor{gray-light} B&
		\cellcolor{white} bis &
		\cellcolor{white} until , up-to \\

		\cellcolor{gray-light} D&
		\cellcolor{white} durch &
		\cellcolor{white} through \\


		\cellcolor{gray-light} E&
		\cellcolor{white} entlang &
		\cellcolor{white} along \\


		\cellcolor{gray-light} F&
		\cellcolor{white} für &
		\cellcolor{white} for \\


		\cellcolor{gray-light} G&
		\cellcolor{white} gegen &
		\cellcolor{white} against / opposite \\


		\cellcolor{gray-light} O&
		\cellcolor{white} ohne &
		\cellcolor{white}  without\\


		\cellcolor{gray-light} U&
		\cellcolor{white} um \ldots herum &
		\cellcolor{white}  around\\


		\cellcolor{gray-light} H&
		\cellcolor{white} hinter &
		\cellcolor{white}  behind\\


		\cellcolor{gray-light} I&
		\cellcolor{white} in &
		\cellcolor{white} in , inside \\


		\cellcolor{gray-light} N&
		\cellcolor{white} neben &
		\cellcolor{white} next to , beside \\


		\cellcolor{gray-light} Ü&
		\cellcolor{white} über &
		\cellcolor{white} over , above \\


		\cellcolor{gray-light} U&
		\cellcolor{white} unter &
		\cellcolor{white} among ,under, below \\


		\cellcolor{gray-light} V&
		\cellcolor{white} vor &
		\cellcolor{white} ahead of, in front of \\


		\cellcolor{gray-light} Z&
		\cellcolor{white} zwischen &
		\cellcolor{white}  between\\


		\cellcolor{gray-light} W&
		\cellcolor{white} wider &
		\cellcolor{white} against \\



	\end{tabularx}
	\end{table-content}

\end{table-bg}

\linenumbers

% }}} End TABLE : prepositions_accusative

% }}} END SUB-SECTION : accusative_prepositions

% 	SUB-SECTION : dative_prepositions {{{
\textbf{\subsection{Dative Prepositions}}
\label{ssec:dative_prepositions}


% TABLE : prepositions_dative {{{

\nolinenumbers

\begin{table-bg}[width=\linewidth]{}

	\begin{table-title}[width=6.5cm]{}
		\captionsetup{labelformat=empty}
		\captionof{table}{Prepositions : Dative}
	\end{table-title}

	\begin{table-theword}
		\footnotesize \begin{mytable}\label{table:prepositions_dative} \end{mytable}
	\end{table-theword}

	\begin{table-content}
	\begin{tabularx}{\textwidth}{l|XX}

		& PREPOSITION & TRANSLATION \\

		\midrule

		\cellcolor{gray-light} V&
		\cellcolor{white} von &
		\cellcolor{white} of / from \\

		\cellcolor{gray-light} Z&
		\cellcolor{white} zu &
		\cellcolor{white} to / for \\


		\cellcolor{gray-light} S&
		\cellcolor{white} seit &
		\cellcolor{white} since \\


		\cellcolor{gray-light} N&
		\cellcolor{white} nach &
		\cellcolor{white} towards / to / past (time) / after\\


		\cellcolor{gray-light} A&
		\cellcolor{white} aus &
		\cellcolor{white} out of / from / made of \\


		\cellcolor{gray-light} M&
		\cellcolor{white} mit &
		\cellcolor{white} with\\


		\cellcolor{gray-light} B&
		\cellcolor{white} bei &
		\cellcolor{white}  with / by\\


		\cellcolor{gray-light} A&
		\cellcolor{white} außer &
		\cellcolor{white}  besides\\


		\cellcolor{gray-light} G&
		\cellcolor{white} gegenüber &
		\cellcolor{white} against \\


	\end{tabularx}
	\end{table-content}


\end{table-bg}

\linenumbers

% }}} End TABLE : prepositions_dative

% }}} END SUB-SECTION : dative_prepositions

% 	SUB-SECTION : wechsel_prepositions {{{
\textbf{\subsection{Wechsel Prepositions}}
\label{ssec:wechsel_prepositions}


Wechsel accusative if the preposition indicates movement.

Wechsel dative if the preposition does not indicate movement.


% TABLE : prepositions_wechsel {{{

\nolinenumbers

\begin{table-bg}[width=\linewidth]{}

\begin{table-title}[width=6.5cm]{}
		\captionsetup{labelformat=empty}
		\captionof{table}{Prepositions : Wechsel}
	\end{table-title}

	\begin{table-theword}
		\footnotesize \begin{mytable}\label{table:prepositions_wechsel} \end{mytable}
	\end{table-theword}

	\begin{table-content}
	\begin{tabularx}{\textwidth}{l|XX}

		\cellcolor{gray-light} Z &
		\cellcolor{white} zwischen &
		\cellcolor{white} between\\

		\cellcolor{gray-light} U&
		\cellcolor{white} unter &
		\cellcolor{white} under\\

		\cellcolor{gray-light} N&
		\cellcolor{white} neben &
		\cellcolor{white} next to\\

		\cellcolor{gray-light} Ü&
		\cellcolor{white} über &
		\cellcolor{white} above\\

		\cellcolor{gray-light} H&
		\cellcolor{white} hinter &
		\cellcolor{white} behind \\

		\cellcolor{gray-light} A&
		\cellcolor{white} an &
		\cellcolor{white} at \\

		\cellcolor{gray-light} V&
		\cellcolor{white} vor &
		\cellcolor{white} in front of \\

		\cellcolor{gray-light} A&
		\cellcolor{white} auf &
		\cellcolor{white} on \\

		\cellcolor{gray-light} I&
		\cellcolor{white} in &
		\cellcolor{white} in \\

	\end{tabularx}
	\end{table-content}

\end{table-bg}

\linenumbers

% }}} End TABLE : prepositions_wechsel

% }}} END SUB-SECTION : wechsel_prepositions

% 	SUB-SECTION : genetiv_prepositions {{{
\textbf{\subsection{Genetive Prepositions}}
\label{ssec:genetiv_prepositions}

während is also used as as connector and a preposition.\\
während : three meanings\\

während : while (connector)\\
während : during (preposition)\\



% TABLE : prepositions_genetive {{{

\nolinenumbers

\begin{table-bg}[width=\linewidth]{}

	\begin{table-title}[width=6.5cm]{}
		\captionsetup{labelformat=empty}
		\captionof{table}{Prepositions : Genetive}
	\end{table-title}

	\begin{table-theword}
		\footnotesize \begin{mytable}\label{table:prepositions_genetive} \end{mytable}
	\end{table-theword}

	\begin{table-content}
	\begin{tabularx}{\textwidth}{l|XX}

	\end{tabularx}
	\end{table-content}

\end{table-bg}

\linenumbers

% }}} End TABLE : prepositions_genetive




% }}} END SUB-SECTION : genetiv_prepositions

\sectionend
% }}} END SECTION : prepositions

% SECTION : class_notes {{{
\section{{Class Notes}}
\label{sec:class_notes}


mit zahlen, wir haben kein nomen deklination.\

ein mittel zum zweck : a means to an end\\
aus dem leben : out of life\\
angst cor verantwortung übernehmen

alle : immer plural , und immer für jemand\\
alles : immer für etwas, und für sachen (etwas)\\

lebensunterhalt\\
übrigens : by the way\\


\nolinenumbers \vspace{0.2cm}
\begin{center} $\blacksquare$ \end{center}
\linenumbers

\clearpage
% }}} END SECTION : class_notes

% SECTION : vocabulary {{{
\onecolumn
\section{{Vocabulary}}
\label{sec:vocabulary}
\nolinenumbers
\noindent

% LIST :  {{{

\begin{tabularx}{0.95\linewidth}{lll|X}

\rowcolor{white} $\bullet$ mutig                 & : & brave , courageous                           & \\
\rowcolor{white} $\bullet$ opfern                & : & to sacrifice , to immolate                   & \\
\rowcolor{white} $\bullet$ mut fassen            & : & to take courage / to pluck up courage        & \\
\rowcolor{white} $\bullet$ vorkommen             & : & to occur / to appear                         & \\
\rowcolor{white} $\bullet$ sich einsetzen        & : & to do extra                                  & \\
\rowcolor{white}                                 &   & sich (fur etwas) einsetzen                   & \\
\rowcolor{white} $\bullet$ weltraum              & : & outer space                                  & \\
\rowcolor{white} $\bullet$ beitrag               & : & contribution / dues / subscription           & leisten \\
\rowcolor{white} $\bullet$ held                  & : & hero                                         & \\
\rowcolor{white} $\bullet$ heldenhaft            & : & heroic / valiant / brave                     & \\
\rowcolor{white} $\bullet$ Ufer                  & : & river bank , shore                           & \\
\rowcolor{white} $\bullet$ retten                & : & to save                                      & (vor + Dat) \\
\rowcolor{white} $\bullet$ leiden an / unten     & : & to suffer , to tolerate                      & \\
\rowcolor{white} $\bullet$ abwechslungsreich     & : & varied , diversified                         & \\
\rowcolor{white} $\bullet$ umfrage               & : & survey / poll / questionaaire                & \\
\rowcolor{white} $\bullet$ betrachten            & : & to consider                                  & (als + Akk)\\
\rowcolor{white} $\bullet$ umfrage               & : & interview                                    & \\
\rowcolor{white} $\bullet$ befragten             & : & interviewee                                  & \\
\rowcolor{white} $\bullet$ vorstellung           & : & condition / state                            & \\
\rowcolor{white} $\bullet$ zustand               & : & current state                                & \\
\rowcolor{white} $\bullet$ sternschnuppe         & : & falling star                                 & \\
\rowcolor{white} $\bullet$ fernhalten            & : & to keep away                                 & \\
\rowcolor{white} $\bullet$ hufeisen              & : & to keep away                                 & \\
\rowcolor{white} $\bullet$ aufgehangen           & : & to hang sth smw                              & \\
\rowcolor{white} $\bullet$ Neid                  & : & Envy                                         & \\
\rowcolor{white} $\bullet$ Eifersucht            & : & jealousy                                     & \\
\rowcolor{white} $\bullet$ Begriff               & : & concept                                      & \\
\rowcolor{white} $\bullet$ wahrnehmen            & : & to percieve                                  & \\
\rowcolor{white} $\bullet$ begrefigen            & : & to grasp / understand sth                    & \\
\rowcolor{white} $\bullet$ regieren              & : & to rule                                      & \\
\rowcolor{white} $\bullet$ dirigieren            & : & to delegate                                  & \\
\rowcolor{white} $\bullet$ sich verabschieden    & : & (von) to depart                              & \\
\rowcolor{white} $\bullet$ decken                & : & to blanket / to cover                        & \\
\rowcolor{white} $\bullet$ entdecken             & : & to discover                                  & \\
\rowcolor{white} $\bullet$ gesicht               & : & face                                         & \\
\rowcolor{white} $\bullet$ gericht (law)         & : & court house  / tribunal                      & \\
\rowcolor{white} $\bullet$ gericht (food)        & : & dish                                         & \\
\rowcolor{white} $\bullet$ gedicht               & : & poem / poetry                                & \\
\rowcolor{white} $\bullet$ beantragen            & : & to request sth / to apply for sth            & \\
\rowcolor{white} $\bullet$ vorstellen sich (Akk) & : & to introduce so / to suggest / introduce sth & \\
\rowcolor{white} $\bullet$ vorstellen sich (Dat) & : &                                              & \\
\rowcolor{white} $\bullet$ umgehen               & : & to circumvent / to elude                     & \\
\rowcolor{white} $\bullet$ Pauschalreise         & : & package trip / holiday                       & \\
\rowcolor{white} $\bullet$ zulassen              & : & to admit / to approve                        & \\
\rowcolor{white} $\bullet$ abgas                 & : & exhaust / exhaust fumes                      & \\
\rowcolor{white} $\bullet$ loben                 & : & to praise                                    & \\
\rowcolor{white} $\bullet$ selbstloben           & : & self prase (germans hate it)                 & \\
\rowcolor{white} $\bullet$ aussage               & : & statement / assertion                        & \\
\rowcolor{white} $\bullet$ verwirklichen         & : & to materialize / to achieve                  & \\
\rowcolor{white} $\bullet$ kaum                  & : & hardly / barely                              & \\
\rowcolor{white} $\bullet$ pflegen               & : & to maintain / foster                         & \\


\end{tabularx}
\newline
% }}} End LIST : 

% LIST :  {{{

\begin{tabularx}{0.95\linewidth}{lllX}
\rowcolor{white} $\bullet$ gelitten        & : & suffered                            & \\
\rowcolor{white} $\bullet$ ereignis        & : & event / incident                    & \\
\rowcolor{white} $\bullet$ scheitern       & : & to fial / to fall                   & \\
\rowcolor{white} $\bullet$ empfindlich     & : & dedicated , sensitive               & \\
\rowcolor{white} $\bullet$ übelnehmen      & : & to resent to miff                   & \\
\rowcolor{white} $\bullet$ geborgen        & : & secure / salvaged / contained       & \\
\rowcolor{white} $\bullet$ geborgenheit    & : & safety / concealment / protection   & \\
\rowcolor{white} $\bullet$ befreien        & : & to free                             & \\
\rowcolor{white} $\bullet$ befrieden       & : & to pacify                           & \\
\rowcolor{white} $\bullet$ eintreten       & : & to occur / to happen                & \\
\rowcolor{white} $\bullet$ betreten        & : & embarrased / abashed                & \\
\rowcolor{white} $\bullet$ seel            & : & soul                                & \\
\rowcolor{white} $\bullet$ eigenschaft     & : & feature / property / characteristic & \\
\rowcolor{white} $\bullet$ merkmal         & : & feature / trait                     & \\
\rowcolor{white} $\bullet$ angriff         & : & attack                              & \\
\rowcolor{white} $\bullet$ angreifen       & : & to attack                           & \\
\rowcolor{white} $\bullet$ verteidigen     & : & to defend                           & \\
\rowcolor{white} $\bullet$ manifestieren   & : & to manifest                         & \\
\rowcolor{white} $\bullet$ vergleichbar    & : & comparable  / similar               & \\
\rowcolor{white} $\bullet$ auswahlen       & : & to choose btw                       & \\
\rowcolor{white} $\bullet$ wahlen          & : & polls / elections                   & \\
\rowcolor{white} $\bullet$ Aufwärmen       & : &                                     & \\
\rowcolor{white} $\bullet$ Ausdruck        & : &                                     & \\
\rowcolor{white} $\bullet$ Rechtschriebung & : &                                     & \\
\rowcolor{white} \bulletpoint Satzzeichen  & : & punctuations                        & \\
\rowcolor{white} $\bullet$fragezeichen     & : &                                     & \\
\rowcolor{white} $\bullet$Ausrufezeichen   & : &                                     & \\
\rowcolor{white} $\bullet$  besprechen     & : &                                     & \\
\rowcolor{white} $\bullet$ zu entnehmen    & : & to get sth out of sth else          & \\
\rowcolor{white} $\bullet$ raten           & : &                                     & \\
\rowcolor{white} $\bullet$ abraten         & : &                                     & \\
\rowcolor{white} $\bullet$  verhältnis     & : &                                     & \\
\rowcolor{white} $\bullet$                 & : &                                     & \\
\rowcolor{white} $\bullet$                 & : &                                     & \\
\rowcolor{white} $\bullet$                 & : &                                     & \\
\rowcolor{white} $\bullet$                 & : &                                     & \\
\rowcolor{white} $\bullet$                 & : &                                     & \\
\rowcolor{white} $\bullet$                 & : &                                     & \\
\rowcolor{white} $\bullet$                 & : &                                     & \\
\rowcolor{white} $\bullet$                 & : &                                     & \\
\rowcolor{white} $\bullet$                 & : &                                     & \\
\rowcolor{white} $\bullet$                 & : &                                     & \\
\rowcolor{white} $\bullet$                 & : &                                     & \\
\rowcolor{white} $\bullet$                 & : &                                     & \\
\rowcolor{white} $\bullet$                 & : &                                     & \\
\rowcolor{white} $\bullet$                 & : &                                     & \\
\rowcolor{white} $\bullet$                 & : &                                     & \\
\rowcolor{white} $\bullet$                 & : &                                     & \\
\rowcolor{white} $\bullet$                 & : &                                     & \\
\rowcolor{white} $\bullet$                 & : &                                     & \\
\rowcolor{white} $\bullet$                 & : &                                     & \\
\rowcolor{white} $\bullet$                 & : &                                     & \\
\rowcolor{white} $\bullet$                 & : &                                     & \\
\rowcolor{white} $\bullet$                 & : &                                     & \\



\end{tabularx}
\newline
% }}} End LIST : 


etwas fallt : something missing\\
punctuations\\
ausdruck\\

machen spaß vs.\ spaß haben\\
be- no prepositions.\\

acc, + acc w.\ prep\\
a reflexive accusative is not the same thing as a normal direct object in the
sentence.\\


the direct,indirect is not always the best way for identification.\\

be careful about the explanation between relative and nebensatze.\\


\nolinenumbers \vspace{0.2cm}
\begin{center} $\blacksquare$ \end{center}
\linenumbers

\clearpage
% }}} END SECTION : vocabulary


% ----------------------------------------------------------------------------- 
% BIBLIOGRAPHY & FIGURE LISTS {{{
\onecolumn
\nolinenumbers
\bibliographystyle{unsrt}
\bibliography{bibliography/references}

% }}}
% -----------------------------------------------------------------------------
\end{document}
% =============================================================================
% - EOF - EOF - EOF - EOF - EOF - EOF - EOF - EOF - EOF - EOF - EOF - EOF -
% =============================================================================

