% =============================================================================

% PACKAGES ---------------------------------------------------------------- {{{

% DOCUMENT CLASS , ENCODING , TITLE -------------------------------------- {{{

\documentclass[a4paper,twocolumn,10pt]{article}
\setlength{\columnsep}{20pt}
%\setlength{\columnseprule}{1pt}
\usepackage[utf8]{inputenc}
\usepackage[english]{babel}

\title{This will be some super impressive title}
\author{Ishan Tiwathia}
\date{\today}

% }}}
% FONTS ------------------------------------------------------------------ {{{

\usepackage{courier}     % courier font
\usepackage{fontawesome} % font awesome glyphs
\usepackage{setspace}    % inter line spacing
\singlespacing           % set spacing to single spaced

% \onehalfspacing
% \doublespacing
% \setstretch{1.1}


% }}}
% PAGE FORMAT ------------------------------------------------------------ {{{

\usepackage[document]{ragged2e} % Text alignment package
\usepackage{enumitem}           %
\usepackage{geometry}           %
\geometry{
	a4paper,
	total = {170mm,257mm},
	top=20mm,
	left=20mm,
	right=20mm,
	bottom=20mm
}

\usepackage[switch,displaymath,mathlines]{lineno}
%\modulolinenumbers[5]
\linenumberfont{\normalfont\large\sffamily}
\renewcommand\thelinenumber{\color{gray}\arabic{linenumber}}
\setlength\linenumbersep{0.5cm}


% }}}
% TEXT FORMAT ------------------------------------------------------------ {{{

\usepackage[activate={true,nocompatibility},
final,
tracking=true,
kerning=true,
spacing=true,
factor=1100,
stretch=10,
shrink=10]{microtype}
% prevents a certiain amount of overfull hbox badness
% helps with other margin stuff

\usepackage{verbatim}
\makeatletter
\newcommand{\verbatimfont}[1]{\def\verbatim@font{#1}}%
\makeatother%\verbatimfont{courier}

\usepackage{type1cm}  % Allows font resizing
\usepackage{lettrine} % Allows font calligraphy (enlarge first char)
\renewcommand{\LettrineTextFont}{\rmfamily}


% }}}
% TITLE FORMAT  ---------------------------------------------------------- {{{

\usepackage[export]{adjustbox}
\usepackage{changepage}
\usepackage[compact,explicit]{titlesec} % Allows customization of section head

% compact : reduces spaces before and after sections
% explicit : allows for expicit positioning of title statement with #1

% title_format : command,shape,format,label,sep,before,after
% title-brackets : {}[]{}{}{}{}[] , basically only shape and aftercode have []

% Section Title Settings {{{
\titleformat {\section}
	[hang]
	{\color{black}\Large\bfseries}
	{}
	{0em}
	{
	\nolinenumbers
	\begin{section-box}
	\thesection. #1
	\end{section-box}
	}
[
\linenumbers
]

% left before-sep after-sep right-sep
\titlespacing{\section}{0cm}{0cm}{0cm}[0cm]

% }}}

% Sub-Section Title Settings {{{
\titleformat {\subsection}
	[hang]
	{\color{black}\small\bfseries}
	{}
	{0em}
	{
	\nolinenumbers
	\begin{subsection-box}
		\thesubsection. #1
	\end{subsection-box}
	}
[
\linenumbers
]

% left before-sep after-sep right-sep
\titlespacing*{\subsection}{0cm}{0cm}{0cm}[0em]

% }}}

% Sub-Sub-Section Title Settings {{{
\titleformat {\subsubsection}
	[hang]
	{\color{black}\small\bfseries}
	{}
	{0em}
	{
	\nolinenumbers
	\begin{subsubsection-box}
		\thesubsubsection. #1
	\end{subsubsection-box}
	}
[
\linenumbers
]

% left before-sep after-sep right-sep
\titlespacing{\subsubsection}{0cm}{0cm}{0cm}[0em]

% }}}

% }}}
% HYPERLINKS ------------------------------------------------------------- {{{

\usepackage{hyperref} % auto hyperlinks toc , refrences
                      % others can be manually specified

\hypersetup{
colorlinks = true,
linktoc    = all,
citecolor  = purple,
filecolor  = black,
linkcolor  = black,
urlcolor   = black
}


% }}}
% HEADER / FOOTER -------------------------------------------------------- {{{

\usepackage{fancyhdr} % allows for header and footer customizations
\pagestyle{fancy}     %
\fancyhf{}            %

\renewcommand{\headrulewidth}{0.2pt} % draw line at header
\lhead{\textit{\leftmark}}           % LEFT  : show section name at header
\rhead{\textit{\thepage}}            % RIGHT : Show page number
% \chead{ }

\renewcommand{\footrulewidth}{0pt} % draw line at footer
%\lfoot{\textit{Last Edited : \today}}
%\cfoot{center foot}
%\rfoot{\textit{tiwathia \thepage}}

\pagenumbering{arabic} % Specify type of number characters to use

% }}}
% COLORS ----------------------------------------------------------------- {{{

\usepackage  {xcolor, colortbl}

% Section Colors {{{

\definecolor {section-bg}         {HTML} { 767676}
\definecolor {subsection-bg}      {HTML} { aaaaaa}
\definecolor {subsubsection-bg}   {HTML} { dddddd}
\definecolor {section-font}       {RGB}  { 0,0,0}
\definecolor {subsection-font}    {RGB}  { 0,0,0}
\definecolor {subsubsection-font} {RGB}  { 0,0,0}

% }}}

% Table Colors {{{

\definecolor {table-topic}        {HTML} {aeb4b8}
\definecolor {table-subtopic}     {HTML} {c2c7ca}
\definecolor {table-subsubtopic}  {HTML} {d6d9db}

\definecolor {table-alternating-blue}    {HTML} {F2F3F4}
\definecolor {table-alternating-white}    {HTML} {FFFFFF} 
\definecolor {table-alternating-gray}  {RGB}  { 228,230,221}

\definecolor {cell-lightblue}     {HTML} { b2cce5}
\definecolor {cell-lightgray}     {HTML} { d8deda}
\definecolor {cell-lightorange}   {HTML} { F6C396}
\definecolor {cell-lightred}      {HTML} { F1A099}
\definecolor {cell-lightgreen}    {HTML} { C6DA7F}
\definecolor {cell-lightpurple}   {HTML} { CCB2E5}
\definecolor {cell-lightyellow}   {HTML} { FFF09A}

% }}}

% Tcolorbox Tables {{{

\definecolor {defn-bg}       {HTML} {F2F3F4}
\definecolor {defn-title}    {HTML} {A9A9A9}
\definecolor {defn-theword}  {HTML} {EDA72D}

\definecolor {note-bg}       {HTML} {F2F3F4}
\definecolor {note-theword}  {HTML} {E25A22}

\definecolor {table-bg}      {HTML} {F2F3F4}
\definecolor {table-title}   {HTML} {A9A9A9}
\definecolor {table-theword} {HTML} {9ACD32}

\definecolor {image-bg}      {HTML} {F2F3F4}
\definecolor {image-title}   {HTML} {A9A9A9}
\definecolor {image-theword} {HTML} {0067A5}

% }}}

\definecolor {gray-dark}          {RGB}  { 71,77,80}
\definecolor {gray-medium}        {RGB}  { 95,103,107}
\definecolor {gray-light}         {RGB}  { 228,230,221}

\definecolor {red-flame}       {HTML}  {E25822}
\definecolor {green-goethe}       {RGB}  {160,200,20}
\definecolor {green-goethe-light} {RGB}  {219,243,134}

% }}}
% TABLES & IMAGES -------------------------------------------------------- {{{

\usepackage{booktabs}   %
\usepackage{multirow}   %
\usepackage{tabularx}   % allow tables to stretch to page length
%\usepackage{longtable}  % allows tables to span pages
%\usepackage{ltablex}    % combination of longtable and tabularx
\usepackage{xtab} % allows page breaking tables inline

\usepackage{graphicx}   %
\usepackage{subcaption} %
\usepackage{wrapfig}    % wrap images around text
\usepackage{capt-of}    % define captions independent of figures
\usepackage{float}
\usepackage{varwidth}
\graphicspath{{images/}} % define folder path for images

% TCOLORBOX -------------------------------------------------------------- {{{

\usepackage[skins,breakable]{tcolorbox}
% skins allows use of enhanced options
% breakable allows breaking boxes between pages

% 	IMAGE TABLES (TCOLORBOX) {{{

\newtcolorbox{image-bg}[2][]{
	enhanced,
	colback           = image-bg,
	colframe          = image-bg,
	fonttitle         = \bfseries,
%	width             = 0.98\linewidth,
	beforeafter skip  = 0.5cm,
	drop fuzzy shadow = gray,
%	boxrule         = 0mm,
%	top             = 0mm,
%	bottom          = 0mm,
%	left            = 0mm,
%	right           = 0mm,
	title = #2,#1
}

\newtcolorbox{image-title}[2][]{
	enhanced,
	colback           = image-title,
	colframe          = image-title,
	fonttitle         = \bfseries,
	height            = 0.6cm,
	drop fuzzy shadow = gray,
	beforeafter skip  = 0pt,
	grow to left by   = 0.7cm,
	boxrule           = 0mm,
	top               = 0.5mm,
	bottom            = 0mm,
	left              = 1mm,
	right             = 0mm,
	sharp corners,
	title = #2,#1
}

\newtcolorbox{image-theword}{
	enhanced,
	colback           = image-theword,
	colframe          = image-theword,
	fonttitle         = \bfseries,
	drop fuzzy shadow = gray,
	width             = 1.8cm,
	height            = 0.5cm,
	beforeafter skip  = 0pt,
	grow to left by   = 0.7cm,
	boxrule           = 0mm,
	top               = 0.5mm,
	bottom            = 0mm,
	left              = 1mm,
	right             = 0mm,
	sharp corners,
}

\newtcolorbox{image-content}{
	enhanced,
	colback         = image-bg,
	colframe        = image-bg,
	fonttitle       = \bfseries,
%	enlarge top by  = -0.5cm,
	enlarge right by = 5cm,
	width           = \linewidth,
	boxrule         = 0mm,
	top             = 2mm,
	bottom          = 0mm,
	left            = 0mm,
	right           = 0mm,
%	show bounding box
}


\newtcolorbox{image-caption}[2][]{
	enhanced,
	colback           = defn-title,
	colframe          = defn-title,
	fonttitle         = \bfseries,
	halign            = center,
	height            = 0.6cm,
	drop fuzzy shadow = gray,
	before skip       = 5pt,
%	grow to right by  = 1.055\linewidth,
	enlarge bottom by = -1cm,
	boxrule           = 0mm,
	top               = 0.5mm,
	bottom            = 0mm,
	left              = 2mm,
	right             = 0mm,
	sharp corners,
	title = #2,#1
}


% }}}

% 	REGULAR TABLES (TCOLORBOX) {{{

\newtcolorbox{table-bg}[2][]{
	enhanced,
%	float,
%	breakable,
	colback           = table-bg,
	colframe          = table-bg,
	fonttitle         = \bfseries,
%	width             = 0.98\linewidth,
	beforeafter skip  = 0.5cm,
	drop fuzzy shadow = gray,
%	boxrule         = 0mm,
%	top             = 0mm,
%	bottom          = 0mm,
%	left            = 0mm,
%	right           = 0mm,
	title = #2,#1
}

\newtcolorbox{table-theword}{
	enhanced,
	colback           = table-theword,
	colframe          = table-theword,
	fonttitle         = \bfseries,
	drop fuzzy shadow = gray,
	width             = 1.5cm,
	height            = 0.5cm,
	beforeafter skip  = 0pt,
	grow to left by   = 0.7cm,
	boxrule           = 0mm,
	top               = 0.5mm,
	bottom            = 0mm,
	left              = 1mm,
	right             = 0mm,
	sharp corners,
}

\newtcolorbox{table-title}[2][]{
	enhanced,
	colback           = table-title,
	colframe          = table-title,
	fonttitle         = \bfseries,
	height            = 0.6cm,
	drop fuzzy shadow = gray,
	beforeafter skip  = 0pt,
	grow to left by   = 0.7cm,
	boxrule           = 0mm,
	top               = 0.5mm,
	bottom            = 0mm,
	left              = 1mm,
	right             = 0mm,
	sharp corners,
	title = #2,#1
}


\newtcolorbox{table-content}{
	enhanced,
	colback         = table-bg,
	colframe        = table-bg,
	fonttitle       = \bfseries,
	before skip = 0.5cm,
%	enlarge top by  = -0.5cm,
%	enlarge right by = 5cm,
	width           = \linewidth,
	boxrule         = 0mm,
	top             = 2mm,
	bottom          = 0mm,
	left            = 0mm,
	right           = 0mm
	%show bounding box
}

% }}}

% 	DEFINITION TABLE (TCOLORBOX) {{{

% \newtcbox[init options]{name}[number][default]{options}

\newtcolorbox{defn-bg}{
	enhanced,
	colback           = defn-bg,
	colframe          = defn-bg,
	fonttitle         = \bfseries,
	drop fuzzy shadow = gray,
	width             = 0.95\linewidth,
	beforeafter skip  = 0.5cm,
	arc is angular,
}

\newtcolorbox{defn-theword}{
	enhanced,
	colback           = defn-theword,
	colframe          = defn-theword,
	fonttitle         = \bfseries,
	drop fuzzy shadow = gray,
	width             = 2.5cm,
	height            = 0.5cm,
	beforeafter skip  = 0pt,
	grow to left by   = 0.7cm,
	boxrule           = 0mm,
	top               = 0.5mm,
	bottom            = 0mm,
	left              = 1mm,
	right             = 0mm,
	sharp corners,
}

\newtcolorbox{defn-title}[2][]{
	enhanced,
	colback           = defn-title,
	colframe          = defn-title,
	fonttitle         = \bfseries,
	height            = 0.6cm,
	drop fuzzy shadow = gray,
	beforeafter skip  = 0pt,
	grow to left by   = 0.7cm,
	boxrule           = 0mm,
	top               = 0.5mm,
	bottom            = 0mm,
	left              = 1mm,
	right             = 0mm,
	sharp corners,
	title = #2,#1
}

\newtcolorbox{defn-content}{
	enhanced,
	standard jigsaw,%allows transparency
	opacityback=1,
	fonttitle       = \bfseries,
%	enlarge top by  = -0.5cm,
%	enlarge right by = -5cm,
%	width           = 13cm,
	boxrule         = 0mm
%	top             = 1mm,
%	bottom          = 0mm,
%	left            = 4mm,
%	right           = 10mm,
%	show bounding box
}

% }}}

% 	NOTE TABLE (TCOLORBOX) {{{

\newtcolorbox{note-bg}{
	enhanced,
	colback           = note-bg,
	colframe          = note-bg,
	fonttitle         = \bfseries,
	drop fuzzy shadow = gray,
	width             = 0.95\linewidth,
	before skip  = 0.5cm,
	after skip  = 0.5cm,
	arc is angular,
}

\newtcolorbox{note-theword}{
	enhanced,
	colback           = note-theword,
	colframe          = note-theword,
	fonttitle         = \bfseries,
	drop fuzzy shadow = gray,
	width             = 0.7cm,
	height            = 0.5cm,
	beforeafter skip  = 0pt,
	grow to left by   = 0.7cm,
	boxrule           = 0mm,
	top               = 0.5mm,
	bottom            = 0mm,
	left              = 1mm,
	right             = 0mm,
	sharp corners,
}

\newtcolorbox{note-content}{
	enhanced,
	colback         = note-bg,
	colframe        = note-bg,
	fonttitle       = \bfseries,
%	enlarge top by  = -0.6cm,
%	enlarge left by = 1.5cm,
%	width           = 13cm,
%	boxrule         = 0mm,
%	top             = 0mm,
%	bottom          = 0mm,
%	left            = 0mm,
%	right           = 0mm
}


% }}}

% 	SECTION TITLES (TCOLORBOX) {{{

\newtcolorbox{section-box}{
	enhanced,
	colback          = section-bg,
	colframe         = section-bg,
	fonttitle        = \bfseries,
	width            = \linewidth,
	height           = 1cm,
	beforeafter skip = 0pt,
	sharp corners 
}

\newtcolorbox{subsection-box}{
	enhanced,
	colback           = subsection-bg,
	colframe          = subsection-bg,
	fonttitle         = \bfseries,
	width             = \linewidth,
	height = 0.6cm,
	sharp corners,
	beforeafter skip  = 0pt,
	boxrule           = 0mm,
	top               = 0.5mm,
	bottom            = 0mm,
	left              = 1mm,
	right             = 0mm
}

\newtcolorbox{subsubsection-box}{
	enhanced,
	colback   = subsubsection-bg,
	colframe  = subsubsection-bg,
	fonttitle = \bfseries,
	width     = \linewidth,
	height = 0.6cm,
	sharp corners,
	beforeafter skip  = 0pt,
	boxrule           = 0mm,
	top               = 0.5mm,
	bottom            = 0mm,
	left              = 1mm,
	right             = 0mm
}

% }}}




% }}}

% }}}
% CODE / CODE DISPLAY ---------------------------------------------------- {{{

\usepackage{listings}


\definecolor{codebackground}{RGB}{239,239,239}
\definecolor{codecomments}{RGB}{169,169,169}
\definecolor{codekeyword}{RGB}{249,38,114}
\definecolor{codestrings}{HTML}{ECE47E}
\definecolor{coderegular}{RGB}{39,40,34}

\lstset{
basicstyle       = \footnotesize\ttfamily,
backgroundcolor  = \color{codebackground},
commentstyle     = \color{codecomments}, % comment style
keywordstyle     = \color{codekeyword},  % keyword style
stringstyle      = \color{codestrings},  % string literal style
rulecolor        = \color{black},        % if not set, the frame-color may be changed on line-breaks
frame            = single,               % adds a frame around the code
basicstyle       = \footnotesize,        % the size of the fonts that are used for the code
keepspaces       = true,                 % keeps spaces in text, useful for keeping indentation of code
tabsize          = 2,                    % sets default tabsize
breaklines       = true,                 % sets automatic line breaking
captionpos       = b,                    % sets the caption-position to bottom
numbers          = left,
numberstyle      = \tiny,
numbersep        = 10pt,
frame            = tb,
columns          = fixed,
showstringspaces = false,
showtabs         = false,
keepspaces,
% escapeinside={\%*}{*)},  % if you want to add LaTeX within your code
framextopmargin=10pt,    % margin for the top background border
framexbottommargin=10pt, % margin for the bot background border
framexleftmargin=0pt,    % margin for the left background border
framexrightmargin=0pt    % margin for the right background border
}


% }}}
% MATH / GRAPHING  ------------------------------------------------------- {{{

\usepackage{amsmath} % basic math package
\usepackage{amssymb} % allows more math symbols
\usepackage{amsthm}  % allows custom therorem,defn,corll etc... definitions
\usepackage{mathrsfs} % some new fonts for math mode

\newtheorem{mydef}{DEFINITION}[section]
\newtheorem{myimage}{IMAGE}[section]
\newtheorem{mytable}{TABLE}[section]

% TIKZ -------------------------------------------------------------------- {{{
\usepackage{tikz}
% }}}

% }}}
% BIBLIOGRAPHY / REFERENCES / FOOTNOTES ---------------------------------- {{{

\usepackage[nottoc,numbib]{tocbibind} % to show references line in toc
\usepackage[super]{natbib}            % superscript the citations

%\usepackage[superscript,biblabel]{cite}
%\usepackage{cleveref}


\renewcommand{\thefootnote}{\roman{footnote}} % footnote style




% }}}
% TABLE OF CONTENTS ------------------------------------------------------ {{{

\usepackage{titletoc}
% margin from RHS
%\contentsmargin{1cm}

% \dottedcontents {section}[left]{above}{label-width}{leader-width}
%\dottedcontents{section}[1.8cm]{\bfseries}{3.2em}{1pc}
%\dottedcontents{subsection}[1.8cm]{}{3.2em}{1pc}
%\dottedcontents{subsubsection}[1.8cm]{}{2.8em}{1pc}


% }}}
% CUSTOM COMMANDS -------------------------------------------------------- {{{

\newcommand{\newpar}
{\par \vspace{0.3cm}}

% New Commands : Sectioning {{{

\newcommand{\sectionend}
{
\nolinenumbers
\begin{center}
%	\textbf{---------}
	$\blacksquare$
%	\textbf{---------}
\end{center}
\clearpage
\linenumbers
}

\newcommand{\subsectionend}
{
\nolinenumbers
%\begin{center}
%	\textbf{---------}
%	\hspace{0.2cm}
%	\textsection 
%	\hspace{0.05cm}
%	\thesubsection
%	\hspace{0.2cm}
%	\textbf{---------}
%\end{center}
\linenumbers
}

\newcommand{\subsubsectionend}
{
\nolinenumbers
%\begin{center}
%	\textbf{---------}
%	\hspace{0.2cm}
%	\textsection
%	\hspace{0.05cm}
%	\thesubsubsection
%	\hspace{0.2cm}
%	\textbf{---------}
%\end{center}
\linenumbers
}

% }}}

% New Commands : Symbols {{{

\newcommand{\bulletpoint}
{ $\bullet$  }

% }}}

% New Commands : Tcolorbox {{{

%	General {{{

\newcommand{\tcolorboxstart}
{
	\nolinenumbers
	\vspace{0.2cm}
	\centering
}

\newcommand{\tcolorboxend}
{
	\justifying
	\vspace{0.2cm}
	\linenumbers
}

% }}}

%	Definition {{{

\newcommand{\tcolorboxdefinition}[3]
{

\tcolorboxstart
\begin{defn-bg}

	\begin{defn-title}[width=7cm]{}
	{
		\normalsize \textbf{\textit{#1}}
	}
	\end{defn-title}

	\begin{defn-theword}
	{
		\footnotesize
		\begin{mydef} #2
%		\label{def:{#2}}
		\end{mydef}
	}
	\end{defn-theword}


	\begin{defn-content}

	\justify
	#3

	\end{defn-content}

\end{defn-bg}
\tcolorboxend
}

% }}}

%	Note {{{

\newcommand{\tcolorboxnote}[1]
{

\tcolorboxstart
\begin{note-bg}

	\begin{note-theword}
		{\footnotesize \textbf{NOTE} }
	\end{note-theword}

	\begin{note-content} \justifying

		#1

	\end{note-content}

\end{note-bg}
\tcolorboxend
}




% }}}

%	Table {{{


\newcommand{\tcolorboxtable}[5]
{
\tcolorboxstart
\begin{table-bg}#3{}

	\begin{table-title}[width=6.5cm]{}
		\captionsetup{labelformat=empty}
		\captionof{table}{#1}
	\end{table-title}

	\begin{table-theword}
		\footnotesize
		\begin{mytable}
		#2
		\end{mytable}
	\end{table-theword}

	\begin{table-content}
	\begin{tabularx}{\textwidth}{#4}

		#5
		
	\end{tabularx}
	\end{table-content}

\end{table-bg}
\tcolorboxend
}

% }}}

%	Image {{{

\newcommand{\tcolorboxfigure}[4]
{
\tcolorboxstart
\begin{image-bg}[width=\linewidth]{}

	\begin{image-title}[width=5cm]{}
		\captionsetup{labelformat=empty}
		\captionof{figure}{#1}
	\end{image-title}

	\begin{image-theword}
		\footnotesize
		\begin{myimage}
		#2
		\end{myimage}
	\end{image-theword}

	\begin{image-content}
		\includegraphics[width=\linewidth]{#3}
		\href{#4}{Source}
	\end{image-content}

\end{image-bg}
\tcolorboxend
}

% }}}

% }}}

% New Commands : Tables {{{

\newcommand{\tabularxtable}[3]
{

	\vspace{0.5cm}
	\nolinenumbers

	\begin{tabularx}{#1}{#2}
		#3
	\end{tabularx}

	\linenumbers
	\vspace{0.5cm}
}

\newcommand{\xtabulartable}[2]
{

	\vspace{0.25cm}
	\nolinenumbers

	\begin{xtabular}{#1}
		#2
	\end{xtabular}

	\vspace{0.25cm}
	\linenumbers
}







% }}}

% New Commands : References {{{

\newcommand{\refsec}[1]
{
	\hyperref[sec:#1]
	{
		(\textsection~\ref{sec:#1})
	}
}

\newcommand{\refssec}[1]
{
	\hyperref[sec:#1]
	{
		(\textsection~\ref{ssec:#1})
	}
}

\newcommand{\refsssec}[1]
{
	\hyperref[sec:#1]
	{
		(\textsection~\ref{sssec:#1})
	}
}

\newcommand{\refdef}[1]
{
	\hyperref[def:#1]
	{
		\textit{(Def.~\ref{def:#1})}
	}
}

\newcommand{\reffig}[1]
{
	\hyperref[fig:#1]
	{
		(Fig.~\ref{fig:#1})
	}
}

\newcommand{\reftable}[1]
{
	\hyperref[table:#1]
	{
		(Table.~\ref{table:#1})
	}
}
% }}}

% }}}
% TESTING ---------------------------------------------------------------- {{{

\usepackage{lipsum}    % generates filler text
\usepackage{blindtext} % generates non-latin filler text

% }}}

\usetikzlibrary{arrows} 
\usetikzlibrary{positioning}



% }}}

% ============================================================================ 
\begin{document}
% ----------------------------------------------------------------------------- 
% TOC & SETUP {{{
\raggedbottom
\onecolumn

\tableofcontents
\pagebreak

\listoftables
\clearpage
\twocolumn
\justifying


\linenumbers

% }}}
% ----------------------------------------------------------------------------- 

% SECTION : nouns {{{
\section{{Nouns}}
\label{sec:nouns}


% DEFINITION : nouns {{{
% {title} {label} {content}

\tcolorboxdefinition
{Nouns}
{\label{def:nouns}}
{


		Substantives/Nouns are people, animals, things, concepts and ideas.

}

% }}} END DEFINITION : nouns

\lettrine[lines=3, findent=3pt, nindent=0pt]{S}{ubstantives}\footnote{In German
	a substantive is spelled as substantiv so over the course of this document I
	will probably end up using both, so yeah, if someone besides me (or more
	probably my future self) is reading this, dont give me shit for having
	spelling mistakes everywhere. This statement applies for a whole bunch of
	words.} are more commonly known as a nouns. Calling something a substantive
is just a more grammatical jargony way of referring to a noun. The reason that
there are two names for the same thing goes back to latin , where the phrase
\textit{Nomen Substantivus} or \textit{the name of substance} was used. I assume
lazy humans just split this into two words meaning the same thing over the
course of history.\newpar

German Nouns / Substantives have two defining characteristics that will help you
identify them in a German sentence. They are :

\begin{enumerate}[noitemsep]
	\item The first character of a noun is always uppercase.
	\item Every noun is preceded by a grammatically gendered article.
	\item German Nouns are declined.
\end{enumerate}

% 	SUB-SECTION : gendered_nouns {{{
\subsection{Gendered Nouns}
\label{ssec:gendered_nouns}

German is a gendered language therefore every substantive comes with one of
three genders. In German the gender is known as \textit{Genus}. German
dileniates between three grammatical genders and they are :

\nolinenumbers
\begin{enumerate}[noitemsep]
	\item \textbf{Maskulin }: der
	\item \textbf{Feminin} : die
	\item \textbf{Neuter}  : das
\end{enumerate}
\linenumbers

The words attached to each gender der , die , and das are what we use to
indicate that a particular noun belongs to a certain gender class. These three
words are called \textit{articles}. These are talked about in more detail in the
next sub section.\newpar

It is very important to note that the gender of a noun is NOT related to its
physical or biological gender, so please keep this in mind. As an exaple a young
girl is : \textit{das Madchen} , which is the article for a neuter noun, even
though we would assume that a young girl would be assigned the feminine article.
It is important to keep the differnce between grammatical gender and physical
gender distict in your mind to avoid making mistakes.


\subsectionend
% }}} END SUB-SECTION : gendered_nouns

% 	SUB-SECTION : articles {{{
\subsection{Articles}
\label{ssec:articles}

As mentioned in Section~\ref{ssec:gendered_nouns}, every German noun has a corresponding
grammatical article. There are two types of article a noun can have  , and they
are :

\nolinenumbers
\begin{itemize}[noitemsep]
	\item \textbf{Definite Articles} : The english equivalent is the word ``{\textit{the}}''
	\item \textbf{In-definite Articles} : The english equivalent is the word ``{\textit{a}}''
\end{itemize}
\linenumbers

There are vairous things that affect what the exact article is for the word that
we are using. The main things to keep into consideration for each German noun
are :

\nolinenumbers
\begin{itemize}[noitemsep]
	\item Grammatical Gender
	\item Count (Singular / Plural)
	\item Case
\end{itemize}
\linenumbers

Based on these three things the article we are using for each noun becomes very
specific and gives a detailed description of the function this noun is serving
in the sentence.\newpar

Since there is no noun without an article in German, the basis for discussing
articles only arises when we understand the German cases. So cases and articles
changes are discussed in Section~\ref{sssec:article_tips_neuter_das_} which is
exclusively about articles and Cases.\newpar

So we need to learn every noun in German along with its corresponding article. I
really don't expect most sane humans will bother sitting around memorizing the
article for each word, so the next couple of subsections have some tricks to
help in guessing them.


% NOTE : Article Importance {{{

\tcolorboxnote
{

		The importance of articles in German cannot be emphasized enough. In no
		correctly formed German sentence will there exist a noun without its
		definite or indefinite article.\newpar

		Every single noun, in every single sentence MUST be written with a
		corresponding grammatical article (in the correct case of course).

}



% }}} END NOTE : article_importance


% 		SUB-SUB-SECTION : article_tips_masculine_der_ {{{
\subsubsection{Article Tips : Masculine (Der)}
\label{sssec:article_tips_masculine_der_}

The following list provides some common roots / endings of words that will
recieve the \textbf{\textit{der (Maskulin)}} article.

% LIST : der (Maskulin) {{{

\tabularxtable
{ 0.95\linewidth }
{ lrX }
{

\rowcolor{white} $\bullet$ -ant   & e.g. & der Konsonant\\
\rowcolor{white} $\bullet$ -ast   & e.g. & der Gast     \\
\rowcolor{white} $\bullet$ -ich   & e.g. & der Teppich  \\
\rowcolor{white} $\bullet$ -ismus & e.g. & der Marxismus\\
\rowcolor{white} $\bullet$ -ling  & e.g. & der Häftling \\
\rowcolor{white} $\bullet$ -us    & e.g. & der Rythmus  \\
\rowcolor{white} $\bullet$ -er    & e.g. & der Sommer   \\

}

% }}} End LIST : der (Maskulin)

A note about the last one with the -er ending. This one not only means that the
grammatical gender of the noun is masculine, but most of the time often is also
referring to the physical gender. E.g.\ der Lehrer (the male teacher), der
Amerikaner (the male American), der Fahrer (the male driver).\newpar

The following things will always be masculine regardless of what the noun ending
is :

% ITEMIZE : always_masculine_nouns {{{

\begin{itemize}[noitemsep]
	\item \textbf{Times of the day} \textit{ (Tageszeiten) }

\noindent
\textit{der Morgen , der Vormittag , der Mittag , der Nachmittag ,
	der Abend , der Nacht}\\
\textcolor{gray} { \textit{( Morning , Late Morning, Noon , Afternoon , Evening
		, Night)} } \newpar

	\item \textbf{Days of the week} \textit{ (Wochentage) }

\noindent
\textit{der Monntag , der Deinstag , der Mittwoch , der Donnerstag , der Freitag
, der Samstag , der Sonntag}\\
\textcolor{gray} { \textit{( Monday , Tuesday , Wednesday, Thursday , Friday ,
		Saturday , Sunday )} } \newpar

	\item \textbf{Months} \textit{ (Monate) }

\noindent \textit{der Januar ,der Februar ,der Marz ,der April ,der Mai ,der
	Juni ,der Juli ,der August , der September ,der Oktober ,der November ,der
	Dezember }\\
\textcolor{gray} { \textit{( January , February , March , April , May , June ,
		July , August , October , November , December  )} } \newpar

	\item \textbf{Seasons} \textit{ (Jahrezeiten) }
		
		\noindent
		\textit{der Sommer , der Frühling , der Herbst , der Winter }\\
		\textcolor{gray} { \textit{( Summer , Spring  , Fall , Winter  )} } \newpar
		
	\item \textbf{Weather} \textit{ (Wetter) }

		
		\noindent
		\textit{der Wind , der Regen , der Schnee , \ldots}\\
		\textcolor{gray} { \textit{( Windy , Rainy , Snowy \ldots )} } \newpar
		
		An excpetion is \textit{die wolke} (the thunder)\\

	\item \textbf{Cardinal Directions} \textit{ (Wochentage) }

A note about the cardinal directions. When we are using them we can basically
have two forms : aus der Norden , or \\



	\item \textbf{Alcoholic drinks} \textit{ (Alkoholische getranke) }

		
		\noindent
		\textit{der Wein , der Schnapps , Der vodka, \ldots}\\
		\textcolor{gray} { \textit{( Wine , Schnapps, Vodka , \ldots )} } \newpar 
		The exception to the alcoholic drinks rule is Beer which is always
		\textit{das Bier}\\


\end{itemize}

% }}} END-ITEMIZE : Always masculine nouns

\subsubsectionend
% }}} END SUB-SUB-SECTION : article_tips_masculine_der_

% 		SUB-SUB-SECTION : article_tips_feminine_die_ {{{
\subsubsection{Article Tips : Feminine (Die)}
\label{sssec:article_tips_feminine_die_}

The following list provides some common roots / endings of words that will
recieve the \textbf{\textit{die (feminin)}} article.

% LIST : die_article_tricks {{{

\nolinenumbers

\vspace{0.2cm}

\begin{xtabular}{l r l}

\rowcolor{white}  $\bullet$ -ung    & e.g. & die Entscheid\textbf{\textcolor{green-goethe}{ung}} \\
\rowcolor{white}  $\bullet$ -tät    & e.g. & die Universi
\textcolor{green-goethe}{\textbf{tät}}  \\
\rowcolor{white}  $\bullet$ -tion   & e.g. & die     \\
\rowcolor{white}  $\bullet$ -sion   & e.g. & die Explo \textcolor{green-goethe}{\textbf{sion}}    \\
\rowcolor{white}  $\bullet$ -schaft & e.g. & die Gesell \textcolor{green-goethe}{\textbf{schaft}} \\
\rowcolor{white}  $\bullet$ -heit   & e.g. & die Schön \textcolor{green-goethe}{\textbf{heit}}    \\
\rowcolor{white}  $\bullet$ -keit   & e.g. & die Schnellig
\textcolor{green-goethe}{\textbf{keit}}\\
\rowcolor{white}  $\bullet$ -ie     & e.g. & die Geograph
\textcolor{green-goethe}{\textbf{ie}}   \\
\rowcolor{white}  $\bullet$ -enz    & e.g. & die              \\
\rowcolor{white}  $\bullet$ -anz    & e.g. & die Toler \textcolor{green-goethe}{\textbf{anz}}     \\
\rowcolor{white}  $\bullet$ -ei     & e.g. & die Schläger
\textcolor{green-goethe}{\textbf{ei}}   \\
\rowcolor{white}  $\bullet$ -ur     & e.g. & die Nat \textcolor{green-goethe}{\textbf{ur}}        \\
\rowcolor{white}  $\bullet$ -in     & e.g. & die Boxer \textcolor{green-goethe}{\textbf{in}}      \\
\rowcolor{white}  $\bullet$ -itis   & e.g. & die Bronch \textcolor{green-goethe}{\textbf{itis}} \\
\rowcolor{white}  $\bullet$ -sis    & e.g. & die Gene \textcolor{green-goethe}{\textbf{sis}}\\
\rowcolor{white}  $\bullet$ -ik     & e.g. & die Fabr \textcolor{green-goethe}{\textbf{ik}}\\
\rowcolor{white}  $\bullet$ -ade    & e.g. & die Limon \textcolor{green-goethe}{\textbf{ade}}\\
\rowcolor{white}  $\bullet$ -age    & e.g. & die Fr \textcolor{green-goethe}{\textbf{age}}\\
\rowcolor{white}  $\bullet$ -ine    & e.g. & die Masch \textcolor{green-goethe}{\textbf{ine}}\\
\rowcolor{white}  $\bullet$ -ere    & e.g. & die Sch \textcolor{green-goethe}{\textbf{ere}}\\
\rowcolor{white}  $\bullet$ -isse   & e.g. & die Kentn \textcolor{green-goethe}{\textbf{isse}}\\
\rowcolor{white}  $\bullet$ -ive    & e.g. & die Alternat
\textcolor{green-goethe}{\textbf{ive}}\\
\rowcolor{white}  $\bullet$ -se     & e.g. & die Ro \textcolor{green-goethe}{\textbf{se}}\\

\end{xtabular}

\vspace{0.2cm}

\linenumbers

% }}} End TABLE : die_article_tricks

Just like the note about the -er in the der section, the last point with the -in
ending not only means that the grammatical gender of the noun is
feminine, but most of the time often is also reffering to the phyisical gender.
E.g.\ die Lehrerin (the female teacher) , die Fahrerin (the female driver), die
Amerikanerin (the female american)\newpar

% NOTE : Die Article Note {{{

\tcolorboxnote
{
	The large majority of nouns which end in -e
		are feminine, e.g. : die Lampe (the lamp), die Rede (the speech), and
		die Bühne (the stage).\newpar

		This is true roughly 80\% of the time opposed to being a grammatical
		rule, which is why this information is in a note as opposed to in the
		list above.



}



% }}} END NOTE : die_article_note

The following things will always be feminine regardless of what the noun ending
is :

% ITEMIZE : always_feminine_nouns {{{

\begin{itemize}[noitemsep]

	\item \textbf{Motobike Brands} \textit{ (Motorradmarken) }


\noindent
\textit{die Yamaha , die Harley-Davidson}\\





\end{itemize}

% }}} END-ITEMIZE : Always feminine nouns



\


\subsubsectionend
% }}} END SUB-SUB-SECTION : article_tips_feminine_die_

% 		SUB-SUB-SECTION : article_tips_neuter_das_ {{{
\subsubsection{Article Tips : Neuter (Das)}
\label{sssec:article_tips_neuter_das_}

The following list provides some common roots / endings of words that will
recieve the \textbf{\textit{das (neuter)}} article.

% LIST : das (neutrum)  {{{

\nolinenumbers

\vspace{0.2cm}

\begin{tabularx}{\linewidth}{l r l}

\rowcolor{white} $\bullet$ -chen & e.g. & Das Häuschen \\
\rowcolor{white} $\bullet$ -lein & e.g. & Das Büchlein \\
\rowcolor{white} $\bullet$ -um & e.g. & Das Wachstum \\

\end{tabularx}

\vspace{0.2cm}

\linenumbers


% }}} End LIST : das (neutrum)

% NOTE : das article note {{{
\nolinenumbers
\vspace{0.2cm}
\centering
\begin{note-bg}

	\begin{note-theword}
		{\footnotesize \textbf{NOTE} }
	\end{note-theword}

	\begin{note-content} \justifying Similar to the note in the femnine article
		tips section, a lot of the German nouns that begin with Ge- are neuter
		but not all, which is why you are reading this in a note right now and
		not the main list above.

	\end{note-content}

\end{note-bg}
\linenumbers
\justifying

% }}} END NOTE : das_article_note

The following things will always be neutral regardless of what the noun ending
is :

% ITEMIZE : always_feminine_nouns {{{

\begin{itemize}[noitemsep]

	\item \textbf{Names of Colors} \textit{ (Farbnamen) }


\noindent
\textit{das Weiß , das Blau , das Rot , das Grau , das Schwarz , \ldots}\\
\textcolor{gray} { \textit{( white , blue , red , gray , black , \ldots )} } \newpar


	\item The exception from the der section on alcoholic drinks : \textit{das
			Bier}

\end{itemize}

% }}} END-ITEMIZE : Always feminine nouns





\subsubsectionend
% }}} END SUB-SUB-SECTION : article_tips_neuter_das_


\subsectionend
% }}} END SUB-SECTION : articles

% 	SUB-SECTION : plurals {{{
\subsection{Plurals}
\label{ssec:plurals}

German just like English build plurals out of nouns by appending certain endings
to the noun. Its a bit more involved than english however, since english has
only the -s ending, German has a few more. The ending appended is realtively
arbitrary so there are few choices but to learn the plural formation along with
the original noun, although after a while you should get a feel for what kind of
word will get what kind of plural formation.\newpar

Below are some general guidelines to building the plural formation for a noun
:\cite{em},\cite{Germanveryeasy},\cite{deutschlingola}

% 		SUB-SUB-SECTION : plurals_masculine_nouns {{{
\textbf{\subsubsection{Plurals : Masculine Nouns}}
\label{sec:plurals_masculine_nouns}

The following list shows some common transformations to make masculine nouns
plural :

% TABLE : plural_masculine_nouns {{{


\nolinenumbers

\vspace{0.2cm}

\begin{xtabular}{lll}


		\rowcolor{white} $\bullet$ -ich  & -e     & \textcolor{gray}{\textit{der Teppich , die Teppiche} }\\
		\rowcolor{white} $\bullet$ -ig   & -e     & \textcolor{gray}{\textit{der König, die Könige} }\\
		\rowcolor{white} $\bullet$ -ling & -e     & \textcolor{gray}{\textit{der Schmetterling, die Schmetterlinge} }\\
		\rowcolor{white} $\bullet$ -är   & -e     & \textcolor{gray}{\textit{der Veterinär,die Veterinäre} }\\
		\rowcolor{white} $\bullet$ -eur   & -e    & \textcolor{gray}{\textit{der Friseur,die Friseure} }\\
		\rowcolor{white} $\bullet$       & \"{-}e & \textcolor{gray}{ \textit{der Platz , die Plätze} }\\
		\rowcolor{white}                 &        & \textcolor{gray}{ \textit{der Kuss , die Küsse} }\\
		\rowcolor{white}                 &        & \textcolor{gray}{ \textit{der Arzt , die Ärzte} }\\
		\rowcolor{white} $\bullet$       & -      & \textcolor{gray}{ \textit{der Schüler , die Schüler} }\\
		\rowcolor{white} $\bullet$ -er   & \"{-}  & \textcolor{gray}{ \textit{der Vater , die Väter} }\\
		\rowcolor{white} $\bullet$ -el   & \"{-}  & \textcolor{gray}{ \textit{der Mantel , die Mäntel} }\\
		\rowcolor{white} $\bullet$ -us   & -usse  & \textcolor{gray}{ \textit{der Bus , die Busse} }\\



\end{xtabular}

\vspace{0.2cm}

\linenumbers

% }}} End TABLE : plural masculine nouns

Please keep in mind that the list above shows trasformations in general and is
not meant to serve as a list of rules for converting nouns with the given
endings into thier respective plural forms. They are only meant to serve as
educated guesses.\newpar

A lot of masculine nouns ending in -e also follow the rules for n-Declination.
So check out \refssec{n_declination}for more details.




\subsubsectionend
% }}} END SUB-SUB-SECTION : plurals_masculine_nouns

% 		SUB-SUB-SECTION : plurals_feminine_nouns {{{
\textbf{\subsubsection{Plurals : Feminine Nouns}}
\label{sec:plurals_feminine_nouns}

The following tables illustrates the plural formations for feminine nouns :

% TABLE : plurals_feminine_nouns {{{

\nolinenumbers

\vspace{0.2cm}

\begin{xtabular}{lll}

		\rowcolor{white} $\bullet$  -ei    & -en & \textcolor{gray}{ \textit{die Datei , die Dateien} }\\
		\rowcolor{white} $\bullet$ -ung    & -en & \textcolor{gray}{ \textit{} }\\
		\rowcolor{white} $\bullet$ -heit   & -en & \textcolor{gray}{ \textit{} }\\
		\rowcolor{white} $\bullet$ -keit   & -en & \textcolor{gray}{ \textit{} }\\
		\rowcolor{white} $\bullet$ -ion    & -en & \textcolor{gray}{ \textit{} }\\
		\rowcolor{white} $\bullet$ -schaft & -en & \textcolor{gray}{ \textit{} }\\
		\rowcolor{white} $\bullet$ -ik     & -en & \textcolor{gray}{ \textit{} }\\
		\rowcolor{white} $\bullet$ -eur    & -en & \textcolor{gray}{ \textit{} }\\
		\rowcolor{white} $\bullet$ -enz    & -en & \textcolor{gray}{ \textit{} }\\
		\rowcolor{white} $\bullet$ -tät    & -en & \textcolor{gray}{ \textit{} }\\
		\rowcolor{white} $\bullet$ -itis   & -en & \textcolor{gray}{ \textit{} }\\
		\rowcolor{white} $\bullet$ -sis    & -en & \textcolor{gray}{ \textit{} }\\
		\rowcolor{white} $\bullet$ -ung    & -en & \textcolor{gray}{ \textit{} }\\
		\rowcolor{white} $\bullet$ -ung    & -en & \textcolor{gray}{ \textit{} }\\

		\midrule
		\rowcolor{white} $\bullet$ -ie  & -n & \textcolor{gray}{ \textit{die Fantasie, die Fantasien} }\\
		\rowcolor{white} $\bullet$ -ade  & -n & \textcolor{gray}{ \textit{..} }\\
		\rowcolor{white} $\bullet$ -age  & -n & \textcolor{gray}{ \textit{..} }\\
		\rowcolor{white} $\bullet$ -ere  & -n & \textcolor{gray}{ \textit{..} }\\
		\rowcolor{white} $\bullet$ -ine  & -n & \textcolor{gray}{ \textit{..} }\\
		\rowcolor{white} $\bullet$ -isse & -n & \textcolor{gray}{ \textit{..} }\\
		\rowcolor{white} $\bullet$ -ive  & -n & \textcolor{gray}{ \textit{..} }\\
		\rowcolor{white} $\bullet$ -se   & -n & \textcolor{gray}{ \textit{..} }\\

		\midrule
		\rowcolor{white} $\bullet$ -in   & -nen    & \textcolor{gray}{ \textit{..} }\\
		\rowcolor{white} $\bullet$ -      & \"{-}e & \textcolor{gray}{ \textit{..} }\\
		\rowcolor{white} $\bullet$ -nis  & -nisse  & \textcolor{gray}{ \textit{..} }\\
		\rowcolor{white} $\bullet$ -xis  & -xien   & \textcolor{gray}{ \textit{..} }\\
		\rowcolor{white} $\bullet$ -itis & -iden   & \textcolor{gray}{ \textit{..} }\\
		\rowcolor{white} $\bullet$ -aus  & -äuse   & \textcolor{gray}{ \textit{..} }\\
		\rowcolor{white} $\bullet$ -      & \"{-}   & \textcolor{gray}{ \textit{die Mutter,die Mütter} }\\
		\rowcolor{white} $\bullet$ -      & \"{-}en & \textcolor{gray}{ \textit{die Werkstatt , die Werkstätten } }\\

	\end{xtabular}

\vspace{0.2cm}

\linenumbers

% }}} End TABLE : plurals_feminine_nouns


% }}} END SUB-SUB-SECTION : plurals_feminine_nouns

% 		SUB-SUB-SECTION : plurals_neuter_nouns {{{
\textbf{\subsubsection{Plurals Neuter Nouns}}
\label{sec:plurals_neuter_nouns}



% }}} END SUB-SUB-SECTION : plurals_neuter_nouns


\subsectionend
% }}} END SUB-SECTION : plurals

% 	SUB-SECTION : cases {{{
\subsection{Cases}
\label{ssec:cases}

There are four ``{ \textbf{\textit{cases}} }'' in German, which correspond to
four different roles a noun can play in a sentence.  There is no such thing as a
case in English, therefore it is difficult to form an equivalence relation with
something that you might already know. The easiest way to go about understanding
cases is to consider the following questions when constructing sentences in
German :

\nolinenumbers
\begin{enumerate}[noitemsep]
	\item Who is doing the action ?
	\item Who or what is being directly affected by the action ?
	\item Who or what is being indirectly affected by the action ?
	\item Who is indicating ownership of what ?
\end{enumerate}
\linenumbers

Based on the answers to the questions above, every German noun falls into a
category called a case \textit{(fall , die falle)} . We already know that German
is a gendered language, and based on the grammatical gender of the noun, the
article of the changes between masculine, feminine and neuter.\newpar

Now, an added degree of complication is that based on the case, the article of
the noun will further change. Essentially cases just serve as extremely specific
articles (instead of just the simple 3 masculine , feminine and neuter) when
talking about German nouns.Basically, in every sentence  when we have a person
or a thing (noun) performing some actions (verbs). Depending on how the noun
(person/thing) is interacting with the verb(action) the article (how we refer to
the person / thing) will slightly change. This slight change is called the
application of a case to that noun.\newpar

A case is called a Falle in German.  In German we have four cases :

% LIST : German_cases_basic_list {{{
\nolinenumbers

\vspace{0.2cm}

\begin{tabularx}{0.95\linewidth}{lllX}

\rowcolor{white} \bulletpoint & \cellcolor{cell-lightred} Nominative    & : & Der Nominativ\\
\rowcolor{white}              &                                         &   & \textcolor{gray}{ \textit{Der Werfall} }\\
\rowcolor{white} \bulletpoint & \cellcolor{cell-lightyellow} Accusative & : & Der Akkusativ\\
\rowcolor{white}              &                                         &   & \textcolor{gray}{ \textit{Der Wenfall} }\\
\rowcolor{white} \bulletpoint & \cellcolor{cell-lightgreen} Dative      & : & Der Dativ\\
\rowcolor{white}              &                                         &   & \textcolor{gray}{ \textit{Der Wemfall} }\\
\rowcolor{white} \bulletpoint & \cellcolor{cell-lightblue} Genetive     & : & Der Genetiv\\
\rowcolor{white}              &                                         &   & \textcolor{gray}{ \textit{Der Wesfall} }\\

\end{tabularx}

\linenumbers

\vspace{0.2cm}
% }}} End LIST : German cases basic list

The cases are color coded to make it easier to identify which object, is in
which case in the examples given in the following sections. The tables in the
pronouns section will also use the same colors in order to maintain consistency
throughout the document.\newpar

A short summary of when to use the cases is in the bullet points below, and a
thorough explanation is further below in the sepreate sub-sections.

% LIST : cases_basic_explanation {{{

\nolinenumbers

\vspace{0.2cm}

\begin{tabularx}{0.95\linewidth}{lllX}


\rowcolor{white} \bulletpoint Nominativ & : &  Subject \\
\rowcolor{white} & &\textcolor{gray}{ \textit{Who is performing the action} }\\

\rowcolor{white} \bulletpoint Akkusativ & : &  Direct object\\
\rowcolor{white} & &\textcolor{gray}{ \textit{Who / what is the action being} }\\
\rowcolor{white} & &\textcolor{gray}{ \textit{being performed on ?} }\\

\rowcolor{white}  \bulletpoint Dativ & : &  Indirect object \\

\rowcolor{white} & &\textcolor{gray}{ \textit{Who / what is the action } }\\
\rowcolor{white} & &\textcolor{gray}{ \textit{ affecting aside from the direct } }\\
\rowcolor{white} & &\textcolor{gray}{ \textit{ object ?} }\\

\rowcolor{white}  \bulletpoint Genetiv & : &  Possession\\

\end{tabularx}

\vspace{0.2cm}

\linenumbers

% }}} End LIST : cases basic explanation

A full table of article changes according to the specific case is shown below.
This table can be super useful when making new sentences as a reference guide.
It will only make complete sense however after you have been through all of the
following sections explaining all four cases in detail.\newpar

A good thing to meintion here would be that cases change articles as both
definite and indefinite articles for the given noun change.


% 		SUB-SUB-SECTION : nominative {{{
\textbf{\subsubsection{Nominative}}
\label{sec:nominative}

To understand the nominative case, and any subsequent cases we have to
understand the two main parts that make up any sentence both in English and in
German. These two things are : \textbf{\textit{the grammatical subject}} , and ,
\textbf{\textit{the predicate}} .  Both of these things are defined below :



% DEFINITION : grammatical_subject {{{
% {title} {label} {content}

\tcolorboxdefinition
{Grammatical Subject}
{\label{def:grammatical_subject}}
{
The grammmatical subject is the person or thing about whom the current statement
is being made.\newpar

In lingustic jargon : The subject is the word or phrase that controls the
verb or the clause.~\footnote{\url{https://en.wikipedia.org/wiki/Subject\_(grammar)}}

}

% }}} END DEFINITION : grammatical_subject

% DEFINITION : predicate {{{
% {title} {label} {content}

\tcolorboxdefinition
{Predicate}
{\label{def:predicate}}
{

The predicate is the part of the sentence (or clause), that tells us where the
subject is, or what the subject does or is.\newpar

Basically the predicate is everything that is not the subject
itself.~\footnote{\url{https://www.grammar-monster.com/glossary/predicate.htm}}

}

% }}} END DEFINITION : predicate


 The subject in the sentence will always take the nominative case and therefore
 the corresponding gendered nominative article. The main subject (as is defined
 above) is the noun that is performing some action. The action here will be
 specified by the verb. The list of both definite and indefinite nominative
 articles in German is shown below :\newpar


% TABLE : Nominative Articles {{{
\nolinenumbers

\vspace{0.5cm}

\begin{tabularx}{0.94\linewidth}{l|XXXX}

		&
		\cellcolor{table-subtopic} \textbf{\textit{MAS.}} &
		\cellcolor{table-subtopic} \textbf{\textit{NEU.}}  &
		\cellcolor{table-subtopic} \textbf{\textit{FEM.}}  &
		\cellcolor{table-subtopic} \textbf{\textit{PLU.}} \\
		\midrule

		\cellcolor{table-subtopic} \textbf{\textit{NOM.}} &
		\cellcolor{cell-lightpurple}  der            &
		\cellcolor{cell-lightorange}  das            &
		\cellcolor{cell-lightblue} die               &
		\cellcolor{cell-lightblue} die \\

		\midrule

		\cellcolor{table-subtopic} \textbf{\textit{NOM.}} &
		\cellcolor{cell-lightpurple}  ein            &
		\cellcolor{cell-lightorange}  ein            &
		\cellcolor{cell-lightblue} eine              &
		\cellcolor{table-bg} - \\

\end{tabularx}

\vspace{0.5cm}

\linenumbers
% }}} END-TABLE : Nominative Articles

An example is shown below to further clarify this concept :\newpar

\noindent
\textit{Der Hund beißt den Mann.}\\
\textcolor{gray} { \textit{( The dog bites the man. )} } \newpar

The action (verb) being performed is beißen (to bite). The thing doing the
action is the dog. Therefore the dog will be in the nominative case and will
have the ‘normal’ masculine article of \textit{der}.\newpar

Overall, the nominative case is used in the following situations :\newpar

\nolinenumbers
\begin{itemize}[noitemsep]
	\item If the word is isolated , e.g. One word answers like
		``{\textit{Name}}''.
	\item If the word makes up part of the subject.
	\item If the word forms part of the object of the predicate, and the
			sentence is formed with the copulative verb
			\textit{(Definition~\ref{def:copulative_verb})}. 
\end{itemize}
\linenumbers


% NOTE : nominative_clarification {{{

\tcolorboxnote
{
		Every single sentence in German will always have a nominative object in
		it. The nominative is the only case where this is fact is true, due to
		fact that there is no such thing as a nominative preposition.


}



% }}} END NOTE : nominative_clarification_






% }}} END SUB-SUB-SECTION : nominative

% 		SUB-SUB-SECTION : accusative {{{
\textbf{\subsubsection{Accusative}}
\label{sec:accusative}

The second German case is called \textbf{\textit{the Accusative (Der Akkusativ /
		Der Wenfall)}} .  The accusative case (or more specifically the
accusative grammatical article) applies on the direct object / person (noun)
that the action (verb) is being performed on by the subject (nominative noun).
Read the last sentence again, because it is a little dense the first time
around, but nonetheless important. \newpar

The accusative gendered articles are mainly the same as the nominative case,
with the only exception being the masculine accusative. All the articles are
shown in a table below :

% TABLE : Accusative Articles {{{

\nolinenumbers
\vspace{0.5cm}
\begin{tabularx}{0.94\linewidth}{l|XXXX}

		&
		\cellcolor{table-subtopic} \textbf{\textit{MAS.}} &
		\cellcolor{table-subtopic} \textbf{\textit{NEU.}}  &
		\cellcolor{table-subtopic} \textbf{\textit{FEM.}}  &
		\cellcolor{table-subtopic} \textbf{\textit{PLU.}} \\
		\midrule

		\cellcolor{table-subtopic} \textbf{\textit{ACC.}} &
		\cellcolor{cell-lightgreen} den              &
		\cellcolor{cell-lightorange}  das            &
		\cellcolor{cell-lightblue}  die              &
		\cellcolor{cell-lightblue} die \\

		\midrule

		\cellcolor{table-subtopic} \textbf{\textit{ACC.}} &
\cellcolor{cell-lightgreen} einen            &
\cellcolor{cell-lightorange}  ein            &
\cellcolor{cell-lightblue}  eine             &
\cellcolor{table-bg} - \\


\end{tabularx}

\vspace{0.5cm}

\linenumbers

% }}} END-TABLE : Accusative Articles

The distinction between the direct object, and the subject is easily explained
through the same example that we dealt with in the nominative section, which is
:\newpar

\noindent
\textit{Der Hund beißt den Mann.}\\
\textcolor{gray} { \textit{( The dog bites the man. )} } \newpar

As mentioned earlier, the thing doing the action is the dog, so the dog gets the
nominative article. The action being done is biting (beißen), but who is the
action being done to , or, who/what is being directly affected by the action ?
In this specific example : Who is being bitten ? : The Mann (Der Mann). The
normal article for the man is the masculine \textit{der} , but in this sentence,
we will use the masculine der, but in its accusative form \textit{der} . The
reason the above is a good example is that we have two nouns, which are both
masculine , both the two take different cases. This allows us to see the
nominative and the accusative case functioning simultaneously.\newpar


Although the accusative object is defined as the direct object, there are also
other ways that an object might be assigned the accusative grammatical article,
even though they are not being directly affected by the verb. The most common
way that this occurs is through certain prepositions in German. When a word is
being affected by these prepositions in a sentence, it will always take the
accusative case, therfore now we have the following situations in which we will
give an object its acccusative grammatival article :\newpar

\nolinenumbers
\begin{itemize}[noitemsep]

	\item If the word is a direct object in the english version of the sentence
		(i.e.\ it is the noun that the verb is acting on), then 90\% of the time
		this word will take the accusative case in German.

	\item If the word is being affected by either an accusative preposition
		\textit{( Table~\ref{table:prepositions_accusative} )} or a wechsel
		preposition \textit{( Table~\ref{table:prepositions_wechsel} )} , then
		it will take the accusative case.~\footnote{More information about how
			to distingush between when to use wechsel prepositions for
			accusative and when for dative is provided in the prepositions
			section.} These prepositions are listed here and then again in the
		Prepositions Section. \textit{(Section~\ref{sec:prepositions})}
	

	\begin{itemize}[noitemsep]

		\item durch, für , entlang, gegen, ohne, um \ldots herum, hinter, in,
			neben, über, unter, vor, zwischen, wider

	\end{itemize}

\end{itemize}
\linenumbers

Rememember that just because there is not direct object in the sentence, does
not mean that there is no accusative object. Since we can have an accusative
object in a sentence as is defined by the preposition.\newpar

This also implies that we can have two ` \textit{Accusative Objects} ' in one
sentnce since, we can have a regular direct object and we also have another
object that is accusative according to one of the accusative prepositions listed
above, or a movement based wechsel preposition.\newpar

For the sake of illustration, here is another example which has two accusative
objects in it :

% }}} END SUB-SUB-SECTION : accusative

% 		SUB-SUB-SECTION : dative {{{
\textbf{\subsubsection{Dative}}
\label{sec:dative}

The third German case is called \textbf{\textit{the Dative Case (Der Dativ Fall
		/ Der Wemfall)}} If you have understood and internalized the accusative
case, then the dative case should not be too much of a stretch to master. The
dative case applies when there is an object in the sentence that is being
indirectly impacted by the action (verb).\newpar

Unlike in the accusative where only the masculine article changes, all the
articles change for an object that is  under the dative case which it is worth
mentioning includes the plural, since up until people learn about the Dative
case most people take for granted that plural article will always be
\textit{`die'}, and this is no longer the case. The dative case articles along
with the articles for all the other cases that we have learned so far are shown
in a table below :\newpar

% TABLE : Dative Articles {{{

\vspace{0.5cm}

\nolinenumbers

\begin{tabularx}{0.94\linewidth}{l|XXXX}

		&
		\cellcolor{table-subtopic} \textbf{\textit{MAS.}} &
		\cellcolor{table-subtopic} \textbf{\textit{NEU.}}  &
		\cellcolor{table-subtopic} \textbf{\textit{FEM.}}  &
		\cellcolor{table-subtopic} \textbf{\textit{PLU.}} \\

		\midrule

		\cellcolor{table-subtopic} \textbf{\textit{DAT.}} &
		\cellcolor{cell-lightred} dem               &
		\cellcolor{cell-lightred} dem               &
		\cellcolor{cell-lightpurple} der               &
		\cellcolor{cell-lightgreen} den \\

		\midrule

		\cellcolor{table-subtopic} \textbf{\textit{DAT.}} &
		\cellcolor{cell-lightred} einem              &
		\cellcolor{cell-lightred} einem              &
		\cellcolor{cell-lightpurple} einer           &
		\cellcolor{table-bg} - \\

\end{tabularx}

\vspace{0.5cm}

\linenumbers

% }}} END-TABLE : Dative Articles

Similar to the accusative case, there are also situations where an object will
take the dative grammatical article and not be the inderect object in the
sentence. This occurs when the noun in question is subject to a dative
preposition. Therefore now we have the following situations in which a noun will
have the dative grammatical article :\newpar

\nolinenumbers
\begin{itemize}[noitemsep]

	\item If the word is an \textbf{\textit{in-direct object}} in the english
		version of the sentence (i.e.\ it is the noun that is being affected by
		the verb, but not the noun that is being directly acted on), then 90\%
		of the time this word will take the Dative case in German.

   \item If the word is being mentioned along with either a Dative preposition
	   \textit{( Table~\ref{table:prepositions_dative} )} or a wechsel
	   preposition \textit{( Table~\ref{table:prepositions_wechsel} )} , then it
	   will take the Dative case.~\footnote{See previous footnote.} 

\end{itemize}
\linenumbers

Just like the previous sections let us consider an example for clarification
:\newpar

\noindent
\textit{Ich schenke dir das Heft.}\\
\textcolor{gray} { \textit{( I gift you the notebook. )} } \newpar

The sentence above can be broken down in the following way :

% TABLE : dative_example_clarification {{{

\nolinenumbers

\vspace{0.2cm}

\begin{tabularx}{0.95\linewidth}{lllX}

\rowcolor{white} \bulletpoint Subject (Nominative) & : & I (ich)   & \\
\rowcolor{white} \bulletpoint Verb                 & : & schenken (to gift) & \\
\rowcolor{white} \bulletpoint Direct Object        & : & das Heft (the notebook)  & \\
\rowcolor{white} \bulletpoint Indirect Object      & : & you (dir)\footnote{more
about pronouns (i.e.\ why dir and not du) in \textit{(Section~\ref{sec:pronouns})}  } & \\

\end{tabularx}

\vspace{0.2cm}

\linenumbers

% }}} End TABLE : dative example clarification

Another example, just in case the one above was not clear is as follows :
\newpar

\noindent
\textit{Wir machen das mit einem Computer}\\
\textcolor{gray} { \textit{( We are doing that with a computer. )} } \newpar

Before ananlysing the sentence, please note that the das used in the example is
not actually a grammatical gender for any noun in the sentence, rather it is the
German equivalent word for that, as is evident in the translation. Das will also
always take the accusative case, since it is always the direct object when the
word that appears in a sentence.\newpar

That being said, this example is a little easier because we know (if we have
already been through the prepositions section) that the preposition mit is a
dative preposition, therefore the thing that is being accompanied with the
dative preposition must take its grammatical gender in the dative case.\newpar

However if we did not know about mit being a dative preposition then the breakup
of the sentence would look something like the following :\newpar

% TABLE : dative_example_clarification_2 {{{

\nolinenumbers

\vspace{0.2cm}

\begin{tabularx}{0.95\linewidth}{llX}

\rowcolor{white} \bulletpoint Subject (Nominative) & : & Wir (we)   \\
\rowcolor{white} \bulletpoint Verb                 & : & machen (to do) \\
\rowcolor{white} \bulletpoint Direct Object        & : & das (that)  \\
\rowcolor{white} \bulletpoint Indirect Object      & : & the computer (einem
Computer) \\

\end{tabularx}

\vspace{0.2cm}

\linenumbers

% }}} End TABLE : dative example clarification_2

So the only thing I think that is worth mentioning here is the computer. The
computer takes the masculine gender in German , so der computer, and it is in
the dative case here so dem computer. Also worth noting in the above example is
that I have used the indefinite (einem) version of the grammatical gender as
opposed to the definite version(dem), just to spice things up a little and to
provide examples for as many scenarios as I can.\newpar

To finish things off, here is an example that has no accusative object, and two
dative objects : \newpar





% }}} END SUB-SUB-SECTION : dative

% 		SUB-SUB-SECTION : genetive {{{
\textbf{\subsubsection{Genetive}}
\label{sec:genetive}

The fourth and last case in German , is called \textbf{\textit{the Genetive Case
		(Der Genetiv Fall /  Der Wesfall)}} . The genetive case is mainly used
when we are trying to indicate posession of something. As is with the previously
mentioned cases (except nominative) , the Genetive has its own set of
prepositoins, the creatively name : genetive prepositions. If an object is being
affected by these prepositions, then it will take the genetive grammtical
article.\newpar

The table for the definite and indefinite genetive articles is as follows
:\newpar

The english equivalent of displaying possession is when we add the apostrophe s
to the end of a word to show belonging.\ e.g.\ John's book , Mary's car , etc
\ldots

% TABLE : Genetive_Articles {{{

\vspace{0.5cm}

\nolinenumbers

\begin{tabularx}{0.94\linewidth}{l|XXXX}

		&
		\cellcolor{table-subtopic} \textbf{\textit{MAS.}} &
		\cellcolor{table-subtopic} \textbf{\textit{NEU.}}  &
		\cellcolor{table-subtopic} \textbf{\textit{FEM.}}  &
		\cellcolor{table-subtopic} \textbf{\textit{PLU.}} \\

		\midrule

		\cellcolor{table-subtopic} \textbf{\textit{GEN.}} &
		\cellcolor{cell-lightyellow} des               &
		\cellcolor{cell-lightyellow} des               &
		\cellcolor{cell-lightpurple} der               &
		\cellcolor{cell-lightpurple} der \\

		\midrule

		\cellcolor{table-subtopic} \textbf{\textit{GEN.}} &
		\cellcolor{cell-lightyellow} eines               &
		\cellcolor{cell-lightyellow} eines               &
		\cellcolor{cell-lightpurple} einer               &
		\cellcolor{cell-lightpurple} einer \\


\end{tabularx}

\vspace{0.5cm}

\linenumbers

% }}} END-TABLE : Genetive_Articles

A noun will recieve Genetive grammatical article in the following situations
:\newpar 
\nolinenumbers
\begin{itemize}[noitemsep]

	\item     If the word is after the word ` \textit{of} ' in English

	\item If it follows a preposition that is Genitive (anstatt, aufgrund,
	außerhalb, dank, statt, während, wegen)

\end{itemize}
\linenumbers

An example to aid in the clarification of the Genetive case is :\newpar


\noindent
\textit{Die Zukunft des Buches ist schwer}\\
\textcolor{gray} { \textit{( The future of the book is difficult )} } \newpar

Just like the previous section here is a breakdown of all of the objects in the
sentence.


*(In English genitive’s expressed with of or by adding an apostrophe to show
possession. Des Buches is translated as of the book or the book’s)

The genitive is not used as often by Germans as the three
other previous cases.


Often, a noun object is made with the preposition von +
Dative and the genitive preposition are sometimes used incorrectly as if they
were dative.


You have to keep in mind that one word can fit the rules of different cases
simultaneously.


For example, it can be a subject while being a
part of a noun object and follow a preposition that is dative. Which case would
it be then?  Nominative because it’s the subject, Genitive, because it’s the
noun object or dative because it is after a preposition?  The answer is that the
priorities are in this order: Following a preposition (governing with
Accusative, Dative or Genitive) Being part of a genitive object (Genitive) The
rest of the rules

% }}} END SUB-SUB-SECTION : genetive

Following we have all the cases with thier correponding articles in two tables,
one for all the definite articles and one for the indefinite ones :\newpar



% TBOX TABLE : definite_articles {{{

\tcolorboxtable
{ Definite Articles }
{ \label{table:definite_articles} }
{ [width=\linewidth] }
{ l|XXXX }
{

		&
		\cellcolor{table-subtopic} \textbf{\textit{MAS.}}  &
		\cellcolor{table-subtopic} \textbf{\textit{NEU.}}  &
		\cellcolor{table-subtopic} \textbf{\textit{FEM.}}  &
		\cellcolor{table-subtopic} \textbf{\textit{PLU.}} \\

		\midrule

		\cellcolor{table-subtopic} \textbf{\textit{NOM.}} &
		\cellcolor{cell-lightpurple}  der            &
		\cellcolor{cell-lightorange}  das            &
		\cellcolor{cell-lightblue}    die            &
		\cellcolor{cell-lightblue}    die \\

		\cellcolor{table-subtopic} \textbf{\textit{ACC.}} &
		\cellcolor{cell-lightgreen}   den            &
		\cellcolor{cell-lightorange}  das            &
		\cellcolor{cell-lightblue}    die            &
		\cellcolor{cell-lightblue}    die \\

		\cellcolor{table-subtopic} \textbf{\textit{DAT.}} &
		\cellcolor{cell-lightred}    dem             &
		\cellcolor{cell-lightred}    dem             &
		\cellcolor{cell-lightpurple} der             &
		\cellcolor{cell-lightgreen}  den \\

		\cellcolor{table-subtopic} \textbf{\textit{GEN.}} &
		\cellcolor{cell-lightyellow} des               &
		\cellcolor{cell-lightyellow} des               &
		\cellcolor{cell-lightpurple} der               &
		\cellcolor{cell-lightpurple} der \\



}

% }}} End TBOX TABLE : definite_articles

% TBOX TABLE : indefinite_articles {{{

\tcolorboxtable
{ Indefinite Articles }
{ \label{table:indefinite_articles} }
{ [width=\linewidth] }
{ l|XXXX }
{

		&
		\cellcolor{table-subtopic} \textbf{\textit{MAS.}} &
		\cellcolor{table-subtopic} \textbf{\textit{NEU.}}  &
		\cellcolor{table-subtopic} \textbf{\textit{FEM.}}  &
		\cellcolor{table-subtopic} \textbf{\textit{PLU.}}~\footnote{In German
			sometimes the word \textit{einige (some) } is used to refer to an indefinite
number of objects in plural.} \\

\midrule

\cellcolor{table-subtopic} \textbf{\textit{NOM.}} &
\cellcolor{cell-lightpurple}  ein            &
\cellcolor{cell-lightorange}  ein            &
\cellcolor{cell-lightblue} eine              &
\cellcolor{table-bg} - \\

\cellcolor{table-subtopic} \textbf{\textit{ACC.}} &
\cellcolor{cell-lightgreen} einen            &
\cellcolor{cell-lightorange}  ein            &
\cellcolor{cell-lightblue}  eine             &
\cellcolor{table-bg} - \\

\cellcolor{table-subtopic} \textbf{\textit{DAT.}} &
\cellcolor{cell-lightred} einem              &
\cellcolor{cell-lightred} einem              &
\cellcolor{cell-lightpurple} einer           &
\cellcolor{table-bg} - \\

\cellcolor{table-subtopic} \textbf{\textit{GEN.}} &
\cellcolor{cell-lightyellow} eines           &
\cellcolor{cell-lightyellow} eines           &
\cellcolor{cell-lightpurple} einer           &
\cellcolor{table-bg} - \\


}

% }}} End TBOX TABLE : indefinite_articles


\subsectionend
% }}} END SUB-SECTION : cases

% 	SUB-SECTION : n_declination {{{
\subsection{n-Declination}
\label{ssec:n_declination}

There are certain nouns in German that are also conjugated / declined just like
verbs. The rules by which these nouns are declined a is known as n-Declination
(n-Deklination) . This is because these nouns get an n ending appended to them.\newpar

Although the actual ` declination ' is pretty easy becuase the ending that is
appended will always be just a -n, we still have to figure out which nous are
going to be declined and which not.\newpar

Most masculine nouns and a new neuter nouns are going to follow the n-declension
structure. Some masculine nouns are ` \textit{weak} '  , which basically means
that they will take a n ending in all cases except the nominative case.\newpar

It is impractical to go around memorizing nouns according to wether they are
weak or strong. We have enough information per noun to deal with articles
attached to each noun.\newpar

Therefore it is best to just look at a whole bunch of examples for weak nouns
and after a period of time you will begin ` recognizing ' these nouns, and
automatically place them in the weak category.\newpar

Weak nouns basically fall into two groups :

\begin{enumerate}[noitemsep]

	\item ending in -e and referring to people / animals\\

		e.g.\ der Kunde , der Neffe , der Russe\\

		this is not a guranteed rule. A counter example is der Käse\\

		Another thing to keep in mind with this group is that some of these -e
		ending nouns will keep the genetive -s ending after the -n ending. These
		are nouns that are basically not referring to a people or animals but
		are still ending in an -e ,  e.g.\ der Wille (the volition, the will) ,
		der Gedanke (the thought).\\

	\item The second group of ` weak ' nouns are ones that have certain Latin /
		Greek endings. This group is super fuzzy and does not have hard
		boundaries like the previous group. Essentially the words in German that
		sound very close to thier English equivalents often fall into the
		category of weak nouns.\\

		e.g.\ der Kapitalist , der Kommunist , der Diplomat , der Astronaut
		\ldots

\end{enumerate}

There are a whole lot more weak nouns , that dont fit into either of the two
categories defined above. As mentioned earlier dont try to memorize all of them,
because that is just not pratical.~\footnote{ If you still insist on
	memorization, there is a full list at :  \url{http://Germanforenglishspeakers.com/nouns/weak-nouns-the-n-declension/} }



All the nouns that fall under one the following rules will be n-declined
:\newpar

\nolinenumbers
\begin{itemize}[noitemsep]
	\item If the noun is masculine and :

	\vspace{0.5cm}
	\begin{tabularx}{0.95\linewidth}{llX}

		\rowcolor{white} \bulletpoint Is a person & , e.g , & \textcolor{gray}{der Junge} \\
		\rowcolor{white} & & \textcolor{gray}{der Kunde} \\
		\rowcolor{white} & & \textcolor{gray}{der Neffe} \\
		\rowcolor{white} \bulletpoint Is a nationality & , e.g , & \textcolor{gray}{der Greiche} \\
		\rowcolor{white} & & \textcolor{gray}{der Chinese} \\
		\rowcolor{white} & & \textcolor{gray}{der Russe} \\
		\rowcolor{white} \bulletpoint Is an animal & , e.g , & \textcolor{gray}{der Rabe} \\
		\rowcolor{white} & & \textcolor{gray}{der Affe} \\
		\rowcolor{white} & & \textcolor{gray}{der Löwe} \\

	\end{tabularx}


	\item All nouns with Latin and Greek roots with the following endings :


\vspace{0.5cm}
\begin{tabularx}{0.95\linewidth}{lllX}

\rowcolor{white} \bulletpoint -ant & , e.g. , & der Elefant  \\
\rowcolor{white} \bulletpoint -and & , e.g. , & der Doktorand\\
\rowcolor{white} \bulletpoint -ent & , e.g. , & der Student, der Präsident\\
\rowcolor{white} \bulletpoint -ist & , e.g. , & der Journalist , der Idealist\\
\rowcolor{white} \bulletpoint -oge & , e.g. , & \\
\rowcolor{white} \bulletpoint -at & , e.g. , & der Bürokrat , der Diplomat\\
\end{tabularx}

	\item All nouns that are internationally derived , e.g. , der Architekt
		(Architecht) , der Fotograf (the photograph) , der Philosoph (the
		philosopher)

		
	\item If the noun is masculine and is a living (lebewesen) 
\end{itemize}
\linenumbers


% NOTE : n-Declination exception {{{

\tcolorboxnote
{
There is one exception to the rules listed above. A non grammatically masculine
noun that takes the n-declination ending : das Herz (the heart).
}



% }}} END NOTE : n_declination_exception






The generale table to keep in mind when trying to figure out n-Declination is
:\newpar







It is important to make a mental distinction between what ending is being given
to a noun for the reason of case, and which due to the reason of
n-declination. Consider the following example : \newpar




\subsectionend
% }}} END SUB-SECTION : n_declination

% 	SUB-SECTION : clauses {{{
\subsection{Clauses}
\label{ssec:clauses}


\subsectionend
% }}} END SUB-SECTION : clauses

% 	SUB-SECTION : noun_contractions {{{
\subsection{Noun Contractions}
\label{ssec:noun_contractions}


\subsectionend
% }}} END SUB-SECTION : noun_contractions

% 	SUB-SECTION : adjectival_nouns {{{
\subsection{Adjectival Nouns}
\label{ssec:adjectival_nouns}

Adjectival nouns are adjectives that act like nouns in a sentence.

I’ll take the usual.

No, wait, the special.

And some wine to go with it—the best one you’ve got.

Have you ever ordered at a restaurant with this sort of lingo?

If you have, then you already know a little bit about adjectival nouns.

In English, they’re so easy to use that we hardly pay attention to them.

But as with so much else, they’re a little tougher in German.

That’s it. Easy-peasy.

The examples above show you how we use them: “usual” is an adjective, but slap a “the” in front of it, and you’ve got the adjectival noun phrase “the usual.”

In English, we also sometimes side-step adjectival nouns by using the word “one.” In the example of “the best one” above, “best” is still functioning as a pure adjective describing the pronoun “one.”

But translate that into German, and “best” would fill the role all by itself. You’ll never hear Germans talk about die beste Eins (literally, the best one). A simple das Beste (the best) will do.

Because adjectival nouns are used everywhere in German, this is yet another reason to practice the dreaded Adjektivendungen (adjective endings).

All adjectives in German inflect for gender, case and number. This is how you end up with an English adjective like “red” having five different forms in German (rot, roter, rote, rotes and roten, if you needed a refresher on all five).

Now that we know what adjectival nouns are and how to use them, let’s look at some examples.

\textbf{1. Wie geht’s dem Kleinen/der Kleinen? (How’s the little one?)}

In German, children are sometimes referred to as Kleine (little ones). We do the exact same thing in English when we call them “little ones.”

(See? If you have an adjective + “one” combo in English, it’s probably gonna need an adjectival noun in German, like I said.)

\textbf{2. Alles Gute zum Geburtstag! (Happy birthday—literally, everything good for your birthday.)}

The German phrase for “happy birthday” uses an adjectival noun? Say what?

Yes, it’s true. That’s because “Alles Gute zum Geburtstag” doesn’t literally translate to “happy birthday.” Strictly speaking, it’s “everything good for your birthday,” a type of generic well-wishing.

And as you might be able to tell by that capital G on Gute, we’re dealing with an adjectival noun. The adjective gut is normally lowercase, but throw an alles (everything) in front of it and you’ve got a noun on your hands, so it’s time to capitalize and decline (gut to Gute).

\textbf{3. Ich gebe immer mein Bestes. (I always do my best.)}

Here’s a better example of that “abstract concept” thing I brought up earlier when mentioning Clint Eastwood.

If you say, “I always do my best,” a question that might follow could be, “Your best what?” What is the adjective “best” describing in this phrase?

Trick question! It’s not an adjective. It’s an adjectival noun referring to the best things overall, the general concept of the best anything.

One extra thing to notice here is that in German, you don’t “do” your best. You “give” your best. That’s why the phrase starts with ich gebe (I give) and not ich mache or ich tue (I do).

\textbf{4. “Der Gefangene von Askaban” (“The Prisoner of Azkaban”)}

I want to keep using this example because Gefangener (prisoner) is one of the most basic and widely-used adjectival nouns in German for something that’s not an adjectival noun in English.

Gefangener (prisoner) comes from the participial adjective gefangen (captured).

Did you catch that? The verb fangen (to catch/capture) becomes an adjective (gefangen) that becomes a noun (der Gefangene).

Parts of speech can morph like that in German, and you’ve got to handle the grammar accordingly.

\textbf{5. Haribo macht Kinder froh und Erwachsene ebenso. (Haribo candy makes kids happy, and adults too.)}

Look at the German word Erwachsene (adults). It comes from the adjective erwachsen (mature).

Unlike children, who are always just Kinder, any time you talk about adults, you’ll need to pay attention to whether you need to say Erwachsene, ein Erwachsener, die Erwachsenen and so on. There’s no non-adjectival word for “adult” that you can use to weasel out of this.

Adjective endings, man. You really need to know them.

\textbf{6. Meine Verwandten sind zu Besuch. (My relatives are visiting.)}

I’m going to throw out one final “this noun is always secretly an adjective” example just to show you that they’re everywhere.

Verwandte (relatives) comes from the adjective verwandt (related).

That means when you’re referring to a bunch of aunts, uncles, cousins and in-laws coming around for holiday visits, the collective noun to refer to them will need some adjective endings on it.

Verwandte? Die Verwandten? Ein Verwandter? I can’t tell you which one you’ll need, because it’ll depend on the context of what you’re saying.

That’s why—I feel like I can’t repeat this enough—you need to practice your adjective endings. Just do it.

Because German adjective endings carry considerable information about case, gender, and number, the noun that they modify can sometimes seem redundant. When Germans refer to Ex-Chancellor Helmut Kohl as der Dicke, they don't need a further noun, since the der, followed by the -e ending on dick tells us that we are dealing with a single masculine subject (in the nominative case). So long as the context is clear, all that's needed to make the noun is to capitalize the first letter.

A number of such nouns constructed in this fashion have become conventional enough to be listed as dictionary entries in their own right. Some adjectives that become such nouns are "bekannt" [= acquainted], "angestellt" [= employed, hired], "verwandt" [= related], "erwachsen" [= grown-up], "heilig" [= holy], and "deutsch" [= German]

Sie ist eine gute Bekannte von mir.  	She is a good acquaintance of mine.
Er ist ein Angestellter dieser Firma.  	He is an employee of this company.
Meine Verwandten sind alle verrückt.  	My relatives are all crazy.
Nur Erwachsene dürfen diesen Film sehen.  	Only adults [grownups] are allowed to see this film.
Der Papst hat sie zur Heiligen erklärt.  	The Pope declared her a saint.
Die Deutschen sind gern pünktlich.  	Germans like to be punctual.

Frequent usage has produced other conventions:
Ich möchte ein Helles.  	I'd like a light beer [a pils].
Und ich nehme ein Dunkles.  	And I'll have a dark beer.
Heute fahren wir ins Blaue.  	Today we're driving into the wild, blue yonder.
Er traf ins Schwarze.  	He hit the bull's-eye.
Mein Alter geht mir auf den Wecker.  	My old man [my father] gets on my nerves.
Meine Alte versteht gar nichts.  	My old lady [my mother] doesn't understand anything.

The examples above are all in the nominative case, but the adjectival inflections hold true in the accusative, dative, and genitive, as well. Here are examples of "the old man," "the rich woman," "the Good", "the poor [poor people]":
          	Masculine 	Feminine 	Neuter 	Plural
 nom. 	der Alte  	die Reiche  	das Gute  	die Armen 
  	* ein Alter  	eine Reiche  	* kein Gutes  	keine Armen 
  	Alter  	Reiche  	Gutes  	Arme 
 acc.  	den Alten  	die Reiche  	das Gute  	die Armen 
  	einen Alten  	eine Reiche  	* ein Gutes  	keine Armen 
  	Alten  	Reiche  	Gutes  	Arme 
 dat.  	dem Alten  	der Reichen  	dem Guten  	den Armen 
  	Altem  	Reicher  	Gutem  	Armen 
 gen.  	des Alten  	der Reichen  	des Guten  	der Armen 
  	Alten  	Reicher  	Guten  	Armer 

Especially when using adjectives that have been derived from present or past participles, it is possible to pack a great deal of information into the adjectival noun:
das Gefundene  	that which has been found
die Gestorbene  	the (female) deceased
ein Studierender  	someone (male) who is studying
ein Studierter  	someone (male) who has studied
die Betende  	the praying woman
der Alternde  	the aging man
das Werdende  	that which is in the process of becoming
der Auserwählte  	the chosen (male) one
das Unverhoffte  	the unexpected
die Leidtragende  	the (female) mourner

A more common appositional structure is formed with the pronouns "etwas" or "nichts"
Ich will dir etwas Schönes zeigen.  	I want to show you something beautiful.
Er führt nichts Gutes im Schilde.  	He's up to no good.
Wir reden von etwas Einmaligem.  	We're talking about something unique.

The adjectives "viel" and "wenig" sometimes look like pronouns, because they normally take no endings in the singular:
Wir haben wenig Interessantes zu berichten.  	We have little of interest to report.
Ihr Boss hat viel Gutes über Sie gesagt.  	Your boss said a lot of good things about you.
Seine Rede enthält wenig Wahres.  	His speech contains little that is true.


\subsectionend
% }}} END SUB-SECTION : adjectival_nouns


\sectionend

% }}} END SECTION : nouns

% SECTION : verbs {{{
\section{{Verbs}}
\label{sec:verbs}

% DEFINITION : verbs {{{

\vspace{0.2cm}
\centering
\nolinenumbers
\begin{defn-bg}

\label{def:verbs_infinitives}
	\begin{defn-title}[width=7cm]{}
		{\normalsize \textbf{\textit{Verbs}} }
	\end{defn-title}

	\begin{defn-theword}
	{
		\footnotesize \begin{mydef} \end{mydef}
	}
	\end{defn-theword}

	\begin{defn-content}
		\justify
	
\textbf {Verbs} are words that refer to actions. These actions are happening to
the nouns (substantives).

	\end{defn-content}

\end{defn-bg}

\justifying
\linenumbers

% }}} END DEFINITION : verbs

\lettrine[lines=3, findent=3pt, nindent=0pt]{V}{erbs} are basically any word that
allows us to describe an action, an ongoing process or a state of being. E.g. :
to walk (laufen), to sing (singen).  Verbs are made up of two parts : the
stem which is the main body of the verb, and the ending.\newpar

\noindent
E.g : Laufen , Singen , Haben , Machen\newpar

Every verb must be \textbf{\textit{Conjugated / Declined}}. To conjugate a verb
means to change it from it's base infinitive form to a form matching the pronoun
and tense that we are using in the current sentence.\newpar

As a note on jargon , some people say we are conjugating the verb , some say we
are declining it. The terminology changes based on where you learned your German
from, but it all means the same thing. I thought I would mention both terms in
order to preempt any sort of confusion that might occur. \newpar

% DEFINITION : infinitive {{{
% {title} {label} {content}

\tcolorboxdefinition
{Infinitive}
{\label{def:infinitive}}
{

		Infinitive is the base form of any verb. This is the unconjugated form.

}

% }}} END DEFINITION : infinitive

When we say we are conjugating / declining a verb, we mean that we are changing
the stems and endings of the verbs in order to suit the pronoun that we are
currently using.\newpar

These two sub-categories of verbs depend on whether we change only the ending
during conjugation , i.e.\ is the verb conjugated according to the regular rules
, or whether we change both the ending and the stem during the conjugation
process , i.e.\ is the verb conjugated according to the irregular rules. All
verbs in German fall into one of the three following categories :\newpar

% ITEMIZE : german_verb_types {{{

\begin{itemize}[noitemsep]
	
	\item \textbf{Regular / Weak Verbs} : Most people just refer to weak /
		regular verbs as just verbs, since this is the most common verb type
		encountered in the German language. A weak verb is a German verb whose
		stem does not change its vowel to form the imperfect tense and the past
		participle. Its past participle \refdef{participle} is formed by adding a -t to the stem of
		the verb. These are also called regular verbs, or just verbs.


	\newpar

	\item \textbf{Irregular / Strong Verbs} : A strong verb is type of germ verb
		, whose stem changes its vowel to form the imperfect tense and the past
		participle. Its past participle is \textbf{not} formed by adding a -t to
		the verb stem. Strong verbs are also known as irregular verbs

	\newpar

	\item \textbf{Mixed Verbs} : A mixed verb is one where the stem changes its
		vowel to form the imperfect tense and the past participle tense, similar
		to strong verbs. The past participle is formed however by adding -t to
		the verb stem just like weak verbs\cite{collins_german_grammar}. So it
		has characteristics of both weak and strong verbs therefore the
		name.\newpar

		Examples of mixed verbs are : 


\end{itemize}

% }}} END-ITEMIZE : German verb types

% LIST : Regular Verbs Conjugation {{{

\vspace{0.3cm}
\begin{tabular}{l l l l}

\rowcolor{white} $\bullet$ Change & e  & to & ie\\
\rowcolor{white} $\bullet$ Change & e  & to & i\\
\rowcolor{white} $\bullet$ Change & el & to & il\\
\rowcolor{white} $\bullet$ Change & eh & to & im\\
\rowcolor{white} $\bullet$ Change & a  & to & ä\\

\end{tabular}
\vspace{0.3cm}
\newline

% }}} End LIST : Regular verbs Conjugation
% LIST : Irregular Verbs Conjugation {{{

\vspace{0.3cm}
\begin{tabular}{l l l l}

\rowcolor{white} $\bullet$ Change & e  & to & ie\\
\rowcolor{white} $\bullet$ Change & e  & to & i\\
\rowcolor{white} $\bullet$ Change & el & to & il\\
\rowcolor{white} $\bullet$ Change & eh & to & im\\
\rowcolor{white} $\bullet$ Change & a  & to & ä\\

\end{tabular}
\vspace{0.3cm}
\newline

% }}} End LIST : Irregular verbs Conjugation

% 	SUB-SECTION : imperative {{{
\subsection{Imperative}
\label{ssec:imperative}

The German imperative is a way of basically ordering someone to do something
instead of asking them politely. The imperative can be formed using three
differnt ways, depending on the count and politeness that we wish to employ. The
pronoun is never used when we are trying to build an imperative sentence. The
conjugation of the verb is also a bit different. The three main forms that will
change are as follows :\newpar


% TABLE : imperative_explanation {{{

\nolinenumbers

\vspace{0.5cm}

\begin{tabularx}{0.95\linewidth}{llX}

\rowcolor{white} \bulletpoint du  & : & The -st(or just -t) ending on the conjugation will
go away.  \\

\rowcolor{white} & e.g  & \textcolor{gray}{iss deine Essen.}\\
\rowcolor{white} \bulletpoint Sie & : & The conjugation stays the same. \\
\rowcolor{white} \bulletpoint ihr & : & The conjugation stays the same. \\

\end{tabularx}

\vspace{0.5cm}

\linenumbers

% }}} End TABLE : Imperative Explanation

if the umlaut is in the infinitive, then the du form will remain the same in the
imperative , i.e.\ it will still have the umlaut.\

if the umlaut is not there in the infinitive, but it appears due to it being a
irregular verb, then it will go away in the imperative form.\

if the umlaut is not there in the infinitive, but it appears due to it being a
irregular verb, then it will go away in the imperative form.\


\subsectionend
% }}} END SUB-SECTION : imperative

% 	SUB-SECTION : irregular_verbs {{{
\subsection{Irregular / Strong Verbs}
\label{ssec:irregular_verbs}




\subsectionend
% }}} END SUB-SECTION : irregular_verbs

% 	SUB-SECTION : tenses {{{
\subsection{Tenses}
\label{ssec:tenses}

% DEFINITION : tense {{{
\tcolorboxdefinition
{Tense}
{\label{def:tense}}
{

A \textbf{\textit{Tense}} is the grammarians’ preferred word for “time.”
Depending when the action that you’re talking about is taking place, you pick a
tense. The ways to look at the concept of time differ.

}

% }}} END DEFINITION : tense

An explanation of the present tense is mainly the same thing as an explanation
of the infinitve, where we first talked about how and we we go about using
verbs. All of those explanations are given in the present tense.\newpar

Basically when we say we are changing the tense in German, we mean that we will
be using a different schema to conjugate the verb in the sentence. That is all a
tense means, a different conjugation. This is why the tenses section is
considered a subset of the verbs section, as well as , why I said any
explanation of the infinitive above can also be considered an explanation of the
present tense in German. 

% DEFINITION : participle {{{
\tcolorboxdefinition
{Participle}
{\label{def:participle}}
{

	Every time we conjugate a verb, we make a new word. These words formed by
	conjugating verbs are called \textbf{\textit{participles}}.

}

% }}} END DEFINITION : participle

As we have already seen a tense is basically what results when a conjugate a
verb differently (the ` normal ' way being the present tense). Therefore based
on how the verb is conjugated we end up with different tenses as well as
participles. So essentially participle is jargon for the verb conjugation that
is being used to implement the current tense.\newpar

Since we have multiple tenses , we also end up with multiple kinds of
participles :

% ITEMIZE : participletypes {{{
\begin{itemize}[noitemsep]

	\item Participle I : Partizip I (Partzip Präsens)
	\item Participle II : Partizip II (Partzip Perfekt)

\end{itemize}
% }}} END-ITEMIZE : participle types

We will discuss both these participle types in thier individual sub-sections
below.\newpar

First lets get started with the most basic of all of the tenses, the present
tense.


% 		SUB-SUB-SECTION : present {{{
\subsubsection{Present (Präsens)}
\label{sssec:present}


Following is a table to indicate the verb conjugation
for making a sentence in the present tense : \newpar

% TABULARX TABLE : present_tense_conjugation {{{

\tabularxtable
{0.95\linewidth}
{l|l|llX}
{
ich & -e  & ich lerne  & , & I learn\\
du  & -st & du lernst  & , & You learn\\
er  & -t  & er lernt   & , & He learns\\
sie & -t  & sie lernt  & , & She learns\\
es  & -t  & es lernt   & , & It learns\\
man & -t  & man lernt  & , & One learns\\
wir & -en & wir lernen & , & We learn\\
ihr & -t  & ihr lernt  & , & They learn\\
sie & -en & sie lernen & , & they learn\\
Sie & -en & Sie lernen & , & You learn\\
}

% }}} End TABULARX TABLE : Present Tense Conjugation

The present tense is the most versatile tense in German. Unlike english, the
present tense has a multitude of uses, and is not not restricted to only one
type of sentence. The biggest differnce that Englsih speakers would notice is
that the German language does not make a distinction between the present and
present continuous tenses. They both fall under the present (Präsens) category.
All the situations where we would use the Präsens tense are shown below along
with examples :\newpar

% ITEMIZE : When to use the present tense {{{
\begin{itemize}[noitemsep]
	\item A fact or a condition in the present. \\

		\textit{Ich denke.}\\
		\textcolor{gray} { \textit{( I think. )} } \\
		\textcolor{gray} { \textit{( I do think. )} } \newpar
	
		Both of the above will be acceptable interpretations of the given German
		sentence.

		\textit{Das ist Felix.}\\
		\textcolor{gray} { \textit{( That is Felix. )} } \\

	\item An ongoing action in the present.\\

		\textit{Ich denke.}\\
		\textcolor{gray} { \textit{( I am thinking. )} } \newpar

	\item An action that takes place in the present once, repeatedly, or never.

		
		\noindent
		\textit{Jeden Dienstag geht er zum Fußballtraining.}\\
		\textcolor{gray} { \textit{( Every teusday I go to football training.  )} } \newpar
		


		\textit{Freitags gehe ich oft ins Kino.}\\
		\textcolor{gray} { \textit{( On Fridays, I often go to the movies. )} } \newpar
		
		
	\item A situation in the past being talked about in the present.\\
	\item To describe a situation that has not yet, but is going to happen.
		
		
		\noindent
		\textit{Morgen fährt meine Freundin nach Dänemark.}\\
		\textcolor{gray} { \textit{( Tomorrow my girlfriend will drive to
				Denkmark. )} } \newpar
		
		A situation that will cocur in the future being talked about through
		the lens of the present. In this case the sentence will always contain
		the verb werden.\\

	\item An action that expresses how long something has been going on.

		
		\noindent
		\textit{Er spielt seit fünf Jahren Fußball.}\\
		\textcolor{gray} { \textit{( He has been playing football for five
				years. )} } \newpar
		

\end{itemize}


% }}}

% NOTE : present tense talking about future {{{

\tcolorboxnote
{

Using a present tense conjugation of the verb werden is a very common way of
talking about future events in German, particularly if there’s a time expression
in the sentence that anchors the action clearly in the future — for example,
nächste Woche (next week) or morgen (tomorrow).\newpar

This formation using werden is therefore considered both talking in the present
tense and talking in the future tense.

}

% }}} END NOTE : present_tense_talking_about_future





\subsubsectionend
% }}} END SUB-SUB-SECTION : present

% 		SUB-SUB-SECTION : partizip_i {{{
\subsubsection{Partizip I (Partizip Präsens)}
\label{sssec:partizip_i}



\subsubsectionend
% }}} END SUB-SUB-SECTION : partizip_i

% 		SUB-SUB-SECTION : partizip_ii {{{
\subsubsection{Partizip II (Partzip Perfekt)}
\label{sssec:partizip_ii}


\subsubsectionend
% }}} END SUB-SUB-SECTION : partizip_i

% 		SUB-SUB-SECTION : present_perfect {{{
\subsubsection{Present Perfect (Perfekt)}
\label{sssec:present_perfect}


% DEFINITION : present_perfect_tense_perfekt_ {{{
\tcolorboxdefinition
{Present Perfect Tense (Perfekt)}
{\label{def:present_perfect_tense_perfekt_}}
{


The simple past tense, also known as the past simple, the past tense or the
preterite, expresses completed actions in the recent and distant past.




}

% }}} END DEFINITION : present_perfect_tense_perfekt_

The present perfect tense is the victim of a very unfortunate naming convention.
I say this because the present perfect tense or the Perfekt tense is actually
used to talk about the past. This can cause some severe headaches if it is not
clarfied at the onset of learing more about this tense. So please bear this in
mind.\newpar

Another thing to clarify / keep in mind is that the present perfect is known as
the simple past tense , or the preterite tense in English.  Please do not
confuse the english terminology with the German tenses though, since there is no
such thing as a preterite / simple past tense in German. There is only the
Perfekt tense.\newpar

O.k.\ got all that squared away and internalized ? Good, cause it caused me a
lot of personal greif when trying to figure all of this out for the first
time.\newpar

The \textbf{\textit{perfect tense}} , also called \textbf{\textit{present
		perfect (Perfekt)}} , is a past tense. We use it to speak about actions
completed in the recent past. In spoken German, the present perfect tense is
often used instead of the past tense. German distinguishes between the way the
perfekt tense is used when writing and how it is used when speaking. As I
meantioned earlier all distinctions between tenses amount to nothing but
changing the conjugation of the verbs. So essentially we have the following
:\newpar

% ITEMIZE : perfect_vs_prateritum {{{

\begin{itemize}[noitemsep]

	\item \textbf{Perfekt} : Used when talking about events in the past. Perfekt is also
		used when we are quoting someone, i.e., when we are writing down what
		someone has said verbally we will not change it to its präteritum
		formation.\\

	\item \textbf{Präteritum} : Used when writing about events in the past.\\

\end{itemize}

% }}} END-ITEMIZE : perfect vs prateritum

The current section will only talk about the perfekt tense in the spoken
context. The following sub-sub-section \refsssec{prateritum} will handle the
Präteritum verb formations.\newpar

The situations that we will use the perfekt tense in are as follows :

% ITEMIZE : present_perfect_usage_situations {{{
\begin{itemize}[noitemsep]

	\item The main past tense used in the spoken form.
	
	
	\noindent
	\textit{Was hast du gestern abend gemacht ?}\\
	\textcolor{gray} { \textit{( What did you do yesterday evening ? )} } \newpar 

	\item Abgeschlossene Vorgänge in der Vergangenheit mit Gegenwartbezug
	
	
	\noindent
	\textit{Seit ihr wegezogen ist sehen wir uns nur noch selten.}\\
	\textcolor{gray} { \textit{( EXAMPLE TRANSLATION )} } \newpar

	\item When talking from the prespective of the past, about the present /
		future.

	
	\noindent
	\textit{Morgen in einer Woche habe ich die Arbeiten an diesem Projekt
		abgeschlossen.}\\
	\textcolor{gray} { \textit{( EXAMPLE TRANSLATION )} } \newpar
	

	\item INSERT CITATIONS FOR THESE EXAMPLES
			
\end{itemize}
% }}} END-ITEMIZE : Present Perfect Situations

\hrule

The way that we will form the present perfect version of a verb, is by using two
things the \textbf{\textit{past pasticiple (Partizip II)}} , and an
\textbf{\textit{auxiliary / helping verb}} .  These helping verbs are always
either \textit{haben} or \textit{sein}. The partizip II formations can be a
little more involved, which is why I dedicated an entire sub-section to Partizip
II.\ Helping verbs as well as how to choose which, are also discussed later in
the same section.\newpar

O.k.\ keeping in mind that whenever we want to form the past participle, the
verb will come with an accompanying helping verb (dont worry about which haben /
sein right now) , we have to figure out verb positioning again. The verbs will
now be arranged into positions as follows :\newpar


% ITEMIZE : past_participle_sentence_structures {{{

\textbf{Past Participle Sentence Structures}
\newpar

\begin{itemize}[noitemsep]

	\item \textbf{Helping / Auxiliary Verb} : The helping verb will take the
position of the regular verb in the sentence , i.e., the helping verb will go in
position 2.\\

	\item \textbf{Regular Verb} : The past participle of the regular verbs will be sent
		to the end of a sentence, in its unconjugatred form, but with a ge-
		appended to the beggining of the verb.\ e.g.\   \\

	\item \textbf{Seperable Verb} : When forming the past participle of a seperable verb,
		we no longer seperate the verb, however the ge- addition that around
		95\% of the verbs get will not be at the starting of the sentence,
		rather it will be after the seperable prefix, but before the main verb.\
		e.g.\ \\

	\item \textbf{Modal Verb} : How do we resolve modal verbs then ? because now we have
		the helping verb in position 2 ? You know I have no fucking clue.\\

\end{itemize}

% }}} END-ITEMIZE : past participle sentence structures

The first thing that we need to know is that , while past participle will go at the absolute end of the sentence.
Here we already have a couple of distinctions based the what the infinitive form
of the verb was, namely :\newpar

Now that the general structure is out of the way let us talk about how we are
going to select which one of the two helping verbs : haben / sein we shouls use
with which verbs. 

% ITEMIZE : choosing_helping_verbs {{{

\textbf{Choosing Helping Verbs}

\begin{itemize}[noitemsep]

	\item \textbf{Sein} :  Sein will be a rarer use case. It will mainly be used
		when, there is some form of change of state or movement involved.\\

		\textit{Meine Freundin ist nach Dänemark gefahren.}\\
		\textcolor{gray} { \textit{( My girlfriend has gone to / went to
				Denkmark. )} } \newpar
		
		\textit{Ich bin in hamburg gewesen.}\\
		\textcolor{gray} { \textit{( I have been to Hamburg. / I was in Hamburg. )} } \newpar
		
		\textit{Du bist mit dem Auto gekommen.}\\
		\textcolor{gray} { \textit{( You came by car. / You have come by car. )} } \newpar
	
	Not in the example above, even though we have the verb sein , the
	translation still reads you have come \ldots

	\item \textbf{Haben} : Most verbs will use haben , as long as there is no
		movement or state change going on.\\

		\textit{David hat mir geholfen.}\\
		\textcolor{gray} { \textit{( David has helped me / has been helping me /
				helped me. )} } \newpar
		
		\textit{Anna hat die Zeitung gelesen.}\\
		\textcolor{gray} { \textit{( Anna has read the newspaper. )} } \newpar
		
		\textit{Ich habe den Film gesehen.}\\
		\textcolor{gray} { \textit{( I have seen the film. / I saw the film. )} } \newpar

\end{itemize}

% }}} END-ITEMIZE : choosing helping verbs

O.k.\ now that we know about the helping verbs, lets give a little more detail
to the past participle formations.


% TABULARX TABLE : present_perfect_tense_conjugation {{{

\tabularxtable
{0.95\linewidth}
{l|l|llX}
{

ich & ich bin  & gegangen  , & ich habe  & \\
du  & du bist  & ,           & du hast   & \\
er  & er ist   & ,           & er hat    & \\
sie & sie ist  & ,           & sie hat   & \\
es  & es ist   & ,           & es hat    & \\
man & man ist  & ,           & man hat   & \\
wir & wir sind & ,           & wir haben & \\
ihr & ihr seid & ,           & ihr habt  & \\
sie & sie sind & ,           & sie haben & \\
Sie & Sie sind & ,           & Sie haben & \\
}

% }}} End TABULARX TABLE : Present Perfect Tense Conjugation

Below is a more involved table for Partizip II formation :\newpar


The past participle (Partizip II) is formed in the following ways:

Regular Verbs also known as weak verbs (schwache Verben) form the past
participle with ge…t and the verb stem.

Example:
    lernen – gelernt

Irregular verbs are verbs that change their verb stem in simlpe past and/or the
participle form (see list of irregular verb). There are two kinds of irregular
verbs in German grammar: strong verbs (starke Verben) and mixed verbs (gemischte
Verben).

Strong Verbs form the past particple with mit ge…en.

Example:
sehen – gesehen (sehen-sah-gesehen)
gehen – gegangen (gehen-ging-gegangen)

Mixed Verbs form the past participle with ge…t.

Example:
haben – gehabt (haben-hatte-gehabt)
bringen – gebracht
(bringen-brachte-gebracht)

Exceptions

We add an -et to weak/mixed verbs when the word stem ends in d/t.

Example:
warten – gewartet

Verbs that end in -ieren form their past participle without
ge.

Example:
studieren – studiert

Inseparable verbs form the past participle
without ge.

Example:
verstehen – verstanden

For separable verbs, the ge
comes after the prefix.

Example:
ankommen –
angekommen

Forming the past participle of a weak verb

Here is the formula for forming the past participle of a weak verb:

ge + verb stem (the infinitive minus -en) + (e) t = past participle

For example, for the verb fragen (to ask), here’s how the formula would play
out:

ge + frag + t = gefragt
Forming the past participle of a strong verb

Here is the formula for forming the past participle of a strong verb:

ge + verb stem (the infinitive minus -en) + en = past participle

For the verb kommen(to come), the past participle would be:

ge + komm + en = gekommen

Writing about the past: Using simple past tense

Simple past tense is used all the time in newspapers, books, and so on, but it
is less common in speech. One exception is the simple past tense of sein (to
be). This is often used in preference to perfect tense in both speech and
writing. Table 1 shows you the various forms of the simple past tense of sein.



\subsubsectionend
% }}} END SUB-SUB-SECTION : present_perfect

% 		SUB-SUB-SECTION : simple_past {{{
\subsubsection{Simple Past (Präteritum)}
\label{sssec:simple_past}
This is your bread and butter tense for writing about events in the past.
However when we want to \textbf{talk} about events in the past then we tend to
use the present perfect (perfekt) tense more often than this one.\newpar


sein , haben , es gibt, gehen,
We use the präteritum for the following two situations :

\begin{enumerate}
	\item A completed action in the past, with a focus on the result of the
		action.
	\begin{itemize}
		\item Gestern hat Michael sein Büro aufgeräumt.

	\vspace{0.25cm}

		When Yesterday ,
		\vspace{0.25cm}
		\textit {Current result}: Office is now clean.
	\end{itemize}
	\item An action that will be completed by a certain point in the future.
\end{enumerate}

Following is a table for the simple past tense conjugation of sein :~\footnote{\url{https://www.dummies.com/languages/German/getting-to-know-german-verb-tenses/} }
% TABULARX TABLE : simple_past_tense_sein_conjugation {{{

\tabularxtable
{0.95\linewidth}
{llX}
{

ich & war  & \\
du  & warst  & \\
er  & war   & \\
sie & war   & \\
es  & war   & \\
man & war   & \\
wir & waren & \\
ihr & wart & \\
sie & waren & \\
Sie & waren & \\
}

% }}} End TABULARX TABLE : simple_past_tense_sein_conjugation

% NOTE : Spoken Prateritum {{{

\tcolorboxnote
{
A thing to remember regarding the usage of the prätertium tense is that although
it is used exlusively when writing there are a couple of verbs that are an
exception to this rule. Namely : sein , haben , all modal verb.\newpar

We use the prateritum formation, while using these verbs to talk about the past.

}



% }}} END NOTE : spoken_prateritum

The stem changes are in \textcolor{red-flame}{this color} , and the ending /
root changes are in \textcolor{green-goethe}{this color} in the following
tables, in order to help distinguish between what is being changed and why
(since stem changes only occur in irregular verbs and the root changes will
occur in all the verbs formations).\newpar

% TABULARX TABLE : prateritum_regular_verb_formation {{{


To conjugate weak or regular verbs in the simple past, add ``{\textit{te}}'' to
the verb ending, and then add the present tense personal conjugation
ending.\newpar


Note that for verbs that end in ``{\textit{t}}'' or ``{\textit{d}}'' , we add an
extra ``{\textit{e}}'' before the 
``{\textit{te}}'' ,  instead of using just the ``{\textit{te}}'' so that
pronunciation of the word is a little easier. The table below shows examples of
verbs in both cases, the -et , and the -ete case.\newpar

\textbf{Präteritum : Regular Verbs}\cite{em}
\tabularxtable
{0.95\linewidth}
{X|XX}
{

	&
	\textit{fragen} &
	\textit{warten} \\
	\midrule

	\cellcolor{table-subtopic} ich &
	frag\textcolor{green-goethe}{\textbf{-te}} &
	wart\textcolor{green-goethe}{\textbf{-e-te}} \\

	\cellcolor{table-subtopic} du &
	\cellcolor{table-alternating-blue}
	frag\textcolor{green-goethe}{\textbf{-te}}-st &
	\cellcolor{table-alternating-blue}
	wart\textcolor{green-goethe}{\textbf{-e-te}}-st \\

	\cellcolor{table-subtopic} er / sie / es &
	frag\textcolor{green-goethe}{\textbf{-te}} &
	wart\textcolor{green-goethe}{\textbf{-e-te}} \\

	\cellcolor{table-subtopic} wir &
	\cellcolor{table-alternating-blue}
	frag\textcolor{green-goethe}{\textbf{-te}}-n &
	\cellcolor{table-alternating-blue}
	wart\textcolor{green-goethe}{\textbf{-e-te}}-n \\

	\cellcolor{table-subtopic} ihr &
	frag\textcolor{green-goethe}{\textbf{-te}}-t &
	wart\textcolor{green-goethe}{\textbf{-e-te}}-t \\


	\cellcolor{table-subtopic} sie / Sie &
	\cellcolor{table-alternating-blue}
	frag\textcolor{green-goethe}{\textbf{-te}}-n &
	\cellcolor{table-alternating-blue}
	wart\textcolor{green-goethe}{\textbf{-e-te}}-n \\


}

% }}} End TABULARX TABLE : Prateritum Regular Verb Formation

% TABULARX TABLE : prateritum_irregular_verb_formation {{{

\textbf{Präteritum : Irregular Verbs}\cite{em}

The irregular präteritum verbs do not follow the same rules as above. We will
not add anything to the root \ ending of the verb. So the overall conjugation
will remain the same , but we will however be making some changes to the stem.
There are no definitive rules regarding the correct präteritum irregular
formations, so the only real option is to memorize them. That being said , here
are a couple of examples :\newpar

\tabularxtable
{0.95\linewidth}
{l|XX}
{
	&
	\textit{kommen} &
	\textit{sehen} \\
	\midrule


	\cellcolor{table-subtopic} ich &
	k\textcolor{green-goethe}{\textbf{a}}m &
	s\textcolor{green-goethe}{\textbf{a}}h \\

	\cellcolor{table-subtopic} du &
	\cellcolor{table-alternating-blue} k\textcolor{green-goethe}{\textbf{a}}m-st &
	\cellcolor{table-alternating-blue} s\textcolor{green-goethe}{\textbf{a}}h-st\\

	\cellcolor{table-subtopic} er / sie / es &
	k\textcolor{green-goethe}{\textbf{a}}m &
	s\textcolor{green-goethe}{\textbf{a}}h \\

	\cellcolor{table-subtopic} wir &
	\cellcolor{table-alternating-blue} k\textcolor{green-goethe}{\textbf{a}}m-en &
	\cellcolor{table-alternating-blue} s\textcolor{green-goethe}{\textbf{a}}h-en \\

	\cellcolor{table-subtopic} ihr &
	k\textcolor{green-goethe}{\textbf{a}}m-t &
	s\textcolor{green-goethe}{\textbf{a}}h-t \\

	\cellcolor{table-subtopic} sie / Sie &
	\cellcolor{table-alternating-blue} k\textcolor{green-goethe}{\textbf{a}}m-en &
	\cellcolor{table-alternating-blue} s\textcolor{green-goethe}{\textbf{a}}h-en \\

}

% }}} End TABULARX TABLE : Prateritum Irregular verb formation

% TABULARX TABLE : prateritum_modal_verb_formation {{{

\textbf{Präteritum : Modal Verbs}\cite{em}
\tabularxtable
{0.95\linewidth}
{l|X}
{

	&
	\textit{können} \\
	\midrule

	\cellcolor{table-subtopic} ich &
	k\textcolor{red-flame}{\textbf{a}}m \\

	\cellcolor{table-subtopic} du &
	\cellcolor{table-alternating-blue} k\textcolor{red-flame}{\textbf{a}}m\textcolor{green-goethe}{\textbf{st}}\\

	\cellcolor{table-subtopic} er / sie / es &
	k\textcolor{red-flame}{\textbf{a}}m\textcolor{green-goethe}{\textbf{ete}} \\

	\cellcolor{table-subtopic} wir &
	\cellcolor{table-alternating-blue}
	k\textcolor{red-flame}{\textbf{a}}m\textcolor{green-goethe}{\textbf{en}} \\

	\cellcolor{table-subtopic} ihr &
	k\textcolor{red-flame}{\textbf{a}}m\textcolor{green-goethe}{\textbf{t}} \\

	\cellcolor{table-subtopic} sie / Sie &
	\cellcolor{table-alternating-blue}
	k\textcolor{red-flame}{\textbf{a}}m\textcolor{green-goethe}{\textbf{en}} \\



}


% NOTE : Modal Präteritum Do not Confuse {{{

\tcolorboxnote
{

The präteritum conjugation of modal verbs is very similar to the konjunktiv II
conjugation, the only difference being the addition of umlauts in konjunktiv
II.\ This however completely changes the meaning of the word , so be very very
careful when conjugating modal verbs in prateritum.

}



% }}} END NOTE : modal_präteritum_do_not_confuse





% }}} End TABULARX TABLE : prateritum modal verb formation

% TABULARX TABLE : prateritum_helping_auxiliary_verb_formation {{{

\textbf{Präteritum : Helping / Auxiliary Verbs}\cite{em}

\textit{Sein} and \textit{haben} are super duper special in German. They havent
really followed any of the regular conjugation rules so far, so the tradition
contiunues and these two troublemakers will get thier own extra special
conjugations even in the Präteritum case. Haben kind of follows some of the
rules, but sein is just goes full loco. These conjugations are listed in the table
below :\newpar

\tabularxtable
{0.95\linewidth}
{l|XX}
{
	&
	\textit{sein} &
	\textit{haben} \\
	\midrule

	\cellcolor{table-subtopic} ich &
	war &
	ha\textcolor{green-goethe}{\textbf{-tte}} \\

	\cellcolor{table-subtopic} du &
	\cellcolor{table-alternating-blue} warst &
	\cellcolor{table-alternating-blue} ha\textcolor{green-goethe}{\textbf{-tte}}-st \\

	\cellcolor{table-subtopic} er / sie / es &
	war&
	ha\textcolor{green-goethe}{\textbf{-tte}} \\

	\cellcolor{table-subtopic} wir &
	\cellcolor{table-alternating-blue} waren  &
	\cellcolor{table-alternating-blue}
	ha\textcolor{green-goethe}{\textbf{-tte}}-n \\

	\cellcolor{table-subtopic} ihr &
	wart &
	ha\textcolor{green-goethe}{\textbf{-tte}}-t \\


	\cellcolor{table-subtopic} sie / Sie &
	\cellcolor{table-alternating-blue} waren &
	\cellcolor{table-alternating-blue} ha\textcolor{green-goethe}{\textbf{-tte}}-n \\

}

% }}} End TABULARX TABLE : prateritum helping / auxiliary verb formation

% TABULARX TABLE : prateritum_mixed_verb_formation {{{

\textbf{Präteritum : Mixed Verbs}\cite{em}
\tabularxtable
{0.95\linewidth}
{l|X}
{
	&
	\textit{denken} \\
	\midrule

	\cellcolor{table-subtopic} ich &
	d\textcolor{green-goethe}{\textbf{achte}} \\

	\cellcolor{table-subtopic} du &
	\cellcolor{table-alternating-blue}
	d\textcolor{green-goethe}{\textbf{achtest}} \\

	\cellcolor{table-subtopic} er / sie / es &
	d\textcolor{green-goethe}{\textbf{achte}} \\

	\cellcolor{table-subtopic} wir &
	\cellcolor{table-alternating-blue} d\textcolor{green-goethe}{\textbf{achten}} \\

	\cellcolor{table-subtopic} ihr &
	d\textcolor{green-goethe}{\textbf{achtet}} \\

	\cellcolor{table-subtopic} sie / Sie &
	\cellcolor{table-alternating-blue} d\textcolor{green-goethe}{\textbf{achten}} \\



}

% }}} End TABULARX TABLE : prateritum mixed verb formation



\subsubsectionend
% }}} END SUB-SUB-SECTION : simple_past

% 		SUB-SUB-SECTION : past_perfect {{{
\subsubsection{Past Perfect (Plusquamperfekt)}
\label{sssec:past_perfect}



\subsubsectionend
% }}} END SUB-SUB-SECTION : past_perfect

% 		SUB-SUB-SECTION : future_i {{{
\subsubsection{Future (Futur I)}
\label{sssec:future_i}

In German, the future tense is not used as consistently as it is in English. In
many situations, you can use the present tense instead. When talking about
things that are going to take place in the future, you can, of course, use
future tense. The way to form future tense in German is pretty similar to
English. You take the verb werden(to become) and add an infinitive.

Table 2 shows you the forms of werden in the present tense.

Table 2 Present Tense Forms of werden And this is how you incorporate future
tense into sentences:

    Ich werde anrufen.(I am going to call.)
	Wir werden morgen kommen.(We will come tomorrow.)
	Es wird regnen.    (It will rain. / It’s going to rain.)


% TABULARX TABLE : present_tense_werden_conjugation {{{

\tabularxtable
{0.95\linewidth}
{llX}
{

ich & werde  & \\
du  & wirst  & \\
er  & wird   & \\
sie & wird   & \\
es  & wird   & \\
man & wird   & \\
wir & werden & \\
ihr & werdet & \\
sie & werden & \\
Sie & werden & \\
}

% }}} End TABULARX TABLE : present_tense_werden_conjugation



\subsubsectionend
% }}} END SUB-SUB-SECTION : future_

% 		SUB-SUB-SECTION : future_ii {{{
\subsubsection{Future Perfect (Futur II)}
\label{sssec:future_ii}



\subsubsectionend
% }}} END SUB-SUB-SECTION : future_

\subsectionend
% }}} END SUB-SECTION : tenses

% 	SUB-SECTION : konjunktiv_ii {{{
\subsection{Konjunktiv II}
\label{ssec:konjunktiv_ii}

The Konjuntiv II is known in English as the subjunctive II.\ Generally
Konjunktiv II is used when we are trying to do the following things :\newpar


% ITEMIZE : konjunktiv_ii_uses {{{

\begin{itemize}[noitemsep]

	\item \textbf{Politeness \textit{ \textcolor{gray}{(Bitten höflich
					ausdrücken)} }} : When we are trying to be super polite and
		say ,, could you please do \ldots `` instead of ,, do \ldots ``\newpar


	\noindent
	Example :\\
	\textit{Könnten Sie mir das Problem bitte genau beschreiben ?}\\
	\textcolor{gray} { \textit{( Could you please describe the specific problem to me ? )} } \newpar


	\item \textbf{Wishes \textit{ \textcolor{gray}{(Irreales ausdrücken)} }}
	When we are talking about wishes we have about something. These wishes are
	always irreal , since they have'nt been realized yet. These wishes can be
	about something that you could have done in the past (vergangenheit) or
	present (gegenwartig).\newpar


\noindent
Example :\\
\textit{Hätten Sie die Ware doch früher abgeschickt.}\\
\textcolor{gray} { \textit{( Had you sent the wares earlier. )} } \newpar
While this is not explicity saying I wish \ldots , it is still expressing it as
such.\newpar




	\item \textbf{Guesses \textit{ \textcolor{gray}{(Vermutungen ausdrücken)} }}
: We can use Konjunctive II when we are trying to guess something. This is
basically like saying , ,,I think it could be \ldots `` When we are using
Konjunktiv II in this way , the only verb formation we can use is könnten.\newpar


\noindent
Example :\\
\textit{Es könnte sein, dass der Laptop ein Defekt hat.}\\
\textcolor{gray} { \textit{( it could be that the laptop has a defect. )} } \newpar 
	\item \textbf{Suggestions \textit{ \textcolor{gray}{(Vorschläge ausdrücken)}
			}} : This use case is when we are making suggestions about
		something.


\noindent
\textit{Ich könnte Ihnen ein Leihgerät anbieten.}\\
\textcolor{gray} { \textit{( I can offer you a loan unit. )} } \newpar




\end{itemize}

% }}} END-ITEMIZE : Konjunktiv ii uses

Now that we understand when to use Konkunktiv II, we need to understand how do
we actually build sentences in this way ?\newpar

A konjunktive II sentence is formed by changing the main verb in the sentence to
their Konjunktiv II formation. This basically means that some sort of
transformation rules need to be in place in order to create this Konjunktive II
,, word ``. These transformation rules change based on the time form , which is
the same thing as what I mentioned earlier when we were takling about using
Konjunktiv II to express wishes. These specfic definite ways of forming
Konjunktiv II are explained below :\newpar

% ITEMIZE : konkunktiv_ii_uses {{{

\begin{itemize}[noitemsep]

	\item \textbf{Present \textit{ \textcolor{gray}{(Gegenwartig)} } }: 
		In this form the konjunktive II ,, word `` is formed by first forming
		the präteritum version of the verb and then adding umlauts. In most of
		the cases we will form a gegenwartig konkunktiv II sentence using the
		konjunktive II formation of sein as a helping verb , and another verb
		which will be in the infinitive.\newpar

		So the konj. II of sein is : sein > waren (Präteritum) > wären. A
		sentence using wären + Infinitive is shown below.\newpar

		\noindent
		\textit{EXAMPLE GERMAN LINE}\\
		\textcolor{gray} { \textit{( EXAMPLE TRANSLATION )} } \newpar

		There are situations where we will not use the \textit{wären +
			Infinitive} formation for the gegenwart Konjunktiv II.\ These
		situations are when we are dealing with one of the following verbs /
		verb types :\newpar


		% ITEMIZE : gegenwartig_konjunktiv_ii {{{

		\begin{itemize}[noitemsep]
			\item \textbf{Hilfsverben} : sein , haben\newpar

\noindent
\textit{EXAMPLE GERMAN LINE}\\
\textcolor{gray} { \textit{( EXAMPLE TRANSLATION )} } \newpar

			\item \textbf{Modal verben}\newpar


\noindent
\textit{EXAMPLE GERMAN LINE}\\
\textcolor{gray} { \textit{( EXAMPLE TRANSLATION )} } \newpar


			\item \textbf{Unregelmäßige verben}\newpar


\noindent
\textit{EXAMPLE GERMAN LINE}\\
\textcolor{gray} { \textit{( EXAMPLE TRANSLATION )} } \newpar




		\end{itemize} 
		% }}} END-ITEMIZE : gegenwartig konjunktiv ii



		

	\item When we are talking aobut a wish that we have related to past events.

		E.g. : If I had more time, I would read more.\newpar

		The same sentence , but this time in only one Hauptsatz , instead of a
		nebensatz (If you dont understand what those words mean , refer
		\refsec{sentence_structures} ).\newpar

			I would read more if I could.

\end{itemize}

% }}} END-ITEMIZE : Konkunktiv II Uses

\subsectionend
% }}} END SUB-SECTION : konjunktiv_ii

% 	SUB-SECTION : special_verbs {{{
\subsection{Special Verbs}
\label{ssec:special_verbs}

% 		SUB-SUB-SECTION : werden {{{
\subsubsection{Werden}
\label{sssec:werden}



\subsubsectionend
% }}} END SUB-SUB-SECTION : werden
% 		SUB-SUB-SECTION : lassen {{{
\subsubsection{Lassen}
\label{sssec:lassen}



\subsubsectionend
% }}} END SUB-SUB-SECTION : lassen



\subsectionend
% }}} END SUB-SECTION : special_verbs

% 	SUB-SECTION : modal_verbs {{{
\subsection{Modal Verbs}
\label{ssec:modal_verbs}


% DEFINITION : modal_verbs {{{
\tcolorboxdefinition
{Modal Verbs}
{\label{def:modal_verbs}}
{

Modal Verbs are verbs that allow us to change / modify the original sentence to
add degrees of ability, permission and necessity.


}

% }}} END DEFINITION : modal_verbs

Modal Verbs are always used in conjunction with a another verb. The modal verbs
action is to indication to what extent the action specified by the other verb is
necessary or allowed in the current sentence.\newpar

Modal verbs are also often called \textbf{\textit{auxiliary / helping verbs}}
(Die Hilfsverben)\newpar

When we are building a sentence with a modal verb, we put the modal verb in its
conjugated form in the second place of the sentence, and send the original verb
that was supposed to be in the second place all the way to the end of the
sentence. Another thing to note is that we should not conjugate the second
verb.\newpar

The general sentence structure in the present tense is as follows :\newpar

Pronoun - Modal Verb (conj.) - frequency/time - other words - second verb
(unconj.)\newpar

In the case of W-questions , or Ja / Nien Questions the sentence structure
remains the same as before with the modal verb in place 1 for Ja / Nien
Questions, and in the second position in W-Style questions. The most common modal verbs
are :

% LIST : modal_verbs_list {{{

\vspace{0.3cm}
\begin{tabular}{l l}

\rowcolor{white} \bulletpoint dürfen & (to be allowed)  \\
\rowcolor{white} \bulletpoint können & (to be able)  \\
\rowcolor{white} \bulletpoint mögen  & (would like to)  \\
\rowcolor{white} \bulletpoint wollen & (want to)  \\
\rowcolor{white} \bulletpoint sollen & (should)\\
\rowcolor{white} \bulletpoint mussen & (must) \\

\end{tabular}
\vspace{0.3cm}

% }}} End LIST : modal_verbs_list

Be careful with saying „ \textit{Ich will} “ as it sounds impolite if you are
asking for something. It is more appropriate to say “Ich möchte“ or “Ich hätte
gern“.\newpar

A table of all modal verbs with conjugations is shown below : 



% TBOX TABLE : modal_verbs {{{

\tcolorboxtable
{ Modal Verbs }
{ \label{table:modal_verbs} }
{ [width=\textwidth] }
{ X|X|X|X|X|X|X }
{

	&
	\cellcolor{gray-light} \textbf{ich} &
	\cellcolor{gray-light} \textbf{du} &
	\cellcolor{gray-light} \textbf{er/sie/es} &
	\cellcolor{gray-light} \textbf{wir} &
	\cellcolor{gray-light} \textbf{ihr} &
	\cellcolor{gray-light} \textbf{sie/Sie} \\

	\midrule

	\cellcolor{gray-light} \textbf{\textit{dürfen}} &
	\cellcolor{white} darf                          &
	\cellcolor{white} darfst                        &
	\cellcolor{white} darf                          &
	\cellcolor{white} dürfen                        &
	\cellcolor{white} dürft                         &
	\cellcolor{white} dürfen \\

	\cellcolor{gray-light} \textbf{\textit{können}} &
	\cellcolor{white} kann                          &
	\cellcolor{white} kannst                        &
	\cellcolor{white} kann                          &
	\cellcolor{white} können                        &
	\cellcolor{white} könnt                         &
	\cellcolor{white} können\\

	\cellcolor{gray-light} \textbf{\textit{mögen}} &
	\cellcolor{white} mag                          &
	\cellcolor{white} magst                        &
	\cellcolor{white} mag                          &
	\cellcolor{white} mögen                        &
	\cellcolor{white} mögt                         &
	\cellcolor{white} mögen\\

	\cellcolor{gray-light} \textbf{\textit{wollen}} &
	\cellcolor{white} will                          &
	\cellcolor{white} willst                        &
	\cellcolor{white} will                          &
	\cellcolor{white} wollen                        &
	\cellcolor{white} wollt                         &
	\cellcolor{white} wollen\\

	\cellcolor{gray-light} \textbf{\textit{sollen}} &
	\cellcolor{white} soll                          &
	\cellcolor{white} sollst                        &
	\cellcolor{white} soll                          &
	\cellcolor{white} sollen                        &
	\cellcolor{white} sollt                         &
	\cellcolor{white} sollen\\

	\cellcolor{gray-light} \textbf{\textit{mussen}} &
	\cellcolor{white} muss                          &
	\cellcolor{white} musst                         &
	\cellcolor{white} muss                          &
	\cellcolor{white} müssen                        &
	\cellcolor{white} müsst                         &
	\cellcolor{white} müssen\\

}

% }}} End TBOX TABLE : modal_verbs

% NOTE : lassen as a modal {{{

\tcolorboxnote
{

Lassen is also considered a modal verb, since it functions exactly like one, but
I have not included it in this table (or section) , since it is a little bit
more complicated than your regular modal verb. Therefore it has been given its
own complete subsection \refsssec{lassen}

}



% }}} END NOTE : lassen_as_a_modal

\subsectionend
% }}} END SUB-SECTION : modal_verbs

\pagebreak

% 	SUB-SECTION : reflexive_verbs {{{
\subsection{Reflexive Verbs}
\label{ssec:reflexive_verbs}


Reflexive verbs are a subset of the verb category, that are used when the
subject and the direct object are the same.\\

Reflexive verbs are very rarely seen in English , but are quite common in
german given the prevalence of cases. There are two distinctions that we should
make , i.e., between Reflexive verbs and Reflexive pronouns. The only way that
we would be able to manifest a reflexive verb is through the use of a reflexive
pronoun. A reflexive pronoun can exist without the use of a reflexive verb in a
sentence, however a reflexive verb cannot exist without the reflexive
pronoun.\newpar

I think this difference will get clearer when we talk about reflexive verbs a
little more. Essentially reflexive verbs can be defined as :

% DEFINITION : reflexive_verbs {{{
\tcolorboxdefinition
{Reflexive Verbs}
{\label{def:reflexive_verbs}}
{



}

% }}} END DEFINITION : reflexive_verbs

There are various types of reflexive verbs in german. They are :

\bulletpoint These are reflexive verbs that do not exist without the reflexive component. If
they are used in a sentence the reflexive pronoun MUST be attached to them
irregardless of whether that makes sense in terms of the sentence or not. These
immer reflexive verbs can have reflexive pronouns attached in both the
accusative and dative forms however.  Basically , if we have an accusative
object in the sentence , then the reflexive pronoun will operate as the dative
object , and will take the dative reflexive pronoun forms. If there is no
accusative object (Akkusative ergänzung) then we will use the accusative
reflexive pronoun.\newpar

\subsectionend

% }}} END SUB-SECTION : reflexive_verbs

% 	SUB-SECTION : verb_contraction {{{
\subsection{Verb Contraction}
\label{ssec:verb_contraction}


\subsectionend
% }}} END SUB-SECTION : verb_contraction

\sectionend
% }}} END SECTION : verbs

% SECTION : pronouns {{{
\section{{Pronouns}}
\label{sec:pronouns}

% 	SUB-SECTION : personal {{{
\subsection{Personal}
\label{ssec:personal}

Like articles and adjectives, pronouns in German vary according to gender and
case. But this time it should be slightly more familiar, as English has kept
some of these distinctions too. Here are the personal pronouns in English, which
hopefully look familiar:

As you can see from the table below, German pronouns are a little more
complicated. Three important things to notice:

    German pronouns often distinguish between the accusative and the dative
	case, while English pronouns never do. Old English did have this
	distinction, but even by the time of Chaucer it was gone (e.g.\ thee was both
	accusative and dative).
	    German has a second person plural (ihr) that’s different from the
		singular (du); English uses you for both, except in casual/regional
		plurals like y’all or you guys.
		    German adds a formal “you” (Sie), which is both singular and plural.
			These ie forms share the same conjugation as the third person
			plural, but are capitalized.

The genitive forms (last row) are grayed out because they're almost never used.
We've included them mainly because they give you the stems of the possessive
articles (mein, dein, sein, etc.) that are used instead (see II.3). Indeed,
saying der Hund meiner instead of mein Hund would be just as awkward as
saying the dog of me in English.

When to use the Sie form rather than du or ihr is one of the most common
questions for German learners, and there's no simple answer. Like most proper
forms of address (“sir/ma’am” in English, vous in French) it’s no longer used in
every situation where the teachers and textbooks suggest that it is. However,
it’s still important to use Sie with police officers (in Germany this is
actually the law) and other authority figures. It’s also polite to use it with
anyone in a service position, like waiters, clerks or salespeople.

After that, it’s largely a matter of familiarity and age – both the absolute age
of the person you’re talking to (older people are more likely to expect Sie),
and their age relative to yours. It can be about the setting, too: sometimes the
same two people will address each other with Sie in the office and du in the bar
after work. It’s also about the tone you’re trying to adopt, and sometimes even
a touch of politics; for example, it was kind of a Hippie thing to use du with
everyone as a statement of egalitarian values. Anyway, you should never use Sie
with children, but otherwise it’s safer to fall back on Sie whenever you’re not
sure.

\subsectionend
% }}} END SUB-SECTION : personal

% 	SUB-SECTION : possessive {{{
\subsection{Possessive}
\label{ssec:possessive}


\subsectionend
% }}} END SUB-SECTION : possessive

% 	SUB-SECTION : reflexive {{{
\subsection{Reflexive}
\label{ssec:reflexive}
Reflexive pronouns (myself, yourself, etc) are more common in German than in
English, because there are many more verbs that require them. (Reflexive verbs
will be covered in Section V.12.) By default, a reflexive pronoun is the direct
or indirect object of a verb, so it can only take the accusative or dative case.
As you can see, there's a great deal of overlap between the reflexive pronouns
and the personal pronouns:
	Singular 	Plural 	Formal
		1st 	2nd 	3rd 	1st 	2nd 	3rd 	2nd
		ACC 	mich 	dich 	sich 	uns 	euch 	sich 	sich
		DAT 	mir 	dir
		English 	myself 	yourself 	him-/her-/
		itself 	ourselves 	yourselves 	themselves 	[yourself/
		yourselves]

		By the way, those plural forms can also be used to mean “each other” or
		“one another.” For example, “wir sehen uns” doesn’t mean that we'll see
		ourselves, it means we’ll see each other – or translating more
		idiomatically, “see you later.”

\subsectionend
% }}} END SUB-SECTION : reflexive

% 	SUB-SECTION : indefinite_pronouns {{{
\subsection{Indefinite Pronouns}
\label{ssec:indefinite_pronouns}

This is a very small group of pronouns containing words that are not referring
to something / someone in particular. The most common ones are :


% ITEMIZE : indefinite_pronouns {{{

\begin{itemize}[noitemsep]
	\item everything 
	\item anybody
	\item nobody
\end{itemize}

% }}} END-ITEMIZE : indefinite pronouns


\subsectionend
% }}} END SUB-SECTION : indefinite_pronouns

% 	SUB-SECTION : relative_pronouns {{{
\subsection{Relative Pronouns}
\label{ssec:relative_pronouns}

\lettrine[lines=3, findent=3pt, nindent=0pt]{R}{elative} pronouns are used when
we are trying to refer to objects that have already been introduced / defined in
different clauses in the sentence. This allows us to spice up the language a bit
, instead of constantly using the noun over and over again. The function of
relative pronouns in English is usually served by “that,” “who / whom” or
“which.” The following are a couple of examples in English , where the reflexive
pronouns are emboldened :\newpar

\textit{,, I will no longer repeat the arguments , \textbf{which} have already
	been established. ``}
\newpar
The which is referring to the arguments in the first clause here.\newpar

\textit{,, I am he , \textbf{who} is trying to explain reflexive pronouns. ``}\\
The who is referring to the he here.\newpar


A thing worth clarification is the relative pronoun term , what I really mean is
that we will be using the article of the word as a pronoun instead of the
repeating the word itself. I thought this was worth mentioning since these are
not pronouns in the regular sense that one would think about them.\newpar

Since we are using articles , therefore like all article related things in
german there will be a different declination table for these relative , pronouns
`. A table with all the relative pronouns is shown below :\newpar


% TBOX TABLE : relative_pronouns {{{

\tcolorboxtable
{ Relative Pronouns }
{ \label{table:relative_pronouns} }
{ [width=\linewidth] }
{ l|XXXX }
{

		&
		\cellcolor{table-subtopic} \textbf{\textit{MAS.}}  &
		\cellcolor{table-subtopic} \textbf{\textit{NEU.}}  &
		\cellcolor{table-subtopic} \textbf{\textit{FEM.}}  &
		\cellcolor{table-subtopic} \textbf{\textit{PLU.}} \\

		\midrule

		\cellcolor{table-subtopic} \textbf{\textit{NOM.}} &
		\cellcolor{cell-lightpurple}  der            &
		\cellcolor{cell-lightorange}  das            &
		\cellcolor{cell-lightblue}    die            &
		\cellcolor{cell-lightblue}    die \\

		\cellcolor{table-subtopic} \textbf{\textit{ACC.}} &
		\cellcolor{cell-lightgreen}   den            &
		\cellcolor{cell-lightorange}  das            &
		\cellcolor{cell-lightblue}    die            &
		\cellcolor{cell-lightblue}    die \\

		\cellcolor{table-subtopic} \textbf{\textit{DAT.}} &
		\cellcolor{cell-lightred}    dem             &
		\cellcolor{cell-lightred}    dem             &
		\cellcolor{cell-lightpurple} der             &
		\cellcolor{cell-lightgreen}  denen \\

		\cellcolor{table-subtopic} \textbf{\textit{GEN.}} &
		\cellcolor{cell-lightyellow} dessen               &
		\cellcolor{cell-lightyellow} dessen               &
		\cellcolor{cell-lightpurple} deren               &
		\cellcolor{cell-lightpurple} deren \\

}
% }}} End TBOX TABLE : relative_pronouns

Most of the declination remains the same as a regular case based declination of
a german article.  The main difference in declination to note here is the plural of the dative case
, as well as the Genetive case.\newpar

The clause which uses a relative pronoun is called a \textbf{\textit{relative
		clause}}.  There are two ways in which a relative clause can appear in a
sentence. The first is as a regular nebensatz an example of which is shown
below :\newpar

\noindent
\textit{Das ist der Mann, \textcolor{green-goethe}{\textbf{dem}} wir das Buch gegeben haben.}\\
\textcolor{gray} { \textit{( That is the man , \textbf{to whom} we have given the book. )} } \newpar

In the sentence above , the first half has der Mann as the nominative
object. In the second half we are referring to the man again but this time using
the relative dative pronoun \textit{dem} instead. Why dative ? because
nominative = wir , accusative = das Buch so dative is the only choice left.
Below is another example of a relativ satz :\newpar

\noindent
\textit{Es gibt wenige Ärzte , denen ich vertraue.}\\
\textcolor{gray} { \textit{( There are few doctors that I trust. )} } \newpar

This was probably mentioned in the verbs section but it is worth repeating that
there is a rule in german where there are some verbs that are just dative or
accusative. So yeah \ldots you just have to learn these verbs and wether they
should have an accusative or dative pronoun accompaniment.The previous example
sentence is an example of that.\newpar

Another thing that can influence the outcome of the pronoun in the relative
sentence is that we might have a preposition. As was mentioned in the
prepositions section , every preposition in german case sensitive (or wechsel).
Which means that with a dative preposition like \textit{mit} we will always have
a dative pronoun , which applies even in the case of the relative sentences with
the only differnce being that the pronouns themselves go by the rules in the
table above.\newpar

Another more exciting way of forming a relative sentence is achieved by
splitting the hauptsatz into two smaller mini hauptsatzeii\footnote{thats a joke
	using the english plural formation on the german word. Hauptsatzeii is not a
	real german word}.\ and the relative clause itself will sit in between these
two mini hauptsatzii.\ Examples of this are shown below :\newpar

\noindent
\textit{Der Mann , dem du gestern die Hand gestern gegeben hat, heißt Richard.}\\
\textcolor{gray} { \textit{( The man whose hand you shook yesterday is called
		Richard. )} } \newpar

\noindent
\textit{Der Mann , dessen Hunde du hörst , ist mein Nachbar}\\
\textcolor{gray} { \textit{( The man whose dogs you hear is my neighbor )} } \newpar

Aside from using articles as pronouns to form relative clauses , we can also use
some w-style question words in place of these articles. The situations where we
can use these as relative pronouns are listed below :


% ITEMIZE : w_style_relative_pronouns {{{

\begin{itemize}[noitemsep]

	\item \textbf{\textit{Wo}} : place

\noindent
\textit{Ich habe Anne in der Stadt kennengelernt , wo wir gearbeitet haben. }\\
\textcolor{gray} { \textit{( EXAMPLE TRANSLATION )} } \newpar

	\item \textbf{\textit{Wohin}} : direction

\textit{Ich habe Anne in der Stadt kennengelernt , wohin ich gezogen bin. }\\
\textcolor{gray} { \textit{( EXAMPLE TRANSLATION )} } \newpar


	\item \textbf{\textit{Woher}} : origin

\textit{Ich habe Anne in der Stadt kennengelernt , woher mein Kollege kommt. }\\
\textcolor{gray} { \textit{( EXAMPLE TRANSLATION )} } \newpar


	\item \textbf{\textit{Was}} : If the main clause uses the words :
		\textit{Das} , \textit{alles} , \textit{nichts} and \textit{etwas} , and
		want to use the relative pronoun to refer to these words we can use
		\textbf{\textit{was}} .  \end{itemize}


\noindent
\textit{Das , was du suchst , gibt es nicht}\\
\textcolor{gray} { \textit{( EXAMPLE TRANSLATION )} } \newpar


\noindent
\textit{Meine Beziehung ist etwas , was mir viel bedeutet.}\\
\textcolor{gray} { \textit{( EXAMPLE TRANSLATION )} } \newpar


\noindent
\textit{Alles , was er mir erzahlt hat , habe ich schon gewusst.}\\
\textcolor{gray} { \textit{( EXAMPLE TRANSLATION )} } \newpar


% }}} END-ITEMIZE : w-style relative pronouns

\subsectionend
% }}} END SUB-SECTION : relative_pronouns

\sectionend
% }}} END SECTION : pronouns

% SECTION : prepositions {{{
\section{{Prepositions}}
\label{sec:prepositions}


% DEFINITION : preposition {{{
% {title} {label} {content}

\tcolorboxdefinition
{Preposition}
{\label{def:preposition}}
{
A preposition is any word that indicates a spatial or temporal relationship
between two nouns or pronouns in a sentence.

}

% }}} END DEFINITION : preposition

\lettrine[lines=3, findent=3pt, nindent=0pt]{P}{repositions} indicate the
relationship of a noun (or pronoun) to another element in the sentence.
Prepositions tend to be some of the most commonly used words in a language.
Following are some examples of prepositions :\newpar


We are going to the Apartment. (Wir gehen zu die Wohnung)
The food is inside the refrigerator. (Das Essen ist innen die Kuhlschrank.)
It is behind the chair. (Es ist hinter den Stühl)


Prepositions work in much the same way in German, except for the added
complication that the nouns and pronouns that the prepositions are acting on
will be declined , i.e.\ , they will take different endings depending on the
preposition in question. The prepositions in German fall into one of four cases
, and they must be memorized as such. There are few easy ways to be able to
logically discern the case of a preposition on occurrence. The four different
types of prepositions are :\newpar

% ITEMIZE : preposition_cases {{{

\begin{itemize}[noitemsep]

	\item \textbf{Accusative prepositions \refssec{accusative_prepositions}} :
		The object that the preposition is acting on will always take the
		Accusative case.

	\item \textbf{Dative prepositions \refssec{dative_prepositions}} : The
		object that the preposition is acting on will always take the dative
		case.

	\item \textbf{Genetive Prepositions \refssec{genetiv_prepositions}} : The
		object that the preposition is acting on will always take the Genetive
		case.

	\item \textbf{Wechsel Prepositions \refssec{wechsel_prepositions}}  : The
		object that the preposition is acting on can take either the genetive
		case or the accusative case.

\end{itemize}

% }}} END-ITEMIZE : preposition cases

Within each one these categories , we can further subdivide the prepositions
into one of the following two categories :\newpar


% ITEMIZE : preposition_cases_two {{{

\begin{itemize}[noitemsep]

	\item \textbf{Temporal Prepositions} : These are the group of prepositions from all
		of the cases above which relate to time.

	\item \textbf{Local / Spatial Prepositions} : The are the group of prepositions that
		relate to physical presence and movement.

\end{itemize}

% }}} END-ITEMIZE : preposition cases two

% NOTE : prepostion rules {{{

\tcolorboxnote
{
In german , there is ALWAYS a noun , or a articel , or a adjective after a
preposition. 
}



% }}} END NOTE : prepostion_rules

% 	SUB-SECTION : accusative_prepositions {{{
\subsection{Accusative prepositions}
\label{ssec:accusative_prepositions}


% TBOX TABLE : prepositions_accusative {{{

\tcolorboxtable
{ Prepositions : Accusative }
{ \label{table:prepositions_accusative} }
{ [width=\linewidth] }
{ l|XX }
{

		\cellcolor{gray-light} B&
		\cellcolor{white} bis &
		\cellcolor{white} until , up-to \\

		\cellcolor{gray-light} D&
		\cellcolor{white} durch &
		\cellcolor{white} through \\


		\cellcolor{gray-light} E&
		\cellcolor{white} entlang &
		\cellcolor{white} along \\


		\cellcolor{gray-light} F&
		\cellcolor{white} für &
		\cellcolor{white} for \\


		\cellcolor{gray-light} G&
		\cellcolor{white} gegen &
		\cellcolor{white} against / opposite \\


		\cellcolor{gray-light} O&
		\cellcolor{white} ohne &
		\cellcolor{white}  without\\


		\cellcolor{gray-light} U&
		\cellcolor{white} um \ldots herum &
		\cellcolor{white}  around\\


		\cellcolor{gray-light} H&
		\cellcolor{white} hinter &
		\cellcolor{white}  behind\\


		\cellcolor{gray-light} I&
		\cellcolor{white} in &
		\cellcolor{white} in , inside \\


		\cellcolor{gray-light} N&
		\cellcolor{white} neben &
		\cellcolor{white} next to , beside \\


		\cellcolor{gray-light} Ü&
		\cellcolor{white} über &
		\cellcolor{white} over , above \\


		\cellcolor{gray-light} U&
		\cellcolor{white} unter &
		\cellcolor{white} among ,under, below \\


		\cellcolor{gray-light} V&
		\cellcolor{white} vor &
		\cellcolor{white} ahead of, in front of \\


		\cellcolor{gray-light} Z&
		\cellcolor{white} zwischen &
		\cellcolor{white}  between\\


		\cellcolor{gray-light} W&
		\cellcolor{white} wider &
		\cellcolor{white} against \\


}

% }}} End TBOX TABLE : prepositions_accusative

% }}} END SUB-SECTION : accusative_prepositions

% 	SUB-SECTION : dative_prepositions {{{
\subsection{Dative Prepositions}
\label{ssec:dative_prepositions}

% TBOX TABLE : prepositions_dative {{{

\tcolorboxtable
{ Prepositions : Dative }
{ \label{table:prepositions_dative} }
{ [width=\linewidth] }
{ l|XX }
{

		\cellcolor{gray-light} V&
		\cellcolor{white} von &
		\cellcolor{white} of / from \\

		\cellcolor{gray-light} Z&
		\cellcolor{white} zu &
		\cellcolor{white} to / for \\


		\cellcolor{gray-light} S&
		\cellcolor{white} seit &
		\cellcolor{white} since \\


		\cellcolor{gray-light} N&
		\cellcolor{white} nach &
		\cellcolor{white} towards / to / past (time) / after\\


		\cellcolor{gray-light} A&
		\cellcolor{white} aus &
		\cellcolor{white} out of / from / made of \\


		\cellcolor{gray-light} M&
		\cellcolor{white} mit &
		\cellcolor{white} with\\


		\cellcolor{gray-light} B&
		\cellcolor{white} bei &
		\cellcolor{white}  with / by\\


		\cellcolor{gray-light} A&
		\cellcolor{white} außer &
		\cellcolor{white}  besides\\


		\cellcolor{gray-light} G&
		\cellcolor{white} gegenüber &
		\cellcolor{white} against \\


}

% }}} End TBOX TABLE : prepositions_dative

% }}} END SUB-SECTION : dative_prepositions

% 	SUB-SECTION : wechsel_prepositions {{{
\subsection{Wechsel Prepositions}
\label{ssec:wechsel_prepositions}

Wechsel accusative if the preposition indicates movement.

Wechsel dative if the preposition does not indicate movement.


Unlike dative or accusative prepositions that we learned earlier, which can only
be used in their respective cases, Wechsel Prepositions are prepositions that
can be used in two different cases, namely wechsel prepositions can be used with
objects that are in the dative case (indirect objects) and in the accusative
case (direct objects). Wechsel prepositions are known as dual prepositions in
English since they can be used with two cases.
The easiest way to determine if in a given sentence we are using the dative or
the accusative version of the wechsel preposition is by looking the question
that the sentence is answering. Extremely simply If the sentence is answering
a wo (where) question about the object, then we are using the wechsel
preposition in the accusative case. If the sentence is answering a wohin (where
to) question about the object then we are using the wechsel preposition in the
dative case.
One thing to clarify when talking about wechsel prepositions is – the fact that
when we say we are using the preposition in the accusative or dative case, the
preposition itself is not changing. What we actually mean is that the article of
the noun that the preposition is talking about will get changed into either it’s
helps to think about the movement of the object in the sentence. A way to think
about it in English is using the two phrases “he jumps into the water”
versus “he is swimming in the water.” The first answers a “where to” question:
Where is he jumping? Into the water. Or in German, in das Wasser or ins Wasser.
He is changing location by moving from the land into the water. The second
phrase represents a “where” situation. Where is he swimming? In the water.
In German, in dem Wasser or im Wasser. He is swimming inside the body of water
and not moving in and out of that one location.


So basically :
Use the accusative if there is a significant change of location / position
happening to the object in the sentence, i.e., if the action (verb) is resulting
in the object being moved from one place to a different place then we will use
the accusative case with the wechsel preposition.
If there is no significant change in movement then, the action is occurring in a
confined space and little or no movement is taking place. Then we will use the
dative article for the object with the wechsel preposition.



% TBOX TABLE : prepositions_wechsel {{{

\tcolorboxtable
{ Prepositions : Wechsel }
{ \label{table:prepositions_wechsel} }
{ [width=\linewidth] }
{ l|XX }
{
		\cellcolor{gray-light} Z &
		\cellcolor{white} zwischen &
		\cellcolor{white} between\\

		\cellcolor{gray-light} U&
		\cellcolor{white} unter &
		\cellcolor{white} under\\

		\cellcolor{gray-light} N&
		\cellcolor{white} neben &
		\cellcolor{white} next to\\

		\cellcolor{gray-light} Ü&
		\cellcolor{white} über &
		\cellcolor{white} above\\

		\cellcolor{gray-light} H&
		\cellcolor{white} hinter &
		\cellcolor{white} behind \\

		\cellcolor{gray-light} A&
		\cellcolor{white} an &
		\cellcolor{white} at \\

		\cellcolor{gray-light} V&
		\cellcolor{white} vor &
		\cellcolor{white} in front of \\

		\cellcolor{gray-light} A&
		\cellcolor{white} auf &
		\cellcolor{white} on \\

		\cellcolor{gray-light} I&
		\cellcolor{white} in &
		\cellcolor{white} in \\


}

% }}} End TBOX TABLE : prepositions_wechsel

% }}} END SUB-SECTION : wechsel_prepositions

% 	SUB-SECTION : genetiv_prepositions {{{
\subsection{Genetive Prepositions}
\label{ssec:genetiv_prepositions}

während is also used as as connector and a preposition.\\
während : three meanings\\

während : while (connector)\\
während : during (preposition)\\


% TBOX TABLE : prepositions_genetive {{{

\tcolorboxtable
{ Prepositions : Genetive }
{ \label{table:prepositions_genetive} }
{ [width=\linewidth] }
{ l|XX }
{

}

% }}} End TBOX TABLE : prepositions_genetive


% }}} END SUB-SECTION : genetiv_prepositions

% 	SUB-SECTION : temporal_prepositions {{{
\subsection{Temporal Prepositions}
\label{ssec:temporal_prepositions}

Temporal Prepositions as mentioned earlier are all the prepositions from
accusative , dative , wechsel and genetive that refrence an object in time. The
temporal prepositions have two categories depending upon whether they answer the
question of When (Wann) , or Till when (Bis wann) , from when or how long (Wie
lange). Basically between a definite time point and a time frame.  The list of
all temporal prepositions is as follows :\newpar

% TBOX TABLE : temporal_prepositions_wann {{{

\tcolorboxtable
{ Prepositions : Temporal Wann}
{ \label{table:temporal_prepositions_wann} }
{ [width=\linewidth] }
{ X|X }
{

\rowcolor{table-topic} \textbf{Preposition} & \textbf{Translation}\\

\midrule

\rowcolor{table-alternating-white} \textbf{\textit{an}}        &      \\
\rowcolor{table-alternating-white} \textbf{\textit{aus}}       &      \\
\rowcolor{table-alternating-white} \textbf{\textit{bei}}       &      \\
\rowcolor{table-alternating-white} \textbf{\textit{in}}        &      \\
\rowcolor{table-alternating-white} \textbf{\textit{nach}}      &      \\
\rowcolor{table-alternating-white} \textbf{\textit{vor}}       &      \\
\rowcolor{table-alternating-white} \textbf{\textit{zu}}        &      \\
\rowcolor{table-alternating-white} \textbf{\textit{zwischen}}  & between     \\
\rowcolor{table-alternating-white} \textbf{\textit{auf}}       &  \\
\rowcolor{table-alternating-white} \textbf{\textit{um}}        &  \\
\rowcolor{table-alternating-white} \textbf{\textit{gegen}}     & against \\
\rowcolor{table-alternating-white} \textbf{\textit{außerhalb}} & outside of    \\
\rowcolor{table-alternating-white} \textbf{\textit{innerhalb}} & within    \\
\rowcolor{table-alternating-white} \textbf{\textit{während}}   & During    \\


}

% }}} End TBOX TABLE : temporal_prepositions_wann

All the situations when these temporal prepositions are supposed to be used are shown below :


% TABULARX TABLE : temporal_prepositions_wann {{{

\tabularxtable
{0.95\linewidth}
{llX}
{

\rowcolor{table-topic}
\textbf{Preposition} & \textbf{Case} & \textbf{Usage}\\


\rowcolor{table-topic}
{\textit{Präposition}} &
{\textit{Kasus}} &
{\textit{Verwendung}}\\

\midrule

\rowcolor{table-alternating-white} \textbf{\textit{an}}        & Dative     & Tagteile\\
\rowcolor{table-alternating-white}                             &            & Tage\\
\rowcolor{table-alternating-white}                             &            & Datum\\
\rowcolor{table-alternating-white}                             &            & Wochenende\\
\rowcolor{table-alternating-white} \textbf{\textit{aus}}       & Dative     & zeitliche Herkunft\\
\rowcolor{table-alternating-white} \textbf{\textit{bei}}       & Dative     & parallel laufende Handlungen\\
\rowcolor{table-alternating-white} \textbf{\textit{in}}        & Dative     & Moment\\
\rowcolor{table-alternating-white}                             &            & nacht\\
\rowcolor{table-alternating-white}                             &            & Wochen\\
\rowcolor{table-alternating-white}                             &            & Monate\\
\rowcolor{table-alternating-white}                             &            & Jahrezeiten\\
\rowcolor{table-alternating-white}                             &            & Jahrzente\\
\rowcolor{table-alternating-white}                             &            & Jahrhunderte / Epochen\\
\rowcolor{table-alternating-white}                             &            & im Sinne von Innerhalb\\
\rowcolor{table-alternating-white}                             &            & zukünfitger Zeitpunkt\\
\rowcolor{table-alternating-white} \textbf{\textit{nach}}      & Dative     & zeitliche abfolge\\
\rowcolor{table-alternating-white} \textbf{\textit{vor}}       & Dative     & zeitliche abfolge\\
\rowcolor{table-alternating-white} \textbf{\textit{zu}}        & Dative     & kirchliche feiertage\\
\rowcolor{table-alternating-white}                             &            & beginn\\
\rowcolor{table-alternating-white}                             &            & bestimmter zeitpunkt\\
\rowcolor{table-alternating-white} \textbf{\textit{zwischen}}  & Dative     & begrenzter zeitraum \\
\rowcolor{table-alternating-white} \textbf{\textit{auf}}       & Akkusative & zeitpunkt \\
\rowcolor{table-alternating-white} \textbf{\textit{um}}        & Akkusative & genaue Uhrzeit \\
\rowcolor{table-alternating-white} \textbf{\textit{gegen}}     & Akkusative & ungenaue Zeitangabe \\
\rowcolor{table-alternating-white} \textbf{\textit{außerhalb}} & Genetiv    & ungenaue Zeitangabe \\
\rowcolor{table-alternating-white} \textbf{\textit{innerhalb}} & Genetiv    & ungenaue Zeitangabe \\
\rowcolor{table-alternating-white} \textbf{\textit{während}}   & Genetiv    & ungenaue Zeitangabe \\



}

% }}} End TABULARX TABLE : temporal prepositions wann


% TBOX TABLE : prepositions_temporal_duration {{{

\tcolorboxtable
{ Prepositions : Temporal Duration }
{ \label{table:prepositions_temporal_duration} }
{ [width=\linewidth] }
{ XX }
{

\rowcolor{table-topic} \textbf{Preposition} & \textbf{Translation}\\

\midrule

\rowcolor{table-alternating-white} \textbf{\textit{ab}}             & \\
\rowcolor{table-alternating-white} \textbf{\textit{seit}}           & \\
\rowcolor{table-alternating-white} \textbf{\textit{bis}}            & \\
\rowcolor{table-alternating-white} \textbf{\textit{von \ldots bis}} & \\
\rowcolor{table-alternating-white} \textbf{\textit{von \ldots}}     & \\
\rowcolor{table-alternating-white} \textbf{\textit{bis zu}}         & \\
\rowcolor{table-alternating-white} \textbf{\textit{für}}            & \\
\rowcolor{table-alternating-white} \textbf{\textit{über}}           & \\


}

% }}} End TBOX TABLE : prepositions_temporal_duration


\subsectionend
% }}} END SUB-SECTION : temporal_prepositions

% 	SUB-SECTION : spatial_prepositions {{{
\subsection{Spatial Prepositions}
\label{ssec:spatial_prepositions}


\subsectionend
% }}} END SUB-SECTION : spatial_prepositions

% 	SUB-SECTION : pronouns_with_prepositions {{{
\subsection{Pronouns with Prepositions}
\label{ssec:pronouns_with_prepositions}

daran , darüber , \ldots
\subsectionend
% }}} END SUB-SECTION : pronouns_with_prepositions

% 	SUB-SECTION : verbs_with_prepositions {{{
\subsection{Verbs with prepositions}
\label{ssec:verbs_with_prepositions}


\subsectionend
% }}} END SUB-SECTION : verbs_with_prepositions

% 	SUB-SECTION : zu {{{
\subsection{Zu}
\label{ssec:zu}

\textbf{Zu} is a preposition that might create a fair amount of confusion whent
rying to learn german. So I thought I would give it its own sub-section to
clarify and all all questions that at least came up in mind.\newpar

, Zu ` can be used as :\newpar

% ITEMIZE : zu_uses {{{

\begin{itemize}[noitemsep]
	\item Locative preposition 
	\item Temporal preposition 
	\item Causal preposition
	\item An adverb
	\item A conjunction
\end{itemize}

% }}} END-ITEMIZE : zu uses

Given the number of use cases listed above , it is no surprise that , zu `
causes so many problems in comprehension.\newpar

\subsectionend
% }}} END SUB-SECTION : zu

% 	SUB-SECTION : contractions {{{
\subsection{Contractions}
\label{ssec:contractions}

This section should be read after going through the prepositions
section\refsec{prepositions}. Contractions in German are basically the fusion of
two words into one. The most common occurance of this is between articles and
prepositions. To be more specific definite articles and some preopsitions. Not
all preposition and noun combinations can be fused into a contracted word.
Following is a list of all the most common contractions in use in German today : 

% TABULARX TABLE : contractions {{{

\tabularxtable
{0.95\linewidth}
{llllX}
{

an    & + & das & = & ans \\
an    & + & dem & = & am\\
auf   & + & das & = & aufs\\
bei   & + & dem & = & beim\\
durch & + & das & = & durchs\\
für   & + & das & = & fürs\\
in    & + & das & = & ins\\
in    & + & dem & = & im\\
um    & + & das & = & ums\\
von   & + & dem & = & vom\\
zu    & + & der & = & zur\\
zu    & + & dem & = & zum\\



}

% }}} End TABULARX TABLE : Contractions

Aside from organizing prepositions into the above cases , prepositions in german
also undergo a process called contraction \ concatenation. This basically means
that the preposition will fuse with the article of the word that it is acting on
in order to form a new word. Some of the most common contractions are shown
below : \newpar

% ITEMIZE : contraction_examples {{{

\begin{itemize}[noitemsep]

	\item bei + dem = beim
	\item von + dem = vom
	\item zu + dem = zum
	\item zu + der = zur

\end{itemize}

% }}} END-ITEMIZE : contraction examples

A more exhaustive list of contractions in german is in the table below :\newpar

Of course not all prepositions and articles are contracted in german. So , if we
have any other article preposition combination except for the ones listed above,
just write the full preposition and article out. For example :\newpar

zu + die =/=  zuie , it will remain as zu die\\
bei + der =/= bier , it will remain as bei der\\

\subsectionend

% }}} END SUB-SECTION : contractions

\sectionend
% }}} END SECTION : prepositions

% SECTION : adjectives {{{
\section{{Adjectives}}
\label{sec:adjectives}

% DEFINITION : adjective {{{
\tcolorboxdefinition
{Adjective}
{\label{def:adjective}}
{

An adjective is any word that serves to describe a given noun.

}

% }}} END DEFINITION : adjective

\lettrine[lines=3, findent=3pt, nindent=0pt]{A}{djectives} are words like young,
old, tall, thin, \ldots . All of these words serve to add a description to the
noun (or pronoun), in some way. They most commonly appear right before a noun in
a sentence , but can also be seperate by a verb like \textit{sein
	\textcolor{gray}{(to be)} , ansehen \textcolor{gray}{(to look)}, fühlen
	\textcolor{gray}{(to feel)} }\ldots \newpar

The difference between German and English adjectives is that like everything
else in German, the adjectives change according to the article of the noun that
they are being used to describe , i.e.\ , the adjectives are also declined. So
essentially based on gender, count and case the formation of the adjectives will
change. Wether an adjective is declined or not depends on the positioning of the
adjective in the sentence relative to the noun and the verb.\newpar


% ITEMIZE : adjective_positioning {{{

\begin{itemize}[noitemsep]
	\item If the adjective is \textbf{before} the noun but \textbf{after} the
		verb, then it will be declined according to the gender of the noun.\\

E.g.\ :

\noindent
\textit{Ich kann das \textbf{alt\textcolor{green-goethe}{e}} Haus sehen. \ Ich
	kann ein \textbf{alt\textcolor{green-goethe}{es}} Haus sehen.}\\
\textcolor{gray} { \textit{( I can see the old house. \ I can see an old house. )} } \newpar

	\item If the adjective comes directly \textbf{after} the verb then it will
		not be declined and it will stay in its base form \textit{(grund
			form)}\\

\noindent
\textit{Das Buch ist \textbf{neu}}\\
\textcolor{gray} { \textit{( The book is new )} } \newpar




\end{itemize}

% }}} END-ITEMIZE : adjective positioning

Keep in mind that both the article itself as well as the adjective will
change, according to the function of the noun in the sentence.\newpar

% Adjectives definite articles {{{
\textbf{Adjective Endings : Definite Articles}\newpar

When we are using nouns with the definite articles : \textbf{\textit{der}} ,
\textbf{\textit{die}} , \textbf{\textit{das}} , the endings for the adjectives
will be as follows :

% TBOX TABLE : adjective_endings_definite {{{

\tcolorboxtable
{ Adjective Endings : Definite }
{ \label{table:adjective_endings_definite} }
{ [width=\linewidth] }
{l|XXXX }
{

&
\cellcolor{table-subtopic} \textbf{\textit{MAS.}} &
\cellcolor{table-subtopic} \textbf{\textit{NEU.}}  &
\cellcolor{table-subtopic} \textbf{\textit{FEM.}}  &
\cellcolor{table-subtopic} \textbf{\textit{PLU.}} \\
\midrule

\cellcolor{table-subtopic} \textbf{\textit{NOM.}} &
\cellcolor{cell-lightred}  -e                &
\cellcolor{cell-lightred}  -e                &
\cellcolor{cell-lightred}  -e                &
\cellcolor{cell-lightblue} -en \\

\cellcolor{table-subtopic} \textbf{\textit{ACC.}} &
\cellcolor{cell-lightblue} -en               &
\cellcolor{cell-lightred}  -e                &
\cellcolor{cell-lightred}  -e                &
\cellcolor{cell-lightblue} -en \\

\cellcolor{table-subtopic} \textbf{\textit{DAT.}} &
\cellcolor{cell-lightblue} -en               &
\cellcolor{cell-lightblue} -en               &
\cellcolor{cell-lightblue} -en               &
\cellcolor{cell-lightblue} -en \\

\cellcolor{table-subtopic} \textbf{\textit{GEN.}} &
\cellcolor{cell-lightblue} -en               &
\cellcolor{cell-lightblue} -en               &
\cellcolor{cell-lightblue} -en               &
\cellcolor{cell-lightblue} -en \\



}

% }}} End TBOX TABLE : adjective_endings_definite

A thing to keep in mind is that some adjectives will change spelling (for
phonetical reasons) based on which declination the adjective is under. An
example is with the adjective \textit{hoch \textcolor{gray}{(high)}}.

\noindent
\textit{Das Gebäude ist hoch.}\\
\textcolor{gray} { \textit{( The building is high. )} } \newpar

\noindent
\textit{Das ist ein hohes Gebäude.}\\
\textcolor{gray} { \textit{( That is a high building. )} } \newpar

The general rule for spelling changes are as follows :


% ITEMIZE : adjective_spelling_changes {{{

\begin{itemize}[noitemsep]
	\item Adjectives ending in -el lose -e when declined.
\end{itemize}

% }}} END-ITEMIZE : adjective spelling changes




% }}}

% Adjectives indefinite articles {{{

\textbf{Adjective Declination : Indefinite Articles}\newpar

The adjective endings used along with \textbf{\textit{ein}} ,
\textbf{\textit{irgendein}} , \textbf{\textit{kein}} are as follows :\newpar


% TBOX TABLE : adjective_endings_indefinite {{{

\tcolorboxtable
{ Adjective Endings : Indefinite }
{ \label{table:adjective_endings_indefinite} }
{ [width=\linewidth] }
{ l|XXXX }
{

&
\cellcolor{table-subtopic} \textbf{\textit{MAS.}} &
\cellcolor{table-subtopic} \textbf{\textit{NEU.}}  &
\cellcolor{table-subtopic} \textbf{\textit{FEM.}}  &
\cellcolor{table-subtopic} \textbf{\textit{PLU.}} \\
\midrule

\cellcolor{table-subtopic} \textbf{\textit{NOM.}} &
\cellcolor{cell-lightgreen}   -er             &
\cellcolor{cell-lightorange} -es             &
\cellcolor{cell-lightred}    -e              &
\cellcolor{cell-lightblue}   -en \\

\cellcolor{table-subtopic} \textbf{\textit{ACC.}} &
\cellcolor{cell-lightblue}   -en             &
\cellcolor{cell-lightorange} -es             &
\cellcolor{cell-lightred}    -e              &
\cellcolor{cell-lightblue}   -en \\

\cellcolor{table-subtopic} \textbf{\textit{DAT.}} &
\cellcolor{cell-lightblue} -en               &
\cellcolor{cell-lightblue} -en               &
\cellcolor{cell-lightblue} -en               &
\cellcolor{cell-lightblue} -en \\

\cellcolor{table-subtopic} \textbf{\textit{GEN.}} &
\cellcolor{cell-lightblue} -en               &
\cellcolor{cell-lightblue} -en               &
\cellcolor{cell-lightblue} -en               &
\cellcolor{cell-lightblue} -en \\



}

% }}} End TBOX TABLE : adjective_endings_indefinite

% }}}

% Adjectives no articles {{{

\textbf{Adjective Declination : Nouns with no articles}\newpar

There are also times when adjectives are used independent of articles, at that
point the declination of the adjective will be applied according to the
following table :\newpar

% TBOX TABLE : adjective_endings_no_article {{{

\tcolorboxtable
{ Adjective Endings : No Article }
{ \label{table:adjective_endings_no_article} }
{ [width=\linewidth] }
{ l|XXXX }
{
&
\cellcolor{table-subtopic} \textbf{\textit{MAS.}} &
\cellcolor{table-subtopic} \textbf{\textit{NEU.}}  &
\cellcolor{table-subtopic} \textbf{\textit{FEM.}}  &
\cellcolor{table-subtopic} \textbf{\textit{PLU.}} \\
\midrule

\cellcolor{table-subtopic} \textbf{\textit{NOM.}} &
\cellcolor{cell-lightgreen}   -er             &
\cellcolor{cell-lightorange} -es             &
\cellcolor{cell-lightred}    -e              &
\cellcolor{cell-lightred}   -e\\

\cellcolor{table-subtopic} \textbf{\textit{ACC.}} &
\cellcolor{cell-lightblue}   -en             &
\cellcolor{cell-lightorange} -es             &
\cellcolor{cell-lightred}    -e              &
\cellcolor{cell-lightred}   -e \\

\cellcolor{table-subtopic} \textbf{\textit{DAT.}} &
\cellcolor{cell-lightyellow} -em               &
\cellcolor{cell-lightyellow} -em               &
\cellcolor{cell-lightgreen} -er              &
\cellcolor{cell-lightblue} -en \\

\cellcolor{table-subtopic} \textbf{\textit{GEN.}} &
\cellcolor{cell-lightblue} -en               &
\cellcolor{cell-lightblue} -en               &
\cellcolor{cell-lightgreen} -er               &
\cellcolor{cell-lightgreen} -er \\



}

% }}} End TBOX TABLE : adjective_endings_no_article

Besides just appllying to adjectives that modify a noun which is not preceded by
a definite or indefinte article, the no article (or strong declension) adjective
endings will also apply to the following words, as long as the noun that the
following words are referring to are not preceded by an article :\newpar


% ITEMIZE : adjective_no_article_ending_extra {{{

\begin{itemize}[noitemsep]
	\item ein bisschen (a little, a bit of)

	\item ein wenig (a little)

		\textit{Morgen hätte ich \textbf{ein wenig}
			frei\textbf{\textcolor{green-goethe}{e}} Zeit für dich.}\\
		\textcolor{gray} { \textit{(I could spare you some time tomorrow.)} }
		\newpar

	\item ein paar (a few , a pair , a couple)

		\textit{Sie hat mir \textbf{ein paar}
			gut\textbf{\textcolor{green-goethe}{e}} tips gegeben.}\\
		\textcolor{gray} { \textit{(She gave me a couple of good tips.)} }
		\newpar


	\item weinger (fewer , less) 

		\textit{Er isst \textbf{weniger}
			frisch\textbf{\textcolor{green-goethe}{es}} Obst als ich.}\\
		\textcolor{gray} { \textit{(He eats less fresh fruit than me.)} }
		\newpar


	\item einige (plural forms only)

	\item etwas (some,any(singular))

	\item mehr (more)

		\textit{Heutzutage wollen \textbf{mehr}
			jung\textbf{\textcolor{green-goethe}{e}} Frauen Ingenieurin werden.}\\
		\textcolor{gray} { \textit{(Nowadays, more young women want to be
				engineers.)} }
		\newpar


	\item lauter (nothing but, sheer , pure)

	\item solch (such)

		\textit{\textbf{Solche} lecker\textbf{\textcolor{green-goethe}{e}}
			Schokolade habe ich schon lange nicht mehr gegessen.}\\
		\textcolor{gray} { \textit{(I haven't had such good chocolate for a long
				time.)} } \newpar

	\item was für (what , what kind of )

		\newpar

	\item viel (much , many , a lot of )

		\textit{Wir haben \textbf{viel}
			kostbar\textbf{\textcolor{green-goethe}{e}} Zeit verschwendet.}\\
		\textcolor{gray} { \textit{(We have wasted a lot of valuable time.)} }
		\newpar

	\item welch \ldots ! (what \ldots ! , what a \ldots !)

		\textit{\textbf{Welch} herrlich\textbf{\textcolor{green-goethe}{es}}
			Wetter.}\\
		\textcolor{gray} { \textit{(What wonderful weather.)} }
		\newpar

	\item manch (many a)

	\item wenig (little few not much)
 
	\item zwei , drei , etc \ldots


\end{itemize}

% }}} END-ITEMIZE : adjective no article ending extra




% }}}

% ITEMIZE : adjective_ending_other_applications {{{

\begin{itemize}[noitemsep]
	\item \textbf{Question Words (Fragewörtern)} : welcher , welches , welche
	\item \textbf{Demonstrative Articles (Demnonstrativartikeln)} \\
		These : dieser , dieses , diese \\
		That \ those : jener , jenes , jene \\
		Such : solcher\\
		Every: jeder
\end{itemize}

% }}} END-ITEMIZE : adjective ending other applications

Adjectives are very versatile in their use cases, which is to say that with just
a little bit of tweaking we can use adjectives as nouns and even as adverbs.
These adjective conversions are talked about in more detail in sections
\refssec{adjectives_as_nouns} and \refssec{adjectives_as_adverbs} \newpar

Besides just article endings, there is another way that adjectives can be used.
Adjectives can be used to compare multiple differnt nouns. These are called
\textbf{\textit{comparatives(komparativ)}} and \textbf{\textit{superlatives
		(Superlativ)}}. Each one of these has different uses. Together these
three are called \textbf{\textit{steigerungsformen}}.  Both of which are talked
about in the following sections.\newpar

% 	SUB-SECTION : comparatives {{{
\subsection{Comparatives}
\label{ssec:comparatives}


% DEFINITION : comparatives {{{
% {title} {label} {content}

\tcolorboxdefinition
{Comparative}
{\label{def:comparative}}
{

A \textbf{Comparative} (Komparativ) is a word that is used to compare a
subjunctive (noun) with another subjunctive (noun). Comparatives are formed out
of either adjectives or adverbs. E.g. : higher , shorter , etc \ldots \\

}

% }}} END DEFINITION : comparatives

Essentially a comparative is used when we are comparing two nouns togther
without going to the extremity. If we wish to tlak about which one of the two
nouns personifies the quality in question to the highest extent then we would
use a superlative.

To form a comparative we add the ending : \textbf{\textit{-er}} to the
adjective. Here are a couple of examples :

% TABULARX TABLE : comparative_formation_examples {{{

\tabularxtable
{0.95\linewidth}
{XX}
{
\rowcolor{table-topic} adjective  & comparative \\

\midrule

\rowcolor{table-alternating-white} schnell &
schnell\textbf{\textcolor{green-goethe}{-er}} \\

\rowcolor{table-alternating-blue} lang  & lang\textbf{\textcolor{green-goethe}{-er}} \\

\rowcolor{table-alternating-white} alt & ält\textbf{\textcolor{green-goethe}{-er}} \\

\rowcolor{table-alternating-blue} klein  &
klein\textbf{\textcolor{green-goethe}{-er}} \\



}

% }}} End TABULARX TABLE : comparative formation examples

\textbf{Umlauts in the adjective formation}\\

The third example above illustrates that sometimes , not only do we add an
`\textit{-er}' ending to the adjective in question, sometimes we will also add
an umlaut in the comparative formatiion in the respective position, as is the
case in : alt , älter~\footnote{as a side note the german youth use alter
	similar to way that english speaker would use `dude', keep this is mins,
	becuase when someone says that to you, they are probably not calling you
	old.}\newpar

\textbf{Comparatives for adjectives with -er endings}\\

The second thing to note is the comparative formation for when we already have
an adjective with an ` \textit{-er} ' ending.  In this case instead of writing
\textit{-erer} as the ending, we write ` \textit{-rer} ' deleting the `e' in the
`-er' ending that comes from the adjective. \newpar



One thing that trips people up (it certainly
tripped me up) is that when we are using comparatives there are two types of
situations based on the positioning of the adjective that we are trying to use
as a comparative


% TBOX TABLE : comparative_and_superlative_adjectives {{{

\tcolorboxtable
{ Comparative and Superlative Adjectives }
{ \label{table:comparative_and_superlative_adjectives} }
{ [width=\linewidth] }
{ llllll }
{

\rowcolor{table-topic} \multicolumn{6}{l}{ \textbf{beim Verb} }\\

\cellcolor{table-subtopic} \textbf{\textit{Grund.\ }} &
\cellcolor{table-alternating-white} \ldots ist        &
\cellcolor{table-alternating-white}                   &
\cellcolor{table-alternating-blue}dick                &
\cellcolor{table-alternating-white}                   &
\cellcolor{table-alternating-white} \\

\cellcolor{table-subtopic} \textbf{\textit{Kom.\ }} &
\cellcolor{table-alternating-white}\ldots ist       &
\cellcolor{table-alternating-white}                 &
\cellcolor{table-alternating-blue}dick              &
\cellcolor{table-alternating-white}-er              &
\cellcolor{table-alternating-white}\\

\cellcolor{table-subtopic} \textbf{\textit{Sup.\ }} &
\cellcolor{table-alternating-white}\ldots ist       &
\cellcolor{table-alternating-white}am               &
\cellcolor{table-alternating-blue}dick              &
\cellcolor{table-alternating-white}-st              &
\cellcolor{table-alternating-white}-en \\

\multicolumn{6}{l}{}\\

\rowcolor{table-topic} \multicolumn{6}{l}{ \textbf{beim Nomen} }\\
\cellcolor{table-subtopic} \textbf{\textit{Grund.\ }} &
\cellcolor{table-alternating-white}ein        &
\cellcolor{table-alternating-white}                   &
\cellcolor{table-alternating-blue}dick                &
\cellcolor{table-alternating-white}                   &
\cellcolor{table-alternating-white}-er \\

\cellcolor{table-subtopic} \textbf{\textit{Kom.\ }} &
\cellcolor{table-alternating-white}ein       &
\cellcolor{table-alternating-white}                 &
\cellcolor{table-alternating-blue}dick              &
\cellcolor{table-alternating-white}-er              &
\cellcolor{table-alternating-white}-er\\

\cellcolor{table-subtopic} \textbf{\textit{Sup.\ }} &
\cellcolor{table-alternating-white}der       &
\cellcolor{table-alternating-white}                 &
\cellcolor{table-alternating-blue}dick              &
\cellcolor{table-alternating-white}-st              &
\cellcolor{table-alternating-white}-e \\




}

% }}} End TBOX TABLE : comparative_and_superlative_adjectives


% ITEMIZE : comparative_situations {{{

\begin{enumerate}[noitemsep]
	\item The adjective is before the noun , i.e.\ , the adjective is modifying
		the noun. In this case we will give the adjective ending.
	\item The adjective is after the noun, i.e.\ , it is not modifying the noun.
\end{enumerate}

% }}} END-ITEMIZE : Comparative situations

Mostly the sentence structure for a comparative sentence looks something like
this :\\


% TABULARX TABLE : comparative_irregularities {{{

\tabularxtable
{0.95\linewidth}
{XlXlX}
{
\bulletpoint edel     & - & edler     & - & am edelsten \\
\bulletpoint sensibel & - & sensibler & - & am sensibelsten \\
\bulletpoint dunkel   & - & dunkler   & - & am dunkelsten\\
\bulletpoint flexibel & - & flexibler & - & am flexibelsten\\
}

% }}} End TABULARX TABLE : comparative irregularities

Look at the following examples :\newpar


\noindent
\textit{Peter ist groß, aber Hubert ist größer}\\
\textcolor{gray} { \textit{( Peter is tall , but Hubert is taller )} } \newpar

In the above sentence the adjective is groß , but we are comparing two things by
size, therefore one must be bigger. So just like english we add an -en to the
ending of the adjective to indicate a comparative. Another example with schön
:\newpar


\noindent
\textit{Gestern war das Wetter schön, und heute wird es noch schöner.}\\
\textcolor{gray} { \textit{( Yesterday the weather was beautiful, and today it
		will be even more beautiful. )} } \newpar


\textbf{Adjective ending in : -el}\\
If the adjective ends in -el , then the e in the
second to last position is deleted when writing the adjective in the comparative
form. However this deletion of the e is not true in the superlative form.

\textbf{Adjective ending in : -er}\\
If the adjective ends in -er and we have
a vowel (a,e,i,o,u) as the last character before the -er ending, then similar to
the -el ending the e in the second to last position in the word is deleted when
forming the comparative. If the character before the -er ending is a consonant
then the comparative form does not change.  Again this is not true for the
superlative form.


% TABULARX TABLE : comparative_irrgularities {{{

\tabularxtable
{0.95\linewidth}
{XlXlX}
{
\rowcolor{white} $\bullet$ teuer  & - & teurer   & - & am teuersten \\
\rowcolor{white} $\bullet$ sauer  & - & saurer   & - & am sauersten \\
\rowcolor{white} $\bullet$ sauber & - & sauberer & - & am saubersten \\
}

% }}} End TABULARX TABLE : comparative_irrgularities

\textbf{Single Syllable Adjectives} \\
When we have single syllable adjectives and they have a vowel then we use the
umlaut version of that vowel in the comparative and the superlative formation.


% TABULARX TABLE : comparative_irrgulartities {{{

\tabularxtable
{0.95\linewidth}
{XlXlX}
{
\rowcolor{white} $\bullet$ groß & - & größer & - & am größten \\
\rowcolor{white} $\bullet$ klug & - & klüger & - & am klügsten \\
\rowcolor{white} $\bullet$ alt  & - & älter  & - & am ältesten \\


}

% }}} End TABULARX TABLE : Comparative Irrgulartities



\subsectionend
% }}} END SUB-SECTION : comparatives

% 	SUB-SECTION : superlatives {{{
\subsection{Superlatives}
\label{ssec:superlatives}
% DEFINITION : superlatives {{{
% {title} {label} {content}

\tcolorboxdefinition
{Superlatives}
{\label{def:superlatives}}
{
A \textbf {Superlative} is a word that is used when we a comparing a subjunctive
to a group of subjunctives. The superlative is used to delineate the upper or
lower limits of the quality that we are comparing. E.g. : highest, shortest,
etc \ldots 
}

% }}} END DEFINITION : superlatives

Similarly, superlatives are used in dealing with extremes of a certain
comparison. Consider the following example :\\
\\
\textit{
Ute ist klein, Petra ist kleiner und Martina ist die kleinste.\\
\color{gray} ( Ute is small, Petra is smaller and Martina is the smallest)  \color{black}}
\\





\subsectionend
% }}} END SUB-SECTION : superlatives

% 	SUB-SECTION : indefinite_adjectives {{{
\subsection{Indefinite Adjectives}
\label{ssec:indefinite_adjectives}

A a very small group of adjectives that are used to talk about people or things
in a general way\cite{collins_german_grammar} . They are words like several ,
all , every.

\subsectionend
% }}} END SUB-SECTION : indefinite_adjectives

% 	SUB-SECTION : demonstrative_adjectives {{{
\subsection{Demonstrative Adjectives}
\label{ssec:demonstrative_adjectives}

\textbf{\textit{Demonstrative adjectives}} are words like \textit{this , that ,
	these} and \textit{those}. All of these words are used with a noun to poinmt
out a particular person , thing or groups of things.

\subsectionend
% }}} END SUB-SECTION : demonstrative_adjectives

% 	SUB-SECTION : adjectives_as_adverbs {{{
\subsection{Adjectives as Adverbs}
\label{ssec:adjectives_as_adverbs}


\subsectionend
% }}} END SUB-SECTION : adjectives_as_adverbs

% 	SUB-SECTION : adjectives_as_nouns {{{
\subsection{Adjectives as Nouns}
\label{ssec:adjectives_as_nouns}

Often times adjectives can be converted into nouns. There are no hard and fast
rules to doing this conversion, should you want to, however there are some
general guidelines that you can follow :\newpar

\subsectionend
% }}} END SUB-SECTION : adjectives_as_nouns

% 	SUB-SECTION : adjective_contractoins {{{
\subsection{Adjective Contractoins}
\label{ssec:adjective_contractoins}


\subsectionend
% }}} END SUB-SECTION : adjective_contractoins

\sectionend
% }}} END SECTION : adjectives

% SECTION : adverbs {{{
\section{{Adverbs}}
\label{sec:adverbs}


% DEFINITION : adverbs {{{
% {title} {label} {content}

\tcolorboxdefinition
{Adverbs}
{\label{def:adverbs}}
{

An \textbf{adverb }is any word that is used with verbs,adjectives or other
adverbs in order to give more information about when , where , how or in what
circumstances the situation is taking place.\cite{collins_german_grammar}

}

% }}} END DEFINITION : adverbs

\lettrine[lines=3, findent=3pt, nindent=0pt]{W}{hen} we use a word to further
accentuate the action that is being performed by a verb, or when we use a word
to accentuate an adjective that is being used to desribe a noun, this word is
called an adverb ( I like to think because it is adding to the verb.)\newpar

When an adverb modifies a verb, it usually tells us :

\begin{itemize}[noitemsep]
	\item \textbf {When} : He ran yesterday.
	\item \textbf {Where} : He ran here.
	\item \textbf{How }: He ran barefoot.
	\item \textbf{To what extent }: He ran fastest.
\end{itemize}

Adverbs are very versatile und are broadly divided into the following categories
in German :

% 	SUB-SECTION : temporal_adverbs {{{
\subsection{Temporal Adverbs}
\label{ssec:temporal_adverbs}

Temporal Adverbs or adverbs of time function to tell us \textbf {when} the event
or the current action is happening. These can include the day that the action
took place or will take place, as well as an abstract time like
``{\textit{soon}}'' . The most commonly used temporal adverbs shown below.
Following are all the temporal adverbs as they relate to days : \newpar

% TABULARX TABLE : temporal_adverbs_days {{{

\tabularxtable { 0.95\linewidth  } {lX} { \rowcolor{white}      vorgestern & day
	before yesterday \\ \rowcolor{gray-light}  gestern    & yesterday \\
	\rowcolor{white}      heute      & today \\ \rowcolor{gray-light}  morgen &
	tomorrow\\ \rowcolor{white}      übermorgen & day after tomorrow\\


}

% }}} End TABULARX TABLE : temporal_adverbs-days

Following are all the temporal adverbs as they relate to abstract amounts of
time\newpar

% TABULARX TABLE : temporal_adverbs_abstract_time {{{

\tabularxtable
{ 0.95\linewidth  }
{lX}
{
\rowcolor{white}     damals & then \\
\rowcolor{gray-light} früher & earlier \\
\rowcolor{white}     jetzt  & now \\
\rowcolor{gray-light} sofort & immediately \\
\rowcolor{white}     gleich & immediately \\
\rowcolor{gray-light} bald   & soon \\
\rowcolor{white}     später & later \\
\rowcolor{gray-light} dann   & then / after \\



}

% }}} End TABULARX TABLE : temporal_adverbs_abstract_time

The adverb \textit{gerade} is used to make a sentence in the present continuous.
For example we have the sentences shown below. The second version ensures
clarity in that the listener will know that the action is still continiuing as
opposed to the dual meaning of the first sentence.

\subsectionend
% }}} END SUB-SECTION : temporal_adverbs

% 	SUB-SECTION : frequency_adverbs {{{
\subsection{Frequency Adverbs}
\label{ssec:frequency_adverbs}


Frequency Adverbs function to tell us \textbf {how often} the event or the
current action is happening. Frequency adverbs can be formed out of any time
period by just adding a -s, or a -lich to the end of the word. This allows us to
use adverbs like daily, monthly, evenings and so forth. Some examples are shown
below :


% TABULARX TABLE : frequency_adverbs_abstract {{{

\tabularxtable
{ 0.95\linewidth  }
{lX}
{
\rowcolor{white}     immer      & always \\
\rowcolor{lightgray} fast immer & almost always \\
\rowcolor{white}     meistens   & most of the time \\
\rowcolor{lightgray} häufig     & frequently \\
\rowcolor{white}     oft        & often \\
\rowcolor{lightgray} ab und zu  & once in a while \\
\rowcolor{white}     manchmal   & sometimes \\
\rowcolor{lightgray} selten     & rarely \\
\rowcolor{white}     fast nie   & almost never \\
\rowcolor{lightgray} nie        & never \\


}

% }}} End TABULARX TABLE : Frequency adverbs abstract

% TABULARX TABLE : frequency_adverbs_specific {{{

\tabularxtable
{ 0.95\linewidth  }
{lX}
{
\rowcolor{white}      morgens     & mornings \\
\rowcolor{lightgray}  nachmittags & afternoons  \\
\rowcolor{white}      montags     & on mondays \\
\rowcolor{lightgray}  täglich     & daily \\
\rowcolor{white}      wochentlich & weekly \\
\rowcolor{lightgray}  monatlich   & monthly \\
\rowcolor{white}      jährlich    & yearly / anually \\
\rowcolor{lightgray}  halbtags    & half days \\
\rowcolor{white}      feiertags   & all holidays  \\


}

% }}} End TABULARX TABLE : Frequency Adverbs Specific



\subsectionend
% }}} END SUB-SECTION : frequency_adverbs

% 	SUB-SECTION : spatial_locative_adverbs {{{
\subsection{Spatial / Locative Adverbs}
\label{ssec:spatial_locative_adverbs}

Spatial Adverbs as is indicated in the name , serve to describe locations and
spaces. Therefore they are used to indicate the position of a person or object
relative to your current position. They can change based on wether you or the
object are moving, and can also change based on wether you and the object are
moving away from or towards each other. The most common spatial adverbs for both
static and movement based scenarios are shown below :



% TABULARX TABLE : spatial_adverbs_static {{{

\tabularxtable
{0.95\linewidth}
{lX}
{

\rowcolor{white}     vorn / vorne & in front\\
\rowcolor{lightgray} hinten       & behind\\
\rowcolor{white}     links        & left\\
\rowcolor{lightgray} rechts       & right\\
\rowcolor{white}     oben         & over\\
\rowcolor{lightgray} unten        & under\\
\rowcolor{white}     innen        & inside\\
\rowcolor{lightgray} außen        & outside\\
\rowcolor{white}     hier         & hier\\
\rowcolor{lightgray} da           & here/there\\
\rowcolor{white}     dort         & over there/there\\
\rowcolor{lightgray} überall      & everywhere\\
\rowcolor{white}     nirgends     & nowhere\\
\rowcolor{lightgray} fort      & away\\


}

% }}} End TABULARX TABLE : Spatial Adverbs Static
% TABULARX TABLE : spatial_adverbs_movement {{{

\tabularxtable
{0.95\linewidth}
{lX}
{
\rowcolor{white}     aufwärts  & upwards\\
\rowcolor{lightgray} abwärts   & downwards\\
\rowcolor{white}     vorwärts  & forwards\\
\rowcolor{lightgray} rückwärts & backwards\\
\rowcolor{white}     heimwärts & homeward\\
\rowcolor{lightgray} westwärts & westward\\
\rowcolor{white}     bergauf   & uphill\\
\rowcolor{lightgray} bergab    & downhill\\


}

% }}} End TABULARX TABLE : Spatial Adverbs Movement



% 		SUB-SUB-SECTION : hin_and_her_ {{{
\subsubsection{Hin- and Her-}
\label{sssec:hin_and_her_}

The prefixes \textit{Hin- }and \textit{Her-} cause a bit of confusion because
there are no equivalencies in English. German delineates between movement away
and towards the speaker in a way that english does not. It is to describe this
delineation that we use the aforementioned prefixes.\newpar

\textbf{\textit{Hin-}} generally indicates movement in a direction away from the speaker toward a
particular destination. Examples of sentences using the \textit{hin-} prefix are
shown below :
\\\\
\noindent
Wir gehen zum Hafen hin. : We are going to the harbor.\newpar
Schau mal hin!           : Look (over there)!
\\\\
\textbf{\textit{Her-}} generally indicates movement from a point of origin in a direction toward
the speaker.
\\\\
\noindent
Komm mal her!                 : Come over here (from there)!\newpar
Wo bekommen wir das Geld her? : Where will we get the money (from)?

Hin- and her- are used in their most literal sense with verbs of movement (e.g.,
gehen to go, kommen to come) or activity that involves direction (e.g., sehen to
look, geben to give, reichen to hand over). Often they appear as separable
prefixes (e.g., herkommen , herholen, hinlegen, hinschreiben). More specific
directional adverbs are created through a number of compounds that comine hin
and her with prepositions that denote direction (e.g., herauf, herab, heraus,
herein, hinauf, hinüber, hindurch, hinzu) or with other adverbs (e.g., hierher,
woher, dahin, überallhin).\newpar

\noindent
Er geht die Treppe hinauf.hinaufHe is going up the stairs.\\
Er kommt die Treppe herunter.herunterHe is coming down the stairs.\\
Der Apfel fiel vom Baum herab.herabThe apple fell (down) from the tree.\\
Der Apfel fiel ins Gras hinunter.hinunterThe apple fell (down) into the grass.
\newpar
\noindent
The most commonly used adverbs with hin and her are shown below :


% TABULARX TABLE : spatial_adverbs_hin_her {{{

\tabularxtable
{0.95\linewidth}
{lX}
{
\rowcolor{white}     heraus & go out (towards the speaker) \\
\rowcolor{lightgray} herein & go in (towards the speaker)\\
\rowcolor{white}     hinein & go in (away from speaker)\\
\rowcolor{lightgray} hinaus & go out (away from the speaker) \\


}

% }}} End TABULARX TABLE : Spatial Adverbs Hin Her


% NOTE : Hin and Her {{{

\tcolorboxnote
{
Even though I havent written the
verb gehen in the table above , the adverbs above are almost always used with
some movement related verb which is why the translation column has the words go.


}



% }}} END NOTE : hin_and_her


\subsubsectionend
% }}} END SUB-SUB-SECTION : hin_and_her_

\subsectionend
% }}} END SUB-SECTION : spatial_locative_adverbs

% 	SUB-SECTION : causal_conjunctional_adverbs {{{
\subsection{Causal / Conjunctional Adverbs}
\label{ssec:causal_conjunctional_adverbs}

Causal adverbs help in explaining a previous action. They are indicating the
cause for the previous action occuring. Therefore they act almost exactly like
conjunctions between two sentences are are also known as conjunctional adverbs.
The most common ones are :


% TABULARX TABLE : causal_adverbs {{{

\tabularxtable
{0.95\linewidth}
{lX}
{
\rowcolor{white}     also / so         & so / therefore \\
\rowcolor{lightgray} anstandshalber    & for decencys sake\\
\rowcolor{white}     dadurch           & through that/ because of that\\
\rowcolor{lightgray} darum             & therefore / because of that\\
\rowcolor{white}     demnach           & thus / according to that\\
\rowcolor{lightgray} demzufolge        & whereby / accordingly\\
\rowcolor{white}     deshalb           & therefore\\
\rowcolor{lightgray} folglich          & consequently\\
\rowcolor{white}     sicherheitshalber & preventatively\\
\rowcolor{lightgray} somit             & thus / therefore\\
\rowcolor{white}     trotzdem          & despite that\\
\rowcolor{lightgray} daher             & therefore\\


}

% }}} End TABULARX TABLE : Causal Adverbs

\subsectionend
% }}} END SUB-SECTION : causal_conjunctional_adverbs

% 	SUB-SECTION : interrogative_adverbs {{{
\subsection{Interrogative adverbs}
\label{ssec:interrogative_adverbs}

Used when trying to ask questions. These are basically the W-question words.
Therefore the main explanation for this topic is provided under the questions
section.


% TABULARX TABLE : interrogative_adverbs {{{

\tabularxtable
{0.95\linewidth}
{lX}
{
\rowcolor{white}     Wer   & Who\\
\rowcolor{lightgray} Was   & What\\
\rowcolor{white}     Wann  & When\\
\rowcolor{lightgray} Wo    & Where\\
\rowcolor{white}     Warum & Why\\


\rowcolor{lightgray} Wofür   & \\
\rowcolor{white}     Woher &  \\
\rowcolor{lightgray} Wieso &  \\
\rowcolor{white}     Wessen &  \\
\rowcolor{lightgray} Welcher &  \\
\rowcolor{white}     Wen &  \\
\rowcolor{lightgray} Wem &  \\

\rowcolor{white}     Wo &  \\
\rowcolor{lightgray} Woran &  \\
\rowcolor{white}     Worauf &  \\
\rowcolor{lightgray} Woraus &  \\
\rowcolor{white}     Wobei &  \\
\rowcolor{lightgray} Wogegen &  \\
\rowcolor{white}     Worin &  \\
\rowcolor{lightgray} Womit &  \\
\rowcolor{white}     Worüber &  \\
\rowcolor{lightgray} Worum &  \\
\rowcolor{white}     Wozu &  \\


}

% }}} End TABULARX TABLE : Interrogative Adverbs

\subsectionend
% }}} END SUB-SECTION : interrogative_adverbs

% 	SUB-SECTION : adverbs_of_manner {{{
\subsection{Adverbs of Manner}
\label{ssec:adverbs_of_manner}
Modal adverbs or adverbs of manner answer questions of how much of a
certain thing we are dealing with. So these can be broken up according to how
much of what we are dealing with. Then we get the following three categories :

\begin{enumerate}[noitemsep]
	\item Manner : 
	\item Quantity :
	\item Restrictive :
\end{enumerate}

beinahe -- nearly, almost
besonders -- especially
bloß -- merely, simply, just
daneben -- besides, in addition
ebenfalls -- likewise, also
ebenso -- equally, similarly
eigentlich -- actually, in fact
fast -- almost
gemeinsam -- in common, jointly
gern, gerne -- gladly
hoffentlich -- hopefully
insgesamt -- in total, altogether
kaum -- hardly
leider -- unfortunately
mindestens -- at least, at minimum
nämlich -- namely
natürlich -- naturally
nebenbei -- by the way, incidentally
schließlich -- finally
sehr -- very
sogar -- even
sonst -- otherwise
teilweise -- partially
übrigens -- by the way
ungefähr -- approximately
ursprünglich -- originally
vielleicht -- perhaps, maybe
wahrscheinlich -- probably, likely
wirklich -- really, truly
ziemlich -- rather, quite
zufällig -- accidentally, by chance
zurück -- back
zusammen -- together



\subsectionend
% }}} END SUB-SECTION : adverbs_of_manner

% 	SUB-SECTION : pronoun_adverbs {{{
\subsection{Pronoun Adverbs}
\label{ssec:pronoun_adverbs}

Pronomial adverbs are words like : daran , damit, darüber, hierbei, hiermit,
wovon.  Even though they are referred to as Pronoun Adverbs, they are actually a
combination of a preposition and a pronoun.\\

\noindent
Each Pronomialadverbien can be broken up into two parts : The stem and
The root.\\

\noindent The stem of pronomial adverbs is always one of the following three :
\textbf{\textit{da-}} , \textbf{\textit{hier-}} , \textbf{\textit{wo-}}.\\

\noindent The endings are determined by what we are actually referring to. The
adverbs ending will change according to wether we are referring to a person or
to a thing.\\


A complete list of all the pronomial adverbs are shown below, although this is
just meant as a reference. I would not advise memorizing all of these, and
rather knowing which preposition is accompanied with which verb we are using,
and using those to form the appropriate pronoun. Anyway here they are :\\

\textbf {Da-} : 
dabei
dadurch
dafür
dagegen
dahinter
damit
danach
daneben
daran
darauf
daraus
darin
darum
darunter
darüber
davon
davor
dazu
dazwischen

\textbf {Hier-} : 
hierfür
hiermit
hiervon

\textbf {Wo-}  
wobei
wodurch
wofür
wogegen
wohinter
womit
wonach
woneben
woran
worauf
woraus
worin
worum
worunter
worüber
wovon
wovor
wozu
wozwischen


\subsectionend
% }}} END SUB-SECTION : pronoun_adverbs

\sectionend
% }}} END SECTION : adverbs

% SECTION : sentence_structures {{{
\section{{Sentence Structures}}
\label{sec:sentence_structures}

% 	SUB-SECTION : clauses {{{
\subsection{Clauses}
\label{ssec:clauses}


% DEFINITION : clause {{{
\tcolorboxdefinition
{Clause}
{\label{def:clause}}
{

	A clause is any group of words which contain a
	verb.\cite{collins_german_grammar}


}

% }}} END DEFINITION : clause

Sentences as we understand them in english are groups of words delineated by
periods / fullstops. Each sentence in english can be made up of multiple
clauses, which are subsets of sentences sepereated by some form of punctuation
or a connecting word (connectors). In german , there are the following types of
clauses :\newpar

A \textbf{\textit{coordinate clause}} is \ldots \newpar

A \textbf{\textit{subordinate clause}} is \ldots \newpar

A \textbf{\textit{hauptsatz}} in german is any sentence that has follows a
regular sentence strucutre. By regular sentence structure I mean , noun /
pronoun at position 1 , the verb at position 2 and so on \ldots . As the term
implies the haupt satz can also be considered the `main clause' of the sentence,
although I would be careful with remembering the hauptsatz as the \textit{main}
sentence , since a sentence in german can contain multiple haupt clauses, which
can cause problems if you are a native english speaker (how can one sentence
have multiple main clauses ?).\newpar

The nebensatz is a sentence structure that is defined according to the
positioning of the verb as well. Nebensatze have two qualities :

% ENUMERATE : nebensatz_qualities {{{

\begin{enumerate}[noitemsep]
	\item There is always a connector at the beggining of a nebensatz.
	\item The verb is sent to the end in a nebensatz.
	\item A nebensatz is always accompanied with a hauptsatz.
\end{enumerate}

% }}} END-ENUMERATE : nebensatz qualities

A \textbf{\textit{nebensatz}} is \ldots \newpar
The nebensatz can come before or after a hauptsatz , which is why I have been
refraining from calling a nebensatz a neighboring sentence even though that is
what the literal translation would imply. We can begin sentences in german with
some connectors unlike in english. The list as well as examples of the main
connectors are given in section \refssec{connectors}. \newpar


A \textbf{\textit{relative clause}} is \ldots \newpar

A \textbf{\textit{Finale nebensatz}} is \ldots \newpar
\subsectionend
% }}} END SUB-SECTION : clauses

% 	SUB-SECTION : conjunctions {{{
\subsection{Conjunctions}
\label{ssec:conjunctions}

\subsectionend
% }}} END SUB-SECTION : conjunctions

% 	SUB-SECTION : cojunctions_connectors {{{
\subsection{Cojunctions / Connectors}
\label{ssec:cojunctions_connectors}

% DEFINITION : conjunction {{{
\tcolorboxdefinition
{Conjunction}
{\label{def:conjunction}}
{

Words that links together two clauses in a sentence are called conjunctions.\cite{collins_german_grammar}
}

% }}} END DEFINITION : conjunction

German has two types of conjuctions :

% ITEMIZE : conjunction_types {{{

\begin{itemize}[noitemsep]
	\item Subordinating Conjunctions  : 
	\item Coordinating Conjunctions : aber , denn , oder , sondern
\end{itemize}

% }}} END-ITEMIZE : Conjunction Types

\textbf{aber vs.\ sondern}

Both aber and sondern mean `but' , however the difference between these two
conjunctions is that sondern is always accompanied with a negation in the
hauptsatz. An example is provided below :


\noindent
\textit{Ich habe sehr viel Erklärt , aber niemand hat mir verstanden.}\\
\textcolor{gray} { \textit{( EXAMPLE TRANSLATION )} } \newpar


\noindent
\textit{Das ist keine Erklärung , sondern ein Beispiel.}\\
\textcolor{gray} { \textit{( That is not an explanation , but an example. )} } \newpar

aber can also mean ` however ' , in which case it will combe between the subject
and the clause.


% TABULARX TABLE : connectors {{{

\tabularxtable
{0.95\linewidth}
{llX}
{
deshalb & & \\
darum & & \\
deswegen& & \\
so \ldots dass& & \\
sodass& & \\
wegen& & \\
trotz& & \\
trotzdem& & \\
ob& & \\
obwohl& & \\
weil& & \\
wenn& & \\
als& & \\
denn& & \\
}

% }}} End TABULARX TABLE : connectors

\subsectionend
% }}} END SUB-SECTION : cojunctions_connectors

% 	SUB-SECTION : negation {{{
\subsection{Negation}
\label{ssec:negation}


\subsectionend
% }}} END SUB-SECTION : negation

% 	SUB-SECTION : passive {{{
\subsection{Passive}
\label{ssec:passive}

Passive sentences are so called because they are used mainly when we are putting
more emphasis on the action (verb / tätigkeit) than the person / thing that is
doing it. The main use cases for passive sentences are in the following
situations :\newpar

% TABULARX TABLE : passive_use_cases {{{
\textbf{Passive Use Cases}
\tabularxtable
{0.95\linewidth}
{lX}
{

}

% }}} End TABULARX TABLE : Passive use cases

This is a very fundamental type of german sentence structure. It is
formed in the following ways :\newpar






% TABULARX TABLE : vorgangspassiv {{{

\tabularxtable
{0.95\linewidth}
{lllllX}
{


\cellcolor{table-title} \textbf{Präs}  & : & Ich & \textbf{\textcolor{green-goethe}{werde}} & informiert & \\
\cellcolor{table-title} \textbf{Prät}  & : & Ich & \textbf{\textcolor{green-goethe}{wurde}} & informiert & \\
\cellcolor{table-title} \textbf{Perf}  & : & Ich & \textbf{\textcolor{green-goethe}{bin}} & informiert & worden \\
\cellcolor{table-title} \textbf{Plqu}  & : & Ich &\textbf{\textcolor{green-goethe}{war}} & informiert & worden \\
\cellcolor{table-title} \textbf{Mod} & : & Ich & \textbf{\textcolor{green-goethe}{kann}}  & informiert & werden \\
\cellcolor{table-title} \textbf{Fut} & : & Ich & \textbf{\textcolor{green-goethe}{werde}} & informiert & werden \\

}

% }}} End TABULARX TABLE : Vorgangspassiv

All the fo





\subsectionend
% }}} END SUB-SECTION : passive

% 	SUB-SECTION : infinitve_zu {{{
\subsection{Infinitve + Zu}
\label{ssec:infinitve_zu}

The \textbf{\textit{infinitive + zu}} formation is a sentence structure used in
german. As the name indicates , it basically is the use of a verb in its
infinitive form along with a zu. As a quick reminder an infinitve is a any verb
in it's unconjugated form, and zu is a prepostion which means ` to '. The only
things that needs explaining are : when we will use a infinitive+zu structure vs
when not , and what the sentence structure will look like.\newpar

First thing to note is that the Infinitive+zu formation will always appear in
the nebensatz.\ Second thing to note is that there are three types of situations
with regard to this formation. Namely :


% ENUMERATE : infinitive_zu_types_of_situations {{{

\begin{enumerate}[noitemsep]

	\item The sentence \textbf{must be} formed with infinitive zu.
	\item The sentence \textbf{can be }formed with or without infinitive zu.
	\item The sentence \textbf{must not be} formed with infinitive zu.

\end{enumerate}

% }}} END-ENUMERATE : infinitive zu types of situations

% TABULARX TABLE : infinitive_zu_will_not_be_used {{{

\textbf{Infinitive + Zu : Will not be used cases}\newpar

The situations where we will not use infinitve+zu are the following :\newpar

\tabularxtable
{0.95\linewidth}
{lX}
{

	Modal Verbs & since we are already using the main (non modal) verb as an
	infinitive.\\

	Verbs with movement & gehen  , kommen , fahren , laufen\\

	Verbs with speaking & sagen , fragen , antworten , berichten , erzählen ,
	informieren\\

	 & With verbs like shen , hören , spüren , and fühlen\\

	finden/haben + location & The vebs finden and haben in connection with locations\\ 
	lassen , schicken\\

	Verben mit wahrnehmung & sehen , hören , riechen , spüren , bemerken
	,lesen\\

	Verbs with knowledge & wissen , zweifeln , vermuten , kennen \\

}

% }}} End TABULARX TABLE : infinitive zu will not be used

% TABULARX TABLE : infinitive_zu_can_be_used {{{

\textbf{Infinitive + Zu : Can be used cases}

The situations where the sentence can be formed with or without infintive zu are
as follows :\newpar 

\tabularxtable
{0.95\linewidth}
{lX}
{

lernen / helfen & \\

& \textit{Das Kind lernt laufen.}\\
& \textit{Ich lerne , fehlerfrei \textbf{\textcolor{green-goethe}{zu schreiben}}.}\\
& \textit{Ich helfe dir tragen.}\\
& \textit{Er hilft Susi , das Auto zu reparieren.}\\


}

% }}} End TABULARX TABLE : infinitive zu can be used

% TABULARX TABLE : infinitive_zu {{{

\textbf{Infinitive + Zu : Must be used cases}\newpar

The list of situations in which we will use it are as follows :\newpar

\tabularxtable
{0.95\linewidth}
{XX}
{

Nomen + Haben              & Angst / Lust / Zeit / den Plan haben\\
iunpersonlichen ausdrucken & es ist wichtig  , we ist schwerig \\
Partizip + sein            & verboten / erlaubt / beabsichtigt sein \\
Verben : Erlaubnis         & erlauben , verbeiten \\
Verben : Angfang           & anfangen  , beginnen , aufhören\\
Verben : Absicht           & versuchen , vorhaben , sich vornehmen\\
Verben : Gefühl            & bedauern , befürchten , hoffen , sich freuen\\
anderen                    & sein , haben , erinnern , vergessen \\
                           & bitten , einladen , gefallen \\
}

% }}} End TABULARX TABLE : Infinitive zu

\subsectionend
% }}} END SUB-SECTION : infinitve_zu

% 	SUB-SECTION : um_zu_ {{{
\subsection{Damit / Um .. Zu .. / Zum}
\label{ssec:um_zu_}

\textbf{\textit{Damit}} , \textbf{\textit{Um \ldots Zu}} and
\textbf{\textit{zum}} are a couple of formations that we can use in order to
help answer the question of wozu in german sentences. This means that if the
first clause in a sentence is asking a question akin to : ,, Why are we / am I
doing this ? `` , then the second clause in the sentence must answer this
question , through the means of the three aforementioned words. Now we have to
figure out when to use which word.\newpar

\bulletpoint \textbf{\textit{Um \ldots Zu \ldots}} : When the subject in the
nebensatz stays the same as the subject in the hauptsatz. This also means that
since the subject is the same then we no longer need to (and by german
grammatical rules must not) mention the subject again when we are trying to use
Um \ldots zu. Another thing to note when using this formation is that the verb
will appear in its infinitive form at the end of the sentence. A couple of
examples showing how to use this structure are given below :\newpar


\noindent
\textit{Klingeln \textbf{Sie}, \textcolor{green-goethe}{\textbf{um}} auf sich aufmerksam
	\textcolor{green-goethe}{\textbf{zu}} machen.}\\
\textcolor{gray} { \textit{( EXAMPLE TRANSLATION )} } \newpar



\bulletpoint \textbf{\textit{Damit}} : We can use damit , when the subject in the nebensatz
stays the same as the hauptsatz , or even when the subject changes. Essentially
you can always use damit. The difference between damit and um \ldots zu is that
when we use damit we need to reaffirm the subject in the nebensatz. Some
examples are shown below , the first is one where the subject stays the same ,
the second where it changes :\newpar

\noindent
\textit{Klingeln \textbf{Sie}, \textcolor{green-goethe}{\textbf{damit}} \textbf{Sie} auf sich aufmerksam machen.}\\
\textcolor{gray} { \textit{( EXAMPLE TRANSLATION )} } \newpar


\noindent
\textit{Klingeln \textbf{Sie} , \textcolor{green-goethe}{\textbf{damit}} \textbf{andere Personen} Sie hören.}\\
\textcolor{gray} { \textit{( EXAMPLE TRANSLATION )} } \newpar




\bulletpoint \textbf{\textit{Zum}} : Zum can be used as an alternative to um
\ldots zu , when answering the wozu question with the subject remaining the
same. There are a couple of differences as can be seen from the examples below
:\newpar

\noindent
\textit{Ich nehme ein feuchtes Taschen \textcolor{green-goethe}{\textbf{zum}} reinigen.}\\
\textcolor{gray} { \textit{( EXAMPLE TRANSLATION )} } \newpar

Compare this to the same sentence made using um \ldots zu in order to help
yourself understand the difference.\newpar

\noindent
\textit{ \textcolor{green-goethe}{\textbf{Um}} die Tastatur
	\textcolor{green-goethe}{\textbf{zu}} reinigen , nehme \textbf{ich} ein feuchtes Taschentuch.}\\
\textcolor{gray} { \textit{( EXAMPLE TRANSLATION )} } \newpar




\subsectionend
% }}} END SUB-SECTION : um_zu_

% 	SUB-SECTION : double_connectors {{{
\subsection{Double Connectors}
\label{ssec:double_connectors}


% TABULARX TABLE : zweiteilige_konnektoren {{{

\tabularxtable
{0.95\linewidth}
{lllX}
{
sowohl     & \ldots & als auch     & das eine und das andere \\
nicht nur  & \ldots & sondern auch & das eine oder das andere \\
entweder   & \ldots & oder         & das eine nicht und das andere nicht\\
weder      & \ldots & noch         & \\
zwar       & \ldots & aber         & das eine mit Einschränkungen\\
einerseits & \ldots & andererseits & ein Sache hat zwei seiten\\
je & \ldots & desto & \\



}~\cite{netzwerk_b1}
% }}} End TABULARX TABLE : Zweiteilige Konnektoren



\subsectionend
% }}} END SUB-SECTION : double_connectors




\sectionend
% }}} END SECTION : sentence_structures

% ----------------------------------------------------------------------------- 
% BIBLIOGRAPHY & FIGURE LISTS {{{

% https://www.fluentu.com/blog/german/german-adjectival-nouns/

\onecolumn
\nolinenumbers
\bibliographystyle{unsrt}
\bibliography{bibliography/references}


% }}}
% -----------------------------------------------------------------------------
\end{document}
% =============================================================================
% - EOF - EOF - EOF - EOF - EOF - EOF - EOF - EOF - EOF - EOF - EOF - EOF -
% =============================================================================

