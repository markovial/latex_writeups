% =============================================================================

% PACKAGES ---------------------------------------------------------------- {{{

% DOCUMENT CLASS , ENCODING , TITLE -------------------------------------- {{{

\documentclass[a4paper,twocolumn,10pt]{article}
\setlength{\columnsep}{10pt}
%\setlength{\columnseprule}{1pt}
\usepackage[utf8]{inputenc}
\usepackage[english]{babel}

\title{This will be some super impressive title}
\author{Ishan Tiwathia}
\date{\today}

% }}}
% FONTS ------------------------------------------------------------------ {{{

\usepackage{courier}     % courier font
\usepackage{fontawesome} % font awesome glyphs
\usepackage{setspace}    % inter line spacing
\singlespacing           % set spacing to single spaced

% \onehalfspacing
% \doublespacing
% \setstretch{1.1}


% }}}
% PAGE FORMAT ------------------------------------------------------------ {{{

\usepackage[document]{ragged2e} % Text alignment package
\usepackage{enumitem}           %
\usepackage{geometry}           %
\geometry{
	a4paper,
	total = {170mm,257mm},
	top=5mm,
	left=5mm,
	right=5mm,
	bottom=5mm
}

\usepackage[switch,displaymath,mathlines]{lineno}
%\modulolinenumbers[5]
\linenumberfont{\normalfont\large\sffamily}
\renewcommand\thelinenumber{\color{gray}\arabic{linenumber}}
\setlength\linenumbersep{0.5cm}


% }}}
% TEXT FORMAT ------------------------------------------------------------ {{{

\usepackage[activate={true,nocompatibility},
final,
tracking=true,
kerning=true,
spacing=true,
factor=1100,
stretch=10,
shrink=10]{microtype}
% prevents a certiain amount of overfull hbox badness
% helps with other margin stuff

\usepackage{verbatim}
\makeatletter
\newcommand{\verbatimfont}[1]{\def\verbatim@font{#1}}%
\makeatother%\verbatimfont{courier}

\usepackage{type1cm}  % Allows font resizing
\usepackage{lettrine} % Allows font calligraphy (enlarge first char)
\renewcommand{\LettrineTextFont}{\rmfamily}


% }}}
% TITLE FORMAT  ---------------------------------------------------------- {{{

\usepackage[export]{adjustbox}
\usepackage{changepage}
\usepackage[compact,explicit]{titlesec} % Allows customization of section head

% compact : reduces spaces before and after sections
% explicit : allows for expicit positioning of title statement with #1

% title_format : command,shape,format,label,sep,before,after
% title-brackets : {}[]{}{}{}{}[] , basically only shape and aftercode have []

% Section Title Settings {{{
\titleformat {\section}
	[hang]
	{\color{black}\Large\bfseries}
	{}
	{0em}
	{
	\nolinenumbers
	\begin{section-box}
	\thesection. #1
	\end{section-box}
	}
[
\linenumbers
]

% left before-sep after-sep right-sep
\titlespacing{\section}{0cm}{0cm}{0cm}[0cm]

% }}}

% Sub-Section Title Settings {{{
\titleformat {\subsection}
	[hang]
	{\color{black}\small\bfseries}
	{}
	{0em}
	{
	\nolinenumbers
	\begin{subsection-box}
		\thesubsection. #1
	\end{subsection-box}
	}
[
\linenumbers
]

% left before-sep after-sep right-sep
\titlespacing*{\subsection}{0cm}{0cm}{0cm}[0em]

% }}}

% Sub-Sub-Section Title Settings {{{
\titleformat {\subsubsection}
	[hang]
	{\color{black}\small\bfseries}
	{}
	{0em}
	{
	\nolinenumbers
	\begin{subsubsection-box}
		\thesubsubsection. #1
	\end{subsubsection-box}
	}
[
\linenumbers
]

% left before-sep after-sep right-sep
\titlespacing{\subsubsection}{0cm}{0cm}{0cm}[0em]

% }}}

% }}}
% HYPERLINKS ------------------------------------------------------------- {{{

\usepackage{hyperref} % auto hyperlinks toc , refrences
                      % others can be manually specified

\hypersetup{
colorlinks = true,
linktoc    = all,
citecolor  = purple,
filecolor  = black,
linkcolor  = black,
urlcolor   = black
}


% }}}
% HEADER / FOOTER -------------------------------------------------------- {{{

\usepackage{fancyhdr} % allows for header and footer customizations
\pagestyle{fancy}     %
\fancyhf{}            %

\renewcommand{\headrulewidth}{0.2pt} % draw line at header
\lhead{\textit{\leftmark}}           % LEFT  : show section name at header
\rhead{\textit{\thepage}}            % RIGHT : Show page number
% \chead{ }

\renewcommand{\footrulewidth}{0pt} % draw line at footer
%\lfoot{\textit{Last Edited : \today}}
%\cfoot{center foot}
%\rfoot{\textit{tiwathia \thepage}}

\pagenumbering{arabic} % Specify type of number characters to use

% }}}
% COLORS ----------------------------------------------------------------- {{{

\usepackage  {xcolor, colortbl}

% Section Colors {{{

\definecolor {section-bg}         {HTML} { 767676}
\definecolor {subsection-bg}      {HTML} { aaaaaa}
\definecolor {subsubsection-bg}   {HTML} { dddddd}
\definecolor {section-font}       {RGB}  { 0,0,0}
\definecolor {subsection-font}    {RGB}  { 0,0,0}
\definecolor {subsubsection-font} {RGB}  { 0,0,0}

% }}}

% Table Colors {{{

\definecolor {table-topic}        {HTML} {aeb4b8}
\definecolor {table-subtopic}     {HTML} {c2c7ca}
\definecolor {table-subsubtopic}  {HTML} {d6d9db}

\definecolor {table-alternating-blue}    {HTML} {F2F3F4}
\definecolor {table-alternating-white}    {HTML} {FFFFFF} 
\definecolor {table-alternating-gray}  {RGB}  { 228,230,221}

\definecolor {cell-lightblue}     {HTML} { b2cce5}
\definecolor {cell-lightgray}     {HTML} { d8deda}
\definecolor {cell-lightorange}   {HTML} { F6C396}
\definecolor {cell-lightred}      {HTML} { F1A099}
\definecolor {cell-lightgreen}    {HTML} { C6DA7F}
\definecolor {cell-lightpurple}   {HTML} { CCB2E5}
\definecolor {cell-lightyellow}   {HTML} { FFF09A}

% }}}

% Tcolorbox Tables {{{

\definecolor {defn-bg}       {HTML} {F2F3F4}
\definecolor {defn-title}    {HTML} {A9A9A9}
\definecolor {defn-theword}  {HTML} {EDA72D}

\definecolor {note-bg}       {HTML} {F2F3F4}
\definecolor {note-theword}  {HTML} {E25A22}

\definecolor {table-bg}      {HTML} {F2F3F4}
\definecolor {table-title}   {HTML} {A9A9A9}
\definecolor {table-theword} {HTML} {9ACD32}

\definecolor {image-bg}      {HTML} {F2F3F4}
\definecolor {image-title}   {HTML} {A9A9A9}
\definecolor {image-theword} {HTML} {0067A5}

% }}}

\definecolor {gray-dark}          {RGB}  { 71,77,80}
\definecolor {gray-medium}        {RGB}  { 95,103,107}
\definecolor {gray-light}         {RGB}  { 228,230,221}

\definecolor {red-flame}       {HTML}  {E25822}
\definecolor {green-goethe}       {RGB}  {160,200,20}
\definecolor {green-goethe-light} {RGB}  {219,243,134}

% }}}
% TABLES & IMAGES -------------------------------------------------------- {{{

\usepackage{booktabs}   %
\usepackage{multirow}   %
\usepackage{tabularx}   % allow tables to stretch to page length
%\usepackage{longtable}  % allows tables to span pages
%\usepackage{ltablex}    % combination of longtable and tabularx
\usepackage{xtab} % allows page breaking tables inline

\usepackage{graphicx}   %
\usepackage{subcaption} %
\usepackage{wrapfig}    % wrap images around text
\usepackage{capt-of}    % define captions independent of figures
\usepackage{float}
\usepackage{varwidth}
\graphicspath{{images/}} % define folder path for images

% TCOLORBOX -------------------------------------------------------------- {{{

\usepackage[skins,breakable]{tcolorbox}
% skins allows use of enhanced options
% breakable allows breaking boxes between pages

% 	IMAGE TABLES (TCOLORBOX) {{{

\newtcolorbox{image-bg}[2][]{
	enhanced,
	colback           = image-bg,
	colframe          = image-bg,
	fonttitle         = \bfseries,
%	width             = 0.98\linewidth,
	beforeafter skip  = 0.5cm,
	drop fuzzy shadow = gray,
%	boxrule         = 0mm,
%	top             = 0mm,
%	bottom          = 0mm,
%	left            = 0mm,
%	right           = 0mm,
	title = #2,#1
}

\newtcolorbox{image-title}[2][]{
	enhanced,
	colback           = image-title,
	colframe          = image-title,
	fonttitle         = \bfseries,
	height            = 0.6cm,
	drop fuzzy shadow = gray,
	beforeafter skip  = 0pt,
	grow to left by   = 0.7cm,
	boxrule           = 0mm,
	top               = 0.5mm,
	bottom            = 0mm,
	left              = 1mm,
	right             = 0mm,
	sharp corners,
	title = #2,#1
}

\newtcolorbox{image-theword}{
	enhanced,
	colback           = image-theword,
	colframe          = image-theword,
	fonttitle         = \bfseries,
	drop fuzzy shadow = gray,
	width             = 1.8cm,
	height            = 0.5cm,
	beforeafter skip  = 0pt,
	grow to left by   = 0.7cm,
	boxrule           = 0mm,
	top               = 0.5mm,
	bottom            = 0mm,
	left              = 1mm,
	right             = 0mm,
	sharp corners,
}

\newtcolorbox{image-content}{
	enhanced,
	colback         = image-bg,
	colframe        = image-bg,
	fonttitle       = \bfseries,
%	enlarge top by  = -0.5cm,
	enlarge right by = 5cm,
	width           = \linewidth,
	boxrule         = 0mm,
	top             = 2mm,
	bottom          = 0mm,
	left            = 0mm,
	right           = 0mm,
%	show bounding box
}


\newtcolorbox{image-caption}[2][]{
	enhanced,
	colback           = defn-title,
	colframe          = defn-title,
	fonttitle         = \bfseries,
	halign            = center,
	height            = 0.6cm,
	drop fuzzy shadow = gray,
	before skip       = 5pt,
%	grow to right by  = 1.055\linewidth,
	enlarge bottom by = -1cm,
	boxrule           = 0mm,
	top               = 0.5mm,
	bottom            = 0mm,
	left              = 2mm,
	right             = 0mm,
	sharp corners,
	title = #2,#1
}


% }}}

% 	REGULAR TABLES (TCOLORBOX) {{{

\newtcolorbox{table-bg}[2][]{
	enhanced,
%	float,
%	breakable,
	colback           = table-bg,
	colframe          = table-bg,
	fonttitle         = \bfseries,
%	width             = 0.98\linewidth,
	beforeafter skip  = 0.5cm,
	drop fuzzy shadow = gray,
%	boxrule         = 0mm,
%	top             = 0mm,
%	bottom          = 0mm,
%	left            = 0mm,
%	right           = 0mm,
	title = #2,#1
}

\newtcolorbox{table-theword}{
	enhanced,
	colback           = table-theword,
	colframe          = table-theword,
	fonttitle         = \bfseries,
	drop fuzzy shadow = gray,
	width             = 1.5cm,
	height            = 0.5cm,
	beforeafter skip  = 0pt,
	grow to left by   = 0.7cm,
	boxrule           = 0mm,
	top               = 0.5mm,
	bottom            = 0mm,
	left              = 1mm,
	right             = 0mm,
	sharp corners,
}

\newtcolorbox{table-title}[2][]{
	enhanced,
	colback           = table-title,
	colframe          = table-title,
	fonttitle         = \bfseries,
	height            = 0.6cm,
	drop fuzzy shadow = gray,
	beforeafter skip  = 0pt,
	grow to left by   = 0.7cm,
	boxrule           = 0mm,
	top               = 0.5mm,
	bottom            = 0mm,
	left              = 1mm,
	right             = 0mm,
	sharp corners,
	title = #2,#1
}


\newtcolorbox{table-content}{
	enhanced,
	colback         = table-bg,
	colframe        = table-bg,
	fonttitle       = \bfseries,
	before skip = 0.5cm,
%	enlarge top by  = -0.5cm,
%	enlarge right by = 5cm,
	width           = \linewidth,
	boxrule         = 0mm,
	top             = 2mm,
	bottom          = 0mm,
	left            = 0mm,
	right           = 0mm
	%show bounding box
}

% }}}

% 	DEFINITION TABLE (TCOLORBOX) {{{

% \newtcbox[init options]{name}[number][default]{options}

\newtcolorbox{defn-bg}{
	enhanced,
	colback           = defn-bg,
	colframe          = defn-bg,
	fonttitle         = \bfseries,
	drop fuzzy shadow = gray,
	width             = 0.95\linewidth,
	beforeafter skip  = 0.5cm,
	arc is angular,
}

\newtcolorbox{defn-theword}{
	enhanced,
	colback           = defn-theword,
	colframe          = defn-theword,
	fonttitle         = \bfseries,
	drop fuzzy shadow = gray,
	width             = 2.5cm,
	height            = 0.5cm,
	beforeafter skip  = 0pt,
	grow to left by   = 0.7cm,
	boxrule           = 0mm,
	top               = 0.5mm,
	bottom            = 0mm,
	left              = 1mm,
	right             = 0mm,
	sharp corners,
}

\newtcolorbox{defn-title}[2][]{
	enhanced,
	colback           = defn-title,
	colframe          = defn-title,
	fonttitle         = \bfseries,
	height            = 0.6cm,
	drop fuzzy shadow = gray,
	beforeafter skip  = 0pt,
	grow to left by   = 0.7cm,
	boxrule           = 0mm,
	top               = 0.5mm,
	bottom            = 0mm,
	left              = 1mm,
	right             = 0mm,
	sharp corners,
	title = #2,#1
}

\newtcolorbox{defn-content}{
	enhanced,
	standard jigsaw,%allows transparency
	opacityback=1,
	fonttitle       = \bfseries,
%	enlarge top by  = -0.5cm,
%	enlarge right by = -5cm,
%	width           = 13cm,
	boxrule         = 0mm
%	top             = 1mm,
%	bottom          = 0mm,
%	left            = 4mm,
%	right           = 10mm,
%	show bounding box
}

% }}}

% 	NOTE TABLE (TCOLORBOX) {{{

\newtcolorbox{note-bg}{
	enhanced,
	colback           = note-bg,
	colframe          = note-bg,
	fonttitle         = \bfseries,
	drop fuzzy shadow = gray,
	width             = 0.95\linewidth,
	before skip  = 0.5cm,
	after skip  = 0.5cm,
	arc is angular,
}

\newtcolorbox{note-theword}{
	enhanced,
	colback           = note-theword,
	colframe          = note-theword,
	fonttitle         = \bfseries,
	drop fuzzy shadow = gray,
	width             = 0.7cm,
	height            = 0.5cm,
	beforeafter skip  = 0pt,
	grow to left by   = 0.7cm,
	boxrule           = 0mm,
	top               = 0.5mm,
	bottom            = 0mm,
	left              = 1mm,
	right             = 0mm,
	sharp corners,
}

\newtcolorbox{note-content}{
	enhanced,
	colback         = note-bg,
	colframe        = note-bg,
	fonttitle       = \bfseries,
%	enlarge top by  = -0.6cm,
%	enlarge left by = 1.5cm,
%	width           = 13cm,
%	boxrule         = 0mm,
%	top             = 0mm,
%	bottom          = 0mm,
%	left            = 0mm,
%	right           = 0mm
}


% }}}

% 	SECTION TITLES (TCOLORBOX) {{{

\newtcolorbox{section-box}{
	enhanced,
	colback          = section-bg,
	colframe         = section-bg,
	fonttitle        = \bfseries,
	width            = \linewidth,
	height           = 1cm,
	beforeafter skip = 0pt,
	sharp corners 
}

\newtcolorbox{subsection-box}{
	enhanced,
	colback           = subsection-bg,
	colframe          = subsection-bg,
	fonttitle         = \bfseries,
	width             = \linewidth,
	height = 0.6cm,
	sharp corners,
	beforeafter skip  = 0pt,
	boxrule           = 0mm,
	top               = 0.5mm,
	bottom            = 0mm,
	left              = 1mm,
	right             = 0mm
}

\newtcolorbox{subsubsection-box}{
	enhanced,
	colback   = subsubsection-bg,
	colframe  = subsubsection-bg,
	fonttitle = \bfseries,
	width     = \linewidth,
	height = 0.6cm,
	sharp corners,
	beforeafter skip  = 0pt,
	boxrule           = 0mm,
	top               = 0.5mm,
	bottom            = 0mm,
	left              = 1mm,
	right             = 0mm
}

% }}}




% }}}

% }}}
% CODE / CODE DISPLAY ---------------------------------------------------- {{{

\usepackage{listings}


\definecolor{codebackground}{RGB}{239,239,239}
\definecolor{codecomments}{RGB}{169,169,169}
\definecolor{codekeyword}{RGB}{249,38,114}
\definecolor{codestrings}{HTML}{ECE47E}
\definecolor{coderegular}{RGB}{39,40,34}

\lstset{
basicstyle       = \footnotesize\ttfamily,
backgroundcolor  = \color{codebackground},
commentstyle     = \color{codecomments}, % comment style
keywordstyle     = \color{codekeyword},  % keyword style
stringstyle      = \color{codestrings},  % string literal style
rulecolor        = \color{black},        % if not set, the frame-color may be changed on line-breaks
frame            = single,               % adds a frame around the code
basicstyle       = \footnotesize,        % the size of the fonts that are used for the code
keepspaces       = true,                 % keeps spaces in text, useful for keeping indentation of code
tabsize          = 2,                    % sets default tabsize
breaklines       = true,                 % sets automatic line breaking
captionpos       = b,                    % sets the caption-position to bottom
numbers          = left,
numberstyle      = \tiny,
numbersep        = 10pt,
frame            = tb,
columns          = fixed,
showstringspaces = false,
showtabs         = false,
keepspaces,
% escapeinside={\%*}{*)},  % if you want to add LaTeX within your code
framextopmargin=10pt,    % margin for the top background border
framexbottommargin=10pt, % margin for the bot background border
framexleftmargin=0pt,    % margin for the left background border
framexrightmargin=0pt    % margin for the right background border
}


% }}}
% MATH / GRAPHING  ------------------------------------------------------- {{{

\usepackage{amsmath} % basic math package
\usepackage{amssymb} % allows more math symbols
\usepackage{amsthm}  % allows custom therorem,defn,corll etc... definitions
\usepackage{mathrsfs} % some new fonts for math mode

\newtheorem{mydef}{DEFINITION}[section]
\newtheorem{myimage}{IMAGE}[section]
\newtheorem{mytable}{TABLE}[section]

% TIKZ -------------------------------------------------------------------- {{{
\usepackage{tikz}
% }}}

% }}}
% BIBLIOGRAPHY / REFERENCES / FOOTNOTES ---------------------------------- {{{

\usepackage[nottoc,numbib]{tocbibind} % to show references line in toc
\usepackage[super]{natbib}            % superscript the citations

%\usepackage[superscript,biblabel]{cite}
%\usepackage{cleveref}


\renewcommand{\thefootnote}{\roman{footnote}} % footnote style




% }}}
% TABLE OF CONTENTS ------------------------------------------------------ {{{

\usepackage{titletoc}
% margin from RHS
%\contentsmargin{1cm}

% \dottedcontents {section}[left]{above}{label-width}{leader-width}
%\dottedcontents{section}[1.8cm]{\bfseries}{3.2em}{1pc}
%\dottedcontents{subsection}[1.8cm]{}{3.2em}{1pc}
%\dottedcontents{subsubsection}[1.8cm]{}{2.8em}{1pc}


% }}}
% CUSTOM COMMANDS -------------------------------------------------------- {{{

\newcommand{\newpar}
{\par \vspace{0.3cm}}

% New Commands : Sectioning {{{

\newcommand{\sectionend}
{
\nolinenumbers
\begin{center}
%	\textbf{---------}
	$\blacksquare$
%	\textbf{---------}
\end{center}
\clearpage
\linenumbers
}

\newcommand{\subsectionend}
{
\nolinenumbers
%\begin{center}
%	\textbf{---------}
%	\hspace{0.2cm}
%	\textsection 
%	\hspace{0.05cm}
%	\thesubsection
%	\hspace{0.2cm}
%	\textbf{---------}
%\end{center}
\linenumbers
}

\newcommand{\subsubsectionend}
{
\nolinenumbers
%\begin{center}
%	\textbf{---------}
%	\hspace{0.2cm}
%	\textsection
%	\hspace{0.05cm}
%	\thesubsubsection
%	\hspace{0.2cm}
%	\textbf{---------}
%\end{center}
\linenumbers
}

% }}}

% New Commands : Symbols {{{

\newcommand{\bulletpoint}
{ $\bullet$  }

% }}}

% New Commands : Tcolorbox {{{

%	General {{{

\newcommand{\tcolorboxstart}
{
	\nolinenumbers
	\vspace{0.2cm}
	\centering
}

\newcommand{\tcolorboxend}
{
	\justifying
	\vspace{0.2cm}
	\linenumbers
}

% }}}

%	Definition {{{

\newcommand{\tcolorboxdefinition}[3]
{

\tcolorboxstart
\begin{defn-bg}

	\begin{defn-title}[width=7cm]{}
	{
		\normalsize \textbf{\textit{#1}}
	}
	\end{defn-title}

	\begin{defn-theword}
	{
		\footnotesize
		\begin{mydef} #2
%		\label{def:{#2}}
		\end{mydef}
	}
	\end{defn-theword}


	\begin{defn-content}

	\justify
	#3

	\end{defn-content}

\end{defn-bg}
\tcolorboxend
}

% }}}

%	Note {{{

\newcommand{\tcolorboxnote}[1]
{

\tcolorboxstart
\begin{note-bg}

	\begin{note-theword}
		{\footnotesize \textbf{NOTE} }
	\end{note-theword}

	\begin{note-content} \justifying

		#1

	\end{note-content}

\end{note-bg}
\tcolorboxend
}




% }}}

%	Table {{{


\newcommand{\tcolorboxtable}[5]
{
\tcolorboxstart
\begin{table-bg}#3{}

	\begin{table-title}[width=6.5cm]{}
		\captionsetup{labelformat=empty}
		\captionof{table}{#1}
	\end{table-title}

	\begin{table-theword}
		\footnotesize
		\begin{mytable}
		#2
		\end{mytable}
	\end{table-theword}

	\begin{table-content}
	\begin{tabularx}{\textwidth}{#4}

		#5
		
	\end{tabularx}
	\end{table-content}

\end{table-bg}
\tcolorboxend
}

% }}}

%	Image {{{

\newcommand{\tcolorboxfigure}[4]
{
\tcolorboxstart
\begin{image-bg}[width=\linewidth]{}

	\begin{image-title}[width=5cm]{}
		\captionsetup{labelformat=empty}
		\captionof{figure}{#1}
	\end{image-title}

	\begin{image-theword}
		\footnotesize
		\begin{myimage}
		#2
		\end{myimage}
	\end{image-theword}

	\begin{image-content}
		\includegraphics[width=\linewidth]{#3}
		\href{#4}{Source}
	\end{image-content}

\end{image-bg}
\tcolorboxend
}

% }}}

% }}}

% New Commands : Tables {{{

\newcommand{\tabularxtable}[3]
{

	\vspace{0.5cm}
	\nolinenumbers

	\begin{tabularx}{#1}{#2}
		#3
	\end{tabularx}

	\linenumbers
	\vspace{0.5cm}
}

\newcommand{\xtabulartable}[2]
{

	\vspace{0.25cm}
	\nolinenumbers

	\begin{xtabular}{#1}
		#2
	\end{xtabular}

	\vspace{0.25cm}
	\linenumbers
}







% }}}

% New Commands : References {{{

\newcommand{\refsec}[1]
{
	\hyperref[sec:#1]
	{
		(\textsection~\ref{sec:#1})
	}
}

\newcommand{\refssec}[1]
{
	\hyperref[sec:#1]
	{
		(\textsection~\ref{ssec:#1})
	}
}

\newcommand{\refsssec}[1]
{
	\hyperref[sec:#1]
	{
		(\textsection~\ref{sssec:#1})
	}
}

\newcommand{\refdef}[1]
{
	\hyperref[def:#1]
	{
		\textit{(Def.~\ref{def:#1})}
	}
}

\newcommand{\reffig}[1]
{
	\hyperref[fig:#1]
	{
		(Fig.~\ref{fig:#1})
	}
}

\newcommand{\reftable}[1]
{
	\hyperref[table:#1]
	{
		(Table.~\ref{table:#1})
	}
}
% }}}

% }}}
% TESTING ---------------------------------------------------------------- {{{

\usepackage{lipsum}    % generates filler text
\usepackage{blindtext} % generates non-latin filler text

% }}}

\usetikzlibrary{arrows} 
\usetikzlibrary{positioning}



% }}}

% ============================================================================ 
\begin{document}
% ----------------------------------------------------------------------------- 
% TOC & SETUP {{{
\raggedbottom
%\onecolumn

%\tableofcontents
%\pagebreak

%\listoftables
\clearpage
\twocolumn
%\justifying


%\linenumbers

% }}}
% ----------------------------------------------------------------------------- 


% TABULARX TABLE : definite_articles {{{

\textbf {Definite Articles}
\tabularxtable
{0.99\linewidth}
{l|XXXX}
{

		&
		\cellcolor{table-subtopic} \textbf{\textit{MAS.}}  &
		\cellcolor{table-subtopic} \textbf{\textit{NEU.}}  &
		\cellcolor{table-subtopic} \textbf{\textit{FEM.}}  &
		\cellcolor{table-subtopic} \textbf{\textit{PLU.}} \\

		\midrule

		\cellcolor{table-subtopic} \textbf{\textit{NOM.}} &
		\cellcolor{cell-lightpurple}  der            &
		\cellcolor{cell-lightorange}  das            &
		\cellcolor{cell-lightblue}    die            &
		\cellcolor{cell-lightblue}    die \\

		\cellcolor{table-subtopic} \textbf{\textit{ACC.}} &
		\cellcolor{cell-lightgreen}   den            &
		\cellcolor{cell-lightorange}  das            &
		\cellcolor{cell-lightblue}    die            &
		\cellcolor{cell-lightblue}    die \\

		\cellcolor{table-subtopic} \textbf{\textit{DAT.}} &
		\cellcolor{cell-lightred}    dem             &
		\cellcolor{cell-lightred}    dem             &
		\cellcolor{cell-lightpurple} der             &
		\cellcolor{cell-lightgreen}  den \\

		\cellcolor{table-subtopic} \textbf{\textit{GEN.}} &
		\cellcolor{cell-lightyellow} des               &
		\cellcolor{cell-lightyellow} des               &
		\cellcolor{cell-lightpurple} der               &
		\cellcolor{cell-lightpurple} der \\



}

% }}} End TABLE : definite_articles

% TABULARX TABLE : indefinite_articles {{{
\textbf {Indefinite Articles}
\tabularxtable
{0.99\linewidth}
{l|XXXX}
{

		&
		\cellcolor{table-subtopic} \textbf{\textit{MAS.}} &
		\cellcolor{table-subtopic} \textbf{\textit{NEU.}}  &
		\cellcolor{table-subtopic} \textbf{\textit{FEM.}}  &
		\cellcolor{table-subtopic} \textbf{\textit{PLU.}}\\

\midrule

\cellcolor{table-subtopic} \textbf{\textit{NOM.}} &
\cellcolor{cell-lightpurple}  ein            &
\cellcolor{cell-lightorange}  ein            &
\cellcolor{cell-lightblue} eine              &
\cellcolor{table-bg} - \\

\cellcolor{table-subtopic} \textbf{\textit{ACC.}} &
\cellcolor{cell-lightgreen} einen            &
\cellcolor{cell-lightorange}  ein            &
\cellcolor{cell-lightblue}  eine             &
\cellcolor{table-bg} - \\

\cellcolor{table-subtopic} \textbf{\textit{DAT.}} &
\cellcolor{cell-lightred} einem              &
\cellcolor{cell-lightred} einem              &
\cellcolor{cell-lightpurple} einer           &
\cellcolor{table-bg} - \\

\cellcolor{table-subtopic} \textbf{\textit{GEN.}} &
\cellcolor{cell-lightyellow} eines           &
\cellcolor{cell-lightyellow} eines           &
\cellcolor{cell-lightpurple} einer           &
\cellcolor{table-bg} - \\



}

% }}} End TABLE : indefinite_articles

% TABULARX TABLE : adjective_endings_definite {{{

\textbf {Adjective Endings : Definite}
\tabularxtable
{0.99\linewidth}
{l|XXXX}
{

&
\cellcolor{table-subtopic} \textbf{\textit{MAS.}} &
\cellcolor{table-subtopic} \textbf{\textit{NEU.}}  &
\cellcolor{table-subtopic} \textbf{\textit{FEM.}}  &
\cellcolor{table-subtopic} \textbf{\textit{PLU.}} \\
\midrule

\cellcolor{table-subtopic} \textbf{\textit{NOM.}} &
\cellcolor{cell-lightred}  -e                &
\cellcolor{cell-lightred}  -e                &
\cellcolor{cell-lightred}  -e                &
\cellcolor{cell-lightblue} -en \\

\cellcolor{table-subtopic} \textbf{\textit{ACC.}} &
\cellcolor{cell-lightblue} -en               &
\cellcolor{cell-lightred}  -e                &
\cellcolor{cell-lightred}  -e                &
\cellcolor{cell-lightblue} -en \\

\cellcolor{table-subtopic} \textbf{\textit{DAT.}} &
\cellcolor{cell-lightblue} -en               &
\cellcolor{cell-lightblue} -en               &
\cellcolor{cell-lightblue} -en               &
\cellcolor{cell-lightblue} -en \\

\cellcolor{table-subtopic} \textbf{\textit{GEN.}} &
\cellcolor{cell-lightblue} -en               &
\cellcolor{cell-lightblue} -en               &
\cellcolor{cell-lightblue} -en               &
\cellcolor{cell-lightblue} -en \\




}

% }}} End TABLE : adjective_endings_definite

% TABULARX TABLE : adjective_endings_indefinite {{{

\textbf {Adjective Endings : Indefinite}
\tabularxtable
{0.99\linewidth}
{l|XXXX}
{

&
\cellcolor{table-subtopic} \textbf{\textit{MAS.}} &
\cellcolor{table-subtopic} \textbf{\textit{NEU.}}  &
\cellcolor{table-subtopic} \textbf{\textit{FEM.}}  &
\cellcolor{table-subtopic} \textbf{\textit{PLU.}} \\
\midrule

\cellcolor{table-subtopic} \textbf{\textit{NOM.}} &
\cellcolor{cell-lightgreen}   -er             &
\cellcolor{cell-lightorange} -es             &
\cellcolor{cell-lightred}    -e              &
\cellcolor{cell-lightblue}   -en \\

\cellcolor{table-subtopic} \textbf{\textit{ACC.}} &
\cellcolor{cell-lightblue}   -en             &
\cellcolor{cell-lightorange} -es             &
\cellcolor{cell-lightred}    -e              &
\cellcolor{cell-lightblue}   -en \\

\cellcolor{table-subtopic} \textbf{\textit{DAT.}} &
\cellcolor{cell-lightblue} -en               &
\cellcolor{cell-lightblue} -en               &
\cellcolor{cell-lightblue} -en               &
\cellcolor{cell-lightblue} -en \\

\cellcolor{table-subtopic} \textbf{\textit{GEN.}} &
\cellcolor{cell-lightblue} -en               &
\cellcolor{cell-lightblue} -en               &
\cellcolor{cell-lightblue} -en               &
\cellcolor{cell-lightblue} -en \\




}

% }}} End TABLE : adjective)endings_indefinite

% TABULARX TABLE : adjective_endings_no_article {{{

\textbf {Adjective Endings : No Article}
\tabularxtable
{0.99\linewidth}
{l|XXXX}
{

&
\cellcolor{table-subtopic} \textbf{\textit{MAS.}} &
\cellcolor{table-subtopic} \textbf{\textit{NEU.}}  &
\cellcolor{table-subtopic} \textbf{\textit{FEM.}}  &
\cellcolor{table-subtopic} \textbf{\textit{PLU.}} \\
\midrule

\cellcolor{table-subtopic} \textbf{\textit{NOM.}} &
\cellcolor{cell-lightgreen}   -er             &
\cellcolor{cell-lightorange} -es             &
\cellcolor{cell-lightred}    -e              &
\cellcolor{cell-lightred}   -e\\

\cellcolor{table-subtopic} \textbf{\textit{ACC.}} &
\cellcolor{cell-lightblue}   -en             &
\cellcolor{cell-lightorange} -es             &
\cellcolor{cell-lightred}    -e              &
\cellcolor{cell-lightred}   -e \\

\cellcolor{table-subtopic} \textbf{\textit{DAT.}} &
\cellcolor{cell-lightyellow} -em               &
\cellcolor{cell-lightyellow} -em               &
\cellcolor{cell-lightgreen} -er              &
\cellcolor{cell-lightblue} -en \\

\cellcolor{table-subtopic} \textbf{\textit{GEN.}} &
\cellcolor{cell-lightblue} -en               &
\cellcolor{cell-lightblue} -en               &
\cellcolor{cell-lightgreen} -er               &
\cellcolor{cell-lightgreen} -er \\




}

% }}} End TABLE : adjective_endings no article

\pagebreak

% TABULARX TABLE : pronouns_realtive {{{

\textbf {Pronouns: Relative}
\tabularxtable
{0.99\linewidth}
{l|XXXX}
{

		&
		\cellcolor{table-subtopic} \textbf{\textit{MAS.}}  &
		\cellcolor{table-subtopic} \textbf{\textit{NEU.}}  &
		\cellcolor{table-subtopic} \textbf{\textit{FEM.}}  &
		\cellcolor{table-subtopic} \textbf{\textit{PLU.}} \\

		\midrule

		\cellcolor{table-subtopic} \textbf{\textit{NOM.}} &
		\cellcolor{cell-lightpurple}  der            &
		\cellcolor{cell-lightorange}  das            &
		\cellcolor{cell-lightblue}    die            &
		\cellcolor{cell-lightblue}    die \\

		\cellcolor{table-subtopic} \textbf{\textit{ACC.}} &
		\cellcolor{cell-lightgreen}   den            &
		\cellcolor{cell-lightorange}  das            &
		\cellcolor{cell-lightblue}    die            &
		\cellcolor{cell-lightblue}    die \\

		\cellcolor{table-subtopic} \textbf{\textit{DAT.}} &
		\cellcolor{cell-lightred}    dem             &
		\cellcolor{cell-lightred}    dem             &
		\cellcolor{cell-lightpurple} der             &
		\cellcolor{cell-lightgreen}  denen \\

		\cellcolor{table-subtopic} \textbf{\textit{GEN.}} &
		\cellcolor{cell-lightyellow} dessen               &
		\cellcolor{cell-lightyellow} dessen               &
		\cellcolor{cell-lightpurple} deren               &
		\cellcolor{cell-lightpurple} deren \\

}

% }}} End TABLE : Pronouns realtive

% TABULARX TABLE : pronouns_possessive {{{

\textbf {Pronouns: Possessive}
\tabularxtable
{0.99\linewidth}
{l|XXXXXXXX}
{

		&
		\cellcolor{table-subtopic}  ich                   &
		\cellcolor{table-subtopic}  du                    &
		\cellcolor{table-subtopic}  er                    &
		\cellcolor{table-subtopic}  sie                   &
		\cellcolor{table-subtopic}  es                    &
		\cellcolor{table-subtopic}  wir                   &
		\cellcolor{table-subtopic}  ihr                   &
		\cellcolor{table-subtopic}  sie \\

		\cellcolor{table-subtopic} \textbf{\textit{NOM.}} &
		\cellcolor{white}  mein                            &
		\cellcolor{white}  dein                             &
		\cellcolor{white}  sein                             &
		\cellcolor{white}  ihr                            &
		\cellcolor{white}  sein                             &
		\cellcolor{white}  unser                            &
		\cellcolor{white}  euer                            &
		\cellcolor{white}  ihr \\

		\cellcolor{table-subtopic} \textbf{\textit{ACC.}} & \\

		\cellcolor{table-subtopic} \textbf{\textit{DAT.}} & \\
		\cellcolor{table-subtopic} \textbf{\textit{GEN.}} & \\

}

% }}} End TABLE : Pronouns Possessive

% TABULARX TABLE : pronouns_reflexive {{{

\textbf {Pronouns: Reflexive}
\tabularxtable
{0.99\linewidth}
{l|XXXXXXXX}
{

		&
		\cellcolor{table-subtopic}  ich                   &
		\cellcolor{table-subtopic}  du                    &
		\cellcolor{table-subtopic}  er                    &
		\cellcolor{table-subtopic}  sie                   &
		\cellcolor{table-subtopic}  es                    &
		\cellcolor{table-subtopic}  wir                   &
		\cellcolor{table-subtopic}  ihr                   &
		\cellcolor{table-subtopic}  sie \\

		\cellcolor{table-subtopic} \textbf{\textit{NOM.}} &
		\cellcolor{white}  mein                           &
		\cellcolor{white}  dein                           &
		\cellcolor{white}  sein                           &
		\cellcolor{white}  ihr                            &
		\cellcolor{white}  sein                           &
		\cellcolor{white}  unser                          &
		\cellcolor{white}  euer                           &
		\cellcolor{white}  ihr \\

		\cellcolor{table-subtopic} \textbf{\textit{ACC.}} & \\
		\cellcolor{table-subtopic} \textbf{\textit{DAT.}} & \\
		\cellcolor{table-subtopic} \textbf{\textit{GEN.}} & \\

}

% }}} End TABLE : Pronouns reflexive

% TABULARX TABLE : verbs_vorgangspassiv {{{

\textbf {Verbs : Vorgangspassiv}
\tabularxtable
{0.95\linewidth}
{lllllX}
{


\cellcolor{table-title} \textbf{Präs}  & : & Ich & \textbf{\textcolor{green-goethe}{werde}} & informiert & \\
\cellcolor{table-title} \textbf{Prät}  & : & Ich & \textbf{\textcolor{green-goethe}{wurde}} & informiert & \\
\cellcolor{table-title} \textbf{Perf}  & : & Ich & \textbf{\textcolor{green-goethe}{bin}} & informiert & worden \\
\cellcolor{table-title} \textbf{Plqu}  & : & Ich &\textbf{\textcolor{green-goethe}{war}} & informiert & worden \\
\cellcolor{table-title} \textbf{Mod} & : & Ich & \textbf{\textcolor{green-goethe}{kann}}  & informiert & werden \\
\cellcolor{table-title} \textbf{Fut} & : & Ich & \textbf{\textcolor{green-goethe}{werde}} & informiert & werden \\

}

% }}} End TABULARX TABLE : Vorgangspassiv

% TABULARX TABLE : verbs_modal {{{
\textbf {Verbs : Modal}
\tabularxtable
{0.99\linewidth}
{X|X|X|X|X|X|X}
{

	&
	\cellcolor{gray-light} \textbf{ich} &
	\cellcolor{gray-light} \textbf{du} &
	\cellcolor{gray-light} \textbf{er/sie/es} &
	\cellcolor{gray-light} \textbf{wir} &
	\cellcolor{gray-light} \textbf{ihr} &
	\cellcolor{gray-light} \textbf{sie/Sie} \\

	\midrule

	\cellcolor{gray-light} \textbf{\textit{dürfen}} &
	\cellcolor{white} darf                          &
	\cellcolor{white} darfst                        &
	\cellcolor{white} darf                          &
	\cellcolor{white} dürfen                        &
	\cellcolor{white} dürft                         &
	\cellcolor{white} dürfen \\

	\cellcolor{gray-light} \textbf{\textit{können}} &
	\cellcolor{white} kann                          &
	\cellcolor{white} kannst                        &
	\cellcolor{white} kann                          &
	\cellcolor{white} können                        &
	\cellcolor{white} könnt                         &
	\cellcolor{white} können\\

	\cellcolor{gray-light} \textbf{\textit{mögen}} &
	\cellcolor{white} mag                          &
	\cellcolor{white} magst                        &
	\cellcolor{white} mag                          &
	\cellcolor{white} mögen                        &
	\cellcolor{white} mögt                         &
	\cellcolor{white} mögen\\

	\cellcolor{gray-light} \textbf{\textit{wollen}} &
	\cellcolor{white} will                          &
	\cellcolor{white} willst                        &
	\cellcolor{white} will                          &
	\cellcolor{white} wollen                        &
	\cellcolor{white} wollt                         &
	\cellcolor{white} wollen\\

	\cellcolor{gray-light} \textbf{\textit{sollen}} &
	\cellcolor{white} soll                          &
	\cellcolor{white} sollst                        &
	\cellcolor{white} soll                          &
	\cellcolor{white} sollen                        &
	\cellcolor{white} sollt                         &
	\cellcolor{white} sollen\\

	\cellcolor{gray-light} \textbf{\textit{mussen}} &
	\cellcolor{white} muss                          &
	\cellcolor{white} musst                         &
	\cellcolor{white} muss                          &
	\cellcolor{white} müssen                        &
	\cellcolor{white} müsst                         &
	\cellcolor{white} müssen\\

}

% }}} End TABLE : verbs modal

% TABULARX TABLE : verbs_konjunktiv_ii {{{
\textbf {Verbs : Konjunktiv II}
\tabularxtable
{0.99\linewidth}
{X|X|X|X|X|X|X}
{

	&
	\cellcolor{gray-light} \textbf{ich} &
	\cellcolor{gray-light} \textbf{du} &
	\cellcolor{gray-light} \textbf{er/sie/es} &
	\cellcolor{gray-light} \textbf{wir} &
	\cellcolor{gray-light} \textbf{ihr} &
	\cellcolor{gray-light} \textbf{sie/Sie} \\

	\midrule

	\cellcolor{gray-light} \textbf{\textit{haben}} &
	\cellcolor{white} hätte                        &
	\cellcolor{white} hättest                      &
	\cellcolor{white} hätte                        &
	\cellcolor{white} hätten                       &
	\cellcolor{white} hättet                       &
	\cellcolor{white} hätten \\

	\cellcolor{gray-light} \textbf{\textit{sein}} &
	\cellcolor{white} wäre                        &
	\cellcolor{white} wär(e)st                    &
	\cellcolor{white} wäre                        &
	\cellcolor{white} wären                       &
	\cellcolor{white} wäre(e)t                    &
	\cellcolor{white} wären \\

	\cellcolor{gray-light} \textbf{\textit{werden}} &
	\cellcolor{white} würde                         &
	\cellcolor{white} würdest                       &
	\cellcolor{white} würde                         &
	\cellcolor{white} würden                        &
	\cellcolor{white} würdet                        &
	\cellcolor{white} würden \\

	\cellcolor{gray-light} \textbf{\textit{können}} &
	\cellcolor{white} könnte                        &
	\cellcolor{white} könntest                      &
	\cellcolor{white} könnte                        &
	\cellcolor{white} könnten                       &
	\cellcolor{white} könntet                       &
	\cellcolor{white} können \\

	\cellcolor{gray-light} \textbf{\textit{sollen}} &
	\cellcolor{white} sollte                        &
	\cellcolor{white} solltest                      &
	\cellcolor{white} sollte                        &
	\cellcolor{white} sollten                       &
	\cellcolor{white} solltet                       &
	\cellcolor{white} sollten \\


}

% }}} End TABLE : Verbs Konjunktiv II

% TABULARX TABLE : verbs_positions {{{

\tabularxtable
{0.99\linewidth}
{llX}
{

	falls   & \\
	wenn    & \\
	dass    & \\
	dadurch & \\
	indem   & \\
	wodurch & \\
	da      & \\
	ob    & \\
	sofern    & \\

	aber    & \\
	denn    & \\
	und    & \\
	sondern    & \\
	oder    & \\


	sodass    & \\
	so \ldots dass    & \\

	dermaßen \ldots dass    & \\
	derart \ldots dass    & \\
	zu \ldots als dass    & \\

	weshalb    & \\
	wegen    & \\
	weswegen    & \\
	deswegen    & \\

	trotz    & \\
	trotzdem    & \\



	obwohl    & \\
	zumal    & \\
	auch wenn    & \\
	wobei    & \\
	ungeachtet dessen , dass  & \\

	während    & \\
	nachdem    & \\
	bevor    & \\
	ehe   & \\
	bis    & \\
	seit(dem)    & \\
	wenn    & \\
	als    & \\
	solange    & \\
	sobald    & \\
	sowie    & \\
	sooft    & \\
	    & \\
	    & \\
	    & \\
	    & \\
	    & \\
	    & \\
	    & \\
	    & \\

und 	and
aber 	but
denn 	because
oder 	or
sondern 	but (as in but rather)
beziehungsweise 	or, or more precisely
doch 	but, however
jedoch 	but, however
allein (rare expression) 	but unfortunately
bevor 	before
nachdem 	after
ehe 	before
seit, seitdem 	since (indicating time, not a causality)
während 	while, during, whereas
als 	when (in describing past events)
wenn 	when (describing present and future), if, whenever
wann 	when (for questions only)
bis 	until, by
obwohl 	although
als ob, als wenn, als 	as if
sooft 	as often as (whenever)
sobald 	as soon as
solange 	as long as
da 	because
indem 	by … -ing
weil 	because
ob 	whether*, if (*only use when you could say “whether” in English as well)
falls 	in case, if
wenn 	if, when
um … zu 	in order to
dass 	that
sodass 	so that
damit 	so that
}

% }}} End TABLE : verbs_positions

% TABULARX TABLE : konnektoren_zweiteilige {{{

\tabularxtable
{0.99\linewidth}
{llX}
{
entweder … oder 	either … or
sowohl … als auch 	both … and
weder … noch 	neither … nor
einerseits, … andererseits 	on the one hand … on the other hand
mal … mal 	sometimes … sometimes
teils … teils 	partly … partly

}

% }}} End TABLE : Konnektoren zweiteilige

% TABULARX TABLE : passive {{{

\tabularxtable
{0.99\linewidth}
{llX}
{


}

% }}} End TABLE : Passive

% TABULARX TABLE : prepositions {{{

\tabularxtable
{0.99\linewidth}
{XXX}
{

	Deutsch & Fall & Englisch  \\
		% Akkusativ
		\midrule


		\cellcolor{white} bis &
		\cellcolor{white} until , up-to \\

		\cellcolor{white} durch &
		\cellcolor{white} through \\

		\cellcolor{white} entlang &
		\cellcolor{white} along \\

		\cellcolor{white} für &
		\cellcolor{white} for \\

		\cellcolor{white} gegen &
		\cellcolor{white} against / opposite \\

		\cellcolor{white} ohne &
		\cellcolor{white}  without\\

		\cellcolor{white} um \ldots herum &
		\cellcolor{white}  around\\

		\cellcolor{white} hinter &
		\cellcolor{white}  behind\\

		\cellcolor{white} in &
		\cellcolor{white} in , inside \\

		\cellcolor{white} neben &
		\cellcolor{white} next to , beside \\

		\cellcolor{white} über &
		\cellcolor{white} over , above \\

		\cellcolor{white} unter &
		\cellcolor{white} among ,under, below \\


		\cellcolor{white} vor &
		\cellcolor{white} ahead of, in front of \\

	\cellcolor{white} zwischen &
		\cellcolor{white}  between\\



		\cellcolor{white} wider &
		\cellcolor{white} against \\

		\midrule

		% dative

		\cellcolor{white} von &
		\cellcolor{white} of / from \\

		\cellcolor{white} zu &
		\cellcolor{white} to / for \\

		\cellcolor{white} seit &
		\cellcolor{white} since \\

		\cellcolor{white} nach &
		\cellcolor{white} towards / to / past (time) / after\\

		\cellcolor{white} aus &
		\cellcolor{white} out of / from / made of \\

		\cellcolor{white} mit &
		\cellcolor{white} with\\

		\cellcolor{white} bei &
		\cellcolor{white}  with / by\\

		\cellcolor{white} außer &
		\cellcolor{white}  besides\\

		\cellcolor{white} gegenüber &
		\cellcolor{white} against \\

		% wechsel
		% akksativ : major movement
		% dativ : minimal / no movement
		% prep is same only the noun changes

		\cellcolor{white} zwischen &
		\cellcolor{white} between\\


		\cellcolor{white} unter &
		\cellcolor{white} under\\

		\cellcolor{white} neben &
		\cellcolor{white} next to\\

		\cellcolor{white} über &
		\cellcolor{white} above\\

		\cellcolor{white} hinter &
		\cellcolor{white} behind \\

		\cellcolor{white} an &
		\cellcolor{white} at \\

		\cellcolor{white} vor &
		\cellcolor{white} in front of \\

		\cellcolor{white} auf &
		\cellcolor{white} on \\

		\cellcolor{white} in &
		\cellcolor{white} in \\



}

% }}} End TABLE : Prepositions

% Konjunktiv I : sein , haben , können {{{

\textbf{Konjunktiv I : Sein , Haben , Können}
\begin{table}[htpb]
	\centering
%	\caption{Konjunktiv I : Sein , Haben , Können}
%	\label{tab:label}
	\begin{tabular}{llll}

ich  & sei
     & hab\textcolor{cell-lightgreen}{\textbf{e}} \textcolor{cell-lightred}{\textbf{(hätte)}}
     & könn\textcolor{cell-lightgreen}{\textbf{e}}
	 \\


du & sei\textcolor{cell-lightgreen}{\textbf{est}}
   & hab\textcolor{cell-lightgreen}{\textbf{est}} \textcolor{cell-lightred}{\textbf{(hättest)}}
& könn\textcolor{cell-lightgreen}{\textbf{est}}
\\


er & sei
   & hab\textcolor{cell-lightgreen}{\textbf{e}}
   & könn\textcolor{cell-lightgreen}{\textbf{e}}
\\

wir & sei\textcolor{cell-lightgreen}{\textbf{en}}
    & hab\textcolor{cell-lightgreen}{\textbf{en}} \textcolor{cell-lightred}{\textbf{(hätten)}}
    & könn\textcolor{cell-lightgreen}{\textbf{en}} \textcolor{cell-lightred}{\textbf{(könnten)}}
\\

ihr & sei\textcolor{cell-lightgreen}{\textbf{et}}
    & hab\textcolor{cell-lightgreen}{\textbf{et}} \textcolor{cell-lightred}{\textbf{(hättet)}}
    & könn\textcolor{cell-lightgreen}{\textbf{est}}
\\

Sie & sei\textcolor{cell-lightgreen}{\textbf{en}}
    & hab\textcolor{cell-lightgreen}{\textbf{en}} \textcolor{cell-lightred}{\textbf{(hätten)}}
    & könn\textcolor{cell-lightgreen}{\textbf{en}} \textcolor{cell-lightred}{\textbf{(könnten)}}
\\

	\end{tabular}
\end{table}
% }}}

% Konjunktiv I : regular verbs {{{

\textbf{Konjunktiv I : Regular Verbs}
\begin{table}[htpb]
	\centering
%	\caption{Konjunktiv I : Sein , Haben , Können}
%	\label{tab:label}
	\begin{tabular}{llll}

ich & brauch\textcolor{cell-lightgreen}{\textbf{e}}
&  \textcolor{cell-lightred}{\textbf{Konj II}}
& würd\textcolor{cell-lightgreen}{\textbf{e}} brauchen

\\


du & brauch\textcolor{cell-lightgreen}{\textbf{est}}
& \textcolor{cell-lightred}{\textbf{Konj II}}
& würd\textcolor{cell-lightgreen}{\textbf{est}} brauchen

\\


er & brauch\textcolor{cell-lightgreen}{\textbf{e}}
& 
&
\\

wir & brauch\textcolor{cell-lightgreen}{\textbf{en}}
& \textcolor{cell-lightred}{\textbf{Konj II}}
& würd\textcolor{cell-lightgreen}{\textbf{en}} brauchen

\\

ihr & brauch\textcolor{cell-lightgreen}{\textbf{et}}
& \textcolor{cell-lightred}{\textbf{Konj II}}
& würd\textcolor{cell-lightgreen}{\textbf{et}} brauchen

\\

Sie & brauch\textcolor{cell-lightgreen}{\textbf{en}} 
& \textcolor{cell-lightred}{\textbf{Konj II}}
& würd\textcolor{cell-lightgreen}{\textbf{en}} brauchen
\\

	\end{tabular}
\end{table}


% }}}


trennbar

immer untrennbar
miss , ent , be , ge , zer , ver , er

both tren and untrenn
durch , hinter, über , um , unter , voll , wider , wieder






% ----------------------------------------------------------------------------- 
% BIBLIOGRAPHY & FIGURE LISTS {{{

% https://www.fluentu.com/blog/german/german-adjectival-nouns/

\onecolumn
\nolinenumbers
\bibliographystyle{unsrt}
\bibliography{bibliography/references}


% }}}
% -----------------------------------------------------------------------------
\end{document}
% =============================================================================
% - EOF - EOF - EOF - EOF - EOF - EOF - EOF - EOF - EOF - EOF - EOF - EOF -
% =============================================================================

