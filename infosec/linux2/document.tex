% =============================================================================
% DOCUMENT CLASS , ENCODING , TITLE -------------------------------------- {{{

\documentclass[a4paper,twocolumn,10pt]{article}
\setlength{\columnsep}{20pt}
%\setlength{\columnseprule}{1pt}
\usepackage[utf8]{inputenc}
\usepackage[english]{babel}

\title{This is title for this document}
\author{John Doestar}
\date{\today}

% }}}
% FONTS ------------------------------------------------------------------ {{{

\usepackage{courier}     % courier font
%\usepackage{fontawesome} % font awesome glyphs
\usepackage{setspace}    % inter line spacing
\singlespacing           % set spacing to single spaced

% \onehalfspacing
% \doublespacing
% \setstretch{1.1}


% }}}
% PAGE FORMAT ------------------------------------------------------------ {{{

\usepackage[document]{ragged2e} % Text alignment package
\usepackage{enumitem}           %
\usepackage{geometry}           %
\geometry{
	a4paper,
	total = {170mm,257mm},
	top=20mm,
	left=20mm,
	right=20mm,
	bottom=20mm
}


\usepackage[switch]{lineno} % switch allows line numbers on alternating sides
%\usepackage[switch,displaymath,mathlines]{lineno}
%\modulolinenumbers[5]
\linenumberfont{\normalfont\large\sffamily}
\renewcommand{\thelinenumber}{\color{gray}\arabic{linenumber}}

\setlength\linenumbersep{0.5cm}


% }}}
% TEXT FORMAT ------------------------------------------------------------ {{{

\usepackage[activate={true,nocompatibility},
final,
tracking=true,
kerning=true,
spacing=true,
factor=1100,
stretch=10,
shrink=10]{microtype}
% prevents a certiain amount of overfull hbox badness
% helps with other margin stuff

\usepackage{verbatim}
\makeatletter
\newcommand{\verbatimfont}[1]{\def\verbatim@font{#1}}%
\makeatother%\verbatimfont{courier}

\usepackage{type1cm}  % Allows font resizing
\usepackage{lettrine} % Allows font calligraphy (enlarge first char)
\renewcommand{\LettrineTextFont}{\rmfamily}

\renewcommand{\baselinestretch}{1} % Line Spacing

%1.0 	single spacing
%1.3 	one-and-a-half spacing
%1.6 	double spacing 

\setlength\parskip{1em} % Space Between Paragraphs
\setlength\parindent{0em} % Indent Spacing 

% }}}
% TITLE FORMAT (TITLESEC) ------------------------------------------------ {{{

\usepackage[export]{adjustbox}
\usepackage{changepage}
\usepackage[compact,explicit]{titlesec} % Allows customization of section head

% compact : reduces spaces before and after sections
% explicit : allows for expicit positioning of title statement with #1

% title_format : command,shape,format,label,sep,before,after
% title-brackets : {}[]{}{}{}{}[] , basically only shape and aftercode have []

% Section Title Settings {{{
\titleformat {\section}
	[hang]
	{\Large\bfseries}
	{}
	{0em}
	{
	\nolinenumbers
	\vspace{-0.5cm}
	\begin{section-box}
		\color{white} \thesection. #1
	\end{section-box}
	}
[
\linenumbers
]
% left before-sep after-sep right-sep
\titlespacing{\section}{0.1cm}{0cm}{0cm}[0cm]

% }}}

% Sub-Section Title Settings {{{
\titleformat {\subsection}
	[hang]
	{\small\bfseries}
	{}
	{0em}
	{
	\nolinenumbers 
	\vspace{-0.5cm} 
	\begin{subsection-box} 
	\begin{subsection-box-2} \end{subsection-box-2} 
	\vspace{-0.5cm} \hspace{0.35cm}
		\color{black} \thesubsection. #1
	\end{subsection-box}
	}
[
\linenumbers
]

% left before-sep after-sep right-sep
\titlespacing*{\subsection}{0cm}{0cm}{0cm}[0em]

% }}}

% Sub-Sub-Section Title Settings {{{
\titleformat {\subsubsection}
	[hang]
	{\small\bfseries}
	{}
	{0em}
	{
	\nolinenumbers 
	\vspace{-0.5cm} 
	\begin{subsubsection-box} 
	\begin{subsubsection-box-2} 
	\begin{subsubsection-box-3}
	\end{subsubsection-box-3} 
	\end{subsubsection-box-2}
	\vspace{-0.4cm} \hspace{0.8cm} 
		\color{black} \thesubsubsection. #1
	\end{subsubsection-box}
	}
[
\linenumbers
]

% left before-sep after-sep right-sep
\titlespacing{\subsubsection}{0cm}{0cm}{0cm}[0em]

% }}}

% }}}
% HYPERLINKS ------------------------------------------------------------- {{{

\usepackage[unicode,bookmarks]{hyperref} % auto hyperlinks toc , refrences
                      % others can be manually specified
					  % unicode + bookmarks for japanese characters

\hypersetup{
colorlinks = true,
linktoc    = all,
citecolor  = purple,
filecolor  = black,
linkcolor  = black,
urlcolor   = black
}


% }}}
% HEADER / FOOTER -------------------------------------------------------- {{{

\usepackage{fancyhdr} % allows for header and footer customizations
\pagestyle{fancy}     %
\fancyhf{}            %

\renewcommand{\headrulewidth}{0.2pt} % draw line at header
\lhead { \textit{ \leftmark } }           % LEFT  : show section name at header
\rhead{\textit{\thepage}}            % RIGHT : Show page number
% \chead{ }

\renewcommand{\footrulewidth}{0pt} % draw line at footer
%\lfoot{\textit{Last Edited : \today}}
%\cfoot{center foot}
%\rfoot{\textit{tiwathia \thepage}}

\pagenumbering{arabic} % Specify type of number characters to use

% }}}
% COLORS ----------------------------------------------------------------- {{{

\usepackage [table]{xcolor}
%\rowcolors{<starting row index>}{<odd row color>}{<even row color>}

% Section Colors {{{

% 464444 is a tint color of 181616
\definecolor {section-bg}         {HTML} {464444}
\definecolor {subsection-bg}      {HTML} {747373}
\definecolor {subsubsection-bg}   {HTML} {8b8a8a}
\definecolor {section-font}       {RGB}  { 0,0,0}
\definecolor {subsection-font}    {RGB}  { 0,0,0}
\definecolor {subsubsection-font} {RGB}  { 0,0,0}

% }}}

% Table Colors {{{

% 404b4b is ananlagous color of 181616
% 534e4e is tint color of 404b4b
\definecolor {table-topic}        {HTML} {747373}
\definecolor {table-subtopic}     {HTML} {8b8a8a}
\definecolor {table-subsubtopic}  {HTML} {a2a1a1}

% e7e7e7 is tint color of 181616
\definecolor {table-alternating-1}    {HTML} {e7e7e7} % 1 = gray
\definecolor {table-alternating-2}    {HTML} {FFFFFF} % 2 = white

\definecolor {cell-lightblue}     {HTML} { b2cce5}
\definecolor {cell-lightgray}     {HTML} { d8deda}
\definecolor {cell-lightorange}   {HTML} { F6C396}
\definecolor {cell-lightred}      {HTML} { F1A099}
\definecolor {cell-lightgreen}    {HTML} { C6DA7F}
\definecolor {cell-lightpurple}   {HTML} { CCB2E5}
\definecolor {cell-lightyellow}   {HTML} { FFF09A}

% }}}

% Tcolorbox Tables {{{


% B/W DEFS

\definecolor {defn-bg}       {HTML} {e7e7e7}
\definecolor {defn-title}    {HTML} {747373}
\definecolor {defn-theword}  {HTML} {a2a1a1}

\definecolor {note-bg}       {HTML} {e7e7e7}
\definecolor {note-theword}  {HTML} {a2a1a1}

\definecolor {table-bg}      {HTML} {FFFFFF}
\definecolor {table-title}   {HTML} {747373}
\definecolor {table-theword} {HTML} {a2a1a1}

\definecolor {image-bg}      {HTML} {FFFFFF} 
\definecolor {image-title}   {HTML} {747373}
\definecolor {image-theword} {HTML} {a2a1a1}

% COLOR DEFS

%\definecolor {defn-bg}       {HTML} {e7e7e7}
%\definecolor {defn-title}    {HTML} {A9A9A9}
%\definecolor {defn-theword}  {HTML} {EDA72D}

%\definecolor {note-bg}       {HTML} {e7e7e7}
%\definecolor {note-theword}  {HTML} {E25A22}

%\definecolor {table-bg}      {HTML} {e7e7e7}
%\definecolor {table-title}   {HTML} {F2F3F4}
%\definecolor {table-theword} {HTML} {9ACD32}

%\definecolor {image-bg}      {HTML} {e7e7e7}
%\definecolor {image-title}   {HTML} {A9A9A9}
%\definecolor {image-theword} {HTML} {0067A5}
% }}}

% Code {{{

\definecolor{code-bg}      {HTML}{fbfbfb}
\definecolor{code-comment} {HTML}{A4E400}
\definecolor{code-keyword} {HTML}{FC1A70}
\definecolor{code-string}  {HTML}{62d8f1}
\definecolor{code-regular} {RGB}{39,40,34}

% }}}

% }}}
% TABLES ----------------------------------------------------------------- {{{

\usepackage{booktabs}    % nicer line drawing with toprule,midrule,hrule
\usepackage{multicol}    % allows column merging in tables
\usepackage{multirow}    % allows row merging in tables

% was causing issues where equations , matrices , merged cells etc... , were
% appearing white / not displaying / overwriting each other
%\usepackage{tabularx}    % allow tables to stretch to page length

\usepackage{xtab}        % allows page breaking tables inline
\usepackage{makecell}    % allows linebreaking within cells of tables

% longtables tend to break in two column mode
%\usepackage{longtable}  % allows tables to span pages
%\usepackage{ltablex}    % combination of longtable and tabularx


% }}}
% TCOLORBOX -------------------------------------------------------------- {{{

\usepackage[skins,breakable]{tcolorbox}
% skins allows use of enhanced options
% breakable allows breaking boxes between pages

% 	IMAGE TABLES (TCOLORBOX) {{{

\newtcolorbox{image-bg}[2][]{
	enhanced,
	colback           = image-bg,
	colframe          = image-bg,
	fonttitle         = \bfseries,
	width             = \linewidth,
	beforeafter skip  = 0.5cm,
	sharp corners,
%	drop fuzzy shadow = gray,
	boxrule         = 0mm,
	top             = 0mm,
	bottom          = 0mm,
	left            = 0mm,
	right           = 0mm,
	title = #2,#1
}

\newtcolorbox{image-title}[2][]{
	enhanced,
	colback           = image-title,
	colframe          = image-title,
	fonttitle         = \bfseries, 
	width             = \linewidth,
	height            = 0.6cm,
%	drop fuzzy shadow = gray,
	beforeafter skip  = 0pt,
%	grow to left by   = 0.7cm,
	boxrule           = 0mm,
	top               = 0.5mm,
	bottom            = 0mm,
	left              = 1mm,
	right             = 0mm,
	sharp corners,
	title = #2,#1
}

\newtcolorbox{image-theword}{
	enhanced,
	colback           = image-theword,
	colframe          = image-theword,
	fonttitle         = \bfseries,
%	drop fuzzy shadow = gray,
	width             = 3cm,
	height            = 0.5cm,
	beforeafter skip  = 0pt,
%	grow to left by   = 0.7cm,
	boxrule           = 0mm,
	top               = 0.5mm,
	bottom            = 0mm,
	left              = 1mm,
	right             = 0mm,
	sharp corners
}

\newtcolorbox{image-content}{
	enhanced,
	colback         = image-bg,
	colframe        = image-bg,
	fonttitle       = \bfseries,
%	enlarge top by  = -0.5cm,
%	enlarge right by = 5cm,
	width           = \linewidth,
	boxrule         = 0mm,
	top             = 0mm,
	bottom          = 0mm,
	left            = 0mm,
	right           = 0mm,
%	show bounding box
}


\newtcolorbox{image-caption}{
	enhanced,
	colback           = image-theword,
	colframe          = image-theword,
	fonttitle         = \bfseries, 
	width             = 0.18\linewidth,
	height            = 0.5cm,
	beforeafter skip  = 0pt,
%	grow to left by   = 0.7cm,
	boxrule           = 0mm,
	top               = 0.5mm,
	bottom            = 0mm,
	left              = 1mm,
	right             = 0mm,
	sharp corners
}


% }}}

% 	REGULAR TABLES (TCOLORBOX) {{{

\newtcolorbox{table-bg}[2][]{
	enhanced,
%	float,
%	breakable,
	colback           = table-bg,
	colframe          = table-bg,
	fonttitle         = \bfseries,
%	width             = 0.98\linewidth,
	beforeafter skip  = 0cm,
%	drop fuzzy shadow = gray,
	boxrule         = 0mm,
	top             = 0mm,
	bottom          = 0mm,
	left            = 0mm,
	right           = 0mm,
	sharp corners,
	title = #2,#1
}

\newtcolorbox{table-theword}{
	enhanced,
	colback           = table-theword,
	colframe          = table-theword,
	fonttitle         = \bfseries,
%	drop fuzzy shadow = gray,
	width             = 2.2cm,
	height            = 0.5cm,
	beforeafter skip  = 0pt,
%	grow to left by   = 0.7cm,
	boxrule           = 0mm,
	top               = 0.5mm,
	bottom            = 0mm,
	left              = 1mm,
	right             = 0mm,
	sharp corners,
}

\newtcolorbox{table-title}[2][]{
	enhanced,
	colback           = table-title,
	colframe          = table-title,
	fonttitle         = \bfseries,
	width             = 0.8\linewidth,
	height            = 0.6cm,
%	width = 0.5\linewidth,
%	drop fuzzy shadow = gray,
	beforeafter skip  = 0pt,
%	grow to left by   = 0.7cm,
	boxrule           = 0mm,
	top               = 0.5mm,
	bottom            = 0mm,
	left              = 1mm,
	right             = 0mm,
	sharp corners,
	title = #2,#1
}


\newtcolorbox{table-content}{
	enhanced,
	colback         = table-bg,
	colframe        = table-bg,
	fonttitle       = \bfseries,
	before skip = 0.5cm,
%	enlarge top by  = -0.5cm,
%	enlarge right by = 5cm,
	width           = \linewidth,
	boxrule         = 0mm,
	top             = 2mm,
	bottom          = 0mm,
	left            = 0mm,
	right           = 0mm 
}

% }}}

% 	DEFINITION TABLE (TCOLORBOX) {{{

% \newtcbox[init options]{name}[number][default]{options}

\newtcolorbox{defn-bg}{
	enhanced,
	colback           = defn-bg,
	colframe          = defn-bg,
	fonttitle         = \bfseries,
%	drop fuzzy shadow = gray,
	width             = \linewidth,
	beforeafter skip  = 0.3cm, 
	boxrule           = 0mm,
	top               = 0mm,
	bottom            = 0mm,
	left              = 0mm,
	right             = 0mm,
%	sharp corners
	arc is angular,
}

\newtcolorbox{defn-theword}{
	enhanced,
	colback           = defn-theword,
	colframe          = defn-theword,
	fonttitle         = \bfseries,
	width             = 3.3cm,
	height            = 0.5cm,
	beforeafter skip  = 0pt,
	boxrule           = 0mm,
	top               = 0.5mm,
	bottom            = 0mm,
	left              = 1mm,
	right             = 0mm,
	drop fuzzy shadow = gray,
	% comment following line to align
	grow to left by   = 0.3cm,
	sharp corners
}

\newtcolorbox{defn-title}[2][]{
	enhanced,
	colback           = defn-title,
	colframe          = defn-title,
	fonttitle         = \bfseries,
	width             = 0.5\linewidth,
	height            = 0.6cm,
	beforeafter skip  = 0pt,
	boxrule           = 0mm,
	top               = 0.5mm,
	bottom            = 0mm,
	left              = 1mm,
	right             = 0mm,
	drop fuzzy shadow = gray,
	% comment following line to align
	grow to left by   = 0.3cm,
	sharp corners,
	title = #2,#1
}

\newtcolorbox{defn-content}{
	enhanced,
	colback         = defn-bg,
	colframe        = defn-bg,
	fonttitle       = \bfseries,
	beforeafter skip = 0.5cm,
%	enlarge top by  = -0.5cm,
%	enlarge right by = 5cm,
	width           = \linewidth,
	boxrule         = 2.5mm,
	top             = 2mm,
	bottom          = 0mm,
	left            = 0mm,
	right           = 0mm 
}

% }}}

% 	NOTE TABLE (TCOLORBOX) {{{

\newtcolorbox{note-bg}{
	enhanced,
	colback           = note-bg,
	colframe          = note-bg,
	fonttitle         = \bfseries,
%	drop fuzzy shadow = gray,
	width             = 0.95\linewidth,
	before skip       = 0cm,
	after skip        = 0cm,
	boxrule           = 0mm,
	top               = 0mm,
	bottom            = 0mm,
	left              = 0mm,
	right             = 0mm,
%	sharp corners, 
	arc is angular,
}

\newtcolorbox{note-theword}{
	enhanced,
	colback           = note-theword,
	colframe          = note-theword,
	fonttitle         = \bfseries,
%	drop fuzzy shadow = gray,
	width             = 1.4cm,
	height            = 0.5cm,
	beforeafter skip  = 0pt,
	boxrule           = 0mm,
	top               = 0.5mm,
	bottom            = 0mm,
	left              = 1mm,
	right             = 0mm,
%	grow to left by   = 0.7cm,
	sharp corners,
}

\newtcolorbox{note-content}{
	enhanced,
	colback         = note-bg,
	colframe        = note-bg,
	fonttitle       = \bfseries,
	boxrule         = 0mm,
	top             = 2mm,
	bottom          = 0mm,
	left            = 0mm,
	right           = 0mm 
%	enlarge top by  = -0.6cm,
%	enlarge left by = 1.5cm,
%	width           = 13cm,
}


% }}}

% 	SECTION TITLES (TCOLORBOX) {{{

\newtcolorbox{section-box}{
	enhanced,
	colback          = section-bg,
	colframe         = section-bg,
	fonttitle        = \bfseries,
	width            = \linewidth,
	height           = 1cm,
	beforeafter skip = 0pt, 
	sharp corners 
}

\newtcolorbox{subsection-box}{
	enhanced,
	colback          = white,
	colframe         = white,
	fonttitle        = \bfseries,
	width            = \linewidth,
	height           = 0.8cm,
	beforeafter skip = 0pt,
	boxrule          = 0mm, 
	top               = -1mm,
	bottom            = 0mm,
	left              = -1mm,
	right             = 0mm,
	sharp corners
}

\newtcolorbox{subsubsection-box}{
	enhanced,
	colback   = white,
	colframe  = white,
	fonttitle = \bfseries,
	width     = \linewidth,
	height = 0.6cm,
	sharp corners,
	beforeafter skip  = 0pt,
	boxrule           = 0mm,
	top               = -1mm,
	bottom            = 0mm,
	left              = -1mm,
	right             = 0mm
}

% }}}

% 	SECTION TITLES 2 (TCOLORBOX) {{{

\newtcolorbox{subsection-box-2}{
	enhanced,
	colback          = section-bg,
	colframe         = section-bg,
	fonttitle        = \bfseries,
	width            = 0.35cm,
	height           = 0.8cm,
	beforeafter skip = 0pt, 
	top               = 0mm,
	bottom            = 0mm,
	left              = 0mm,
	right             = 0mm,
	sharp corners 
}

\newtcolorbox{subsubsection-box-2}{
	enhanced,
	colback          = section-bg,
	colframe         = section-bg,
	fonttitle        = \bfseries,
	width            = 0.35cm,
	height           = 0.6cm,
	beforeafter skip = 0pt, 
	top               = -1.5mm,
	bottom            = 0mm,
	left              = 2mm,
	right             = 0mm,
	sharp corners 
}


\newtcolorbox{subsubsection-box-3}{
	enhanced,
	colback          = subsection-bg,
	colframe         = subsection-bg,
	fonttitle        = \bfseries,
	width            = 0.35cm,
	height           = 0.6cm,
	beforeafter skip = 0pt,
	boxrule          = 0mm, 
	top               = 0mm,
	bottom            = 0mm,
	left              = 0mm,
	right             = 0mm,
	sharp corners
}

% }}}

% }}}
% IMAGES ----------------------------------------------------------------- {{{

\usepackage{wrapfig, graphicx}
%\usepackage{graphicx}   %
\usepackage{subcaption} %
%\usepackage{wrapfig}    % wrap images around text
\usepackage{capt-of}    % define captions independent of figures
\usepackage{float}
\usepackage{varwidth}
\graphicspath{{2_images/}} % define folder path for images

% }}}
% CODE / CODE DISPLAY ---------------------------------------------------- {{{

\usepackage{listings}

\lstset{
basicstyle       = \footnotesize\ttfamily,
backgroundcolor  = \color{code-bg},
commentstyle     = \color{code-comment}, % comment style
keywordstyle     = \color{code-keyword},  % keyword style
stringstyle      = \color{code-string},  % string literal style
rulecolor        = \color{black},        % if not set, the frame-color may be changed on line-breaks
frame            = single,               % adds a frame around the code
basicstyle       = \footnotesize,        % the size of the fonts that are used for the code
keepspaces       = true,                 % keeps spaces in text, useful for keeping indentation of code
tabsize          = 2,                    % sets default tabsize
breaklines       = true,                 % sets automatic line breaking
captionpos       = b,                    % sets the caption-position to bottom
%numbers          = left,
%numberstyle      = \tiny,
%numbersep        = 10pt,
frame            = tb, % top bottom
columns          = fixed,
showstringspaces = false,
showtabs         = false,
keepspaces,
% escapeinside={\%*}{*)},  % if you want to add LaTeX within your code
framextopmargin=10pt,    % margin for the top background border
framexbottommargin=10pt, % margin for the bot background border
framexleftmargin=0pt,    % margin for the left background border
framexrightmargin=0pt    % margin for the right background border
}


% }}}
% MATH / GRAPHING  ------------------------------------------------------- {{{

\usepackage{amsmath}  % basic math package
\usepackage{amssymb}  % allows more math symbols
\usepackage{amsthm}   % allows custom therorem,defn,corll etc... definitions
\usepackage{mathrsfs} % some new fonts for math mode

\newtheorem{mydef}{DEFINITION}[section] % to get auto numbering
\newtheorem{myfigure}{FIGURE}[section]  % to get auto numbering
\newtheorem{mytable}{TABLE}[section]    % to get auto numbering

% tex.stackexchange.com/questions/461186/how-to-use-lineno-with-amsmath-align
% amsmath should come after lineno for auto math line numbering
% linenumbers for amsmath (lineno+ams) ,
% honestly i can live without it ... , it creates more problems than its worth

	% TIKZ --------------------------------------------------------------- {{{
	\usepackage{tikz}
	%\documentclass[tikz, border=1mm]{standalone} % only tikz image document

	% https://latexdraw.com/exploring-tikz-arrows/

	\usetikzlibrary{calc} % to sum coordinates
	\usetikzlibrary{shapes.geometric}
	\usetikzlibrary{shapes.arrows}
	\usetikzlibrary{arrows.meta} % custom arrow heads for vectors
	\usetikzlibrary{positioning}
	\usetikzlibrary{backgrounds}
	\usetikzlibrary{decorations.pathreplacing}
	\usetikzlibrary{angles}
	\usetikzlibrary{quotes}
	\usetikzlibrary{intersections}
	\usetikzlibrary{calligraphy}


	\usepackage{pgfplots}
	\pgfplotsset{compat=newest}
	\usepgfplotslibrary{fillbetween}
	\usepackage{array}

	% }}}

% }}}
% BIBLIOGRAPHY / REFERENCES / FOOTNOTES ---------------------------------- {{{

\usepackage[nottoc,numbib]{tocbibind} % to show references line in toc
\usepackage{natbib}                   % natbib bibliography manager
\bibliographystyle{rusnat}            % natbib bibliography style
\setcitestyle{notesep={; },round,aysep={},yysep={;}} % natbib inline citation style

%\usepackage[superscript,biblabel]{cite}

%\usepackage{cleveref} % enhances latex referencing features
                      % stuff like page refrences , formats etc
% it might help get rid of all those custom secrefs we have going on

\renewcommand{\thefootnote}{\roman{footnote}} % footnote style

% force footnotes to bottom of page
\def\footnoterule{\vfill % added this
   \kern-3pt\hrule width 2truein \kern 2.6pt} % the \hrule is .4pt high

% Dont forget to generate the .bbl and .blg files using :
% pdflatex document.tex
% bibtex document
% pdflatex document.tex


% }}}
% TABLE OF CONTENTS ------------------------------------------------------ {{{

\usepackage{titletoc}
% margin from RHS
%\contentsmargin{1cm}

%\dottedcontents {section}[left]{above}{label-width}{leader-width}

%\dottedcontents{section}[1.8cm]{\bfseries}{3.2em}{1pc}
%\dottedcontents{subsection}[1.8cm]{}{3.2em}{1pc}
%\dottedcontents{subsubsection}[1.8cm]{}{2.8em}{1pc}


% }}}
% CUSTOM COMMANDS -------------------------------------------------------- {{{

% New Commands : Sectioning {{{

\newcommand{ \sectionend}
{
\nolinenumbers
\begin{center} 
{\large \textbf{-§-}  }
\end{center} 

% was causing problems , i havent figured out the cause
%\justify

\linenumbers
\pagebreak
}

\newcommand{ \subsectionend}
{
\nolinenumbers 

%\begin{center} 
%\rule{0.7em}{0.7em} \rule{0.7em}{0.7em}
%\end{center} 

\linenumbers
}

\newcommand{ \subsubsectionend}
{
\nolinenumbers

%\begin{center} 
%\rule{0.7em}{0.7em}
%\end{center} 

\linenumbers
}

% }}}

% New Commands : Symbols {{{

\newcommand{ \bulletpoint}
{ $\bullet$  }

% }}}

% New Commands : Tcolorbox {{{

%	General {{{

\newcommand { \tcolorboxstart }
{
	\nolinenumbers
	\vspace{0.2cm}
	\centering
}

\newcommand{ \tcolorboxend}
{
	\justifying
	\vspace{0.2cm}
	\linenumbers
}

% }}}

%	Definition {{{

\newcommand{ \tcolorboxdefinition}[3]
{

\tcolorboxstart
\begin{defn-bg}

	\begin{defn-title}[width=7cm]{}
	{
		\normalsize \textbf{#1}
	}
	\end{defn-title}

	\begin{defn-theword}
	{
		\footnotesize
		\begin{mydef} #2
%		\label{def:{#2}}
		\end{mydef}
	}
	\end{defn-theword}

	\begin{defn-content}
		\justify
		\vspace{-0.6cm}
		#3
	\end{defn-content}

\end{defn-bg}
\tcolorboxend
}

% }}}

%	Note {{{

\newcommand{ \tcolorboxnote}[1]
{

\tcolorboxstart
\begin{note-bg}

	\begin{note-theword}
		{\footnotesize \textbf{NOTE} }
	\end{note-theword}

	\begin{note-content} \justifying

		#1

	\end{note-content}

\end{note-bg}
\tcolorboxend
}




% }}}

%	Table {{{


\newcommand{ \tcolorboxtable}[5]
{
\tcolorboxstart
\begin{table-bg}#3{}

	\begin{table-title}[width=\linewidth]{}
		\captionsetup{labelformat=empty}
		\captionof{table}{ \textbf{#1} }
	\end{table-title}

	\begin{table-theword}
		\footnotesize
		\begin{mytable}
		#2
		\end{mytable}
	\end{table-theword}

	\begin{table-content}
	\vspace{-0.5cm}
	#4
	\end{table-content}

\end{table-bg}
\tcolorboxend
}

% }}}

%	Image {{{

\newcommand{ \tcolorboxfigure}[4]
{
\tcolorboxstart
\begin{image-bg}[width=\linewidth]{}

	\begin{image-title}[width=\linewidth]{}
		\captionsetup{labelformat=empty}
		\captionof{figure}{#1 \cite{#4}}
	\end{image-title}

	\begin{image-theword} 
	\footnotesize 
		\begin{myfigure}
		#2
		\end{myfigure}
	\end{image-theword}

	\begin{image-content}
		\includegraphics[width=\linewidth]{#3} 
	\end{image-content}



\end{image-bg}
\tcolorboxend
}

% }}}

% }}}

% New Commands : Tables {{{



\newcommand{ \tabulartable }[3]
{
	\nolinenumbers
	\begin{tabular*}{#1}{@{\extracolsep{\fill}} #2 @{}}

		#3

	\end{tabular*}
	\linenumbers
}

\newcommand{ \alternatingtabulartable }[2]
{
	\nolinenumbers

	\rowcolors{2}{table-alternating-1}{table-alternating-2}
	\begin{tabular*}{\columnwidth}{@{\extracolsep{\fill}} #1 @{}}

		#2

	\end{tabular*}

	\linenumbers
}

\newcommand{ \alternatingtabularxtable}[3]
{

	\nolinenumbers

	\rowcolors{2}{table-alternating-1}{table-alternating-2}
	\begin{tabularx}{#1}{#2}
		#3
	\end{tabularx}

	\linenumbers

}

\newcommand{ \tabularxtable}[3]
{
	\nolinenumbers

	\rowcolors{2}{table-alternating-2}{table-alternating-2}
	\begin{tabularx}{#1}{#2}
		#3
	\end{tabularx}

	\linenumbers
}

\newcommand{ \alternatingxtabulartable}[2]
{
	\nolinenumbers

	\rowcolors{2}{table-alternating-1}{table-alternating-2}
	\begin{xtabular}{#1}
		#2
	\end{xtabular}

	\vspace{0.25cm}
	\linenumbers
}

\newcommand{ \xtabulartable}[2]
{
	\nolinenumbers

	\rowcolors{2}{table-alternating-2}{table-alternating-2}
	\begin{xtabular}{#1}
		#2
	\end{xtabular}

	\vspace{0.25cm}
	\linenumbers
}

\newcommand{ \twocolumnxtabulartable}[2]
{
	\nolinenumbers

	\rowcolors{2}{table-alternating-2}{table-alternating-2}
	\begin{xtabular}{#1}
		#2
	\end{xtabular}

	\vspace{0.25cm}
	\linenumbers
}

\newcommand{ \pagewidetable}[3]
{

	\nolinenumbers

	\begin{table*}[ht]
	\caption{#1}
	\centering
	\begin{tabular} {#2}
		#3
	\end{tabular}
	\end{table*}

	\linenumbers
}
% }}}

% New Commands : References {{{

\newcommand{ \refsec}[1]
{
	\hyperref[sec:#1]
	{
		(\textsection~\ref{sec:#1})
	}
}

\newcommand{ \refssec}[1]
{
	\hyperref[sec:#1]
	{
		(\textsection~\ref{ssec:#1})
	}
}

\newcommand{ \refsssec}[1]
{
	\hyperref[sec:#1]
	{
		(\textsection~\ref{sssec:#1})
	}
}

\newcommand{ \refdef}[1]
{
	\hyperref[def:#1]
	{
		(Def.~\ref{def:#1})
	}
}

\newcommand{ \reffig}[1]
{
	\hyperref[fig:#1]
	{
		(Fig.~\ref{fig:#1})
	}
}

\newcommand{ \reftable}[1]
{
	\hyperref[table:#1]
	{
		(Table.~\ref{table:#1})
	}
}
% }}}

% }}}
% TESTING ---------------------------------------------------------------- {{{ 

\usepackage{lipsum}    % generates filler text
\usepackage{blindtext} % generates non-latin filler text

%\usepackage{showframe} % allows showing frames within which stuff is printed
%\renewcommand*\ShowFrameColor{\color{red}}


% }}}
% JAPANESE --------------------------------------------------------------- {{{

\usepackage[whole]{bxcjkjatype} % Allows Japanese Characters , CJK incompatible with pdflatex
\usepackage{ruby}               % used for kanji furigana placement
                                % \ruby{学}{がく}

\renewcommand\rubysep{0ex}      % Kanji and furigana placement location
\let\temp\ruby                  %
\renewcommand\ruby[2]{          % Kanji and furigana size / spacing / color
	\temp
	{\fontsize{10}{10}#1 }
	{\fontsize{6}{6}\textcolor{gray}{#2}}
}
% }}}
% KOREAN {{{
\usepackage{CJKutf8}

% text has to inside block :
%\begin{CJK}{UTF8}{}
%\end{CJK}


% }}}


% ============================================================================ 
\begin{document}
%% Suppresses headers and footers on the title page
\begin{titlepage}

	% Centre everything on the title page
	\centering

	% Use small caps for all text on the title page
	\scshape

	% White space at the top of the page
	\vspace*{\baselineskip}

	%------------------------------------------------
	%	Title
	%------------------------------------------------

	% Thick horizontal rule
	\rule{\textwidth}{1.6pt}\vspace*{-\baselineskip}\vspace*{2pt}

	% Thin horizontal rule
	\rule{\textwidth}{0.4pt}

	% Whitespace above the title
	\vspace{0.75\baselineskip}

	% Title
	{\LARGE TITLE \\}

	% Whitespace below the title
	\vspace{0.75\baselineskip}

	% Thin horizontal rule
	\rule{\textwidth}{0.4pt}\vspace*{-\baselineskip}\vspace{3.2pt}

	% Thick horizontal rule
	\rule{\textwidth}{1.6pt}

	\vspace{2\baselineskip} % Whitespace after the title block

	%------------------------------------------------
	% Subtitle
	%------------------------------------------------

	% Kurt Wolff Verlag, Leipzig, 1927
	\vspace*{3\baselineskip} % Whitespace under the subtitle

	%------------------------------------------------
	% Author(s)
	%------------------------------------------------
	
%	Author
	by

	% Whitespace before the editors
	\vspace{0.5\baselineskip}

	% Author list
	{\scshape\Large John Doe  \\}

	\vspace*{3\baselineskip} % Whitespace under the subtitle

	%------------------------------------------------
	%	Editor(s)
	%------------------------------------------------
	
%	Edited By

	% Whitespace before the editors
%	\vspace{0.5\baselineskip}

	% Editor list
%	{\scshape\Large John Smith \\ Jane Smith \\}

	% Whitespace below the editor list
%	\vspace{0.5\baselineskip}

	% Editor affiliation
%	\textit{The University of California \\ Berkeley}

	% Whitespace between editor names and publisher logo
	\vfill

	%------------------------------------------------
	% Publisher
	%------------------------------------------------

	% Whitespace under the publisher logo
	\vspace{0.3\baselineskip}

	% Publisher
%	{\large TheVirtualLibrary.org} \\

	% Publication year
%	2019

	{\large Last Edited : } \today \\

\end{titlepage}

\raggedbottom
\twocolumn
%\onecolumn

\tableofcontents
\pagebreak

%\listoftables
%\vspace{1cm}
%\listoffigures
%\clearpage

%\onecolumn
\justifying 
\linenumbers


%\input{3_content/file.tex}

% SECTION : linux_sysadmin_course {{{
\section{Linux SysAdmin Course}
\label{sec:linux_sysadmin_course}

% 	SUB-SECTION : std_o_i_err {{{
\subsection{STD O I ERR}
\label{ssec:std_o_i_err}

Everything in linux is treated as ordinary files. This includes data streams.
Each file gets a file-descriptor assigned to it that can then be used to acces
the particular type of data stream that it is carrying. A file-descriptor is
basically in integer. When executing a command on linux we will mainly be
concerned with the following three tupes of file descriptors :


% tabular TABLE : file_descriptors {{{

\tabulartable
{0.95\linewidth}
{t}
{llll}
{

\textbf{STDIN}  & : & Standard input has file decriptor  & 0 \\
\textbf{STDOUT} & : & Standard output has file decriptor & 1 \\
\textbf{STDERR} & : & Standard error has file decriptor  & 2 \\

}

% }}} End TABLE : file descriptors

every process that is running in the linux kernel has three channels , which are
STDIN , STDOUT and STDERR . Each one of these channels have numbers that
correspond to them. If we want to redirect ouputs of certain processes into
others.\\


% IMAGE : std_in_out_err {{{

\tcolorboxfigure
{ \textbf{STD IN / OUT / ERR} }
{\label{fig:std_in_out_err}}
{std.png}
{stdioerr}

% }}} End IMAGE : std_in_out_err


\noindent
echo ``this should be in a file'' 1 \textgreater \hspace{0.2cm} somefile.txt\\

\noindent
output redirection uses STDOUT by default so we dont even need the 1 in the
above command.\\

\noindent
echo ``this should be in a file'' \textgreater \hspace{0.1cm} somefile.txt \\

\noindent
this will however overwrite everything in the original file.\ so instead of
overwrite , we append using \textgreater \textgreater \\

\noindent
echo ``this should be in a file'' \textgreater \textgreater \hspace{0.1cm}
somefile.txt\\

\noindent
similarly to print out error , we use the filedescriptor 2\\

\noindent
ls thisdoesnotexist 2 \textgreater \hspace{0.1cm} somefile.txt\\

\noindent
to redirect input we use the \textless symbol\\


\subsectionend
% }}} END SUB-SECTION : std_o_i_err

% 	SUB-SECTION : package_management {{{
\subsection{Package Management}
\label{ssec:package_management}

% 		SUB-SUB-SECTION : pacman {{{
\subsubsection{pacman}
\label{sssec:pacman}



\subsubsectionend
% }}} END SUB-SUB-SECTION : pacman

% 		SUB-SUB-SECTION : aur {{{
\subsubsection{AUR}
\label{sssec:aur}



\subsubsectionend
% }}} END SUB-SUB-SECTION : aur

% 		SUB-SUB-SECTION : source_installation {{{
\subsubsection{Source Installation}
\label{sssec:source_installation}



\subsubsectionend
% }}} END SUB-SUB-SECTION : source_installation

\subsectionend
% }}} END SUB-SECTION : package_management

% 	SUB-SECTION : grep {{{
\subsection{GREP : \textit{Global Regular Expression Print} }
\label{ssec:grep}

\noindent
Grep is a pattern matching tool. It basically returns things based on your set
conditions from a large ouput.  Grep is most often used in conjuction with other
file manipulation commands such as less , cut , cat , sort \ldots in order to
reduce the information that we as a human need to parse through to get at what
we really wanted.

Grep uses POSIX regular expressions in evaluation. This is different from PERL
compatible regular expressions that are used in python and some other
programming languages. So dont get that confused.

Grep use examples and flags are shown below :

% XTABULAR TABLE :  {{{

\xtabulartable
{lllll}
{

\rowcolor{table-alternating-1}
grep &           & John            & notes.txt        \\
\multicolumn{4}{l}{ basic usecase } \\
&&&\\

\rowcolor{table-alternating-1}
grep &           & ``abc pqr''     & \textit{filename} \\
\multicolumn{4}{l}{ use double quotes to search for phrases }\\
\multicolumn{4}{l}{ with spaces }\\
&&&\\

\rowcolor{table-alternating-1}
grep & -w        & \textit{phrase} & \textit{filename} \\
\multicolumn{4}{l}{ exact matches  }\\
&&&\\

\rowcolor{table-alternating-1}
grep & -i        & \textit{phrase} & \textit{filename} \\
\multicolumn{4}{l}{ non case sensitive matches  }\\
&&&\\

\rowcolor{table-alternating-1}
grep & -n        & \textit{phrase} & \textit{filename} \\
\multicolumn{4}{l}{ display line numbers  }\\
&&&\\

\rowcolor{table-alternating-1}
grep & -B 4      & \textit{phrase} & \textit{filename} \\
\multicolumn{4}{l}{ display lines \textbf{before} matched word  }\\
&&&\\

\rowcolor{table-alternating-1}
grep & -A 4      & \textit{phrase} & \textit{filename} \\
\multicolumn{4}{l}{ display lines \textbf{after} matched word   }\\
&&&\\

\rowcolor{table-alternating-1}
grep & -win -C 4 & \textit{phrase} & \textit{filename} \\
\multicolumn{4}{l}{ display lines \textbf{context} of word \textbf{(before and
		after) } }\\
%\multicolumn{4}{l}{ \textbf{after)} }\\
&&&\\

\rowcolor{table-alternating-1}
grep & -r        & \textit{phrase} & \textit{filename} \\
\multicolumn{4}{l}{ search for phrase recursively in all subdirectories }\\
&&&\\

\rowcolor{table-alternating-1}
grep & -l        & \textit{phrase} & \textit{filename} \\
\multicolumn{4}{l}{ display files that contain match }\\
&&&\\

\rowcolor{table-alternating-1}
grep & -c        & \textit{phrase} & \textit{filename} \\

\multicolumn{4}{l}{ \makecell[l]{return count of the number of lines (not \\
		times) the phrase occurs in the specified files} } \\

&&&\\

\rowcolor{table-alternating-1}
grep & -v        & \textit{phrase} & \textit{filename} \\

\multicolumn{4}{l}{ \makecell[l]{Do an inverse match , i.e.\, return all \\
		lines that do not contain the given phrase}{} }\\

&&&\\

\rowcolor{table-alternating-1}
grep &           & \textit{phrase} & ./* \\
\multicolumn{4}{l}{ search all files (*) in current dir (./) }\\
&&&\\

\rowcolor{table-alternating-1}
grep &           & \textit{phrase} & ./*.txt \\
\multicolumn{4}{l}{ search all .txt files (*.txt) in current dir (./) }\\
&&&\\


}


% }}} End XTABULAR TABLE : 

\subsectionend
% }}} END SUB-SECTION : grep

% 	SUB-SECTION : file_permissions {{{
\subsection{File Permissions}
\label{ssec:file_permissions}

\vspace{0.5cm}

Each file in linux has 9 permission bits.The first 9 bits in a file are used to
describe the kinds of permissions that are allowed for a file.  Traditionally
there are three types of permission bits :

% ITEMIZE : permission_bits {{{

\begin{itemize}[noitemsep]

	\item \textbf{r} gives permission to read the file
	\item \textbf{w} gives permission to write the file , this lets you edit a
		file , move a file , rename a file and a bunch of other stuff.
	\item \textbf{x} gives permission to execute the file
	\item \textbf{-} means the file has no permission

\end{itemize}

% }}} END-ITEMIZE : permission bits

\noindent
The nine bits can be grouped into threes since they specify \textit{rwx}
permissions for the owner , the next three for the owners group , the last three
are for everyone else respectively.

\noindent
Usually when you are printing with a command like ls with the -l flag , then we
see an additional 10th bit at the front of the permission bits. This bit sets
the entry type, and describes the type of the file we are dealing with. The most
common types you might see the first bit being are :

% ITEMIZE : entry_bit_types {{{

\begin{itemize}[noitemsep]

	\item \textbf{-} means it is a file 
	\item \textbf{d} means it is a directory 
	\item \textbf{l} means it is a link
	\item \textbf{b} means it is a block device type (hard disk)

\end{itemize}

% }}} END-ITEMIZE : Entry bit Types

% tabular TABLE : file_permissions {{{

\tabulartable {0.95\linewidth}{t} {lllllll} { \rowcolor{table-subtopic}
	\textbf{Permissions} &   & \textbf{Owner} & \textbf{Owner Group} &
	\textbf{PID} &  & \textbf{Filename} \\ \midrule \rowcolor{white}  &   &  &
	&  &  &  \\

-rw-r--r-- & 1  & ishan & users & 11707 & Jun 15 17:59 & .vimrc \\ drwxr-xr-x &
15 & ishan & users & 4096  & Jun 15 20:55 & Documents \\

}

% }}} End TABLE : file permissions

\noindent
The permissions are goverened by the command \textit{chmod}

\noindent
The rwx permission also have numerical representations that are sometime used.
These also follow the same orders mentioned above , i.e.\ , the first numerical
representation will be for the user account. The second for the user group and
the third for the everyone else.

\noindent
These numbers are :


% ITEMIZE : file_permissions_numerical {{{

\begin{itemize}[noitemsep]

	\item 7  :  r w x
	\item 6  :  r w -
	\item 5  :  r - x
	\item 4  :  r - -
	\item 3  :  - w x
	\item 2  :  - w -
	\item 1  :  - - x
	\item 0  :  - - -

\end{itemize}

% }}} END-ITEMIZE : file_permissions_numerical

\noindent
If you dont want to go around manually editing file permissions then edit (very
very carefully) the file : /etc/login.defs in debian based systems.


\subsectionend
% }}} END SUB-SECTION : file_permissions

% 	SUB-SECTION : file_access_control {{{
\subsection{File Access Control}
\label{ssec:file_access_control}

Everything in linux is an object.\\

Permissions of a process are goverened by the user that started the process.\\

Usually these permissions are set-up not on an individual user basis , but
rather at a user group basis. (you might have heard of the WHEEL group). In a
production or company environment these are groups like Human Resources group ,
the managers group etc \ldots. One user can be part of multiple groups which
allows for easy modularity between user privilege levels.\\

The biggest baddest user is called root.

\vspace{0.5cm}

% 	SUB-SECTION : root {{{
\subsection{Root}
\label{ssec:root}

Root in linux can mean one of three things : 

\noindent


% ENUMERATE : roottypes {{{

\begin{enumerate}[noitemsep]

	\item Root User
	\item Root Folder
	\item Root Home

\end{enumerate}

% }}} END-ENUMERATE : Roottypes

\noindent
\textbf{Root User}


\noindent
The root user is the most powerful account user , and is also known as a
\textbf{\textit{superuser}}. There are many commands such as installing new
packages, or mounting / unmounting filesystems that require this privelege level (sudo == super user do ). 


\noindent 
\textbf{Root Directory}


\noindent
The root diretory is the place where the linux filesystem starts. It is the
directory that is indicated by only a / . The root directory contains all
folders in the linux sub directory.

\subsectionend
% }}} END SUB-SECTION : root

\subsectionend
% }}} END SUB-SECTION : file_access_control

% 	SUB-SECTION : users {{{
\subsection{Users}
\label{ssec:users}

A user is the person who is currently using the machine. You can see who you are
in many ways. First you can just open up the shell and look at the prompt. It
will look something like \textit{username@machinename:~} . You can also type in
the \textit{who} and \textit{w} commands. If you want more information than that
about just your current user account then you can see all current users by
catting the file \textbf{/etc/passwd} , or the file \textbf{/etc/shadow} .

\noindent
In the \textbf{/etc/shadows} file you might see two anomailes in the password
field : * and ! . The root user has password status ! which means that they can
only be accessed through sudo. And all processes acting as users have no ability
to log in whatsoever. This is specified by *. All other passwords should look
like hashes. If a user has been created , but thier password has not been set
yet, then they will also have ! in the password field in this file.

\noindent
To list user groups instead of just the users , we can cat : \textbf{/etc/group}

\noindent
These files also contains services that are running. Therefore you will see a lot
more ` Users ' than you might have on the machine. This is because a lot of
services also run as users within a linux machine. These services are listed
here albeit with much fewer permission than those that a normal human user would
have.

\noindent
The feilds in \textbf{/etc/passwd} will look something like :

\noindent
Username  :  x  :  USER ID  :  GROUP ID  :  Human Readable Info  :  Home Dir  :  Default Shell \\

\noindent
The feilds in \textbf{/etc/group} will look something like :

\noindent
Group Name  :  x  :  GROUP ID  :  Members of that Group \\

\noindent
\textbf{\textit{Adding a user}}

\noindent
To add a user we use the \textit{useradd} command , with the following flags

% ITEMIZE : add_user_flags {{{

\begin{itemize}[noitemsep]


	\item -m create home directory
	\item -d /home/username
	\item -u create user id
	\item -g define group id
	\item -s define default shell


\end{itemize}

% }}} END-ITEMIZE : add user flags

\noindent We can also create a bunch of new users simultaneously using the
\textit{newusers} command.  After creating the user , use the \textit{passwd}
command to set the password for the user. This will remove the locked ! status
from /etc/shadows mentioned earlier. If you want to relock the user then we can
use the \textit{usermod -L username} command. And to unlock him/her user
\textit{usermod -U username} .

\noindent
\textbf{\textit{Deleting a user}}

\noindent
Finally use \textit{userdel username} to delete the user. After deleting the
user remember to find and clean up their home directory or anything else thay
they have done.



% NOTE : Email Origins {{{

\tcolorboxnote
{

Emails originated from the ishan@linuxmachine sort of format , where the
username would follow the machine name and that is the convention that people
were used to in servers back in like the 70s.
}

% }}} END NOTE : email_origins

\subsectionend
% }}} END SUB-SECTION : users

% 	SUB-SECTION : filesystem {{{
\subsection{Filesystem}
\label{ssec:filesystem}

So when we are storing data it is essentially stored in binary bits (0 and 1).
But that is essentially just one large body of text in some storage device and
there is no way for us or the computer tell when the bits from one piece of data
end and where the bits of another begin. We need some sort of pre agreed on
system , to say : `` \textit{from `here' to `there' is one thing, and from
	`there' to `another place' is another thing} ''.

By separating the data into pieces and giving each piece a name, the data is
easily isolated and identified.

Since most of the naming conventions came about when paper-based management
systems and filing cabinets were still things that existed , each little group
of data started being called a ``file''. 

And since we are still dealing with old timer terminology , what do you do with
files ? You name them and organize them and put them into filing cabinets
according to some sort of system. Therefore the way these are named , sorted and
organized got called a file system.

So one thing that we know is that not everyone agrees on the logic of how things
should be organized. Some people organize clothes by color , others by season ,
others by occasion and so on. Similary there are various different ways to
organize and categorize the files inside our storage devices, which means there
are a bunch of differnt filesystems that we can use. The most common ones are :

% ITEMIZE : common_filesystems {{{

\begin{itemize}[noitemsep]

	\item FAT32
	\item EXT3
	\item EXT4
	\item NTFS
	\item ZFS
	\item HFS

\end{itemize}

% }}} END-ITEMIZE : common filesystems

There are also certain filesystems that are specific to certain mediums. Such as
the ISO 9660 filsystem which is only for CDs.

Its been said before , and I will say it again. Everything in linux is a process
and every process is manfisted as a file, so everything in linux is essentially
just a file. Which means that files and filesystems are very important in
understanding and dealing with any information and information manipulation
inside Linux.

The filesystems inside linux can be physical / real filesystems that the machine
is dealing with , or they can be virtual filesystems that the kernel / system
creates to make things easier to deal with. An example is the /proc filesystem
that deals with the storage and processing of inter-process signals.


% 		SUB-SUB-SECTION : logical_file_system {{{
\subsubsection{Logical File System}
\label{sssec:logical_file_system}

The logical file system is responsible for interaction with the user
application. It provides the application program interface (API) for file
operations — OPEN, CLOSE, READ, etc., and passes the requested operation to the
layer below it for processing. The logical file system "manage[s] open file
table entries and per-process file descriptors."[8] This layer provides "file
access, directory operations, [and] security and protection."[7]


\subsubsectionend
% }}} END SUB-SUB-SECTION : logical_file_system

% 		SUB-SUB-SECTION : virtual_filesystem {{{
\subsubsection{Virtual Filesystem}
\label{sssec:virtual_filesystem}


The second optional layer is the virtual file system. "This interface allows
support for multiple concurrent instances of physical file systems, each of
which is called a file system implementation."[8]

\subsubsectionend
% }}} END SUB-SUB-SECTION : virtual_filesystem

% 		SUB-SUB-SECTION : physical_filesystem {{{
\subsubsection{Physical Filesystem}
\label{sssec:physical_filesystem}

The third layer is the physical file system. This layer is concerned with the
physical operation of the storage device (e.g. disk). It processes physical
blocks being read or written. It handles buffering and memory management and is
responsible for the physical placement of blocks in specific locations on the
storage medium. The physical file system interacts with the device drivers or
with the channel to drive the storage device.[7] 

\subsubsectionend
% }}} END SUB-SUB-SECTION : physical_filesystem


\textbf{\textit{tmpfs}}


\textbf{\textit{Mounting}}

\textbf{\textit{udevd Daemon}}



\subsectionend
% }}} END SUB-SECTION : filesystem

% 	SUB-SECTION : process {{{
\subsection{Process}
\label{ssec:process}

A process ia basically an application , shell command or any program that is
``doing something'' in linux. It can be as complex as running a browser ,
running a vulnerability scanner , or as simple as running a shell command or
read a file. All of this will spawn a process.

A process is given its own share of memory do go about its business (it has its
own call stack and memory allocated for a heap). This is usually virtual memory
, so it might not be able to see absolute address that it is reading or writing
to. Instead it is given virtual addresses by the kernel.

\textbf{\textit{Kernel}}

Basically , a kernel is the thing that is sitting between a running process and
the physical hardware.  The kernel also keeps track of

% ITEMIZE : things_the_kernel_does {{{

\begin{itemize}[noitemsep]

	\item \textbf{Owner} : Which user is the owner of the process 

	\item \textbf{Parent Process} : Which process spawned the current process ,
		since all processes are born from other processes.

	\item \textbf{Priority} : Does it need to do work quickly (high priority) or
		can it chill for a while (low priority).

	\item \textbf{Memory Allocation} : the address spaces (files , network ports
		\ldots) that the process is allowed to use and what it is currently
		using.

\end{itemize}

% }}} END-ITEMIZE : things the kernel does

Each process has an ID , called the \textbf{\textit{Process ID}} or
\textbf{\textit{PID}}. This is a unique number assigned to each process. The
first process (PID 1) is always \textit{init}. It is the parent that gives birth
to all other processes on the system. The kernel spawns this process during boot
, and it runs all the startup scripts that setup linux. If any process dies ,
then all the children that it spawns are not killed , rather they are adopted by
init.

What init specifically does differs according to some distributions, and is
outside the scope of this text. Just remember it is the first process and it is
spawned by the kernel.

% NOTE : a note about init {{{

\tcolorboxnote
{

There are some virtual machines that think that they are real machines. Which
means that they will have thier own \textit{init} with PID 1, while the host
machine also has a similar \textit{init} with PID 1.

} 

% }}} END NOTE : a_note_about_init

There are two kinds of process permissions , namely id based permissions or
effective id based permissions. More formally UID (user id) define permissions
based on user privilege level of the user that spawns the process.  On the other
hand , the EUID (effective user id) , gives permissions more sparingly (or just
differently) based upon a foundation of the original permissions.

A good example why this is needed is if root spawns a process , and this process
goes crazy , we dont want it walking around in our system doing whatever it
wants with root privileges , because that pretty much means deth. So EUID gives
permissions different than the ones granted by user spawn.

Similarly there are also group ids GID and effective group ids EGIDs.

% 		SUB-SUB-SECTION : signals {{{
\subsubsection{Signals}
\label{sssec:signals}

Signals are what processes use to communicate with each other. You might also
already know them as software interrupts from your operating systems class.
Signals can be sent by the kernel as well when a process does something heinous
like dividing by or zero, but they are more commonly the kernel uses signals to
indicate the state of the system to the processes. Some common signals are
notifying the parent of the death of the child process, or notifying a process
about the availability and readiness of some hardware like a CD-drive.

As has been mentioned before everything in linux is a file. So what a signal
essentially is read and writing from files. A process has something to say to
another process so it will write to a file, and the other process will read fomr
this file. This `process' of reading and writing is called a signal.

The following are examples of signals :
~\footnote{man page is : man 7 signal}


% XTABULAR TABLE : process_signals {{{

\tabulartable
{\columnwidth}
{t}
{ll}
{

	SIG \textbf{HUP} & Hang up \\

	SIG \textbf{INIT} & interrupt \\
	SIG \textbf{QUIT} & quit \\
	SIG \textbf{ILL} & illegal instruction \\
	SIG \textbf{TRAP} & trace trap \\
	SIG \textbf{ABRT} & abort process \\
	SIG \textbf{BUS} & bus error \\
	SIG \textbf{FPE} & floating point exception \\
	SIG \textbf{KILL} & kill (cannot be ignored) \\
	SIG \textbf{USR1} & custom user defined signal \\
	SIG \textbf{SEGV} & Segmentation Fault / Violation \\
	SIG \textbf{USR2} & custom user defined signal \\
	SIG \textbf{PIPE} & Write to pipe fault  \\
	SIG \textbf{ALRM} & Alarm \\
	SIG \textbf{TERM} & Terminate (can be ignored by process) \\
	SIG \textbf{STKFLT} & Stack fault \\
	SIG \textbf{CHLD} & \makecell{Child process terminted \\ / interrupted / resumed.} \\
	SIG \textbf{CONT} & Continue executing after stop \\
	SIG \textbf{STOP} & Stop a process for later \\
	SIG \textbf{TSTP} & Temporaily stop \\
	SIG \textbf{TTIN} & tty in \\
	SIG \textbf{TTOU} & tty out \\
	SIG \textbf{URG} & urgent data aviable for read\\
	SIG \textbf{XCPU} & You have overused your CPU time. This is a warning to gracefully exit before SIGKILL shows up to bounce yo ass. \\
	SIG \textbf{XFSZ} & X-Filesize , Grew file larger than maximum allowed size. \\
	SIG \textbf{VTALRM} & Virtual alarm clock \\
	SIG \textbf{PROF} & Profling alarm clock \\
	SIG \textbf{WINCH} & window change , or when terminal / window manager changes size \\
	SIG \textbf{IO} & Input Output now possible \\
	SIG \textbf{PWR} & Power failure detected \\
	SIG \textbf{LOST} & Power lost , synonym for SIGPWR \\
	SIG \textbf{UNUSED} & Unused signal \\
}


% }}} End XTABULAR TABLE : process signals


\textbf{\textit{Sending signals from the keyboard}}

Signals may be sent from the keyboard. Several standard defaults are listed below. Default key combinations for sending interrupt signals can be defined with the stty command.


% tabular TABLE : common_singal_interrupts {{{

\tabulartable
{0.99\linewidth}
{t}
{ll}
{


Ctrl-C & Send SIG INT (Interrupt).  \\
&  By default, this causes a process to terminate.  \\

Ctrl-Z & Send SIG TSTP (Suspend).  \\
&  By default, this causes a process to suspend all operation. \\

Ctrl-\ & Send SIG QUIT (Quit).  \\
&  By default, this causes a process to terminate immediately and dump the
core. \\

Ctrl-T & Send SIG INFO (Info).  \\
&  By default, this causes the operating system to display information about
the command. Not supported on all systems. \\

}

% }}} End TABLE : Common Singal interrupts

To send a signal we can also use the signal number assigned to each signal as
opposed to typing / remebering the whole signal name. Similarly for the
processes themselves. Although getting the process ID will usually require a
grep. As examples :

% tabular TABLE : common_signal_commands {{{

\tabulartable
{0.99\linewidth}
{t}
{lll}
{

\textit{kill 15 firefox} & : & this will send SIGTERM to firefox\\

\textit{ps aux \textbar grep firefox} & : & Greps for the named process
~\footnote{
This particular grep command will always return at least 1 result , since a
process would be started to look for the other process containing the same name.
If firefox is actually running then we will see at least 2 processes in the
returned results. One for the actual process and the other for the search
process.
}

\\

\textit{kill 9 3460} &  : & this will send SIGKILL to process number 3460\\

}

% }}} End TABLE : common signal commands


\subsubsectionend
% }}} END SUB-SUB-SECTION : signals

% 		SUB-SUB-SECTION : process_states {{{
\subsubsection{Process : States}
\label{sssec:process_states}

Everything on the machine is basically some combination of calculations that
need to get done, in order to get some answer. This answer is the raison d'être
~\footnote{reason or justification for existence} for a process , and to
calculate it they all need CPU time. However not every process needs CPU time at
every given second. Which is why we have scheduling algorithms like round robin.
So for example if a process is waiting for keyboard input , or for something to
be read from the hard drive or CD-Drive then it doesnt need CPU time because it
can't really do much (or anything) with it anyway. So the processes can exist in
the following states :

% ITEMIZE : process_states {{{

\begin{itemize}[noitemsep]

	\item \textbf{Runnable} : The process has all the information and input it
		needs , and it can just get run (or be scheduled to run) the next time
		the CPU has any free time.

	\item \textbf{Sleeping} : It can be waiting for something.

	\item \textbf{Zombie State} : It has finished doing whatever it needed to do
		, but now it is just waiting to give back the information and then be
		killed.

	\item \textbf{Stopped State} : A process that may have been in the middle of
		doing something but the user or another process sent it SIGSTOP so it is
		waiting for SIGCONT to resume where it left off.

\end{itemize}

% }}} END-ITEMIZE : Process states

The processes also have a \textbf{\textit{niceness property}}, which basically
details how much they allows other processes to run , i.e.\ does this process
hog system resources or not. This is goverened by it's priority level, so the
higher the priority the more of a dick the process is going to be about sharing
system resources.

Linux processes have niceness ranger from -20 to 19 , where the higher the
number the lower the priority. So if something is -20 nice then it is ultra
critical and it needs to get run before everything else. This will also probably
cause your screen freeze because as much CPU time as possible is going to this
process , and the CPU wont even have enough time to refresh your screen.


\subsubsectionend
% }}} END SUB-SUB-SECTION : process_states

% 		SUB-SUB-SECTION : proc_filesystem {{{
\subsubsection{/proc Filesystem}
\label{sssec:proc_filesystem}

One of the reasons Linux is really popular as a server side operating system is
that we (specifically System Admnisistrators) can change the underlying values
of the operating systen while still keeping the server and the system up and
running.

Since we know that all actions on linux run as an underlying process , and we
also know that all things in linux are essentially files. This means that all
procsses are also represented as a file `somewhere' on the system.

This `somewhere' is the \textbf{/proc} folder. This folder is a virtual
filsystem which can be accessed by the system administrarors to view and change
underlying system values.

A /proc filesystem is not a real filsystem because it resides only in the
computer's memory and does not use and space on the hard disk (i.e.\ it is a
virtual filesystem). This filesystem is a map to the running kernel processes.

It is mounted in /proc during system initialization and has an entry in the
/etc/fstab file\footnote{fstab file is the operating systems system table}

If you ls /proc you will see a bunch of subdirs and files. Most of these have
numbers , and these numbers are actually PIDs for running processes. Other
directories deal with system hardware, networking settings , activity and memory
usage and so on.

If you list the files under /proc, you will find that all the files have a size
of 0 - this is because they are not really files and directories in the typical
sense. Basically these files are beaing created and updated on the fly.  This
also means that if you want to `see' one of these files you can't just cat them.

% NOTE : warning {{{

\tcolorboxnote
{

\textbf{\textit{WARNING}} \\

Do not use `cat' on /proc/kcore as this is a special file which is an
image of the running kernel's memory at that particular moment - cat'ting this
file will leave your terminal unusable.\\

}



% }}} END NOTE : warning

Another thing to keep in mind is that a lot of the information contained in
these files is either raw data , and sometimes even just binary. So looking
through these files for useful information is not always easy. This information
if often meant to parsed and interpreted through other programs , but it is
still good to know where the programs like top and htop are getting their
information from, and if you are a madlad read it directly from source.

Within each process's (es) folder there will be some common files that will be
of interest :

% ITEMIZE : proc_init_files {{{

\begin{itemize}[noitemsep]

	\item \textbf{cmd} : Contains the command that the process is currently
		executing

	\item \textbf{cmdline} : Contains information about how the process was
		called , i.e.\ the command line argument that was inputted to run the
		current process 

	\item \textbf{cwd} : Contains information about the directory within which
		the process is opertaing from

	\item \textbf{environ} : shows the environment variables

	\item \textbf{exe} : a link to the executable

\end{itemize}

% }}} END-ITEMIZE : proc init files

Some of the key files in the top-level directory are as follows :

% ITEMIZE : proc_key_directories {{{

\begin{itemize}[noitemsep]

	\item \textbf{/proc/interrupts} - View IRQ settings

	\item \textbf{/proc/cpuinfo} - Information about the system's CPU(s)

	\item \textbf{/proc/dma} - Direct Memory Access (DMA) settings

	\item \textbf{/proc/ioports} - I/O settings.

	\item \textbf{/proc/meminfo} - Information on available memory, free memory,
		swap, cache memory and buffers. You can also get the same information
		using the utilities free and vmstat.

	\item \textbf{/proc/loadavg} - System load average

	\item \textbf{/proc/uptime} - system uptime and idle time. Can also be
		obtained using utility uptime.

	\item \textbf{/proc/version} - Linux kernel version, build host, build date
		etc. Can also be obtained by executing `uname -a`.

\end{itemize}

% }}} END-ITEMIZE : proc key directories

Beneath the top-level /proc directory are a number of important subdirectories containing files with useful information. These include :

% ITEMIZE : proc_subdirectories {{{

\begin{itemize}[noitemsep]

	\item \textbf{/proc/scsi} - Gives information about SCSI devices

	\item \textbf{/proc/ide} - information about IDE devices

	\item \textbf{/proc/net} - information about network activity and
		configuration

	\item \textbf{/proc/sys} - Kernel configuration parameters. The values in
		files in this directory are editable by root, which I will further
		explain below.

	\item \textbf{/proc/} - information about process PID.

\end{itemize}

% }}} END-ITEMIZE : proc subdirectories

\subsubsectionend
% }}} END SUB-SUB-SECTION : proc_filesystem

\subsectionend
% }}} END SUB-SECTION : process

\onecolumn

% TBOX TABLE : commands_overview {{{

\tcolorboxtable
{ Commands Overview }
{ \label{table:commands_overview} }
{ [width=\linewidth] }
{

\tabulartable
{\linewidth}
{t}
{lll}
{

\rowcolor{table-subtopic}
\multicolumn{3}{c}{ \textbf{Navigation \& Creation \& Destruction}}  \\
\midrule
\rowcolor{white} \multicolumn{3}{c}{ }  \\

\textbf{clear}   & : & clear the terminal \\
\textbf{pwd}     & : & Print working Directory \\
\textbf{ls}      & : & list everything in current dir\\
                 & : & can also list everything the dir if path is specified\\
\textbf{cd}      & : & change directory \\
\textbf{touch}   & : & creates an empty file \\
\textbf{mkdir}   & : & makes a directory inside current directory with given name \\
\textbf{rmdir}   & : & removes directory inside current directory with given name , but only works if specified dir is empty \\
\textbf{rm}      & : & remove (delete) the file , will also work with directories , even if they are non empty \\
\textbf{mv}      & : & move the file to the specified destination (Regex works). \\
                 &   & Is also used to rename files. Just move the file into the same folder with a different name \\

\rowcolor{white} \multicolumn{3}{c}{ }  \\
\rowcolor{table-subtopic} \multicolumn{3}{c}{ \textbf{Manipulation}}  \\
\midrule
\rowcolor{white} \multicolumn{3}{c}{ }  \\


\textbf{cat}      & : & prints out everything in the file \\
\textbf{grep}     & : & Allows STDOUT manipulation , see\refssec{grep} \\
\textbf{less}     & : & allows pagination of STDOUT \\
\textbf{cut}      & : & remove section from STDOUT based on args\\
\textbf{sort}     & : & sort the output produced on STDOUT \\
\textbf{wc}       & : & gives char / byte / newline count of STDOUT \\
\textbf{uniq}     & : & removed repeated lines from STDOUT\\
\textbf{tail}     & : & print the last 10 lines of each file to STDOUT \\
\textbf{wc}       & : & print the number of lines / character / bytes in a file depending on args \\
\textbf{\textbar} & : & pipes all STDOUT from the first command into STDIN for second command \\

\textbf{\&\&}     & : & runs command 2 iff command 1 runs succesfully ,i.e.\ returns true\\

\rowcolor{white} \multicolumn{3}{c}{ }  \\
\rowcolor{table-subtopic} \multicolumn{3}{c}{ \textbf{System}}  \\
\midrule
\rowcolor{white} \multicolumn{3}{c}{ }  \\

\textbf{man}     & : & shows manual for command , as well as all avialable flags\\
\textbf{ps}      & : & lists all the running processes \\
\textbf{who}     & : & shows list of logged on users \\
\textbf{w}       & : & shows list of logged on users and their processes\\
\textbf{.}       & : & indicates current working directory \\
\textbf{..}      & : & indicates parent of current working directory \\


\textbf{find}     & : & \\
\textbf{locate}   & : & \\
\textbf{awk}      & : & \\
\textbf{ack}      & : & \\
\textbf{rename}   & : & rename files \\

\rowcolor{white} \multicolumn{3}{c}{ }  \\
\rowcolor{table-subtopic} \multicolumn{3}{c}{ \textbf{Users}}  \\
\midrule
\rowcolor{white} \multicolumn{3}{c}{ }  \\

\textbf{useradd}  & : & Add a new user \\
\textbf{userdel}  & : & Delete an existing user \\
\textbf{usermod}  & : & Modify existing user somehow \\
\textbf{newusers} & : & create multiple users simultaneously \\
\textbf{passwd}   & : & set the password for a user \\

}
}

% }}} End TBOX TABLE : commands_overview

% TBOX TABLE : commands_overview {{{

\tcolorboxtable
{ Commands Overview 2 }
{ \label{table:commands_overview_2} }
{ [width=\linewidth] }
{

\tabulartable
{\linewidth}
{t}
{lll}
{

\rowcolor{table-subtopic} \multicolumn{3}{c}{ \textbf{Process \& Signals}}  \\
\midrule
\rowcolor{white} \multicolumn{3}{c}{ }  \\

\textbf{kill}   & : & tries to terminate a process , default SIGTERM (15) \\
\textbf{kill} 9  & : & sends SIGKILL to brutally murder a process where it
stands\\
\textbf{killall} & : & SIGTERM all processes with the specified name\\
\textbf{pkill} & : & kill all processes based on name or other attribute\\
\textbf{pgrep} & : & kill all processes based on name or other attribute\\
\textbf{pkill -u} & : & kill all processes for a certain user\\
\textbf{nice} & : & run a process with specified niceness value\\
\textbf{renice} & : & Change the niceness of a running process \footnote{be very
very careful here. If you give a process too high importance you might not even
be able to kill it since your kill signal will get less CPU time than the
process itself. It might even take minutes just to renice the thing again to be
able to kill it. Its just not a good time.}\\

\textbf{strace} & : & gives a detailed view of system processes\\

\rowcolor{white} \multicolumn{3}{c}{ }  \\
\rowcolor{table-subtopic} \multicolumn{3}{c}{ \textbf{Network}}  \\
\midrule
\rowcolor{white} \multicolumn{3}{c}{ }  \\

\textbf{ifconfig}   & : &  \\

}
}

% }}} End TBOX TABLE : commands_overview


\twocolumn
\sectionend
% }}} END SECTION : linux_sysadmin_course

% SECTION : standards {{{
\section{Standards}
\label{sec:standards}

% 	SUB-SECTION : posix {{{
\subsection{POSIX}
\label{ssec:posix}


\subsectionend
% }}} END SUB-SECTION : posix

% 	SUB-SECTION : ansi {{{
\subsection{ANSI}
\label{ssec:ansi}


\subsectionend
% }}} END SUB-SECTION : ansi

% 	SUB-SECTION : bsd {{{
\subsection{BSD}
\label{ssec:bsd}


\subsectionend
% }}} END SUB-SECTION : bsd

\sectionend
% }}} END SECTION : standards

% SECTION : conventions {{{
\section{Conventions}
\label{sec:conventions}
\parindent=0em


All file names are case sensitive. So filename vivek.txt Vivek.txt VIVEK.txt all
are three different files.

You can use upper and lowercase letters, numbers, “.” (dot), and “\_”
(underscore) symbols.

You can use other special characters such as blank space, but they are hard to
use and it is better to avoid them.

In short, filenames may contain any character except / (root directory), which
is reserved as the separator between files and directories in a pathname. You
cannot use the null character.

No need to use . (dot) in a filename. Some time dot improves readability of
filenames. And you can use dot based filename extension to identify file. For
example:

.sh = Shell file

.tar.gz = Compressed archive

Most modern Linux and UNIX limit filename to 255 characters (255 bytes).
However, some older version of UNIX system limits filenames to 14 characters
only.

A filename must be unique inside its directory. For example, inside /home/vivek
directory you cannot create a demo.txt file and demo.txt directory name.
However, other directory may have files with the same names. For example, you
can create demo.txt directory in /tmp.

Linux / UNIX: Reserved Characters And Words

Avoid using the following characters from appearing in file names:

%    /
%    >
%    <
%    |
%    :
%    &

Please note that Linux and UNIX allows white spaces, <, >, |, \, :, (, ), \&, ;,
as well as wildcards such as ? and *, to be quoted or escaped using \ symbol. 

. is used to separate a filetype extension, e.g. foo.txt.

- or \_ is used to separate logical words, e.g. my-big-file.txt or sometimes
my\_big\_file.txt. - is better because you don't have to press the Shift key (at
least with a standard US English PC keyboard), others prefer \_ because it looks
more like a space.

So if I understand your example, backup-part2-random or backup\_part2\_random
would be closest to the normal Unix convention.

CamelCase is normally not used on Linux/Unix systems. Have a look at file names
in /bin and /usr/bin. CamelCase is the exception rather than the rule on Unix
and Linux systems.

(NetworkManager is the only example I can think of that uses CamelCase, and it
was written by a Mac developer. Many have complained about this choice of name.
On Ubuntu, they have actually renamed the script to network-manager.)


\textbf{\textit{Absolute vs.\ Relative Paths}}


Absolute path means full true path to get to the thing you want to get to. This
means that all aboslute paths must start with / i.e.\ they must all start in the
root directory. So the absolute path to get to file in your Documents directory
would be :

/home/username/Documents/

Relative paths are based on where you are currently in the system. As an example
from your root directory if you cd into Documents, and then run ls. The ls
command will be run relative to where you are and will therefore show the
contents of the folder that you are in ,i.e.\ of Documents. Which means to get
to the same file in the Documents directory as in the absolute path explanation
, the path is now (if we are sitting on home) :

Documents/

Generally you want to be using Absolute paths as much as possible. This avoids
confusion and works on all linux distributions. The reason I say this is because
there are some commands (like ifconfig) that work based on relative paths in
some distributions like Ubuntu , but on others like Debian they will not work
unless the full absolute path is specified , i.e.\ , /sbin/ifconfig , which
would work on both systems.

There are also security reasons to use absolute paths since whenever you use
relative paths to access / run some software some other process / program has to
deal with figuring out the absolute path to it and the returning the relevant
information. The program responsible for this translation might return the
correct information or run the right software but it also has the ability to
store any information that was passed as arguments to the command we are running
since it is a literal man in the middle on your system.

\sectionend
% }}} END SECTION : conventions


% :)

% ----------------------------------------------------------------------------- 
\bibliography{1_bibliography/references}
\end{document}
% =============================================================================
% - EOF - EOF - EOF - EOF - EOF - EOF - EOF - EOF - EOF - EOF - EOF - EOF -
% =============================================================================

