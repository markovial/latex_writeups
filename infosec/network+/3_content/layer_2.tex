
% SECTION : layer_2_data_link {{{
\section{Layer 2 : Data Link}
\label{sec:layer_2_data_link}
\parindent=0em

Layer 2, the Data Link Layer

This layer receives data from the physical layer and compiles it into a transform form called framing or frame. The principal purpose of this layer is to detect transfer errors by adding headers to data packets.

The protocols are used by the Data Link Layer include: ARP, CSLIP, HDLC, IEEE.802.3, PPP, X-25, SLIP, ATM, SDLS and PLIP.

% 	SUB-SECTION : switches {{{
\subsection{Switches}
\label{ssec:switches}


% repeater {{{

repeater : gets a signal , regenerates it and sends it on. Used to extend
range of networks , since cables and topologies have maximum lengths they can
reach. Originally known as a digital regenerator since it only works for digital signals.
the analog equivalent is called a analog amplifier. Reapeaters have an advantage
over amplifiers since the digital signal is regenerated back to original quality
and all noise is removed from the signal. Whereas amplifiers would only boost
whatever analog signal they recieve which includes the noise.

multi-port repeater : this is also known as a hub

% }}}

% hub {{{

% }}}

% switches {{{

bridge : filter by mac between hubs by using MAC table

multilayer switches
layer 3 switch
content switch

frame switching

no loops

spanning tree protocols

flooding for unknown MACS


Switches can be connected to each other over any port using a crossover cable.
Switches can also have an uplink port , which means that this particular port
can be configured ( by pressing a button ) to either be a pre crosslink port or
a normal port. As a crosslink port the switch would treat the plugged in cable
as a cross link which means that we would use a straight through cable because
the switch is doing the crossover for us. If we use the uplink port on the
normal setting then that means we should be using a crossover cable to connect
our other switch. Nowwdays most switches have Auto sensing ports , which will
self configure to the type of cable attached so we dont really have to worry
about all of this straight through and crossover business. What this also
means is that everyone basically just uses straight through TIA 568A standard
cables.

Switching / Bridging Loops : These occur when we have a loop between network switches.
If this is allowed to exist without fixing it then the data will continue to
loop around inside the network causing lots and lots of collisons. There was a
protocoll developed exclusively to avoid this kind of shit. It was called the
spanning tree protcol (STP). STP is built into the switches. What happends
when switches are first connected to each other is that one of them
automatically becomes the `root switch'. This root switch will watch any of the
data that goes through , and upon detection of a bridge loop , it will
disconnect one of its ports that is connected to any one of the switches. This
means that the loop is broken , but all the switches will keep functioning and
will remain connected to each other.

Flood Protection : Works similar to the STP , where when a computer or network
device is flooding the network with traffic ( usually because they are trying a
layer 2 DDoS attack) , then the port connected to that machine will get turned
off by that switch.

console Port : Console ports allow you to go in and manage the switches. These
require rollover (a.k.a Yost cable). THe better idea is that to just use SSH to
this instead of physically connecting.

switch port vs ports : Switch Ports do not use IP adresses or work with Layer 3

root guard :

BPDU Guard : A Cisco method allowing only non-switch devices to connect to the
switch. If there is another switch that is plugged into a port that is marked
with BPDU guard , then the port automatically turns itself off , and can only be
turned back on by an administrator logging in and reenabling it.


DHCP Snooping : Similar to BPDU guard , we can specify certain ports in the
swith as ports that are communicating with a DHCP server. This means that any
other port on the switch that is not the certified DHCP port , will be
automatically shut down if it senses a DHCP server plugged into it. This
prevents things like rouge DHCP servers.


Autosensing in the switch will cause the switch port to re-wire into s crossover
configuration. Straight-through cables can connect switches together. Switch
ports work in full-duplex mode when connected together with either
straight-through or crossover cables. A switch cannot rewire the terminating
connector on a cable, only the internals of the port.

% }}}


\subsectionend
% }}} END SUB-SECTION : switches

% 	SUB-SECTION : mac_eui_addresses {{{
\subsection{MAC / EUI Addresses}
\label{ssec:mac_eui_addresses}

media access control

EUI-48 is the link-layer address of ethernet devices (used almost nowhere else) EUI-64 is the link-layer address used pretty much everywhere else.

Historically, both EUI-48 and MAC-48 were concatenations of a 24-bit OUI
(Organizationally Unique Identifier) assigned by the IEEE and a 24-bit extension
identifier assigned by the organization with that OUI assignment (NIC). The
subtle difference between EUI-48 and MAC-48 was not well understood; as a
result, the term MAC-48 is now obsolete and the term EUI-48 is used for both
(but the terms “MAC” and “MAC address” are still used).

In other words, EUI-48 and the MAC number of a device represent the same thing!
Usually it is represented in 12 hex (e.g. 0023.a34e.abc9), equivalent to 48 bits
or 6 bytes.

\subsectionend
% }}} END SUB-SECTION : mac_eui_addresses

% 	SUB-SECTION : protocols {{{
\subsection{Protocols}
\label{ssec:protocols}

ICMP : works on layer 2 (internet) of TCP/IP and layer 3 (network) on the OSI 


ARP : If two systems are to communicate using IP, the host sending the packet
must map the IP address of the destination host to the hardware address of the
destination host. The Address Resolution Protocol (ARP) is the protocol that
enables this process of local address discovery to take place. Hosts broadcast
ARP messages onto the local network to find out which host MAC address "owns" a
particular IP address. If the destination host responds, the frame can be
delivered. Hosts also cache IP:MAC address mappings for several minutes to
reduce the number of ARP messages that have to be sent. 

RARP



\subsectionend
% }}} END SUB-SECTION : protocols

% 	SUB-SECTION : vlans {{{
\subsection{VLANs}
\label{ssec:vlans}

VLAN : A VLAN (Virtual Local Area Network) splits one broadcast domain into two
or more smaller broadcast domains. In a VLAN , the computers , servers ,a dn
other network devices are logically connected regardless of their physical
location. These are super upseful when you dont want traffic from devices that
are physically on the same network to intermingle (e.g. accounting shouldnt be
able to connect to shipping computers etc ...). To do this we basically break up
the physical network into smaller distinct virtual networks. You can designate
VLANs on a switch by specifiying ports or ranges of ports. Vlans can also help
with traffic management because of the smaller broadcast domain , we can
alleviate the broadcast traffic on the network.


Inter VLAN Routing : If we set up a VLAN we have multiple broadcast domains.
This means that we want the traffic between these domains to remain segregated.
We want this on the switch side , as well as the router side. So one solution is
that we can just plug in one port from each VLAN into the router to create
indipendent default gateways. The problem with this approach is that you are
going to have to keep adding more router cause they usually dont ship with a
whole bunch of ports ( usually only 2 ). So the solution is something called
inter VLAN routing.  This is a virtualization of the functions of the router. So
we are basically  creating new virtual routers to interact with these virtual
LANS.


Trunk Ports : A trunk port is assigned to carry traffic from all the VLANs
connected to a specific network switch.

802.1q adds a tag to the ethernet frame header , labeling it as belonging to a
certian VLAN.

Trunk Line

Port Bonding / NIC Teaming / Link Aggregation / Channel Bonding / Port Trunking
/ NIC Trunking : It is the process of taking two ports and bonding them
together such that they act as on higher speed port. In cisco switches at least
, the way we do this is by creating groups and assigning each individual port we
want bonded onto that group. Use LACP for the trunking protocol and make sure at
least one of the ports is set as active. If both ports are set as passive it
wont work.

LACP

Port Mirroring : Port Mirroring enables the traffic flowing through one port to
be monitored on another port. This feature enables administrators to remotely
insepect traffic from a suspicious machine. It is configured on a switch by
providing a source port and a destination port.


\subsectionend
% }}} END SUB-SECTION : vlans

% 	SUB-SECTION : ppp {{{
\subsection{PPP}
\label{ssec:ppp}

% 		SUB-SUB-SECTION : aaa {{{
\subsubsection{AAA}
\label{sssec:aaa}

ACL : Access Control List

MAC : Mandatory Access Control : uses specific labels to restrict access to
individual files
DAC : Discretionary Access Control : gives / restricts acess to users
RBAC : Role Based Access Control : uses groups that are then assigned to users
to grant permissions

AAA : Authentication , Authorization , Accounting


\subsubsectionend
% }}} END SUB-SUB-SECTION : aaa

% 		SUB-SUB-SECTION : radius {{{
\subsubsection{RADIUS}
\label{sssec:radius}



\subsubsectionend
% }}} END SUB-SUB-SECTION : radius

% 		SUB-SUB-SECTION : tacacs {{{
\subsubsection{TACACS+}
\label{sssec:tacacs}



\subsubsectionend
% }}} END SUB-SUB-SECTION : tacacs

\subsectionend
% }}} END SUB-SECTION : ppp

% 	SUB-SECTION : kerberos {{{
\subsection{Kerberos}
\label{ssec:kerberos}
Single Sign On
Kerberos : Handles authentication and authorization for wired networks. It
relies heavily on timestamps
KDC ( Key Distribution Center )
AS ( Authentication Service )
TGS ( Ticket Granting Service ) 
TGT ( Ticket Granting Ticket / Token )

\subsectionend
% }}} END SUB-SECTION : kerberos

% 	SUB-SECTION : layer_2_vpns {{{
\subsection{Layer 2 VPNs}
\label{ssec:layer_2_vpns}

When it comes to the Data Link Layer VPNs, there are two private networks which are linked or connected on to the Layer 2 of the OSI model while utilizing a suitable protocol like Frame Relay or ATM.

However, these two procedures simultaneously give off a quite suitable way towards the development of VPNs. These layers are often found to be expensive as they require dedicated Layer 2 pathways for its creation and functioning.

Frame Relay and ATM protocols, both of these protocols usually don’t provide encrypting mechanisms. Instead, these two mechanisms are only responsible for allowing the network traffic for the segregation based on how Layer 2 is connected and how it relates to it.

To wrap it up, we could say that for an extra layer of security and protection you would need to develop some encrypting mechanism in its place.
\subsectionend
% }}} END SUB-SECTION : layer_2_vpns

\sectionend
% }}} END SECTION : layer_2_data_link
