
% SECTION : layer_4_transport {{{
\section{Layer 4 : Transport}
\label{sec:layer_4_transport}
\parindent=0em

Layer 4, the Transport Layer

The transport layer works on two determined communication modes: Connection oriented and connectionless. This layer transmits data from source to destination node.

It uses the most important protocols of OSI protocol family, which are: Transmission Control Protocol (TCP), UDP, SPX, DCCP and SCTP.
% 	SUB-SECTION : proxy_servers {{{
\subsection{Proxy Servers}
\label{ssec:proxy_servers}

A proxy server, also known as a "proxy" or "application-level gateway", is a
computer that acts as a gateway between a local network (for example, all the
computers at one company or in one building) and a larger-scale network such as
the internet. Proxy servers provide increased performance and security. In some
cases, they monitor employees' use of outside resources.

A proxy server works by intercepting connections between sender and receiver.
All incoming data enters through one port and is forwarded to the rest of the
network via another port. By blocking direct access between two networks, proxy
servers make it much more difficult for hackers to get internal addresses and
details of a private network.

Some proxy servers are a group of applications or servers that block common
internet services. For example, an HTTP proxy intercepts web access, and an SMTP
proxy intercepts email. A proxy server uses a network addressing scheme to
present one organization-wide IP address to the internet. The server funnels all
user requests to the internet and returns responses to the appropriate users. In
addition to restricting access from outside, this mechanism can prevent inside
users from reaching specific internet resources (for example, certain websites).
A proxy server can also be one of the components of a firewall.

Proxies may also cache web pages. Each time an internal user requests a URL from
outside, a temporary copy is stored locally. The next time an internal user
requests the same URL, the proxy can serve the local copy instead of retrieving
the original across the network, improving performance.

Do not confuse a proxy server with a NAT (Network Address Translation) device. A
proxy server connects to, responds to, and receives traffic from the internet,
acting on behalf of the client computer, while a NAT device transparently
changes the origination address of traffic coming through it before passing it
to the internet.

For those who understand the OSI (Open System Interconnection) model of
networking, the technical difference between a proxy and a NAT is that the proxy
server works on the transport layer (layer 4) or higher of the OSI model,
whereas a NAT works on the network layer (layer 3).

A proxy is any device that acts like an intermedieary between two different
devices tha are in a session. This means that a proxy will sit between a client
and a server when they are talking.

Proxies are all going to be application specific. As an example if we want to
channel HTTP traffic through the proxy then we would have a web proxy. Other
types of proxies , all application specific are :

- Web Proxy
- FTP Proxy
- VoIP Proxy


Transparent Proxy


% Forward Proxy Server {{{

A forward proxy sits behind the network firewall and in front of the client. In
this case the client would be aware of the existence of this proxy. The client
would speak to the proxy and then the proxy after doing whatever it is doing
would forward the clients request to whichever server that the client wanted to
talk to.

Forward Proxies can be a dedicated box , or a peice of software that is running
on any computer somewhere.

Most forward proxy servers would act like firewalls , by providing content
filtering , ad blocking and stuff.

% }}}

% Reverse Proxy Server {{{

% }}}

This is a complete reverse of a forward server. In this case the proxy will
represent the actual server that we are communicating with. This means that the
client will send a request , the proxy server will intercept the request , and
send that request to the actual server on behalf of the client , the server
responds not to the client , but to the request of the proxy server. The proxy
then returns whatever information was requested to the original client.

The main functions of reverse proxy servers is to protect the server instead of
the client. Therefore they have features such as :

- high security
- handle DoS attacks
- load balancing
- caching
- encryption acceleration

\subsectionend
% }}} END SUB-SECTION : proxy_servers

% 	SUB-SECTION : tcp {{{
\subsection{TCP}
\label{ssec:tcp}

NCP : Network Control protocol , precursor to tcp

TCP Acceleration : TCP acceleration is the name of a series of techniques for
achieving better throughput on a network connection than standard TCP achieves,
without modifying the end applications. It is an alternative or a supplement to
TCP tuning. 

\subsectionend
% }}} END SUB-SECTION : tcp

% 	SUB-SECTION : udp {{{
\subsection{UDP}
\label{ssec:udp}


\subsectionend
% }}} END SUB-SECTION : udp

% 	SUB-SECTION : ports_port_numbers {{{
\subsection{Ports / Port Numbers}
\label{ssec:ports_port_numbers}

Yes, its the 4th layer (transport) of the 7 layers for the OSI model.

Reading your last post/question, if you're also asking about the TCP/IP model,
it's the 3rd layer (transport) of the 4 layers for that model.

Port numbers, in IP, are used by both TCP and UDP. Port numbers all quick
"sorting" of received packets, to processes that want them. Some applications
have been "assigned" specific port numbers. For example, HTTP has assigned to it
port 80. So, when a client wants to contact a HTTP server, its uses destination
port of 80 and a source port unique to the process making the request. This
allows the receiving host to send any received packets with a destination of
port 80 to the processes "listening" for those packets, which if there is one,
would normally be a HTTP server process.

When the HTTP server responds, it uses the client's source port as the reply's
destination port and it might use port 80 for the reply packet's source port.
This allows the original client to forward the port quickly to the process that
made the request.

Although many applications have "assigned" ports, applications might use other
port numbers. I.e. the port number doesn't control an application, it's just a
convenience. For example, you could configure a HTTP server to "listen" on port
8080. Now, either the client needs to know that too, or it would need to send
its HTTP request to all possible 65K port numbers.

\subsectionend
% }}} END SUB-SECTION : ports_port_numbers


DCCP
SCTP
RSVP


\sectionend
% }}} END SECTION : layer_4_transport
