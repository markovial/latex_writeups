
% SECTION : layer_6_presentation {{{
\section{Layer 6 : Presentation}
\label{sec:layer_6_presentation}
\parindent=0em

Layer 6, the Presentation Layer

The functions of encryption and decryption are defined on this layer. It ensures that data is transferred in standardized formats by converting data formats into a format readable by the application layer.

The following are the presentation layer protocols: XDR, TLS, SSL and MIME.

% 	SUB-SECTION : dhcp {{{
\subsection{DHCP}
\label{ssec:dhcp}

DHCP : gives a computer a dynamic ip , subnet mask and gateway from its scope

DHCP Reservation : asks for specific mac to be given specific IP everytime from
DHCP server , usually given to network printers and servers , that need to
maintain a constant IP

DHCP Lease : amount of time an IP is assigned to a computer

DHCP Relay : 

BootP

DHCP Client

DHCP servers can be included within the routers you buy , or they can be special
software that is sitting on your system. Most commonly however a DHCP server is
an actual computer server sitting out there on the network , whose only job it
is to be a computer running the DHCP software.

DHCP Discover
DHCP Offer
DHCP Request
DHCP Acknowledge

You want to make sure that each broadcast domain only has one DHCP server
running, because otherwise if you send out a broadcast you can get back
conflicting information from two different DHCP servers that are operating on
the same network. This is one of those so called "bad things". This also means
that the DHCP Server has to be within the broadcast domain , i.e. within your
LAN. It cannot be outside the network.

DHCP Relay
DHCP Server Scope : When you are creating your own DHCP server , one of the
first things you are going to want to do is set the scope. Which means you are
going to set the starting IP address , and an end IP address. As an example , 

Start IP address : 192.168.15.100
End IP address : 192.168.15.105

passes out a total of 6 IP Adresses
In addtion you will also decide which subnet to pass out , as well as
exclusions. This basically means that if there are any special IP adresses
within the scope that you specified earlier , you can specify them and the
server will not hand them out when some device is making a DHCP request. Another
thing you will specify is the lease duration. Basically how long will the host
that got a specific IP address be along to keep it for. On most Windows DHCP
servers , the default is something like 8 days. But if you are in something like
a coffe shop or some other public facing network , you might want to reduce this
setting down to just a couple of hours.


Rogue DHCP Server : If you have a DHCP server running on the local network and
you are not getting an APIPA , and you have an IP address which is different
than what you know yours to actually be , then that means that you have a rogue
DHCP server.  Rogue DHCP servers can assign incompatible IP addresses to hosts
on a network making them unable to communicate with other hosts or the Internet.
Rogue DHCP servers cause IP address incompatibilities or worse - they do not
increase network performance by either increasing DHCP assignment speeds or the
size of IP address pools. The existence of a DHCP server, rogue or approved,
ensures that hosts will not generate APIPA addresses.

WAN DHCP 

APIPA : automatic private IP address assignment ,  if DHCP cannot be reached
then all computers post Windows 98 will assign an IP to themselves which looks
like 169.254.0.0. We can still use the APIPA address to communicate on the
internal network. We cannot however use it to connect to a machine over the
internet.

Dynamic IP addressing : gets IP from DHCP (dynamic host configuration protocol )
, the dhcp has a pool of IPs that it can lease to a particular device for
certain periods of time ,

\subsectionend
% }}} END SUB-SECTION : dhcp

% 	SUB-SECTION : dns_server {{{
\subsection{DNS Server}
\label{ssec:dns_server}

% 		SUB-SUB-SECTION : dns_hierarchy {{{
\subsubsection{DNS Hierarchy}
\label{sssec:dns_hierarchy}


Host File : Before DNS Servers existed , computers used to use something called
a host file to resolve domain names to ip addresses. This was basically a text
file containing a list of domains and ips. Even though DNS Servers can do
everything a host file was supposed to do,  the hosts files still exist on every
computer that uses TCP/IP. What is even more interesting is that the records in
the host file take precedence over any DNS server that you might be using.
The one job that Domain Name System Servers (DNS Servers) have  is
to resolve resource names to IP addresses. In the DNS, the clients are called
resolvers because the are requesting a resolution of domain name to an
IP-Address and the servers are called name servers.

DNS databases are distributed because no one DNS server holds all possible DNS
records. This would be far too much information for a single server to store.
Instead if the particular record that the client is asking for does not exist in
this particular DNS servers table , then it will just ask some other DNS server
for the information. The way they figure out which one of the numerous DNS
servers to ask for this information is by looking at the requested domain name.
Then they use a tree structure where just below the root are a set of Top Level
Domains (TLD) that define broad classes of entities (.com , .gov , \ldots) or
national authorities (.uk , .ca \ldots). Within these top level domains,
companies, universities, non-profits, governments, or even just one guy can all
register individual domains.
\subsubsectionend
% }}} END SUB-SUB-SECTION : dns_hierarchy

% 		SUB-SUB-SECTION : lookup_zones {{{
\subsubsection{Lookup Zones}
\label{sssec:lookup_zones}

Lookup Zone : When we create our own DNS server for a LAN , we need to input the
domain to ip mappings in a table that will allow the server to resolve requests.
These records in the table are called lookup zones.

There are two different types of zones ( or requests ) that you can make to your
local Authoritative DNS server :

- Forward Lookup Zone : This basically resolves the domain name to an ip
address. The server will query its table for some string "name" and then return
the address in the corresponding "data" column in the table.

 - Reverse Lookup : The other type of request we can make from a DNS server is
 to ask for domain names , based on provided IP addresses.  Unfortunately, it is
 a limitation by design that DNS server cannot just lookup at the value on
 “Data” column to find the associated “Name” value. To fix this we store the ip
 addresses as regular "names" in the table and the corresponding "data" column
 would then contain the domain name. So a lookup for a domain based on provided
 ip address is called a reverse lookup.  Reverse Lookup zones are useful for
 mail servers.

\subsubsectionend
% }}} END SUB-SUB-SECTION : lookup_zones

% 		SUB-SUB-SECTION : dns_records {{{
\subsubsection{DNS Records}
\label{sssec:dns_records}

Any computer holding records for a part of the namespace is said to be a name
server. Name servers that contain the requested resource records for a
particular namespace are said to be authoritative. If they are not authoritative
for a namespace, they will have pointers to other name servers which might be
authoritative.

Resolvers are software programs running on client computers. For example, name
resolution is a critical part of web browsing, so web browser software will
implement a resolver.
Authoritative (Local) DNS Servers : One of the few reasons to deal with
implmenting or at least spinning up your own DNS server is to allow computers on
the local network to talk to each other. Usually by convention the domains that
are supposed to be used on LANs are marked .local as opposed to .com / .edu /
.org or something like that. The name server ( local DNS server ) that resolves
local domain names is called an Authoritative DNS Server. If we have multiple
DNS servers on a LAN , then one of them is designated the primary one we should
attempt to use. This sever is called the Start of Authority. The rest of the
local DNS servers will all just be called Name servers. It can in addition
resolve your nomal website domains like google.com by sending an upstream
request from other internet DNS servers.  To be able to resolve these domain
names into ip addresses , we need a table that holds these records. These
records are called lookup zones.  If not then we can just statically enter the
domains and the ips. If we are using IPv4 records that is called a "A" record ,
and if we are using IPv6 then it is reffered to as a "AAAA" record. We can also
have CNAME ( Canonical Names ) and are used as aliases for the host names.
MX Record
SRV Record (Service Location) : 

TXT Records
DKMI
SPF


\subsubsectionend
% }}} END SUB-SUB-SECTION : dns_records

% 		SUB-SUB-SECTION : ddns {{{
\subsubsection{DDNS}
\label{sssec:ddns}

Dynamic DNS : We can either manually go into the DNS servers we have created and
enter the records into our table , or we can get this information dynamically. 
If we are using DHCP to get our IP addresses , then the DNS server would need to
be configured to get the table records from the DHCP server. This is called DDNS
or dynamic DNS.

 Enables you to use a DHCP-assigned IP address for connection.


\subsubsectionend
% }}} END SUB-SUB-SECTION : ddns


\subsectionend
% }}} END SUB-SECTION : dns_server

% 	SUB-SECTION : protocols {{{
\subsection{Protocols}
\label{ssec:protocols}


\subsectionend
% }}} END SUB-SECTION : protocols

\sectionend
% }}} END SECTION : layer_6_presentation
