
% SECTION : security {{{
\section{Security}
\label{sec:security}
\parindent=0em

something you know : password , pin , captcha , security questions
something you have
something about you
somewhere you are
something you do : typing speed

federated trust system

% 	SUB-SECTION : symmetric_encryption {{{
\subsection{Symmetric Encryption}
\label{ssec:symmetric_encryption}

RC4
AES

Symmetric Encryption uses one key for both encryption and decryption.

Super basically symmetric encryption will change some plaintext into cyphertext
using a key and an algorithm.

\subsectionend
% }}} END SUB-SECTION : symmetric_encryption

% 	SUB-SECTION : asymmetric_encryption {{{
\subsection{Asymmetric Encryption}
\label{ssec:asymmetric_encryption}

% 		SUB-SUB-SECTION : hashing {{{
\subsubsection{hashing}
\label{sssec:hashing}

MD5
SHA1

hashes confirm data integrity , they are not forms of encryption.

Hash values are always fixed in size.



\subsubsectionend
% }}} END SUB-SUB-SECTION : hashing

% 		SUB-SUB-SECTION : digital_signatures {{{
\subsubsection{Digital Signatures}
\label{sssec:digital_signatures}

Asymmetric Encryption uses two keys. One key only encrypts and the other key
only decrypts. The key that encrypts is also called the public key. The key that
decrypts is called the private key.

Now that we have two keys , we can share one of them with others. The one we
share is the public encryption key. The guy who has this key can then send over
his own public key. This process is known as a key exchange.

Public keys aernt really protected all that much , since the only thing you can
do with it is encrypt data. The only person who can actually decrypt the data is
the guy who has the exact corresponding private decryption key to that public
encryption key.

Keep in mind though that both the public encryption key and the private
decryption key are basically just strings of binary. There is no rule built into
the key itself saying that you can only decrypt with this key. The simple faact
is that the algorithmn generated two keys ( binary strings ) and we happen to be
using one of those strings arbitrarily as a public encryption key and another
one as a private decryption key.

There are two problems that arise here :

1. How do you know that the domain claiming to send the public key is the real
domain you wished to get the public key for ?

2. How do we verify that the the person who sent the information is actually the
owner of the public key that was sent ?

To solve this problem we use a digital signature.


\subsubsectionend
% }}} END SUB-SUB-SECTION : digital_signatures

% 		SUB-SUB-SECTION : certificates {{{
\subsubsection{Certificates}
\label{sssec:certificates}

Digital Certificates and Anti-phishing

When a web browser communicates with a secure (HTTPS) server, it accepts the
server's digital certificate to use its public key to encrypt communications.
Because of the special way that the keys are linked, the public key cannot be
used to decrypt the message once encrypted. Only the linked private key can be
used to do that. The private key must be kept secret. This is referred to as
asymmetric encryption. 

Having a certificate is not in itself any proof of identity. The browser and
server rely upon a third-party—the Certificate Authority (CA)—to vouch for the
server's identity. This framework is called Public Key Infrastructure (PKI).

A browser is pre-installed with a number of root certificates that are
automatically trusted. These represent the commercial CAs that grant
certificates to most of the companies that do business on the web. 

Digital Signature : When we are trying to have secure communications and verify
the integrity of the files sent as well as the identity of the sender ( because
the sender could be some evil dude ) , we use encryption and hashing together. A
digital signature basically serves to verify the owner / sender of the private
key as well as the information being sent.


what I as a sender do is the following :

1. Participate in public key exchange
2. Encrypt the entire information being sent (e.g. webpage) using a private key
3. Hash the entire encrypted page
4. Send the hashed encrypted page to the guy who has my public key and requested
the info
5. Dude will unencrypt the page using the public key

This is called a digital signature

Digital Certificate : A digital certificate is a collection of a public key , a
digital signature of the sending party (i.e. the guy who owns this particular
public key) to verify that the public / private key belongs to the sender , as
well as a third party signature that verifies that the sender actually is who he
says he is.

Unsigned Certificate :

Self Signed Certificate : A self-signed certificate can throw a 443 error, as
the certificate has not been issued by a certifcate authority.

Any expired certificate can be viewed , then fiexed by getting a new certificate
from the issuer or accepting the certificate in its current state.


Web of Trust :

PKI (Public Key Infrastructure) : 

Certificate Authorities (CA) :

CRL

OCSP : 


\subsubsectionend
% }}} END SUB-SUB-SECTION : certificates


\subsectionend
% }}} END SUB-SECTION : asymmetric_encryption

% symmetric encryption {{{


% }}}

% asymmetric encryption {{{




% }}}




% 	SUB-SECTION : wireless_security {{{
\subsection{Wireless Security}
\label{ssec:wireless_security}


\subsectionend
% }}} END SUB-SECTION : wireless_security





% malware {{{

There are types of malware :

- Viruses and Worms
- Tojan Horse
- Adware
- Spyware
- Randomware / Crypto-Malware
- Logic Bomb
- Rootkit and Backdoors


RAT ( Remote Access Trojan )
Polymorphic Malware
Keyloggers
Armored Viruses








% }}}

% DDoS {{{

Distributed Denial of Service

A Denial of Service Attack Basically Prevents other from accessing a system.

A Distrubuted Denial of Service does the same thing as a DoS attack except this
time using a bunch of different computers.

There are three types od DoS ( Denial of Service ) :

- Volume Attack : Flood the server with ping , or udp etc \ldots so that the
server is not able to handle the amount of network traffic and goes down.
- Protocol Attack
- Application Attack


Slow Loris Attack
Smurf Attack

BotNet


% }}}





\sectionend
% }}} END SECTION : security
