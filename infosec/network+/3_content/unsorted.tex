
% unsorted {{{




% commands windows {{{

ipconfig /all
ipconfig /release
ipconfig /renew
ipconfig /displaydns
ipconfig /flushdns

nslookup

arp -a
netstat

route print
route print is the exact same as : netstat -r

tracert www.google.com
pathping www.google.com

ftp
ftp open <ServerName>

net view
net user
net use
net share <filename>=<filepath>
net share <foldername>=<folderpath>


show current computer name
nbtstat -n

show remote cache table
nbtstat -c



% }}}

% commands linux {{{


dig
ifconfig
arp -a
traceroute
tcpdump

% }}}


% qos {{{

QOS (Quality of Service) : QOS Controls help better manage available bandwidth.
One type of quality of service control is traffic shaping.

Quality of Service is the feature that allows you to prioritize certain types of
network traffic over other types of lesser importance.


Traffic Shaping : We can set the different priorities (based on application /
MAC etc \ldots ) of the network traffic that is coming through to always make
maximum use of our bandwidth. Simple QoS on SOHO routers allows priority setting
for different protocols.
% }}}

% vpn {{{

The various VPN Tunneling Protocols are :

- PPTP : Point to Point Tunneling Protocol
- - Encapsulation : Encapsulates PPP frames in IP datagrams
- - Encryption : PPP frame is encrypted with Microsoft Point to Point Encryption
(MPPE)
- - TCP Port 1723
- - GRE (Protocol 47)
- - Older and does not provide data integrity ( proof that the data was not
modified in transit ) , or data origin authentication ( proof that the data was
sent by the authorized user)
- - Support was dropped with some newer operating systems.

- L2TP : Layer 2 Tunneling Protocol
- - Encapsulation : 2 Layers - PPP frame is wrapped in IP datagram then wrapped
with IPsec Encapsulating Security Protocol (ESP)
- - Encryption : IPsec encryption algorithm
- - UDP Port 500 and 4500
- - ESP (Protocol 50)
- - Will support most older clients and can use certificates or preshared key
for IPsec.
- - Support for 3DES and AES encryption algorithms
- - Considered secure when using AES and not a pre-shared key
- - Some difficulty when NAT is involved

- IKEv2 : Internet key exchange version 2
- - Encapsulation : Uses IPsec and the Encapsulating Security Protcol (ESP)
- - Encryption : IPsec encryption algorithm
- - UDP Port 500 and 4500
- - ESP (Protocol 50)
- - Mainly supported by newer clients
- - Some difficulty when NAT is involved
- - Suppoorts VPN RE-connect , MOBIKE

- SSTP : Secure Socket Tunneling Protocol
- - Encapsulation : Encapsulates PPP frames in IP datagrams over port 443.
- - Encryption : Encryption with the SSL channel of the HTTPS protocol.
- - TCP Port 443
- - Pretty much always works becuase it only uses port 443.
- - Support is limited with operating systems other thant windows because it's a
Microsoft owned protocol.

- OpenVPN : This is an open source technology that uses the OpenSSL library and
the TLS protocols. It can be similar to SSTP in that it can be configures to use
port 443. It needs a 3rd party VPN client. Should use with perfect forward
secrecy (PFS) .

PAP
CHAP
MS-CHAP



% }}}

% Tunneling {{{

IP Tunneling

PPTP ( Point to Point Tunneling Protocol )

VPN : A VPN creates a secure tunnel so that a remote machine or network can be
part of a local network.

L2TP/IPSec ( Layer 2 Tunneling Protocol )

SSTP ( Secure Socket Tunneling Protocol )

VPN Concentrator : This is a dedicated box that acts as an endpoint for the
entire network.

client-to-site VPN : A client-to-site VPN connects a remote computer to a local
network.

site-to-site VPN : A site-to-site VPN connects distant networks into a single
network.


% }}}

% Remote Desktop {{{

Introduced by CITRIX

TightVNC : Port 5900

Remote Desktop Server : RDP : Port 3389


% }}}

% san {{{

Storage Area Network (SAN) : A SAN is a high speed network that stores and
provides access to large amounts of data. This is basically a dedicated networok
(or subnet) that is use donly for data storage. This is in opposition to a NAS (
Network attached storage ) device since a NAS would live on the same network but
comes with the disadvantage of a single point of failure. As an example if the
power supply to the NAS fails , then all the computers on that network cannot
use it. A SAN contains mutiple disk arrays , its own siwtches and servers. Since
there are multiple disks arrays that share the data this makes the SAN a lot
more fault tolerant than a NAS. When a server accesses a drive on the SAN it is
accessed as a local drive , as opposed to a network drive as we would see in
NAS. SANs are also very scalable since we can just add more disks and disk
arrays. Another adavantage is that SANs are not affected by the actual network
traffic , so there is no danger of not being able to access files becuase the
network is clogged up. All this being said SANs cost a lot of money so they are
only used by huge companies.
% }}}

% DMZ demilitarized zone {{{

A dmz allows unsoliticied internet traffic from outside the local network to
enter inside. The difference between this and port forwarding is that the DMZ
will send all traffic that tries to get in to one specified machine. Port
forwarding however works on pre specified ports or port ranges that can span
multiple machines. You are basically placing this machine outside of military
procetion zone of your router.


This is super scary cause all bad evil internet traffic can
get in and wreck you. So do not do this on SOHO routers.

DMZs are used to protect public-facingservers by creating an isolated area for
those devices.

Two firewalls are used in a DMZ : one allowing unsolicited traffic to the public
service , and the second maintaining isolation of the private network.





% }}}

% Load Balancing {{{

Load Balancing : This is basically the act of being able to handle the amount of
traffic that is being requested from them. The easiest way to do this is to just
plug in more servers , all of which are serving the exact same content as the
original. But now we need to spread the traffic out amongst all of these so as
not to cause one to overload. This can be done in a variety of ways such as :


 - DNS Load balancing : This can be done using DNS servers. We can send
the information to the servers using a round robin scheduling algorithmn. If the
servers are in different physical locations though (like different continents)
then we want to use the server that would be physically the closest one.

- Server Side Load balancing :

% }}}

% }}}
