


% SECTION : code {{{
\section{Code}
\label{sec:code}
\parindent=0em

\textbf{Python}
\begin{lstlisting}[language=Python]
import numpy as np

var1 = 'Hello World!' # this vairable is here just to see the string color

def incmatrix(genl1,genl2):
	m = len(genl1)
	n = len(genl2)
	M = None #to become the incidence matrix
	VT = np.zeros((n*m,1), int)  #dummy variable

	#compute the bitwise xor matrix
	M1 = bitxormatrix(genl1)
	M2 = np.triu(bitxormatrix(genl2),1) 

	for i in range(m-1):
		for j in range(i+1, m):
			[r,c] = np.where(M2 == M1[i,j])
				for k in range(len(r)):
				VT[(i)*n + r[k]] = 1;
				VT[(i)*n + c[k]] = 1;
				VT[(j)*n + r[k]] = 1;
				VT[(j)*n + c[k]] = 1;

				if M is None:
					M = np.copy(VT)
				else:
					M = np.concatenate((M, VT), 1)

				VT = np.zeros((n*m,1), int)

	return M
\end{lstlisting}

\lipsum[1]

\textbf{C++}
\begin{lstlisting}[language=C++]

#include <iostream>
#include <cmath>
using namespace std;

int main() {

	float a, b, c, x1, x2, discriminant, realPart, imaginaryPart;
	cout << "Enter coefficients a, b and c: ";
	cin >> a >> b >> c;
	discriminant = b*b - 4*a*c;

	if (discriminant > 0) {
		x1 = (-b + sqrt(discriminant)) / (2*a);
		x2 = (-b - sqrt(discriminant)) / (2*a);
		cout << "Roots are real and different." << endl;
		cout << "x1 = " << x1 << endl;
		cout << "x2 = " << x2 << endl;
	}

	else if (discriminant == 0) {
		cout << "Roots are real and same." << endl;
		x1 = (-b + sqrt(discriminant)) / (2*a);
		cout << "x1 = x2 =" << x1 << endl;
	}

	else {
		realPart = -b/(2*a);
		imaginaryPart =sqrt(-discriminant)/(2*a);
		cout << "Roots are complex and different."  << endl;
		cout << "x1 = " << realPart << "+" << imaginaryPart << "i" << endl;
		cout << "x2 = " << realPart << "-" << imaginaryPart << "i" << endl;
	}

	return 0;
}


\end{lstlisting}


\lipsum[1]


\sectionend
% }}} END SECTION : code

