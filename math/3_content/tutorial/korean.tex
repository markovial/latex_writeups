
% SECTION : korean {{{
\section{Korean}
\label{sec:korean}
\parindent=0em

% https://generator.lorem-ipsum.info/_korean
\begin{CJK}{UTF8}{}
 \CJKfamily{mj}

직전대통령이 없을 때에는 대통령이 지명한다. 대통령은 전시·사변 또는 이에 준하는
국가비상사태에 있어서 병력으로써 군사상의 필요에 응하거나 공공의 안녕질서를
유지할 필요가 있을 때에는 법률이 정하는 바에 의하여 계엄을 선포할 수 있다.
대통령의 임기가 만료되는 때에는 임기만료 70일 내지 40일전에 후임자를 선거한다,
감사원은 원장을 포함한 5인 이상 11인 이하의 감사위원으로 구성한다.

국회는 법률에 저촉되지 아니하는 범위안에서 의사와 내부규율에 관한 규칙을 제정할
수 있다. 새로운 회계연도가 개시될 때까지 예산안이 의결되지 못한 때에는 정부는
국회에서 예산안이 의결될 때까지 다음의 목적을 위한 경비는 전년도 예산에 준하여
집행할 수 있다. 1차에 한하여 중임할 수 있다. 재판관은 대통령이 임명한다.

외교사절을 신임·접수 또는 파견하며. 대통령은 즉시 이를 공포하여야 한다. 피고인의
자백이 고문·폭행·협박·구속의 부당한 장기화 또는 기망 기타의 방법에 의하여 자의로
진술된 것이 아니라고 인정될 때 또는 정식재판에 있어서 피고인의 자백이 그에게
불리한 유일한 증거일 때에는 이를 유죄의 증거로 삼거나 이를 이유로 처벌할 수
없다. 경제주체간의 조화를 통한 경제의 민주화를 위하여 경제에 관한 규제와 조정을
할 수 있다.

국회가 재적의원 과반수의 찬성으로 계엄의 해제를 요구한 때에는 대통령은 이를
해제하여야 한다. 국가안전보장에 관련되는 대외정책·군사정책과 국내정책의 수립에
관하여 국무회의의 심의에 앞서 대통령의 자문에 응하기 위하여 국가안전보장회의를
둔다. 선거에 관한 경비는 법률이 정하는 경우를 제외하고는 정당 또는 후보자에게
부담시킬 수 없다. 제1항의 해임건의는 국회재적의원 3분의 1 이상의 발의에 의하여
국회재적의원 과반수의 찬성이 있어야 한다.

다만, 국회의 폐회중에도 또한 같다. 헌법재판소에서 법률의 위헌결정. 탄핵결정은
공직으로부터 파면함에 그친다.

체포·구속·압수 또는 수색을 할 때에는 적법한 절차에 따라 검사의 신청에 의하여
법관이 발부한 영장을 제시하여야 한다, 모든 국민은 인간다운 생활을 할 권리를
가진다. 헌법에 의하여 체결·공포된 조약과 일반적으로 승인된 국제법규는 국내법과
같은 효력을 가진다, 그 자율적 활동과 발전을 보장한다.

대통령은 제3항과 제4항의 사유를 지체없이 공포하여야 한다. 국가는 과학기술의
혁신과 정보 및 인력의 개발을 통하여 국민경제의 발전에 노력하여야 한다. 제1항의
지시를 받은 당해 행정기관은 이에 응하여야 한다. 정당한 보상을 지급하여야 한다.

헌법개정안이 제2항의 찬성을 얻은 때에는 헌법개정은 확정되며, 선거에 있어서
최고득표자가 2인 이상인 때에는 국회의 재적의원 과반수가 출석한 공개회의에서
다수표를 얻은 자를 당선자로 한다. 징계처분에 의하지 아니하고는 정직·감봉 기타
불리한 처분을 받지 아니한다, 법률이 정한 국무위원의 순서로 그 권한을 대행한다.

모든 국민은 행위시의 법률에 의하여 범죄를 구성하지 아니하는 행위로 소추되지
아니하며. 국가는 농지에 관하여 경자유전의 원칙이 달성될 수 있도록 노력하여야
하며. 헌법재판소의 조직과 운영 기타 필요한 사항은 법률로 정한다. 대통령은 국민의
보통·평등·직접·비밀선거에 의하여 선출한다.

선거에 있어서 최고득표자가 2인 이상인 때에는 국회의 재적의원 과반수가 출석한
공개회의에서 다수표를 얻은 자를 당선자로 한다. 국가안전보장회의는 대통령이
주재한다. 교육의 자주성·전문성·정치적 중립성 및 대학의 자율성은 법률이 정하는
바에 의하여 보장된다. 국가는 전통문화의 계승·발전과 민족문화의 창달에 노력하여야
한다.


\end{CJK}
\sectionend
% }}} END SECTION : korean
%\section{ \begin{CJK}{UTF8}{} \CJKfamily{mj} 한국어 \end{CJK} }

%\sectionend

